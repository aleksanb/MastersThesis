\section{Minimum required reservoir size and optimal input connectivity}
\label{section:required_reservoir_size}

\subsection{Description}

Section \ref{section:pre-thesis-project} and \cite{burkow2015evolving}
show that almost all RBN RC systems with $N=100, K=\{2, 3\}$ are able to solve the temporal parity 3 task.
For temporal parity 5, reservoirs with $N=100, K=3$ are barely able to solve the task,
with their $K=\{1, 2\}$ brethren being unable to classify the task correctly.

To find the minimum required reservoir size, as well as how it changes with the temporal memory requirements and complexity of the task at hand,
we create a number of reservoirs with parameters from \ref{tab:ic-reservoir-parameters}.
The four tasks used to benchmark the generated reservoirs are temporal parity 3 and 5 (the number being the window size),
as well as temporal density 3 and 5.
These will sometimes be referred to as TP3, TP5, TD3, and TD5.

As the temporal density task is computationally less expensive than temporal parity \cite{rbn-reservoir},
one can expect a smaller required reservoir size than for temporal parity.
In addition, there should be a 'significant' bump in required reservoir size for larger task windows, as observed in section \ref{section:pre-thesis-project} and \cite{burkow2015evolving}.
The expected relationship between the difficulties of the four tasks therefore becomes $ TD3 \leq TP3 \leq TD5 \leq TP5 $.

\begin{table}[ht]
    \centering
    \caption{Task parameters. The tasks are explained in detail in chapter \ref{section:tasks}}
    \label{tab:tasks}
    \begin{tabular}{ll}
        \hline
        \textbf{Parameter} & \textbf{Configuration} \\
        \hline
        \hline
        Task type               & Temporal parity and temporal density \\
        Training dataset length & 4 000                       \\
        Test dataset length     & 200                         \\
        $N$ (window size)       & 3 and 5                     \\
        $t$ (offset)            & 0 \\
        \hline
    \end{tabular}
\end{table}

\begin{table}[ht]
    \centering
    \caption{Reservoir parameters for optimal input connectivity}
    \label{tab:ic-reservoir-parameters}
    \begin{tabular}{ll}
        \hline
        \textbf{Parameter} & \textbf{Configuration} \\
        \hline
        \hline
        Nodes               & 10 to (100, 140) for TP, 5 to (35, 65) for TD \\
        Node step size      & 10 for TP, 5 for TD \\
        Connectivity        & 3                              \\
        Input connectivity  & [0..n\_nodes], step size = 5   \\
        Output connectivity & n\_nodes                       \\
        Sample size         & 50 \\
        \hline
    \end{tabular}
\end{table}

To find which input connectivity gives the greatest population accuracy,
one iterates over all input connectivities, generating accuracy distributions for each one.
In section \ref{section:pre-thesis-project},
the optimal input connectivity is empirically observed to lie around $ 0.5 \cdot n\_nodes $.
The chaotic reservoirs with $K=3$ have a slight skew to the right of 0.5.
In addition,
the temporal density task might have different input connectivity requirements than the temporal parity task.

The resulting accuracy distributions will be plotted as box plots (as in section \ref{section:pre-thesis-project}).
This should allow for visual identification of the minimum required reservoir sizes,
as well as the optimal input connectivities.

\subsection{Results}
\label{subsection:reservoir_size-input_connectivity:results}

We define a 'Task accuracy threshold' as the smallest reservoir size where at least two reservoirs have the required accuracy on the task.
These are presented in figure \ref{fig:accuracy-threshold-size} and table \ref{tab:accuracy-thresholds}.
The 90\% accuracy threshold is also included in table \ref{tab:accuracy-thresholds} as it appears quite a bit earlier than the 98\% threshold for tasks with a window size of 5.

\begin{figure}[ht]
    \centering
    \caption[Required reservoir sizes to reach the 98\% accuracy threshold for all tasks]{
        Accuracy plots for the required reservoir sizes to reach the 98\% accuracy threshold for each of the four tasks:
        TP3 (Figure \ref{fig:threshold-TP3}), TP5 (Figure \ref{fig:threshold-TP5}), TD3 (Figure \ref{fig:threshold-TD3}) and TD5 (Figure \ref{fig:threshold-TD5}).
        The x-axis for all plots has been normalized to the largest reservoir size, $N=90$.
    }
    \label{fig:accuracy-threshold-size}
    \resizebox{\textwidth}{!}{
        \subfloat[TP3, N=20]{
            \myboxplot{

\addplot[mark=*, boxplot, boxplot/draw position=0]
table[row sep=\\, y index=0] {
data
0.485 \\
0.465 \\
0.46 \\
0.485 \\
0.465 \\
0.465 \\
0.485 \\
0.485 \\
0.53 \\
0.485 \\
0.515 \\
0.51 \\
0.51 \\
0.51 \\
0.465 \\
0.515 \\
0.52 \\
0.53 \\
0.53 \\
0.575 \\
0.465 \\
0.465 \\
0.485 \\
0.515 \\
0.485 \\
0.55 \\
0.465 \\
0.51 \\
0.485 \\
0.51 \\
0.515 \\
0.51 \\
0.49 \\
0.54 \\
0.435 \\
0.465 \\
0.485 \\
0.465 \\
0.495 \\
0.485 \\
0.51 \\
0.485 \\
0.465 \\
0.545 \\
0.485 \\
0.485 \\
0.465 \\
0.485 \\
0.55 \\
0.595 \\
};

\addplot[mark=*, boxplot, boxplot/draw position=1]
table[row sep=\\, y index=0] {
data
0.67 \\
0.465 \\
0.52 \\
0.93 \\
0.895 \\
0.68 \\
0.865 \\
0.59 \\
0.675 \\
0.79 \\
0.63 \\
0.595 \\
0.565 \\
0.815 \\
0.745 \\
0.735 \\
0.585 \\
0.485 \\
0.8 \\
0.65 \\
0.69 \\
0.83 \\
0.82 \\
0.555 \\
0.64 \\
0.645 \\
0.46 \\
0.695 \\
0.625 \\
0.785 \\
0.745 \\
0.46 \\
0.62 \\
0.545 \\
0.835 \\
0.57 \\
0.785 \\
0.565 \\
0.555 \\
0.435 \\
0.69 \\
0.645 \\
0.58 \\
0.555 \\
0.585 \\
0.53 \\
0.67 \\
0.525 \\
0.705 \\
0.685 \\
};

\addplot[mark=*, boxplot, boxplot/draw position=2]
table[row sep=\\, y index=0] {
data
0.92 \\
0.995 \\
0.9 \\
0.76 \\
0.945 \\
0.575 \\
0.96 \\
0.925 \\
0.96 \\
0.995 \\
0.61 \\
0.625 \\
0.615 \\
0.78 \\
0.995 \\
0.885 \\
0.785 \\
0.955 \\
0.995 \\
0.735 \\
0.87 \\
0.94 \\
0.81 \\
0.63 \\
0.995 \\
0.465 \\
0.525 \\
0.525 \\
0.97 \\
0.72 \\
0.55 \\
0.94 \\
0.965 \\
0.885 \\
0.665 \\
0.935 \\
0.7 \\
0.94 \\
0.8 \\
0.66 \\
0.815 \\
0.81 \\
0.87 \\
0.77 \\
0.895 \\
0.945 \\
0.845 \\
0.955 \\
0.925 \\
0.765 \\
};

\addplot[mark=*, boxplot, boxplot/draw position=3]
table[row sep=\\, y index=0] {
data
0.495 \\
0.825 \\
0.495 \\
0.995 \\
0.76 \\
0.495 \\
0.995 \\
0.48 \\
0.495 \\
0.695 \\
0.475 \\
0.495 \\
0.505 \\
0.495 \\
0.495 \\
0.62 \\
0.995 \\
0.525 \\
0.84 \\
0.735 \\
0.495 \\
0.73 \\
0.37 \\
0.65 \\
0.495 \\
0.76 \\
0.675 \\
0.73 \\
0.995 \\
0.995 \\
0.495 \\
0.495 \\
0.495 \\
0.655 \\
0.495 \\
0.995 \\
0.655 \\
0.615 \\
0.995 \\
0.88 \\
0.695 \\
0.995 \\
0.495 \\
0.76 \\
0.87 \\
0.675 \\
0.495 \\
0.87 \\
0.58 \\
0.995 \\
};

\addplot[mark=*, boxplot, boxplot/draw position=4]
table[row sep=\\, y index=0] {
data
0.495 \\
0.495 \\
0.495 \\
0.495 \\
0.495 \\
0.495 \\
0.495 \\
0.495 \\
0.495 \\
0.495 \\
0.495 \\
0.495 \\
0.495 \\
0.495 \\
0.495 \\
0.495 \\
0.495 \\
0.495 \\
0.495 \\
0.495 \\
0.495 \\
0.495 \\
0.495 \\
0.495 \\
0.495 \\
0.495 \\
0.495 \\
0.495 \\
0.495 \\
0.495 \\
0.495 \\
0.495 \\
0.495 \\
0.495 \\
0.495 \\
0.495 \\
0.495 \\
0.495 \\
0.495 \\
0.495 \\
0.495 \\
0.495 \\
0.495 \\
0.495 \\
0.495 \\
0.495 \\
0.495 \\
0.495 \\
0.495 \\
0.495 \\
};
}{0.2}{Input connectivity}{18}

            \label{fig:threshold-TP3}
        }
        \subfloat[TP5, N=90]{
            \myboxplot{

\addplot[mark=*, boxplot, boxplot/draw position=0]
table[row sep=\\, y index=0] {
data
0.47 \\
0.48 \\
0.515 \\
0.485 \\
0.55 \\
0.505 \\
0.49 \\
0.51 \\
0.53 \\
0.535 \\
0.475 \\
0.53 \\
0.45 \\
0.455 \\
0.43 \\
0.515 \\
0.48 \\
0.56 \\
0.43 \\
0.46 \\
0.48 \\
0.51 \\
0.52 \\
0.5 \\
0.54 \\
0.52 \\
0.465 \\
0.455 \\
0.5 \\
0.475 \\
0.46 \\
0.48 \\
0.48 \\
0.505 \\
0.42 \\
0.49 \\
0.51 \\
0.49 \\
0.55 \\
0.545 \\
0.48 \\
0.455 \\
0.435 \\
0.515 \\
0.485 \\
0.48 \\
0.5 \\
0.47 \\
0.48 \\
0.49 \\
};

\addplot[mark=*, boxplot, boxplot/draw position=1]
table[row sep=\\, y index=0] {
data
0.56 \\
0.41 \\
0.46 \\
0.52 \\
0.45 \\
0.485 \\
0.475 \\
0.545 \\
0.485 \\
0.45 \\
0.55 \\
0.495 \\
0.49 \\
0.475 \\
0.495 \\
0.53 \\
0.5 \\
0.565 \\
0.495 \\
0.56 \\
0.46 \\
0.495 \\
0.51 \\
0.535 \\
0.465 \\
0.485 \\
0.47 \\
0.475 \\
0.53 \\
0.485 \\
0.45 \\
0.5 \\
0.455 \\
0.505 \\
0.54 \\
0.54 \\
0.495 \\
0.455 \\
0.55 \\
0.455 \\
0.45 \\
0.515 \\
0.51 \\
0.515 \\
0.445 \\
0.53 \\
0.535 \\
0.43 \\
0.475 \\
0.49 \\
};

\addplot[mark=*, boxplot, boxplot/draw position=2]
table[row sep=\\, y index=0] {
data
0.47 \\
0.505 \\
0.475 \\
0.495 \\
0.425 \\
0.45 \\
0.43 \\
0.525 \\
0.445 \\
0.455 \\
0.445 \\
0.495 \\
0.485 \\
0.5 \\
0.44 \\
0.485 \\
0.475 \\
0.45 \\
0.47 \\
0.475 \\
0.45 \\
0.445 \\
0.535 \\
0.46 \\
0.455 \\
0.525 \\
0.455 \\
0.5 \\
0.47 \\
0.545 \\
0.51 \\
0.58 \\
0.48 \\
0.43 \\
0.49 \\
0.5 \\
0.495 \\
0.46 \\
0.465 \\
0.505 \\
0.445 \\
0.52 \\
0.47 \\
0.5 \\
0.37 \\
0.52 \\
0.54 \\
0.58 \\
0.54 \\
0.53 \\
};

\addplot[mark=*, boxplot, boxplot/draw position=3]
table[row sep=\\, y index=0] {
data
0.505 \\
0.525 \\
0.455 \\
0.53 \\
0.475 \\
0.4 \\
0.575 \\
0.46 \\
0.47 \\
0.495 \\
0.465 \\
0.5 \\
0.51 \\
0.51 \\
0.52 \\
0.44 \\
0.5 \\
0.525 \\
0.565 \\
0.455 \\
0.51 \\
0.505 \\
0.585 \\
0.485 \\
0.515 \\
0.505 \\
0.51 \\
0.43 \\
0.52 \\
0.59 \\
0.53 \\
0.495 \\
0.52 \\
0.475 \\
0.615 \\
0.58 \\
0.57 \\
0.47 \\
0.47 \\
0.495 \\
0.44 \\
0.585 \\
0.47 \\
0.48 \\
0.455 \\
0.535 \\
0.505 \\
0.45 \\
0.48 \\
0.5 \\
};

\addplot[mark=*, boxplot, boxplot/draw position=4]
table[row sep=\\, y index=0] {
data
0.54 \\
0.715 \\
0.5 \\
0.595 \\
0.595 \\
0.585 \\
0.59 \\
0.525 \\
0.465 \\
0.59 \\
0.48 \\
0.575 \\
0.455 \\
0.54 \\
0.63 \\
0.53 \\
0.605 \\
0.49 \\
0.65 \\
0.515 \\
0.535 \\
0.535 \\
0.52 \\
0.505 \\
0.635 \\
0.52 \\
0.635 \\
0.575 \\
0.61 \\
0.65 \\
0.475 \\
0.495 \\
0.545 \\
0.485 \\
0.595 \\
0.58 \\
0.625 \\
0.51 \\
0.575 \\
0.485 \\
0.545 \\
0.5 \\
0.575 \\
0.485 \\
0.51 \\
0.635 \\
0.465 \\
0.52 \\
0.535 \\
0.61 \\
};

\addplot[mark=*, boxplot, boxplot/draw position=5]
table[row sep=\\, y index=0] {
data
0.58 \\
0.655 \\
0.565 \\
0.595 \\
0.545 \\
0.62 \\
0.55 \\
0.635 \\
0.58 \\
0.49 \\
0.755 \\
0.54 \\
0.575 \\
0.47 \\
0.565 \\
0.53 \\
0.55 \\
0.535 \\
0.54 \\
0.545 \\
0.725 \\
0.53 \\
0.675 \\
0.735 \\
0.575 \\
0.55 \\
0.615 \\
0.64 \\
0.64 \\
0.605 \\
0.575 \\
0.485 \\
0.465 \\
0.61 \\
0.625 \\
0.62 \\
0.41 \\
0.65 \\
0.635 \\
0.585 \\
0.655 \\
0.68 \\
0.575 \\
0.65 \\
0.61 \\
0.575 \\
0.805 \\
0.515 \\
0.59 \\
0.705 \\
};

\addplot[mark=*, boxplot, boxplot/draw position=6]
table[row sep=\\, y index=0] {
data
0.735 \\
0.79 \\
0.81 \\
0.635 \\
0.69 \\
0.66 \\
0.625 \\
0.605 \\
0.765 \\
0.66 \\
0.535 \\
0.45 \\
0.67 \\
0.67 \\
0.79 \\
0.66 \\
0.505 \\
0.555 \\
0.615 \\
0.545 \\
0.86 \\
0.57 \\
0.675 \\
0.475 \\
0.76 \\
0.825 \\
0.535 \\
0.635 \\
0.545 \\
0.695 \\
0.59 \\
0.59 \\
0.655 \\
0.56 \\
0.66 \\
0.505 \\
0.5 \\
0.57 \\
0.58 \\
0.725 \\
0.76 \\
0.71 \\
0.595 \\
0.705 \\
0.615 \\
0.66 \\
0.54 \\
0.605 \\
0.57 \\
0.53 \\
};

\addplot[mark=*, boxplot, boxplot/draw position=7]
table[row sep=\\, y index=0] {
data
0.69 \\
0.695 \\
0.745 \\
0.64 \\
0.545 \\
0.62 \\
0.66 \\
0.9 \\
0.88 \\
0.77 \\
0.885 \\
0.85 \\
0.585 \\
0.585 \\
0.63 \\
0.69 \\
0.75 \\
0.905 \\
0.795 \\
0.71 \\
0.52 \\
0.71 \\
0.51 \\
0.71 \\
0.765 \\
0.65 \\
0.85 \\
0.8 \\
0.75 \\
0.67 \\
0.645 \\
0.84 \\
0.83 \\
0.68 \\
0.7 \\
0.785 \\
0.7 \\
0.915 \\
0.77 \\
0.655 \\
0.665 \\
0.635 \\
0.75 \\
0.685 \\
0.645 \\
0.615 \\
0.675 \\
0.815 \\
0.695 \\
0.79 \\
};

\addplot[mark=*, boxplot, boxplot/draw position=8]
table[row sep=\\, y index=0] {
data
0.84 \\
0.825 \\
0.86 \\
0.825 \\
0.65 \\
0.915 \\
0.73 \\
0.7 \\
0.795 \\
0.465 \\
0.73 \\
0.825 \\
0.665 \\
0.765 \\
0.79 \\
0.765 \\
0.66 \\
0.645 \\
0.615 \\
0.64 \\
0.81 \\
0.64 \\
0.715 \\
0.585 \\
0.8 \\
0.665 \\
0.895 \\
0.605 \\
0.815 \\
0.83 \\
0.66 \\
0.73 \\
0.82 \\
0.755 \\
0.73 \\
0.665 \\
0.67 \\
0.78 \\
0.83 \\
0.735 \\
0.92 \\
0.89 \\
0.78 \\
0.65 \\
0.92 \\
0.635 \\
0.835 \\
0.705 \\
0.875 \\
0.65 \\
};

\addplot[mark=*, boxplot, boxplot/draw position=9]
table[row sep=\\, y index=0] {
data
0.88 \\
0.88 \\
0.69 \\
0.79 \\
0.67 \\
0.735 \\
0.845 \\
0.625 \\
0.715 \\
0.575 \\
0.69 \\
0.82 \\
0.95 \\
0.54 \\
0.66 \\
0.795 \\
0.685 \\
0.635 \\
0.835 \\
0.98 \\
0.735 \\
0.66 \\
0.585 \\
0.695 \\
0.6 \\
0.785 \\
0.765 \\
0.675 \\
0.77 \\
0.615 \\
0.665 \\
0.75 \\
0.73 \\
0.75 \\
0.745 \\
0.705 \\
0.64 \\
0.835 \\
0.89 \\
0.76 \\
0.87 \\
0.825 \\
0.66 \\
0.71 \\
0.83 \\
0.535 \\
0.715 \\
0.615 \\
0.72 \\
0.645 \\
};

\addplot[mark=*, boxplot, boxplot/draw position=10]
table[row sep=\\, y index=0] {
data
0.72 \\
0.745 \\
0.565 \\
0.845 \\
0.835 \\
0.74 \\
0.685 \\
0.79 \\
0.63 \\
0.835 \\
0.515 \\
0.76 \\
0.735 \\
0.755 \\
0.685 \\
0.62 \\
0.885 \\
0.725 \\
0.8 \\
0.695 \\
0.67 \\
0.675 \\
0.935 \\
0.6 \\
0.695 \\
0.76 \\
0.645 \\
0.665 \\
0.745 \\
0.61 \\
0.885 \\
0.945 \\
0.825 \\
0.655 \\
0.745 \\
0.74 \\
0.815 \\
0.71 \\
0.885 \\
0.78 \\
0.775 \\
0.58 \\
0.7 \\
0.68 \\
0.555 \\
0.715 \\
0.765 \\
0.805 \\
0.805 \\
0.83 \\
};

\addplot[mark=*, boxplot, boxplot/draw position=11]
table[row sep=\\, y index=0] {
data
0.68 \\
0.99 \\
0.66 \\
0.895 \\
0.725 \\
0.565 \\
0.69 \\
0.72 \\
0.665 \\
0.77 \\
0.61 \\
0.67 \\
0.72 \\
0.5 \\
0.605 \\
0.935 \\
0.555 \\
0.615 \\
0.745 \\
0.64 \\
0.605 \\
0.74 \\
0.665 \\
0.635 \\
0.76 \\
0.585 \\
0.785 \\
0.475 \\
0.69 \\
0.725 \\
0.7 \\
0.615 \\
0.465 \\
0.645 \\
0.625 \\
0.73 \\
0.6 \\
0.61 \\
0.615 \\
0.515 \\
0.525 \\
0.76 \\
0.4 \\
0.63 \\
0.655 \\
0.795 \\
0.655 \\
0.8 \\
0.585 \\
0.74 \\
};

\addplot[mark=*, boxplot, boxplot/draw position=12]
table[row sep=\\, y index=0] {
data
0.635 \\
0.665 \\
0.455 \\
0.43 \\
0.72 \\
0.69 \\
0.57 \\
0.745 \\
0.815 \\
0.69 \\
0.69 \\
0.555 \\
0.525 \\
0.745 \\
0.585 \\
0.46 \\
0.59 \\
0.45 \\
0.475 \\
0.985 \\
0.65 \\
0.725 \\
0.68 \\
0.43 \\
0.625 \\
0.79 \\
0.76 \\
0.705 \\
0.71 \\
0.57 \\
0.52 \\
0.58 \\
0.605 \\
0.495 \\
0.565 \\
0.575 \\
0.81 \\
0.595 \\
0.575 \\
0.88 \\
0.515 \\
0.495 \\
0.745 \\
0.7 \\
0.615 \\
0.375 \\
0.69 \\
0.665 \\
0.795 \\
0.7 \\
};

\addplot[mark=*, boxplot, boxplot/draw position=13]
table[row sep=\\, y index=0] {
data
0.655 \\
0.45 \\
0.48 \\
0.735 \\
0.5 \\
0.75 \\
0.635 \\
0.545 \\
0.575 \\
0.43 \\
0.755 \\
0.62 \\
0.645 \\
0.47 \\
0.69 \\
0.595 \\
0.58 \\
0.59 \\
0.54 \\
0.465 \\
0.72 \\
0.585 \\
0.74 \\
0.785 \\
0.54 \\
0.46 \\
0.43 \\
0.56 \\
0.64 \\
0.465 \\
0.445 \\
0.465 \\
0.515 \\
0.585 \\
0.6 \\
0.515 \\
0.46 \\
0.435 \\
0.515 \\
0.57 \\
0.61 \\
0.55 \\
0.7 \\
0.47 \\
0.43 \\
0.605 \\
0.56 \\
0.63 \\
0.545 \\
0.705 \\
};

\addplot[mark=*, boxplot, boxplot/draw position=14]
table[row sep=\\, y index=0] {
data
0.515 \\
0.485 \\
0.445 \\
0.445 \\
0.47 \\
0.455 \\
0.56 \\
0.445 \\
0.575 \\
0.43 \\
0.43 \\
0.43 \\
0.43 \\
0.58 \\
0.455 \\
0.43 \\
0.415 \\
0.52 \\
0.445 \\
0.595 \\
0.47 \\
0.455 \\
0.47 \\
0.445 \\
0.455 \\
0.495 \\
0.665 \\
0.63 \\
0.47 \\
0.48 \\
0.465 \\
0.535 \\
0.455 \\
0.43 \\
0.43 \\
0.475 \\
0.46 \\
0.43 \\
0.445 \\
0.43 \\
0.535 \\
0.43 \\
0.58 \\
0.505 \\
0.43 \\
0.575 \\
0.43 \\
0.435 \\
0.695 \\
0.445 \\
};

\addplot[mark=*, boxplot, boxplot/draw position=15]
table[row sep=\\, y index=0] {
data
0.63 \\
0.445 \\
0.455 \\
0.445 \\
0.46 \\
0.43 \\
0.43 \\
0.43 \\
0.575 \\
0.525 \\
0.485 \\
0.445 \\
0.635 \\
0.51 \\
0.515 \\
0.47 \\
0.5 \\
0.43 \\
0.415 \\
0.43 \\
0.45 \\
0.47 \\
0.65 \\
0.43 \\
0.455 \\
0.47 \\
0.485 \\
0.43 \\
0.465 \\
0.47 \\
0.43 \\
0.43 \\
0.47 \\
0.43 \\
0.57 \\
0.505 \\
0.57 \\
0.465 \\
0.445 \\
0.47 \\
0.47 \\
0.43 \\
0.47 \\
0.43 \\
0.43 \\
0.47 \\
0.47 \\
0.47 \\
0.4 \\
0.47 \\
};

\addplot[mark=*, boxplot, boxplot/draw position=16]
table[row sep=\\, y index=0] {
data
0.43 \\
0.6 \\
0.47 \\
0.55 \\
0.47 \\
0.43 \\
0.475 \\
0.47 \\
0.47 \\
0.47 \\
0.47 \\
0.515 \\
0.47 \\
0.47 \\
0.47 \\
0.43 \\
0.47 \\
0.47 \\
0.47 \\
0.47 \\
0.59 \\
0.43 \\
0.47 \\
0.47 \\
0.465 \\
0.47 \\
0.43 \\
0.47 \\
0.42 \\
0.445 \\
0.47 \\
0.47 \\
0.47 \\
0.47 \\
0.47 \\
0.445 \\
0.47 \\
0.47 \\
0.445 \\
0.47 \\
0.43 \\
0.45 \\
0.47 \\
0.47 \\
0.47 \\
0.6 \\
0.47 \\
0.555 \\
0.53 \\
0.47 \\
};

\addplot[mark=*, boxplot, boxplot/draw position=17]
table[row sep=\\, y index=0] {
data
0.47 \\
0.535 \\
0.47 \\
0.47 \\
0.47 \\
0.47 \\
0.47 \\
0.47 \\
0.47 \\
0.47 \\
0.47 \\
0.47 \\
0.47 \\
0.47 \\
0.47 \\
0.47 \\
0.47 \\
0.47 \\
0.47 \\
0.43 \\
0.475 \\
0.5 \\
0.47 \\
0.47 \\
0.47 \\
0.47 \\
0.47 \\
0.47 \\
0.47 \\
0.47 \\
0.47 \\
0.47 \\
0.47 \\
0.47 \\
0.47 \\
0.47 \\
0.47 \\
0.47 \\
0.47 \\
0.47 \\
0.43 \\
0.47 \\
0.47 \\
0.47 \\
0.47 \\
0.47 \\
0.545 \\
0.47 \\
0.47 \\
0.47 \\
};

\addplot[mark=*, boxplot, boxplot/draw position=18]
table[row sep=\\, y index=0] {
data
0.47 \\
0.47 \\
0.47 \\
0.47 \\
0.47 \\
0.47 \\
0.47 \\
0.47 \\
0.47 \\
0.47 \\
0.47 \\
0.47 \\
0.47 \\
0.47 \\
0.47 \\
0.47 \\
0.47 \\
0.47 \\
0.47 \\
0.47 \\
0.47 \\
0.47 \\
0.47 \\
0.47 \\
0.47 \\
0.47 \\
0.47 \\
0.47 \\
0.47 \\
0.47 \\
0.47 \\
0.47 \\
0.47 \\
0.47 \\
0.47 \\
0.47 \\
0.47 \\
0.47 \\
0.47 \\
0.47 \\
0.47 \\
0.47 \\
0.47 \\
0.47 \\
0.47 \\
0.47 \\
0.47 \\
0.47 \\
0.47 \\
0.47 \\
};
}{0.2}{Input connectivity}{18}

            \label{fig:threshold-TP5}
        }
    }
    \resizebox{\textwidth}{!}{
        \subfloat[TD3, N=10]{
            \myboxplot{

\addplot[mark=*, boxplot, boxplot/draw position=0]
table[row sep=\\, y index=0] {
data
0.595 \\
0.595 \\
0.57 \\
0.595 \\
0.595 \\
0.595 \\
0.595 \\
0.55 \\
0.595 \\
0.585 \\
0.57 \\
0.595 \\
0.595 \\
0.595 \\
0.545 \\
0.595 \\
0.595 \\
0.595 \\
0.595 \\
0.57 \\
0.595 \\
0.585 \\
0.57 \\
0.595 \\
0.61 \\
0.595 \\
0.595 \\
0.595 \\
0.595 \\
0.595 \\
0.595 \\
0.595 \\
0.615 \\
0.55 \\
0.57 \\
0.595 \\
0.585 \\
0.595 \\
0.585 \\
0.595 \\
0.57 \\
0.595 \\
0.595 \\
0.595 \\
0.595 \\
0.595 \\
0.595 \\
0.55 \\
0.55 \\
0.595 \\
};

\addplot[mark=*, boxplot, boxplot/draw position=1]
table[row sep=\\, y index=0] {
data
0.9 \\
1.0 \\
0.72 \\
0.95 \\
0.9 \\
0.8 \\
0.87 \\
0.94 \\
0.745 \\
0.805 \\
0.875 \\
0.86 \\
0.885 \\
0.8 \\
0.8 \\
0.82 \\
0.91 \\
0.745 \\
0.84 \\
1.0 \\
0.89 \\
0.805 \\
0.9 \\
0.8 \\
0.89 \\
0.815 \\
0.92 \\
0.815 \\
0.81 \\
0.785 \\
0.8 \\
0.855 \\
0.8 \\
0.745 \\
0.745 \\
0.725 \\
0.745 \\
0.805 \\
0.8 \\
0.9 \\
0.9 \\
0.9 \\
1.0 \\
0.99 \\
0.96 \\
0.935 \\
0.87 \\
0.875 \\
0.835 \\
0.845 \\
};

\addplot[mark=*, boxplot, boxplot/draw position=2]
table[row sep=\\, y index=0] {
data
0.745 \\
0.745 \\
0.745 \\
0.745 \\
0.745 \\
0.745 \\
0.745 \\
0.745 \\
0.745 \\
0.745 \\
0.745 \\
0.745 \\
0.745 \\
0.745 \\
0.745 \\
0.745 \\
0.745 \\
0.745 \\
0.745 \\
0.745 \\
0.745 \\
0.745 \\
0.745 \\
0.745 \\
0.745 \\
0.745 \\
0.745 \\
0.745 \\
0.745 \\
0.745 \\
0.745 \\
0.745 \\
0.745 \\
0.745 \\
0.745 \\
0.745 \\
0.745 \\
0.745 \\
0.745 \\
0.745 \\
0.745 \\
0.745 \\
0.745 \\
0.745 \\
0.745 \\
0.745 \\
0.745 \\
0.745 \\
0.745 \\
0.745 \\
};
}{0.2}{Input connectivity}{18}

            \label{fig:threshold-TD3}
        }
        \subfloat[TD5, N=55]{
            \myboxplot{

\addplot[mark=*, boxplot, boxplot/draw position=0]
table[row sep=\\, y index=0] {
data
0.505 \\
0.475 \\
0.505 \\
0.475 \\
0.55 \\
0.545 \\
0.51 \\
0.52 \\
0.535 \\
0.595 \\
0.54 \\
0.555 \\
0.485 \\
0.48 \\
0.51 \\
0.54 \\
0.53 \\
0.525 \\
0.495 \\
0.56 \\
0.555 \\
0.48 \\
0.525 \\
0.47 \\
0.535 \\
0.535 \\
0.435 \\
0.535 \\
0.495 \\
0.46 \\
0.495 \\
0.52 \\
0.565 \\
0.535 \\
0.485 \\
0.515 \\
0.52 \\
0.55 \\
0.53 \\
0.535 \\
0.535 \\
0.48 \\
0.49 \\
0.51 \\
0.585 \\
0.485 \\
0.42 \\
0.535 \\
0.47 \\
0.52 \\
};

\addplot[mark=*, boxplot, boxplot/draw position=1]
table[row sep=\\, y index=0] {
data
0.78 \\
0.725 \\
0.665 \\
0.75 \\
0.735 \\
0.705 \\
0.71 \\
0.77 \\
0.755 \\
0.715 \\
0.74 \\
0.745 \\
0.74 \\
0.735 \\
0.895 \\
0.71 \\
0.84 \\
0.77 \\
0.79 \\
0.75 \\
0.82 \\
0.9 \\
0.88 \\
0.725 \\
0.77 \\
0.77 \\
0.7 \\
0.75 \\
0.765 \\
0.815 \\
0.75 \\
0.85 \\
0.805 \\
0.745 \\
0.775 \\
0.705 \\
0.735 \\
0.745 \\
0.655 \\
0.725 \\
0.705 \\
0.94 \\
0.725 \\
0.72 \\
0.79 \\
0.755 \\
0.68 \\
0.775 \\
0.72 \\
0.705 \\
};

\addplot[mark=*, boxplot, boxplot/draw position=2]
table[row sep=\\, y index=0] {
data
0.78 \\
0.865 \\
0.77 \\
0.81 \\
0.795 \\
0.775 \\
0.805 \\
0.805 \\
0.855 \\
0.81 \\
0.855 \\
0.735 \\
0.805 \\
0.79 \\
0.785 \\
0.815 \\
0.76 \\
0.835 \\
0.845 \\
0.785 \\
0.755 \\
0.785 \\
0.77 \\
0.76 \\
0.775 \\
0.895 \\
0.82 \\
0.87 \\
0.895 \\
0.87 \\
0.735 \\
0.795 \\
0.805 \\
0.79 \\
0.83 \\
0.885 \\
0.885 \\
0.775 \\
0.835 \\
0.8 \\
0.79 \\
0.84 \\
0.8 \\
0.79 \\
0.835 \\
0.815 \\
0.87 \\
0.715 \\
0.825 \\
0.785 \\
};

\addplot[mark=*, boxplot, boxplot/draw position=3]
table[row sep=\\, y index=0] {
data
0.82 \\
0.82 \\
0.76 \\
0.825 \\
0.85 \\
0.82 \\
0.815 \\
0.97 \\
0.795 \\
0.79 \\
0.77 \\
0.82 \\
0.795 \\
0.865 \\
0.835 \\
0.86 \\
0.805 \\
0.895 \\
0.91 \\
0.895 \\
0.73 \\
0.855 \\
0.805 \\
0.82 \\
0.87 \\
0.935 \\
0.905 \\
0.895 \\
0.855 \\
0.77 \\
0.805 \\
0.915 \\
0.865 \\
0.75 \\
0.855 \\
0.805 \\
0.865 \\
0.845 \\
0.835 \\
0.845 \\
0.935 \\
0.84 \\
0.805 \\
0.875 \\
0.79 \\
0.845 \\
0.855 \\
0.795 \\
0.86 \\
0.89 \\
};

\addplot[mark=*, boxplot, boxplot/draw position=4]
table[row sep=\\, y index=0] {
data
0.915 \\
0.83 \\
0.85 \\
0.76 \\
0.84 \\
0.86 \\
0.905 \\
0.775 \\
0.85 \\
0.76 \\
0.905 \\
0.95 \\
0.99 \\
0.905 \\
0.87 \\
0.815 \\
0.845 \\
0.875 \\
0.82 \\
0.9 \\
0.835 \\
0.815 \\
0.86 \\
0.805 \\
0.92 \\
0.845 \\
0.95 \\
0.855 \\
0.935 \\
0.85 \\
0.885 \\
0.825 \\
0.91 \\
0.755 \\
0.835 \\
0.885 \\
0.9 \\
0.86 \\
0.8 \\
0.795 \\
0.915 \\
0.875 \\
0.885 \\
0.93 \\
0.805 \\
0.945 \\
0.92 \\
0.89 \\
0.89 \\
0.9 \\
};

\addplot[mark=*, boxplot, boxplot/draw position=5]
table[row sep=\\, y index=0] {
data
0.865 \\
0.915 \\
0.855 \\
0.95 \\
0.9 \\
0.885 \\
0.92 \\
0.9 \\
0.86 \\
0.925 \\
0.855 \\
0.885 \\
0.795 \\
0.9 \\
0.85 \\
0.88 \\
0.78 \\
0.96 \\
0.845 \\
0.83 \\
0.825 \\
0.92 \\
0.91 \\
0.845 \\
0.83 \\
0.95 \\
0.76 \\
0.87 \\
0.91 \\
0.925 \\
0.84 \\
0.77 \\
0.885 \\
0.915 \\
0.92 \\
0.835 \\
0.975 \\
0.89 \\
0.86 \\
0.95 \\
0.865 \\
0.91 \\
0.825 \\
0.905 \\
0.855 \\
0.965 \\
0.955 \\
0.915 \\
0.82 \\
0.89 \\
};

\addplot[mark=*, boxplot, boxplot/draw position=6]
table[row sep=\\, y index=0] {
data
0.88 \\
0.875 \\
0.895 \\
0.945 \\
0.91 \\
0.9 \\
0.85 \\
0.82 \\
0.915 \\
0.85 \\
0.84 \\
0.915 \\
0.895 \\
0.98 \\
0.895 \\
0.915 \\
0.825 \\
0.85 \\
0.86 \\
0.81 \\
0.84 \\
0.885 \\
0.855 \\
0.865 \\
0.93 \\
0.805 \\
0.87 \\
0.855 \\
0.965 \\
0.905 \\
0.845 \\
0.865 \\
0.84 \\
0.805 \\
0.82 \\
0.885 \\
0.855 \\
0.905 \\
0.82 \\
0.935 \\
0.925 \\
0.93 \\
0.855 \\
0.825 \\
0.835 \\
0.835 \\
0.855 \\
0.91 \\
0.7 \\
0.915 \\
};

\addplot[mark=*, boxplot, boxplot/draw position=7]
table[row sep=\\, y index=0] {
data
0.87 \\
0.815 \\
0.82 \\
0.805 \\
0.87 \\
0.845 \\
0.795 \\
0.845 \\
0.775 \\
0.84 \\
0.835 \\
0.925 \\
0.96 \\
0.7 \\
0.79 \\
0.82 \\
0.885 \\
0.865 \\
0.82 \\
0.8 \\
0.815 \\
0.89 \\
0.805 \\
0.81 \\
0.7 \\
0.845 \\
0.75 \\
0.795 \\
0.825 \\
0.76 \\
0.81 \\
0.795 \\
0.87 \\
0.83 \\
0.855 \\
0.94 \\
0.865 \\
0.85 \\
0.83 \\
0.83 \\
0.865 \\
0.865 \\
0.845 \\
0.805 \\
0.84 \\
0.805 \\
0.795 \\
0.915 \\
0.82 \\
0.79 \\
};

\addplot[mark=*, boxplot, boxplot/draw position=8]
table[row sep=\\, y index=0] {
data
0.755 \\
0.7 \\
0.755 \\
0.82 \\
0.82 \\
0.82 \\
0.82 \\
0.825 \\
0.82 \\
0.82 \\
0.895 \\
0.805 \\
0.885 \\
0.815 \\
0.805 \\
0.82 \\
0.82 \\
0.82 \\
0.765 \\
0.795 \\
0.835 \\
0.81 \\
0.835 \\
0.805 \\
0.7 \\
0.775 \\
0.82 \\
0.74 \\
0.84 \\
0.765 \\
0.81 \\
0.865 \\
0.795 \\
0.845 \\
0.755 \\
0.82 \\
0.775 \\
0.765 \\
0.82 \\
0.9 \\
0.7 \\
0.82 \\
0.74 \\
0.82 \\
0.82 \\
0.81 \\
0.81 \\
0.755 \\
0.74 \\
0.82 \\
};

\addplot[mark=*, boxplot, boxplot/draw position=9]
table[row sep=\\, y index=0] {
data
0.765 \\
0.765 \\
0.82 \\
0.675 \\
0.755 \\
0.755 \\
0.72 \\
0.82 \\
0.81 \\
0.755 \\
0.755 \\
0.72 \\
0.765 \\
0.7 \\
0.705 \\
0.82 \\
0.735 \\
0.755 \\
0.75 \\
0.82 \\
0.82 \\
0.7 \\
0.765 \\
0.755 \\
0.82 \\
0.7 \\
0.7 \\
0.7 \\
0.82 \\
0.7 \\
0.82 \\
0.75 \\
0.7 \\
0.7 \\
0.84 \\
0.805 \\
0.82 \\
0.765 \\
0.7 \\
0.755 \\
0.725 \\
0.81 \\
0.835 \\
0.72 \\
0.755 \\
0.7 \\
0.76 \\
0.755 \\
0.84 \\
0.82 \\
};

\addplot[mark=*, boxplot, boxplot/draw position=10]
table[row sep=\\, y index=0] {
data
0.7 \\
0.7 \\
0.7 \\
0.7 \\
0.765 \\
0.765 \\
0.68 \\
0.7 \\
0.755 \\
0.7 \\
0.7 \\
0.7 \\
0.765 \\
0.755 \\
0.68 \\
0.82 \\
0.7 \\
0.7 \\
0.82 \\
0.82 \\
0.7 \\
0.7 \\
0.755 \\
0.82 \\
0.7 \\
0.7 \\
0.765 \\
0.7 \\
0.84 \\
0.82 \\
0.7 \\
0.7 \\
0.7 \\
0.755 \\
0.68 \\
0.755 \\
0.7 \\
0.7 \\
0.7 \\
0.7 \\
0.7 \\
0.68 \\
0.81 \\
0.7 \\
0.7 \\
0.7 \\
0.765 \\
0.7 \\
0.7 \\
0.7 \\
};

\addplot[mark=*, boxplot, boxplot/draw position=11]
table[row sep=\\, y index=0] {
data
0.68 \\
0.68 \\
0.68 \\
0.68 \\
0.68 \\
0.68 \\
0.68 \\
0.68 \\
0.68 \\
0.68 \\
0.68 \\
0.68 \\
0.68 \\
0.68 \\
0.68 \\
0.68 \\
0.68 \\
0.68 \\
0.68 \\
0.68 \\
0.68 \\
0.68 \\
0.68 \\
0.68 \\
0.68 \\
0.68 \\
0.68 \\
0.68 \\
0.68 \\
0.68 \\
0.68 \\
0.68 \\
0.68 \\
0.68 \\
0.68 \\
0.68 \\
0.68 \\
0.68 \\
0.68 \\
0.68 \\
0.68 \\
0.68 \\
0.68 \\
0.68 \\
0.68 \\
0.68 \\
0.68 \\
0.68 \\
0.68 \\
0.68 \\
};
}{0.2}{Input connectivity}{18}

            \label{fig:threshold-TD5}
        }
    }
\end{figure}

\begin{table}[ht]
    \centering
    \caption{Accuracy thresholds for all four tasks.}
    \label{tab:accuracy-thresholds}
    \begin{tabular}{lllll}
    \hline
    \hline
                            & \textbf{TP3} & \textbf{TP5} & \textbf{TD3} & \textbf{TD5} \\
    \hline
    90\% accuracy threshold & ~15 & 70  & 5   & 30  \\
    98\% accuracy threshold & 20  & 90  & 10  & 55  \\
    \hline
    \end{tabular}
\end{table}

The number of input connectivity plots resulting from the reservoir (table \ref{tab:ic-reservoir-parameters}) and task parameter (table \ref{tab:tasks}) combinations is quite large.
Therefore only reservoir sizes of $ N=[10...30, 80...100]$ from the accuracy distributions on the temporal parity 3 task is shown here (figure \ref{fig:TP3-IC}).
There is a slight skew to each side of $ 0.5 \cdot n\ _nodes $ dependent on whether the task chosen is temporal parity or temporal density.
This is observable in figure \ref{fig:TP3-IC} for TP3 and appendix \ref{app:reservoir_size-input_connectivity} figures \ref{fig:TP3-IC-1} through \ref{fig:TD5-IC-2} for the remaining tasks.

\begin{figure*}[ht]
    \centering
    \caption[Abridged plots for input connectivity against accuracy for temporal parity 3]{
        Plots of input connectivity against accuracy on TP3. Reservoir sizes $[40..70]$ are omitted for brevity.
        Note that the optimal input connectivity tends slightly to the right of the middle for all reservoir sizes.
        The omitted plots are presented in figures \ref{fig:TP3-IC-1} and \ref{fig:TP3-IC-2} in appendix \ref{app:reservoir_size-input_connectivity}.
        }
    \label{fig:TP3-IC}
    \resizebox{\textwidth}{!}{
        \subfloat[N=10]{
            \myboxplot{

\addplot[mark=*, boxplot, boxplot/draw position=0]
table[row sep=\\, y index=0] {
data
0.515 \\
0.465 \\
0.51 \\
0.51 \\
0.55 \\
0.51 \\
0.485 \\
0.485 \\
0.465 \\
0.485 \\
0.47 \\
0.5 \\
0.55 \\
0.49 \\
0.435 \\
0.485 \\
0.51 \\
0.55 \\
0.515 \\
0.535 \\
0.51 \\
0.51 \\
0.485 \\
0.465 \\
0.485 \\
0.575 \\
0.485 \\
0.485 \\
0.465 \\
0.515 \\
0.535 \\
0.465 \\
0.485 \\
0.51 \\
0.485 \\
0.485 \\
0.485 \\
0.49 \\
0.485 \\
0.565 \\
0.465 \\
0.515 \\
0.485 \\
0.485 \\
0.485 \\
0.51 \\
0.485 \\
0.55 \\
0.485 \\
0.485 \\
};

\addplot[mark=*, boxplot, boxplot/draw position=1]
table[row sep=\\, y index=0] {
data
0.785 \\
0.89 \\
0.48 \\
0.495 \\
0.62 \\
0.635 \\
0.735 \\
0.82 \\
0.495 \\
0.65 \\
0.635 \\
0.635 \\
0.73 \\
0.785 \\
0.735 \\
0.815 \\
0.65 \\
0.825 \\
0.69 \\
0.65 \\
0.515 \\
0.665 \\
0.67 \\
0.495 \\
0.495 \\
0.525 \\
0.645 \\
0.72 \\
0.755 \\
0.71 \\
0.635 \\
0.65 \\
0.535 \\
0.78 \\
0.73 \\
0.755 \\
0.775 \\
0.67 \\
0.715 \\
0.495 \\
0.485 \\
0.705 \\
0.66 \\
0.615 \\
0.495 \\
0.495 \\
0.495 \\
0.495 \\
0.87 \\
0.675 \\
};

\addplot[mark=*, boxplot, boxplot/draw position=2]
table[row sep=\\, y index=0] {
data
0.495 \\
0.495 \\
0.495 \\
0.495 \\
0.495 \\
0.495 \\
0.495 \\
0.495 \\
0.495 \\
0.495 \\
0.495 \\
0.495 \\
0.495 \\
0.495 \\
0.495 \\
0.495 \\
0.495 \\
0.495 \\
0.495 \\
0.495 \\
0.495 \\
0.495 \\
0.495 \\
0.495 \\
0.495 \\
0.495 \\
0.495 \\
0.495 \\
0.495 \\
0.495 \\
0.495 \\
0.495 \\
0.495 \\
0.495 \\
0.495 \\
0.495 \\
0.495 \\
0.495 \\
0.495 \\
0.495 \\
0.495 \\
0.495 \\
0.495 \\
0.495 \\
0.495 \\
0.495 \\
0.495 \\
0.495 \\
0.495 \\
0.495 \\
};
}{0.2}{Input connectivity}{21}

        }
        \subfloat[N=80]{
            \myboxplot{

\addplot[mark=*, boxplot, boxplot/draw position=0]
table[row sep=\\, y index=0] {
data
0.485 \\
0.495 \\
0.5 \\
0.485 \\
0.47 \\
0.45 \\
0.495 \\
0.465 \\
0.535 \\
0.47 \\
0.48 \\
0.505 \\
0.51 \\
0.425 \\
0.485 \\
0.5 \\
0.485 \\
0.52 \\
0.51 \\
0.6 \\
0.555 \\
0.46 \\
0.48 \\
0.495 \\
0.49 \\
0.47 \\
0.485 \\
0.57 \\
0.485 \\
0.495 \\
0.51 \\
0.545 \\
0.465 \\
0.56 \\
0.535 \\
0.515 \\
0.505 \\
0.505 \\
0.465 \\
0.525 \\
0.545 \\
0.455 \\
0.495 \\
0.455 \\
0.51 \\
0.56 \\
0.495 \\
0.505 \\
0.52 \\
0.455 \\
};

\addplot[mark=*, boxplot, boxplot/draw position=1]
table[row sep=\\, y index=0] {
data
0.605 \\
0.46 \\
0.485 \\
0.455 \\
0.55 \\
0.495 \\
0.525 \\
0.475 \\
0.55 \\
0.46 \\
0.475 \\
0.485 \\
0.645 \\
0.62 \\
0.585 \\
0.565 \\
0.665 \\
0.505 \\
0.545 \\
0.55 \\
0.55 \\
0.615 \\
0.47 \\
0.475 \\
0.65 \\
0.43 \\
0.46 \\
0.57 \\
0.45 \\
0.5 \\
0.54 \\
0.5 \\
0.505 \\
0.505 \\
0.575 \\
0.49 \\
0.495 \\
0.565 \\
0.49 \\
0.49 \\
0.505 \\
0.55 \\
0.545 \\
0.465 \\
0.47 \\
0.54 \\
0.45 \\
0.615 \\
0.485 \\
0.53 \\
};

\addplot[mark=*, boxplot, boxplot/draw position=2]
table[row sep=\\, y index=0] {
data
0.51 \\
0.545 \\
0.63 \\
0.82 \\
0.615 \\
0.645 \\
0.59 \\
0.825 \\
0.69 \\
0.57 \\
0.6 \\
0.725 \\
0.69 \\
0.67 \\
0.795 \\
0.6 \\
0.68 \\
0.62 \\
0.565 \\
0.57 \\
0.795 \\
0.685 \\
0.635 \\
0.745 \\
0.895 \\
0.58 \\
0.64 \\
0.565 \\
0.61 \\
0.535 \\
0.535 \\
0.535 \\
0.72 \\
0.695 \\
0.995 \\
0.5 \\
0.765 \\
0.575 \\
0.51 \\
0.765 \\
0.615 \\
0.685 \\
0.565 \\
0.61 \\
0.605 \\
0.51 \\
0.615 \\
0.735 \\
0.595 \\
0.94 \\
};

\addplot[mark=*, boxplot, boxplot/draw position=3]
table[row sep=\\, y index=0] {
data
0.75 \\
0.57 \\
0.73 \\
0.635 \\
0.59 \\
0.75 \\
0.82 \\
0.715 \\
0.8 \\
0.71 \\
0.92 \\
0.815 \\
0.88 \\
0.74 \\
0.8 \\
0.995 \\
0.705 \\
0.69 \\
0.85 \\
0.83 \\
0.65 \\
0.855 \\
0.645 \\
0.67 \\
0.89 \\
0.685 \\
0.875 \\
0.58 \\
0.635 \\
0.78 \\
0.675 \\
0.895 \\
0.79 \\
0.615 \\
0.745 \\
0.475 \\
0.755 \\
0.655 \\
0.805 \\
0.575 \\
0.725 \\
0.67 \\
0.6 \\
0.945 \\
0.905 \\
0.75 \\
0.81 \\
0.69 \\
0.65 \\
0.74 \\
};

\addplot[mark=*, boxplot, boxplot/draw position=4]
table[row sep=\\, y index=0] {
data
0.695 \\
0.7 \\
0.945 \\
0.64 \\
0.935 \\
0.96 \\
0.905 \\
0.945 \\
0.68 \\
0.695 \\
0.77 \\
0.755 \\
0.97 \\
0.65 \\
0.78 \\
0.985 \\
0.745 \\
0.635 \\
0.995 \\
0.81 \\
0.825 \\
0.865 \\
0.9 \\
0.895 \\
0.87 \\
0.81 \\
0.9 \\
0.98 \\
0.885 \\
0.875 \\
0.755 \\
0.995 \\
0.915 \\
0.77 \\
0.89 \\
0.76 \\
0.805 \\
0.59 \\
0.84 \\
0.99 \\
0.725 \\
0.905 \\
0.755 \\
0.92 \\
0.93 \\
0.81 \\
0.985 \\
0.845 \\
0.995 \\
0.92 \\
};

\addplot[mark=*, boxplot, boxplot/draw position=5]
table[row sep=\\, y index=0] {
data
0.955 \\
0.93 \\
0.85 \\
0.835 \\
0.83 \\
0.85 \\
0.93 \\
0.895 \\
0.71 \\
0.94 \\
0.895 \\
0.955 \\
0.81 \\
0.99 \\
0.995 \\
0.97 \\
0.905 \\
0.89 \\
0.875 \\
0.84 \\
0.995 \\
0.995 \\
0.86 \\
0.96 \\
0.87 \\
0.945 \\
0.85 \\
0.99 \\
0.995 \\
0.965 \\
0.95 \\
0.815 \\
0.94 \\
0.995 \\
0.985 \\
0.875 \\
0.83 \\
0.945 \\
0.83 \\
0.895 \\
0.85 \\
0.995 \\
0.985 \\
0.73 \\
0.935 \\
0.95 \\
0.995 \\
0.87 \\
0.995 \\
0.91 \\
};

\addplot[mark=*, boxplot, boxplot/draw position=6]
table[row sep=\\, y index=0] {
data
0.995 \\
0.995 \\
0.865 \\
0.995 \\
0.975 \\
0.97 \\
0.955 \\
0.99 \\
0.95 \\
0.835 \\
0.98 \\
0.995 \\
0.995 \\
0.975 \\
0.96 \\
0.995 \\
0.975 \\
0.995 \\
0.895 \\
0.99 \\
0.995 \\
0.93 \\
0.965 \\
0.995 \\
0.975 \\
0.845 \\
0.995 \\
0.91 \\
0.895 \\
0.975 \\
0.975 \\
0.82 \\
0.805 \\
0.995 \\
0.995 \\
0.945 \\
0.995 \\
0.995 \\
0.97 \\
0.995 \\
0.995 \\
0.995 \\
0.895 \\
0.995 \\
0.955 \\
0.995 \\
0.94 \\
0.945 \\
0.995 \\
0.865 \\
};

\addplot[mark=*, boxplot, boxplot/draw position=7]
table[row sep=\\, y index=0] {
data
0.99 \\
0.985 \\
0.93 \\
0.99 \\
0.995 \\
0.995 \\
0.995 \\
0.995 \\
0.995 \\
0.995 \\
0.91 \\
0.99 \\
0.995 \\
0.965 \\
0.85 \\
0.995 \\
0.97 \\
0.95 \\
0.97 \\
0.995 \\
0.975 \\
0.99 \\
0.99 \\
0.995 \\
0.945 \\
0.975 \\
0.995 \\
0.96 \\
0.975 \\
0.995 \\
0.965 \\
0.93 \\
0.995 \\
0.995 \\
0.995 \\
0.995 \\
0.995 \\
0.995 \\
0.99 \\
0.965 \\
0.995 \\
0.785 \\
0.975 \\
0.995 \\
0.91 \\
0.95 \\
0.96 \\
0.995 \\
0.94 \\
0.975 \\
};

\addplot[mark=*, boxplot, boxplot/draw position=8]
table[row sep=\\, y index=0] {
data
0.845 \\
0.995 \\
0.995 \\
0.995 \\
0.995 \\
0.975 \\
0.955 \\
0.985 \\
0.995 \\
0.995 \\
0.995 \\
0.99 \\
0.995 \\
0.995 \\
0.995 \\
0.995 \\
0.995 \\
0.98 \\
0.995 \\
0.995 \\
0.94 \\
0.995 \\
0.995 \\
0.99 \\
0.995 \\
0.965 \\
0.995 \\
0.885 \\
0.995 \\
0.995 \\
0.995 \\
0.995 \\
0.96 \\
0.995 \\
0.965 \\
0.995 \\
0.98 \\
0.965 \\
0.995 \\
0.99 \\
0.95 \\
0.995 \\
0.995 \\
0.995 \\
0.995 \\
0.985 \\
0.995 \\
0.93 \\
0.995 \\
0.945 \\
};

\addplot[mark=*, boxplot, boxplot/draw position=9]
table[row sep=\\, y index=0] {
data
0.965 \\
0.96 \\
0.985 \\
0.975 \\
0.995 \\
0.965 \\
0.995 \\
0.995 \\
0.89 \\
0.995 \\
0.995 \\
0.995 \\
0.93 \\
0.995 \\
0.995 \\
0.965 \\
0.995 \\
0.995 \\
0.98 \\
0.995 \\
0.995 \\
0.995 \\
0.995 \\
0.995 \\
0.995 \\
0.995 \\
0.995 \\
0.9 \\
0.995 \\
0.995 \\
0.995 \\
0.995 \\
0.995 \\
0.995 \\
0.88 \\
0.925 \\
0.995 \\
0.805 \\
0.995 \\
0.995 \\
0.995 \\
0.885 \\
0.995 \\
0.995 \\
0.995 \\
0.995 \\
0.995 \\
0.995 \\
0.995 \\
0.995 \\
};

\addplot[mark=*, boxplot, boxplot/draw position=10]
table[row sep=\\, y index=0] {
data
0.995 \\
0.995 \\
0.995 \\
0.995 \\
0.97 \\
0.995 \\
0.995 \\
0.995 \\
0.995 \\
0.995 \\
0.995 \\
0.995 \\
0.995 \\
0.9 \\
0.995 \\
0.95 \\
0.995 \\
0.995 \\
0.995 \\
0.99 \\
0.995 \\
0.995 \\
0.97 \\
0.995 \\
0.98 \\
0.995 \\
0.995 \\
0.995 \\
0.995 \\
0.995 \\
0.995 \\
0.995 \\
0.995 \\
0.995 \\
0.995 \\
0.995 \\
0.995 \\
0.995 \\
0.99 \\
0.995 \\
0.995 \\
0.995 \\
0.995 \\
0.995 \\
0.995 \\
0.995 \\
0.985 \\
0.99 \\
0.995 \\
0.98 \\
};

\addplot[mark=*, boxplot, boxplot/draw position=11]
table[row sep=\\, y index=0] {
data
0.95 \\
0.83 \\
0.995 \\
0.995 \\
0.91 \\
0.995 \\
0.79 \\
0.995 \\
0.97 \\
0.995 \\
0.995 \\
0.995 \\
0.87 \\
0.995 \\
0.69 \\
0.975 \\
0.73 \\
0.995 \\
0.995 \\
0.995 \\
0.88 \\
0.995 \\
0.59 \\
0.995 \\
0.995 \\
0.995 \\
0.995 \\
0.96 \\
0.995 \\
0.995 \\
0.87 \\
0.995 \\
0.95 \\
0.995 \\
0.995 \\
0.995 \\
0.915 \\
0.995 \\
0.995 \\
0.995 \\
0.995 \\
0.88 \\
0.995 \\
0.72 \\
0.6 \\
0.995 \\
0.995 \\
0.935 \\
0.995 \\
0.995 \\
};

\addplot[mark=*, boxplot, boxplot/draw position=12]
table[row sep=\\, y index=0] {
data
0.76 \\
0.995 \\
0.955 \\
0.995 \\
0.995 \\
0.96 \\
0.995 \\
0.495 \\
0.995 \\
0.48 \\
0.995 \\
0.995 \\
0.995 \\
0.995 \\
0.995 \\
0.935 \\
0.995 \\
0.995 \\
0.895 \\
0.97 \\
0.995 \\
0.995 \\
0.735 \\
0.995 \\
0.87 \\
0.965 \\
0.995 \\
0.995 \\
0.785 \\
0.955 \\
0.94 \\
0.995 \\
0.915 \\
0.995 \\
0.745 \\
0.72 \\
0.995 \\
0.995 \\
0.985 \\
0.935 \\
0.995 \\
0.765 \\
0.755 \\
0.93 \\
0.83 \\
0.995 \\
0.995 \\
0.87 \\
0.995 \\
0.865 \\
};

\addplot[mark=*, boxplot, boxplot/draw position=13]
table[row sep=\\, y index=0] {
data
0.495 \\
0.495 \\
0.735 \\
0.8 \\
0.37 \\
0.945 \\
0.995 \\
0.995 \\
0.495 \\
0.62 \\
0.88 \\
0.885 \\
0.76 \\
0.87 \\
0.495 \\
0.76 \\
0.995 \\
0.995 \\
0.935 \\
0.995 \\
0.995 \\
0.995 \\
0.995 \\
0.65 \\
0.495 \\
0.995 \\
0.87 \\
0.87 \\
0.89 \\
0.795 \\
0.795 \\
0.88 \\
0.995 \\
0.885 \\
0.995 \\
0.495 \\
0.495 \\
0.995 \\
0.995 \\
0.88 \\
0.995 \\
0.945 \\
0.995 \\
0.995 \\
0.995 \\
0.995 \\
0.9 \\
0.995 \\
0.995 \\
0.495 \\
};

\addplot[mark=*, boxplot, boxplot/draw position=14]
table[row sep=\\, y index=0] {
data
0.76 \\
0.495 \\
0.495 \\
0.995 \\
0.88 \\
0.73 \\
0.945 \\
0.615 \\
0.495 \\
0.495 \\
0.875 \\
0.635 \\
0.495 \\
0.76 \\
0.995 \\
0.88 \\
0.995 \\
0.76 \\
0.995 \\
0.73 \\
0.495 \\
0.87 \\
0.73 \\
0.495 \\
0.73 \\
0.995 \\
0.995 \\
0.995 \\
0.73 \\
0.995 \\
0.495 \\
0.67 \\
0.665 \\
0.495 \\
0.76 \\
0.875 \\
0.76 \\
0.995 \\
0.69 \\
0.9 \\
0.495 \\
0.76 \\
0.995 \\
0.73 \\
0.995 \\
0.73 \\
0.845 \\
0.48 \\
0.48 \\
0.62 \\
};

\addplot[mark=*, boxplot, boxplot/draw position=15]
table[row sep=\\, y index=0] {
data
0.495 \\
0.495 \\
0.495 \\
0.495 \\
0.66 \\
0.495 \\
0.495 \\
0.73 \\
0.495 \\
0.495 \\
0.495 \\
0.495 \\
0.495 \\
0.495 \\
0.635 \\
0.675 \\
0.995 \\
0.495 \\
0.495 \\
0.495 \\
0.495 \\
0.495 \\
0.495 \\
0.76 \\
0.495 \\
0.495 \\
0.495 \\
0.495 \\
0.73 \\
0.635 \\
0.495 \\
0.495 \\
0.495 \\
0.495 \\
0.73 \\
0.495 \\
0.495 \\
0.495 \\
0.495 \\
0.495 \\
0.495 \\
0.635 \\
0.995 \\
0.48 \\
0.665 \\
0.495 \\
0.76 \\
0.495 \\
0.495 \\
0.495 \\
};

\addplot[mark=*, boxplot, boxplot/draw position=16]
table[row sep=\\, y index=0] {
data
0.495 \\
0.495 \\
0.495 \\
0.495 \\
0.495 \\
0.495 \\
0.495 \\
0.495 \\
0.495 \\
0.495 \\
0.495 \\
0.495 \\
0.495 \\
0.495 \\
0.495 \\
0.495 \\
0.495 \\
0.495 \\
0.495 \\
0.495 \\
0.495 \\
0.495 \\
0.495 \\
0.495 \\
0.495 \\
0.495 \\
0.495 \\
0.495 \\
0.495 \\
0.495 \\
0.495 \\
0.495 \\
0.495 \\
0.495 \\
0.495 \\
0.495 \\
0.495 \\
0.495 \\
0.495 \\
0.495 \\
0.495 \\
0.495 \\
0.495 \\
0.495 \\
0.495 \\
0.495 \\
0.495 \\
0.495 \\
0.495 \\
0.495 \\
};
}{0.2}{Input connectivity}{21}

        }
    }
    \resizebox{\textwidth}{!}{
        \subfloat[N=20]{
            \myboxplot{

\addplot[mark=*, boxplot, boxplot/draw position=0]
table[row sep=\\, y index=0] {
data
0.485 \\
0.465 \\
0.46 \\
0.485 \\
0.465 \\
0.465 \\
0.485 \\
0.485 \\
0.53 \\
0.485 \\
0.515 \\
0.51 \\
0.51 \\
0.51 \\
0.465 \\
0.515 \\
0.52 \\
0.53 \\
0.53 \\
0.575 \\
0.465 \\
0.465 \\
0.485 \\
0.515 \\
0.485 \\
0.55 \\
0.465 \\
0.51 \\
0.485 \\
0.51 \\
0.515 \\
0.51 \\
0.49 \\
0.54 \\
0.435 \\
0.465 \\
0.485 \\
0.465 \\
0.495 \\
0.485 \\
0.51 \\
0.485 \\
0.465 \\
0.545 \\
0.485 \\
0.485 \\
0.465 \\
0.485 \\
0.55 \\
0.595 \\
};

\addplot[mark=*, boxplot, boxplot/draw position=1]
table[row sep=\\, y index=0] {
data
0.67 \\
0.465 \\
0.52 \\
0.93 \\
0.895 \\
0.68 \\
0.865 \\
0.59 \\
0.675 \\
0.79 \\
0.63 \\
0.595 \\
0.565 \\
0.815 \\
0.745 \\
0.735 \\
0.585 \\
0.485 \\
0.8 \\
0.65 \\
0.69 \\
0.83 \\
0.82 \\
0.555 \\
0.64 \\
0.645 \\
0.46 \\
0.695 \\
0.625 \\
0.785 \\
0.745 \\
0.46 \\
0.62 \\
0.545 \\
0.835 \\
0.57 \\
0.785 \\
0.565 \\
0.555 \\
0.435 \\
0.69 \\
0.645 \\
0.58 \\
0.555 \\
0.585 \\
0.53 \\
0.67 \\
0.525 \\
0.705 \\
0.685 \\
};

\addplot[mark=*, boxplot, boxplot/draw position=2]
table[row sep=\\, y index=0] {
data
0.92 \\
0.995 \\
0.9 \\
0.76 \\
0.945 \\
0.575 \\
0.96 \\
0.925 \\
0.96 \\
0.995 \\
0.61 \\
0.625 \\
0.615 \\
0.78 \\
0.995 \\
0.885 \\
0.785 \\
0.955 \\
0.995 \\
0.735 \\
0.87 \\
0.94 \\
0.81 \\
0.63 \\
0.995 \\
0.465 \\
0.525 \\
0.525 \\
0.97 \\
0.72 \\
0.55 \\
0.94 \\
0.965 \\
0.885 \\
0.665 \\
0.935 \\
0.7 \\
0.94 \\
0.8 \\
0.66 \\
0.815 \\
0.81 \\
0.87 \\
0.77 \\
0.895 \\
0.945 \\
0.845 \\
0.955 \\
0.925 \\
0.765 \\
};

\addplot[mark=*, boxplot, boxplot/draw position=3]
table[row sep=\\, y index=0] {
data
0.495 \\
0.825 \\
0.495 \\
0.995 \\
0.76 \\
0.495 \\
0.995 \\
0.48 \\
0.495 \\
0.695 \\
0.475 \\
0.495 \\
0.505 \\
0.495 \\
0.495 \\
0.62 \\
0.995 \\
0.525 \\
0.84 \\
0.735 \\
0.495 \\
0.73 \\
0.37 \\
0.65 \\
0.495 \\
0.76 \\
0.675 \\
0.73 \\
0.995 \\
0.995 \\
0.495 \\
0.495 \\
0.495 \\
0.655 \\
0.495 \\
0.995 \\
0.655 \\
0.615 \\
0.995 \\
0.88 \\
0.695 \\
0.995 \\
0.495 \\
0.76 \\
0.87 \\
0.675 \\
0.495 \\
0.87 \\
0.58 \\
0.995 \\
};

\addplot[mark=*, boxplot, boxplot/draw position=4]
table[row sep=\\, y index=0] {
data
0.495 \\
0.495 \\
0.495 \\
0.495 \\
0.495 \\
0.495 \\
0.495 \\
0.495 \\
0.495 \\
0.495 \\
0.495 \\
0.495 \\
0.495 \\
0.495 \\
0.495 \\
0.495 \\
0.495 \\
0.495 \\
0.495 \\
0.495 \\
0.495 \\
0.495 \\
0.495 \\
0.495 \\
0.495 \\
0.495 \\
0.495 \\
0.495 \\
0.495 \\
0.495 \\
0.495 \\
0.495 \\
0.495 \\
0.495 \\
0.495 \\
0.495 \\
0.495 \\
0.495 \\
0.495 \\
0.495 \\
0.495 \\
0.495 \\
0.495 \\
0.495 \\
0.495 \\
0.495 \\
0.495 \\
0.495 \\
0.495 \\
0.495 \\
};
}{0.2}{Input connectivity}{21}

        }
        \subfloat[N=90]{
            \myboxplot{

\addplot[mark=*, boxplot, boxplot/draw position=0]
table[row sep=\\, y index=0] {
data
0.51 \\
0.535 \\
0.56 \\
0.535 \\
0.495 \\
0.465 \\
0.485 \\
0.45 \\
0.485 \\
0.485 \\
0.485 \\
0.455 \\
0.47 \\
0.385 \\
0.505 \\
0.57 \\
0.495 \\
0.485 \\
0.5 \\
0.54 \\
0.495 \\
0.495 \\
0.545 \\
0.515 \\
0.465 \\
0.565 \\
0.445 \\
0.465 \\
0.525 \\
0.505 \\
0.51 \\
0.49 \\
0.47 \\
0.53 \\
0.46 \\
0.48 \\
0.495 \\
0.455 \\
0.49 \\
0.51 \\
0.465 \\
0.555 \\
0.52 \\
0.41 \\
0.455 \\
0.535 \\
0.45 \\
0.485 \\
0.49 \\
0.495 \\
};

\addplot[mark=*, boxplot, boxplot/draw position=1]
table[row sep=\\, y index=0] {
data
0.59 \\
0.855 \\
0.51 \\
0.465 \\
0.575 \\
0.6 \\
0.465 \\
0.53 \\
0.53 \\
0.51 \\
0.595 \\
0.525 \\
0.535 \\
0.58 \\
0.515 \\
0.49 \\
0.57 \\
0.555 \\
0.56 \\
0.445 \\
0.47 \\
0.47 \\
0.505 \\
0.555 \\
0.47 \\
0.575 \\
0.45 \\
0.765 \\
0.47 \\
0.515 \\
0.58 \\
0.52 \\
0.495 \\
0.49 \\
0.515 \\
0.575 \\
0.72 \\
0.535 \\
0.495 \\
0.53 \\
0.48 \\
0.545 \\
0.555 \\
0.54 \\
0.475 \\
0.53 \\
0.565 \\
0.545 \\
0.595 \\
0.655 \\
};

\addplot[mark=*, boxplot, boxplot/draw position=2]
table[row sep=\\, y index=0] {
data
0.725 \\
0.61 \\
0.44 \\
0.58 \\
0.475 \\
0.695 \\
0.585 \\
0.54 \\
0.555 \\
0.595 \\
0.62 \\
0.65 \\
0.625 \\
0.505 \\
0.575 \\
0.82 \\
0.72 \\
0.475 \\
0.77 \\
0.805 \\
0.505 \\
0.655 \\
0.67 \\
0.545 \\
0.5 \\
0.615 \\
0.73 \\
0.545 \\
0.815 \\
0.635 \\
0.515 \\
0.66 \\
0.565 \\
0.56 \\
0.675 \\
0.53 \\
0.665 \\
0.545 \\
0.43 \\
0.62 \\
0.545 \\
0.615 \\
0.565 \\
0.625 \\
0.565 \\
0.73 \\
0.655 \\
0.575 \\
0.63 \\
0.675 \\
};

\addplot[mark=*, boxplot, boxplot/draw position=3]
table[row sep=\\, y index=0] {
data
0.73 \\
0.825 \\
0.99 \\
0.815 \\
0.845 \\
0.755 \\
0.62 \\
0.875 \\
0.69 \\
0.67 \\
0.65 \\
0.95 \\
0.67 \\
0.78 \\
0.705 \\
0.75 \\
0.785 \\
0.825 \\
0.71 \\
0.75 \\
0.615 \\
0.77 \\
0.785 \\
0.91 \\
0.645 \\
0.555 \\
0.53 \\
0.71 \\
0.545 \\
0.605 \\
0.83 \\
0.88 \\
0.795 \\
0.885 \\
0.635 \\
0.765 \\
0.81 \\
0.76 \\
0.685 \\
0.57 \\
0.795 \\
0.91 \\
0.87 \\
0.73 \\
0.54 \\
0.935 \\
0.825 \\
0.695 \\
0.68 \\
0.68 \\
};

\addplot[mark=*, boxplot, boxplot/draw position=4]
table[row sep=\\, y index=0] {
data
0.82 \\
0.74 \\
0.84 \\
0.73 \\
0.98 \\
0.72 \\
0.99 \\
0.895 \\
0.675 \\
0.77 \\
0.91 \\
0.77 \\
0.895 \\
0.705 \\
0.785 \\
0.77 \\
0.77 \\
0.925 \\
0.85 \\
0.97 \\
0.99 \\
0.995 \\
0.995 \\
0.69 \\
0.965 \\
0.87 \\
0.925 \\
0.955 \\
0.915 \\
0.86 \\
0.875 \\
0.76 \\
0.76 \\
0.895 \\
0.915 \\
0.89 \\
0.805 \\
0.7 \\
0.995 \\
0.82 \\
0.785 \\
0.74 \\
0.745 \\
0.89 \\
0.755 \\
0.865 \\
0.64 \\
0.69 \\
0.915 \\
0.745 \\
};

\addplot[mark=*, boxplot, boxplot/draw position=5]
table[row sep=\\, y index=0] {
data
0.975 \\
0.83 \\
0.915 \\
0.98 \\
0.95 \\
0.845 \\
0.795 \\
0.945 \\
0.995 \\
0.77 \\
0.855 \\
0.925 \\
0.905 \\
0.855 \\
0.95 \\
0.995 \\
0.995 \\
0.9 \\
0.905 \\
0.995 \\
0.98 \\
0.855 \\
0.735 \\
0.765 \\
0.93 \\
0.98 \\
0.85 \\
0.94 \\
0.995 \\
0.725 \\
0.985 \\
0.91 \\
0.875 \\
0.875 \\
0.985 \\
0.93 \\
0.93 \\
0.995 \\
0.96 \\
0.84 \\
0.785 \\
0.815 \\
0.995 \\
0.885 \\
0.78 \\
0.96 \\
0.93 \\
0.88 \\
0.96 \\
0.97 \\
};

\addplot[mark=*, boxplot, boxplot/draw position=6]
table[row sep=\\, y index=0] {
data
0.885 \\
0.995 \\
0.87 \\
0.985 \\
0.975 \\
0.88 \\
0.9 \\
0.93 \\
0.99 \\
0.98 \\
0.995 \\
0.995 \\
0.985 \\
0.995 \\
0.995 \\
0.995 \\
0.905 \\
0.865 \\
0.825 \\
0.98 \\
0.965 \\
0.995 \\
0.97 \\
0.86 \\
0.93 \\
0.855 \\
0.99 \\
0.995 \\
0.995 \\
0.915 \\
0.97 \\
0.985 \\
0.98 \\
0.775 \\
0.975 \\
0.985 \\
0.905 \\
0.93 \\
0.93 \\
0.985 \\
0.99 \\
0.85 \\
0.955 \\
0.91 \\
0.945 \\
0.885 \\
0.95 \\
0.995 \\
0.995 \\
0.9 \\
};

\addplot[mark=*, boxplot, boxplot/draw position=7]
table[row sep=\\, y index=0] {
data
0.955 \\
0.975 \\
0.995 \\
0.97 \\
0.995 \\
0.985 \\
0.985 \\
0.97 \\
0.94 \\
0.965 \\
0.935 \\
0.98 \\
0.975 \\
0.985 \\
0.995 \\
0.995 \\
0.955 \\
0.97 \\
0.995 \\
0.99 \\
0.995 \\
0.975 \\
0.995 \\
0.995 \\
0.995 \\
0.995 \\
0.97 \\
0.925 \\
0.995 \\
0.995 \\
0.965 \\
0.945 \\
0.995 \\
0.975 \\
0.995 \\
0.995 \\
0.97 \\
0.995 \\
0.88 \\
0.87 \\
0.925 \\
0.92 \\
0.995 \\
0.985 \\
0.995 \\
0.995 \\
0.945 \\
0.965 \\
0.995 \\
0.98 \\
};

\addplot[mark=*, boxplot, boxplot/draw position=8]
table[row sep=\\, y index=0] {
data
0.995 \\
0.995 \\
0.995 \\
0.97 \\
0.995 \\
0.985 \\
0.995 \\
0.99 \\
0.995 \\
0.995 \\
0.99 \\
0.995 \\
0.99 \\
0.995 \\
0.985 \\
0.945 \\
0.995 \\
0.985 \\
0.995 \\
0.995 \\
0.97 \\
0.995 \\
0.995 \\
0.995 \\
0.935 \\
0.995 \\
0.995 \\
0.97 \\
0.995 \\
0.89 \\
0.965 \\
0.895 \\
0.995 \\
0.96 \\
0.995 \\
0.99 \\
0.995 \\
0.995 \\
0.995 \\
0.995 \\
0.995 \\
0.995 \\
0.995 \\
0.98 \\
0.995 \\
0.995 \\
0.995 \\
0.985 \\
0.995 \\
0.995 \\
};

\addplot[mark=*, boxplot, boxplot/draw position=9]
table[row sep=\\, y index=0] {
data
0.995 \\
0.945 \\
0.995 \\
0.995 \\
0.995 \\
0.995 \\
0.99 \\
0.995 \\
0.995 \\
0.995 \\
0.985 \\
0.995 \\
0.995 \\
0.995 \\
0.995 \\
0.975 \\
0.995 \\
0.98 \\
0.995 \\
0.78 \\
0.995 \\
0.995 \\
0.995 \\
0.995 \\
0.995 \\
0.995 \\
0.995 \\
0.995 \\
0.995 \\
0.995 \\
0.995 \\
0.945 \\
0.995 \\
0.995 \\
0.995 \\
0.995 \\
0.945 \\
0.995 \\
0.99 \\
0.93 \\
0.995 \\
0.99 \\
0.98 \\
0.995 \\
0.995 \\
0.94 \\
0.94 \\
0.995 \\
0.94 \\
0.995 \\
};

\addplot[mark=*, boxplot, boxplot/draw position=10]
table[row sep=\\, y index=0] {
data
0.995 \\
0.995 \\
0.995 \\
0.995 \\
0.995 \\
0.995 \\
0.995 \\
0.995 \\
0.995 \\
0.995 \\
0.995 \\
0.995 \\
0.995 \\
0.98 \\
0.995 \\
0.88 \\
0.995 \\
0.995 \\
0.995 \\
0.995 \\
0.995 \\
0.995 \\
0.995 \\
0.995 \\
0.995 \\
0.995 \\
0.995 \\
0.995 \\
0.995 \\
0.995 \\
0.98 \\
0.975 \\
0.995 \\
0.995 \\
0.995 \\
0.995 \\
0.995 \\
0.995 \\
0.995 \\
0.995 \\
0.995 \\
0.995 \\
0.995 \\
0.995 \\
0.995 \\
0.995 \\
0.995 \\
0.995 \\
0.995 \\
0.995 \\
};

\addplot[mark=*, boxplot, boxplot/draw position=11]
table[row sep=\\, y index=0] {
data
0.995 \\
0.995 \\
0.995 \\
0.995 \\
0.95 \\
0.995 \\
0.995 \\
0.995 \\
0.95 \\
0.995 \\
0.995 \\
0.97 \\
0.995 \\
0.995 \\
0.995 \\
0.995 \\
0.995 \\
0.995 \\
0.995 \\
0.995 \\
0.945 \\
0.995 \\
0.93 \\
0.995 \\
0.995 \\
0.995 \\
0.995 \\
0.995 \\
0.985 \\
0.995 \\
0.995 \\
0.995 \\
0.995 \\
0.995 \\
0.995 \\
0.995 \\
0.995 \\
0.995 \\
0.97 \\
0.995 \\
0.995 \\
0.995 \\
0.995 \\
0.995 \\
0.995 \\
0.995 \\
0.995 \\
0.995 \\
0.995 \\
0.995 \\
};

\addplot[mark=*, boxplot, boxplot/draw position=12]
table[row sep=\\, y index=0] {
data
0.995 \\
0.955 \\
0.995 \\
0.995 \\
0.995 \\
0.97 \\
0.995 \\
0.995 \\
0.73 \\
0.995 \\
0.995 \\
0.995 \\
0.995 \\
0.87 \\
0.995 \\
0.995 \\
0.995 \\
0.995 \\
0.99 \\
0.935 \\
0.995 \\
0.995 \\
0.995 \\
0.995 \\
0.995 \\
0.995 \\
0.845 \\
0.995 \\
0.995 \\
0.995 \\
0.995 \\
0.93 \\
0.995 \\
0.995 \\
0.995 \\
0.995 \\
0.995 \\
0.995 \\
0.855 \\
0.995 \\
0.935 \\
0.995 \\
0.98 \\
0.995 \\
0.995 \\
0.995 \\
0.995 \\
0.995 \\
0.995 \\
0.995 \\
};

\addplot[mark=*, boxplot, boxplot/draw position=13]
table[row sep=\\, y index=0] {
data
0.995 \\
0.995 \\
0.995 \\
0.995 \\
0.935 \\
0.995 \\
0.995 \\
0.995 \\
0.9 \\
0.95 \\
0.995 \\
0.995 \\
0.995 \\
0.995 \\
0.85 \\
0.995 \\
0.995 \\
0.995 \\
0.995 \\
0.995 \\
0.935 \\
0.995 \\
0.995 \\
0.995 \\
0.89 \\
0.995 \\
0.995 \\
0.995 \\
0.97 \\
0.835 \\
0.685 \\
0.525 \\
0.935 \\
0.995 \\
0.89 \\
0.995 \\
0.995 \\
0.995 \\
0.895 \\
0.995 \\
0.99 \\
0.995 \\
0.995 \\
0.995 \\
0.995 \\
0.995 \\
0.995 \\
0.995 \\
0.995 \\
0.995 \\
};

\addplot[mark=*, boxplot, boxplot/draw position=14]
table[row sep=\\, y index=0] {
data
0.905 \\
0.77 \\
0.995 \\
0.76 \\
0.995 \\
0.995 \\
0.995 \\
0.73 \\
0.995 \\
0.965 \\
0.93 \\
0.995 \\
0.995 \\
0.495 \\
0.875 \\
0.88 \\
0.87 \\
0.755 \\
0.995 \\
0.995 \\
0.635 \\
0.76 \\
0.76 \\
0.995 \\
0.995 \\
0.995 \\
0.96 \\
0.995 \\
0.995 \\
0.995 \\
0.88 \\
0.875 \\
0.94 \\
0.995 \\
0.73 \\
0.885 \\
0.995 \\
0.995 \\
0.995 \\
0.87 \\
0.995 \\
0.76 \\
0.975 \\
0.73 \\
0.995 \\
0.94 \\
0.76 \\
0.96 \\
0.995 \\
0.995 \\
};

\addplot[mark=*, boxplot, boxplot/draw position=15]
table[row sep=\\, y index=0] {
data
0.995 \\
0.635 \\
0.87 \\
0.495 \\
0.495 \\
0.945 \\
0.76 \\
0.815 \\
0.495 \\
0.495 \\
0.995 \\
0.76 \\
0.495 \\
0.995 \\
0.635 \\
0.995 \\
0.37 \\
0.735 \\
0.495 \\
0.995 \\
0.745 \\
0.995 \\
0.945 \\
0.825 \\
0.505 \\
0.495 \\
0.995 \\
0.995 \\
0.935 \\
0.935 \\
0.755 \\
0.76 \\
0.76 \\
0.495 \\
0.87 \\
0.995 \\
0.88 \\
0.995 \\
0.77 \\
0.73 \\
0.995 \\
0.73 \\
0.995 \\
0.755 \\
0.995 \\
0.995 \\
0.995 \\
0.76 \\
0.73 \\
0.945 \\
};

\addplot[mark=*, boxplot, boxplot/draw position=16]
table[row sep=\\, y index=0] {
data
0.73 \\
0.995 \\
0.735 \\
0.635 \\
0.495 \\
0.995 \\
0.685 \\
0.935 \\
0.735 \\
0.67 \\
0.76 \\
0.765 \\
0.495 \\
0.495 \\
0.615 \\
0.67 \\
0.495 \\
0.495 \\
0.76 \\
0.935 \\
0.855 \\
0.84 \\
0.8 \\
0.755 \\
0.495 \\
0.785 \\
0.995 \\
0.495 \\
0.635 \\
0.495 \\
0.615 \\
0.495 \\
0.495 \\
0.9 \\
0.61 \\
0.87 \\
0.495 \\
0.915 \\
0.665 \\
0.87 \\
0.615 \\
0.495 \\
0.495 \\
0.95 \\
0.9 \\
0.515 \\
0.73 \\
0.995 \\
0.495 \\
0.67 \\
};

\addplot[mark=*, boxplot, boxplot/draw position=17]
table[row sep=\\, y index=0] {
data
0.495 \\
0.495 \\
0.475 \\
0.495 \\
0.495 \\
0.495 \\
0.495 \\
0.495 \\
0.495 \\
0.495 \\
0.495 \\
0.495 \\
0.495 \\
0.495 \\
0.495 \\
0.995 \\
0.495 \\
0.495 \\
0.495 \\
0.76 \\
0.995 \\
0.495 \\
0.495 \\
0.495 \\
0.495 \\
0.495 \\
0.495 \\
0.495 \\
0.495 \\
0.755 \\
0.495 \\
0.495 \\
0.495 \\
0.495 \\
0.495 \\
0.495 \\
0.495 \\
0.73 \\
0.495 \\
0.495 \\
0.615 \\
0.495 \\
0.495 \\
0.755 \\
0.73 \\
0.495 \\
0.495 \\
0.495 \\
0.495 \\
0.495 \\
};

\addplot[mark=*, boxplot, boxplot/draw position=18]
table[row sep=\\, y index=0] {
data
0.495 \\
0.495 \\
0.495 \\
0.495 \\
0.495 \\
0.495 \\
0.495 \\
0.495 \\
0.495 \\
0.495 \\
0.495 \\
0.495 \\
0.495 \\
0.495 \\
0.495 \\
0.495 \\
0.495 \\
0.495 \\
0.495 \\
0.495 \\
0.495 \\
0.495 \\
0.495 \\
0.495 \\
0.495 \\
0.495 \\
0.495 \\
0.495 \\
0.495 \\
0.495 \\
0.495 \\
0.495 \\
0.495 \\
0.495 \\
0.495 \\
0.495 \\
0.495 \\
0.495 \\
0.495 \\
0.495 \\
0.495 \\
0.495 \\
0.495 \\
0.495 \\
0.495 \\
0.495 \\
0.495 \\
0.495 \\
0.495 \\
0.495 \\
};
}{0.2}{Input connectivity}{21}

        }
    }
    \resizebox{\textwidth}{!}{
        \subfloat[N=30]{
            \myboxplot{

\addplot[mark=*, boxplot, boxplot/draw position=0]
table[row sep=\\, y index=0] {
data
0.51 \\
0.515 \\
0.485 \\
0.49 \\
0.515 \\
0.55 \\
0.485 \\
0.485 \\
0.495 \\
0.465 \\
0.56 \\
0.485 \\
0.465 \\
0.515 \\
0.485 \\
0.5 \\
0.475 \\
0.505 \\
0.515 \\
0.525 \\
0.51 \\
0.495 \\
0.49 \\
0.465 \\
0.505 \\
0.485 \\
0.55 \\
0.515 \\
0.435 \\
0.485 \\
0.575 \\
0.51 \\
0.485 \\
0.585 \\
0.485 \\
0.455 \\
0.485 \\
0.55 \\
0.55 \\
0.535 \\
0.515 \\
0.485 \\
0.495 \\
0.545 \\
0.495 \\
0.545 \\
0.43 \\
0.51 \\
0.495 \\
0.485 \\
};

\addplot[mark=*, boxplot, boxplot/draw position=1]
table[row sep=\\, y index=0] {
data
0.895 \\
0.695 \\
0.535 \\
0.81 \\
0.64 \\
0.67 \\
0.475 \\
0.535 \\
0.73 \\
0.575 \\
0.705 \\
0.61 \\
0.52 \\
0.665 \\
0.73 \\
0.495 \\
0.625 \\
0.59 \\
0.475 \\
0.755 \\
0.515 \\
0.6 \\
0.69 \\
0.665 \\
0.7 \\
0.78 \\
0.565 \\
0.66 \\
0.49 \\
0.65 \\
0.63 \\
0.6 \\
0.575 \\
0.72 \\
0.53 \\
0.535 \\
0.445 \\
0.585 \\
0.73 \\
0.495 \\
0.565 \\
0.615 \\
0.535 \\
0.655 \\
0.59 \\
0.57 \\
0.765 \\
0.69 \\
0.63 \\
0.52 \\
};

\addplot[mark=*, boxplot, boxplot/draw position=2]
table[row sep=\\, y index=0] {
data
0.73 \\
0.49 \\
0.75 \\
0.78 \\
0.995 \\
0.59 \\
0.91 \\
0.65 \\
0.655 \\
0.76 \\
0.81 \\
0.64 \\
0.83 \\
0.86 \\
0.505 \\
0.585 \\
0.975 \\
0.69 \\
0.755 \\
0.715 \\
0.83 \\
0.585 \\
0.855 \\
0.725 \\
0.825 \\
0.71 \\
0.73 \\
0.995 \\
0.73 \\
0.865 \\
0.76 \\
0.92 \\
0.665 \\
0.69 \\
0.61 \\
0.7 \\
0.765 \\
0.875 \\
0.79 \\
0.875 \\
0.855 \\
0.775 \\
0.495 \\
0.79 \\
0.865 \\
0.875 \\
0.8 \\
0.905 \\
0.81 \\
0.625 \\
};

\addplot[mark=*, boxplot, boxplot/draw position=3]
table[row sep=\\, y index=0] {
data
0.99 \\
0.755 \\
0.96 \\
0.85 \\
0.825 \\
0.995 \\
0.995 \\
0.62 \\
0.79 \\
0.77 \\
0.97 \\
0.95 \\
0.915 \\
0.965 \\
0.78 \\
0.885 \\
0.97 \\
0.95 \\
0.835 \\
0.62 \\
0.69 \\
0.995 \\
0.69 \\
0.7 \\
0.955 \\
0.92 \\
0.8 \\
0.785 \\
0.495 \\
0.955 \\
0.995 \\
0.965 \\
0.88 \\
0.85 \\
0.88 \\
0.87 \\
0.82 \\
0.995 \\
0.515 \\
0.86 \\
0.905 \\
0.845 \\
0.88 \\
0.995 \\
0.9 \\
0.555 \\
0.88 \\
0.935 \\
0.78 \\
0.995 \\
};

\addplot[mark=*, boxplot, boxplot/draw position=4]
table[row sep=\\, y index=0] {
data
0.96 \\
0.995 \\
0.89 \\
0.905 \\
0.75 \\
0.85 \\
0.995 \\
0.515 \\
0.84 \\
0.495 \\
0.92 \\
0.995 \\
0.93 \\
0.875 \\
0.995 \\
0.995 \\
0.88 \\
0.975 \\
0.785 \\
0.985 \\
0.885 \\
0.995 \\
0.945 \\
0.93 \\
0.995 \\
0.995 \\
0.915 \\
0.94 \\
0.715 \\
0.88 \\
0.825 \\
0.95 \\
0.73 \\
0.475 \\
0.85 \\
0.87 \\
0.73 \\
0.995 \\
0.625 \\
0.92 \\
0.845 \\
0.755 \\
0.495 \\
0.495 \\
0.78 \\
0.995 \\
0.495 \\
0.995 \\
0.995 \\
0.76 \\
};

\addplot[mark=*, boxplot, boxplot/draw position=5]
table[row sep=\\, y index=0] {
data
0.87 \\
0.59 \\
0.495 \\
0.73 \\
0.855 \\
0.8 \\
0.665 \\
0.76 \\
0.425 \\
0.71 \\
0.76 \\
0.495 \\
0.73 \\
0.685 \\
0.995 \\
0.495 \\
0.495 \\
0.735 \\
0.495 \\
0.495 \\
0.495 \\
0.495 \\
0.495 \\
0.495 \\
0.735 \\
0.495 \\
0.495 \\
0.495 \\
0.495 \\
0.755 \\
0.495 \\
0.495 \\
0.62 \\
0.495 \\
0.495 \\
0.61 \\
0.495 \\
0.825 \\
0.83 \\
0.495 \\
0.5 \\
0.695 \\
0.61 \\
0.495 \\
0.635 \\
0.665 \\
0.495 \\
0.56 \\
0.495 \\
0.525 \\
};

\addplot[mark=*, boxplot, boxplot/draw position=6]
table[row sep=\\, y index=0] {
data
0.495 \\
0.495 \\
0.495 \\
0.495 \\
0.495 \\
0.495 \\
0.495 \\
0.495 \\
0.495 \\
0.495 \\
0.495 \\
0.495 \\
0.495 \\
0.495 \\
0.495 \\
0.495 \\
0.495 \\
0.495 \\
0.495 \\
0.495 \\
0.495 \\
0.495 \\
0.495 \\
0.495 \\
0.495 \\
0.495 \\
0.495 \\
0.495 \\
0.495 \\
0.495 \\
0.495 \\
0.495 \\
0.495 \\
0.495 \\
0.495 \\
0.495 \\
0.495 \\
0.495 \\
0.495 \\
0.495 \\
0.495 \\
0.495 \\
0.495 \\
0.495 \\
0.495 \\
0.495 \\
0.495 \\
0.495 \\
0.495 \\
0.495 \\
};
}{0.2}{Input connectivity}{21}

        }
        \subfloat[N=100]{
            \myboxplot{

\addplot[mark=*, boxplot, boxplot/draw position=0]
table[row sep=\\, y index=0] {
data
0.52 \\
0.47 \\
0.49 \\
0.475 \\
0.475 \\
0.49 \\
0.535 \\
0.495 \\
0.495 \\
0.515 \\
0.475 \\
0.5 \\
0.47 \\
0.505 \\
0.48 \\
0.495 \\
0.43 \\
0.495 \\
0.53 \\
0.495 \\
0.47 \\
0.53 \\
0.51 \\
0.475 \\
0.515 \\
0.585 \\
0.545 \\
0.515 \\
0.485 \\
0.535 \\
0.49 \\
0.465 \\
0.5 \\
0.515 \\
0.51 \\
0.485 \\
0.505 \\
0.51 \\
0.49 \\
0.5 \\
0.475 \\
0.485 \\
0.5 \\
0.49 \\
0.535 \\
0.455 \\
0.465 \\
0.475 \\
0.485 \\
0.56 \\
};

\addplot[mark=*, boxplot, boxplot/draw position=1]
table[row sep=\\, y index=0] {
data
0.46 \\
0.645 \\
0.56 \\
0.6 \\
0.47 \\
0.47 \\
0.385 \\
0.45 \\
0.615 \\
0.54 \\
0.545 \\
0.485 \\
0.65 \\
0.52 \\
0.6 \\
0.505 \\
0.585 \\
0.52 \\
0.51 \\
0.44 \\
0.635 \\
0.515 \\
0.585 \\
0.515 \\
0.59 \\
0.52 \\
0.49 \\
0.5 \\
0.53 \\
0.58 \\
0.465 \\
0.59 \\
0.485 \\
0.56 \\
0.45 \\
0.55 \\
0.495 \\
0.52 \\
0.69 \\
0.48 \\
0.5 \\
0.52 \\
0.48 \\
0.55 \\
0.47 \\
0.515 \\
0.545 \\
0.53 \\
0.485 \\
0.605 \\
};

\addplot[mark=*, boxplot, boxplot/draw position=2]
table[row sep=\\, y index=0] {
data
0.535 \\
0.56 \\
0.63 \\
0.43 \\
0.635 \\
0.665 \\
0.58 \\
0.54 \\
0.685 \\
0.865 \\
0.685 \\
0.545 \\
0.6 \\
0.715 \\
0.735 \\
0.735 \\
0.645 \\
0.56 \\
0.52 \\
0.52 \\
0.845 \\
0.495 \\
0.505 \\
0.525 \\
0.5 \\
0.58 \\
0.585 \\
0.745 \\
0.48 \\
0.47 \\
0.7 \\
0.48 \\
0.67 \\
0.685 \\
0.545 \\
0.47 \\
0.595 \\
0.59 \\
0.66 \\
0.545 \\
0.53 \\
0.595 \\
0.605 \\
0.82 \\
0.495 \\
0.465 \\
0.54 \\
0.815 \\
0.71 \\
0.635 \\
};

\addplot[mark=*, boxplot, boxplot/draw position=3]
table[row sep=\\, y index=0] {
data
0.675 \\
0.685 \\
0.785 \\
0.72 \\
0.8 \\
0.64 \\
0.615 \\
0.745 \\
0.605 \\
0.905 \\
0.59 \\
0.985 \\
0.88 \\
0.7 \\
0.925 \\
0.585 \\
0.61 \\
0.89 \\
0.745 \\
0.895 \\
0.63 \\
0.775 \\
0.83 \\
0.75 \\
0.69 \\
0.685 \\
0.73 \\
0.555 \\
0.76 \\
0.58 \\
0.625 \\
0.695 \\
0.54 \\
0.71 \\
0.54 \\
0.76 \\
0.55 \\
0.805 \\
0.855 \\
0.755 \\
0.74 \\
0.71 \\
0.745 \\
0.805 \\
0.53 \\
0.735 \\
0.52 \\
0.79 \\
0.735 \\
0.855 \\
};

\addplot[mark=*, boxplot, boxplot/draw position=4]
table[row sep=\\, y index=0] {
data
0.715 \\
0.795 \\
0.74 \\
0.87 \\
0.59 \\
0.865 \\
0.805 \\
0.83 \\
0.865 \\
0.8 \\
0.805 \\
0.69 \\
0.86 \\
0.875 \\
0.78 \\
0.775 \\
0.805 \\
0.57 \\
0.835 \\
0.68 \\
0.765 \\
0.775 \\
0.75 \\
0.61 \\
0.905 \\
0.875 \\
0.865 \\
0.905 \\
0.885 \\
0.795 \\
0.62 \\
0.97 \\
0.655 \\
0.805 \\
0.735 \\
0.55 \\
0.695 \\
0.75 \\
0.865 \\
0.54 \\
0.845 \\
0.705 \\
0.88 \\
0.88 \\
0.775 \\
0.805 \\
0.975 \\
0.68 \\
0.795 \\
0.81 \\
};

\addplot[mark=*, boxplot, boxplot/draw position=5]
table[row sep=\\, y index=0] {
data
0.835 \\
0.93 \\
0.71 \\
0.865 \\
0.99 \\
0.78 \\
0.83 \\
0.79 \\
0.98 \\
0.89 \\
0.815 \\
0.98 \\
0.87 \\
0.94 \\
0.91 \\
0.88 \\
0.84 \\
0.94 \\
0.74 \\
0.93 \\
0.995 \\
0.66 \\
0.79 \\
0.825 \\
0.77 \\
0.83 \\
0.935 \\
0.74 \\
0.825 \\
0.85 \\
0.815 \\
0.95 \\
0.72 \\
0.925 \\
0.955 \\
0.855 \\
0.945 \\
0.955 \\
0.855 \\
0.83 \\
0.82 \\
0.85 \\
0.87 \\
0.845 \\
0.95 \\
0.955 \\
0.84 \\
0.905 \\
0.97 \\
0.94 \\
};

\addplot[mark=*, boxplot, boxplot/draw position=6]
table[row sep=\\, y index=0] {
data
0.995 \\
0.895 \\
0.935 \\
0.97 \\
0.805 \\
0.99 \\
0.995 \\
0.935 \\
0.98 \\
0.84 \\
0.77 \\
0.91 \\
0.9 \\
0.995 \\
0.96 \\
0.94 \\
0.975 \\
0.985 \\
0.995 \\
0.795 \\
0.905 \\
0.925 \\
0.89 \\
0.95 \\
0.88 \\
0.855 \\
0.995 \\
0.835 \\
0.83 \\
0.87 \\
0.89 \\
0.995 \\
0.94 \\
0.965 \\
0.91 \\
0.99 \\
0.785 \\
0.995 \\
0.77 \\
0.77 \\
0.865 \\
0.975 \\
0.99 \\
0.9 \\
0.975 \\
0.595 \\
0.955 \\
0.925 \\
0.975 \\
0.845 \\
};

\addplot[mark=*, boxplot, boxplot/draw position=7]
table[row sep=\\, y index=0] {
data
0.95 \\
0.855 \\
0.985 \\
0.94 \\
0.96 \\
0.995 \\
0.93 \\
0.845 \\
0.965 \\
0.87 \\
0.935 \\
0.995 \\
0.9 \\
0.995 \\
0.995 \\
0.84 \\
0.97 \\
0.995 \\
0.79 \\
0.975 \\
0.88 \\
0.995 \\
0.985 \\
0.985 \\
0.945 \\
0.935 \\
0.955 \\
0.94 \\
0.995 \\
0.955 \\
0.99 \\
0.975 \\
0.91 \\
0.85 \\
0.97 \\
0.995 \\
0.995 \\
0.875 \\
0.79 \\
0.995 \\
0.98 \\
0.995 \\
0.855 \\
0.995 \\
0.985 \\
0.865 \\
0.985 \\
0.995 \\
0.895 \\
0.945 \\
};

\addplot[mark=*, boxplot, boxplot/draw position=8]
table[row sep=\\, y index=0] {
data
0.995 \\
0.95 \\
0.985 \\
0.995 \\
0.895 \\
0.955 \\
0.955 \\
0.97 \\
0.995 \\
0.995 \\
0.985 \\
0.965 \\
0.995 \\
0.995 \\
0.96 \\
0.965 \\
0.995 \\
0.915 \\
0.995 \\
0.99 \\
0.96 \\
0.995 \\
0.96 \\
0.985 \\
0.985 \\
0.995 \\
0.995 \\
0.98 \\
0.945 \\
0.995 \\
0.995 \\
0.99 \\
0.995 \\
0.995 \\
0.985 \\
0.945 \\
0.97 \\
0.975 \\
0.955 \\
0.995 \\
0.985 \\
0.975 \\
0.995 \\
0.995 \\
0.99 \\
0.985 \\
0.975 \\
0.995 \\
0.97 \\
0.985 \\
};

\addplot[mark=*, boxplot, boxplot/draw position=9]
table[row sep=\\, y index=0] {
data
0.995 \\
0.995 \\
0.995 \\
0.985 \\
0.995 \\
0.995 \\
0.975 \\
0.995 \\
0.94 \\
0.985 \\
0.995 \\
0.995 \\
0.995 \\
0.965 \\
0.995 \\
0.995 \\
0.98 \\
0.995 \\
0.995 \\
0.995 \\
0.995 \\
0.995 \\
0.975 \\
0.98 \\
0.995 \\
0.995 \\
0.97 \\
0.995 \\
0.995 \\
0.985 \\
0.985 \\
0.945 \\
0.895 \\
0.995 \\
0.995 \\
0.985 \\
0.995 \\
0.995 \\
0.995 \\
0.88 \\
0.875 \\
0.995 \\
0.995 \\
0.97 \\
0.995 \\
0.995 \\
0.995 \\
0.995 \\
0.995 \\
0.995 \\
};

\addplot[mark=*, boxplot, boxplot/draw position=10]
table[row sep=\\, y index=0] {
data
0.995 \\
0.995 \\
0.925 \\
0.995 \\
0.995 \\
0.995 \\
0.995 \\
0.955 \\
0.985 \\
0.99 \\
0.995 \\
0.88 \\
0.995 \\
0.995 \\
0.995 \\
0.995 \\
0.995 \\
0.995 \\
0.995 \\
0.945 \\
0.995 \\
0.995 \\
0.995 \\
0.995 \\
0.885 \\
0.995 \\
0.995 \\
0.97 \\
0.995 \\
0.995 \\
0.995 \\
0.995 \\
0.995 \\
0.995 \\
0.995 \\
0.995 \\
0.85 \\
0.995 \\
0.995 \\
0.995 \\
0.995 \\
0.965 \\
0.995 \\
0.995 \\
0.995 \\
0.995 \\
0.995 \\
0.995 \\
0.995 \\
0.995 \\
};

\addplot[mark=*, boxplot, boxplot/draw position=11]
table[row sep=\\, y index=0] {
data
0.955 \\
0.995 \\
0.995 \\
0.995 \\
0.995 \\
0.995 \\
0.995 \\
0.995 \\
0.995 \\
0.995 \\
0.995 \\
0.995 \\
0.995 \\
0.995 \\
0.995 \\
0.995 \\
0.995 \\
0.995 \\
0.995 \\
0.995 \\
0.985 \\
0.995 \\
0.985 \\
0.995 \\
0.985 \\
0.985 \\
0.995 \\
0.995 \\
0.995 \\
0.97 \\
0.995 \\
0.995 \\
0.995 \\
0.995 \\
0.995 \\
0.995 \\
0.995 \\
0.995 \\
0.995 \\
0.995 \\
0.995 \\
0.995 \\
0.895 \\
0.995 \\
0.995 \\
0.995 \\
0.995 \\
0.995 \\
0.99 \\
0.995 \\
};

\addplot[mark=*, boxplot, boxplot/draw position=12]
table[row sep=\\, y index=0] {
data
0.995 \\
0.91 \\
0.995 \\
0.995 \\
0.995 \\
0.995 \\
0.97 \\
0.995 \\
0.995 \\
0.995 \\
0.995 \\
0.995 \\
0.955 \\
0.995 \\
0.995 \\
0.995 \\
0.995 \\
0.995 \\
0.965 \\
0.995 \\
0.995 \\
0.995 \\
0.995 \\
0.995 \\
0.995 \\
0.995 \\
0.995 \\
0.945 \\
0.995 \\
0.995 \\
0.995 \\
0.985 \\
0.995 \\
0.995 \\
0.995 \\
0.995 \\
0.995 \\
0.995 \\
0.995 \\
0.995 \\
0.995 \\
0.995 \\
0.995 \\
0.995 \\
0.995 \\
0.995 \\
0.995 \\
0.995 \\
0.995 \\
0.995 \\
};

\addplot[mark=*, boxplot, boxplot/draw position=13]
table[row sep=\\, y index=0] {
data
0.995 \\
0.995 \\
0.995 \\
0.995 \\
0.995 \\
0.995 \\
0.995 \\
0.995 \\
0.995 \\
0.995 \\
0.905 \\
0.995 \\
0.995 \\
0.995 \\
0.98 \\
0.995 \\
0.995 \\
0.995 \\
0.995 \\
0.995 \\
0.995 \\
0.995 \\
0.995 \\
0.995 \\
0.995 \\
0.995 \\
0.995 \\
0.995 \\
0.995 \\
0.995 \\
0.995 \\
0.995 \\
0.995 \\
0.995 \\
0.995 \\
0.995 \\
0.995 \\
0.955 \\
0.97 \\
0.995 \\
0.98 \\
0.995 \\
0.995 \\
0.995 \\
0.995 \\
0.995 \\
0.995 \\
0.985 \\
0.995 \\
0.995 \\
};

\addplot[mark=*, boxplot, boxplot/draw position=14]
table[row sep=\\, y index=0] {
data
0.995 \\
0.995 \\
0.8 \\
0.995 \\
0.975 \\
0.995 \\
0.995 \\
0.97 \\
0.955 \\
0.995 \\
0.995 \\
0.995 \\
0.995 \\
0.955 \\
0.995 \\
0.995 \\
0.975 \\
0.995 \\
0.995 \\
0.995 \\
0.995 \\
0.995 \\
0.995 \\
0.87 \\
0.995 \\
0.995 \\
0.91 \\
0.815 \\
0.885 \\
0.995 \\
0.985 \\
0.995 \\
0.995 \\
0.995 \\
0.995 \\
0.995 \\
0.995 \\
0.975 \\
0.97 \\
0.995 \\
0.935 \\
0.995 \\
0.995 \\
0.995 \\
0.965 \\
0.96 \\
0.995 \\
0.995 \\
0.995 \\
0.995 \\
};

\addplot[mark=*, boxplot, boxplot/draw position=15]
table[row sep=\\, y index=0] {
data
0.995 \\
0.995 \\
0.995 \\
0.995 \\
0.995 \\
0.73 \\
0.995 \\
0.495 \\
0.995 \\
0.995 \\
0.87 \\
0.995 \\
0.81 \\
0.995 \\
0.995 \\
0.495 \\
0.855 \\
0.94 \\
0.995 \\
0.995 \\
0.87 \\
0.73 \\
0.99 \\
0.995 \\
0.905 \\
0.995 \\
0.995 \\
0.995 \\
0.935 \\
0.995 \\
0.975 \\
0.96 \\
0.965 \\
0.995 \\
0.905 \\
0.995 \\
0.99 \\
0.87 \\
0.74 \\
0.37 \\
0.995 \\
0.37 \\
0.995 \\
0.995 \\
0.995 \\
0.825 \\
0.93 \\
0.995 \\
0.995 \\
0.995 \\
};

\addplot[mark=*, boxplot, boxplot/draw position=16]
table[row sep=\\, y index=0] {
data
0.73 \\
0.855 \\
0.995 \\
0.88 \\
0.905 \\
0.995 \\
0.73 \\
0.995 \\
0.945 \\
0.86 \\
0.855 \\
0.995 \\
0.69 \\
0.995 \\
0.965 \\
0.87 \\
0.88 \\
0.995 \\
0.995 \\
0.995 \\
0.995 \\
0.82 \\
0.995 \\
0.495 \\
0.62 \\
0.875 \\
0.995 \\
0.995 \\
0.73 \\
0.44 \\
0.995 \\
0.995 \\
0.87 \\
0.995 \\
0.995 \\
0.995 \\
0.995 \\
0.93 \\
0.75 \\
0.915 \\
0.635 \\
0.995 \\
0.9 \\
0.76 \\
0.87 \\
0.88 \\
0.935 \\
0.995 \\
0.995 \\
0.995 \\
};

\addplot[mark=*, boxplot, boxplot/draw position=17]
table[row sep=\\, y index=0] {
data
0.995 \\
0.495 \\
0.495 \\
0.915 \\
0.995 \\
0.995 \\
0.875 \\
0.495 \\
0.995 \\
0.635 \\
0.635 \\
0.62 \\
0.905 \\
0.735 \\
0.76 \\
0.995 \\
0.935 \\
0.76 \\
0.73 \\
0.615 \\
0.495 \\
0.995 \\
0.82 \\
0.79 \\
0.48 \\
0.995 \\
0.995 \\
0.76 \\
0.76 \\
0.73 \\
0.495 \\
0.725 \\
0.495 \\
0.9 \\
0.995 \\
0.88 \\
0.665 \\
0.75 \\
0.48 \\
0.495 \\
0.995 \\
0.87 \\
0.995 \\
0.995 \\
0.875 \\
0.615 \\
0.945 \\
0.76 \\
0.495 \\
0.495 \\
};

\addplot[mark=*, boxplot, boxplot/draw position=18]
table[row sep=\\, y index=0] {
data
0.635 \\
0.995 \\
0.875 \\
0.635 \\
0.93 \\
0.495 \\
0.495 \\
0.495 \\
0.76 \\
0.73 \\
0.495 \\
0.855 \\
0.76 \\
0.495 \\
0.795 \\
0.495 \\
0.76 \\
0.495 \\
0.76 \\
0.495 \\
0.495 \\
0.495 \\
0.87 \\
0.495 \\
0.495 \\
0.67 \\
0.525 \\
0.495 \\
0.495 \\
0.76 \\
0.73 \\
0.875 \\
0.56 \\
0.495 \\
0.76 \\
0.495 \\
0.495 \\
0.495 \\
0.495 \\
0.995 \\
0.53 \\
0.73 \\
0.995 \\
0.495 \\
0.8 \\
0.695 \\
0.495 \\
0.495 \\
0.995 \\
0.48 \\
};

\addplot[mark=*, boxplot, boxplot/draw position=19]
table[row sep=\\, y index=0] {
data
0.495 \\
0.495 \\
0.495 \\
0.495 \\
0.495 \\
0.495 \\
0.495 \\
0.495 \\
0.495 \\
0.495 \\
0.995 \\
0.495 \\
0.61 \\
0.495 \\
0.495 \\
0.495 \\
0.87 \\
0.685 \\
0.495 \\
0.495 \\
0.495 \\
0.495 \\
0.495 \\
0.495 \\
0.495 \\
0.495 \\
0.495 \\
0.495 \\
0.495 \\
0.495 \\
0.495 \\
0.495 \\
0.495 \\
0.495 \\
0.495 \\
0.495 \\
0.495 \\
0.495 \\
0.76 \\
0.495 \\
0.495 \\
0.495 \\
0.495 \\
0.495 \\
0.495 \\
0.995 \\
0.495 \\
0.495 \\
0.495 \\
0.495 \\
};

\addplot[mark=*, boxplot, boxplot/draw position=20]
table[row sep=\\, y index=0] {
data
0.495 \\
0.495 \\
0.495 \\
0.495 \\
0.495 \\
0.495 \\
0.495 \\
0.495 \\
0.495 \\
0.495 \\
0.495 \\
0.495 \\
0.495 \\
0.495 \\
0.495 \\
0.495 \\
0.495 \\
0.495 \\
0.495 \\
0.495 \\
0.495 \\
0.495 \\
0.495 \\
0.495 \\
0.495 \\
0.495 \\
0.495 \\
0.495 \\
0.495 \\
0.495 \\
0.495 \\
0.495 \\
0.495 \\
0.495 \\
0.495 \\
0.495 \\
0.495 \\
0.495 \\
0.495 \\
0.495 \\
0.495 \\
0.495 \\
0.495 \\
0.495 \\
0.495 \\
0.495 \\
0.495 \\
0.495 \\
0.495 \\
0.495 \\
};
}{0.2}{Input connectivity}{21}

        }
    }
\end{figure*}

We confirm this skew by calculating the optimal input connectivity for each task as follows:
\begin{equation} \label{eq:optimal-ic}
optimal\_ic^{task} = average(max\_accuracy\_ic_{n\_nodes}^{task} / n\_nodes)
\end{equation}
where $ max\_accuracy\_ic_{n\_nodes}^{task} $ is the connectivity which gives the highest number of high-accuracy reservoirs for that task and reservoir size.
The results are presented in table \ref{tab:optimal-ic},
with the chosen values of $ max\_accuracy\_ic_{n\_nodes}^{task} $ presented in table \ref{tab:chosen-optimal-ic-values} in appendix \ref{app:reservoir_size-input_connectivity}.

\begin{table}
    \centering
    \caption{Optimal input connectivities as fraction of reservoir size.}
    \label{tab:optimal-ic}
    \begin{tabular}{lllll}
        \hline
        \hline
                         & \textbf{T=3} & \textbf{T=5} \\
        \hline
        Temporal parity  & 0.528          & 0.489 \\
        Temporal density & 0.439          & 0.443 \\
        \hline
    \end{tabular}
\end{table}

\subsection{Discussion}

\subsubsection{Required reservoir size}

The suspected ordering of the task difficulties,
$ TD3 \leq TP3 \leq TD5 \leq TP5 $,
is confirmed in figure \ref{fig:accuracy-threshold-size} and table \ref{tab:accuracy-thresholds}.
A reservoir of size 20 is large enough to correctly predict TP3,
quite a bit lower than the 100 nodes used in section \ref{section:pre-thesis-project} (figure \ref{fig:res:d-100-3-3}).
A reservoir of size 90 appears sufficient for TP5, rather close to the 100 nodes used in used in section \ref{section:pre-thesis-project} (figure \ref{fig:res:d-100-5-3}).
The entire accuracy distribution doesn't increase significantly until 120 nodes,
as seen in figure \ref{fig:TP5-IC-2} in the appendix.
A tiny reservoir of size 10 is sufficient for TD3,
with a reservoir of size 5 achieving 90\% accuracy!
There is a corresponding ramp-up in required reservoir size to 55 to achieve 98\% accuracy on TD5.

For the same window size, temporal parity requires a larger reservoir than temporal density,
and the additional computational requirements of remembering five time steps into the past requires a significant ramp-up in reservoir size.
The memory capacity of an Echo State Network trained on Independent and identically distributed data (as both our tasks are) is bounded by the size of the reservoir \cite{Jaeger:2007}.
A randomly generated reservoir is statistically unlikely to be able to attain this bound,
as additional constraints on the reservoir topology are required \cite{Jaeger:2007}.
One such specially crafted RBN RC system is the one consisting solely of nodes passing their states forward to the next node each time step,
only the first node being perturbed and all nodes read out.
The memory bound for the RBN RC systems used here seem lower than their computationally more powerful ESN forefathers,
which is corroborated by the significant increase in required reservoir size for tasks with window size 5.
This isn't that surprising as the RBN RC system is a much simpler model of computation than the connected sigmoid nodes of the ESN RC system.

When using Reservoir Computing as an abstraction over a physical device (e.g. an evolution-in-materio device),
empirical results from simulation can be used to aid in the suggestion of the required physical reservoir size.
One would need a way to roughly quantify the computational power contained in the simulated RBN as well as that of the computational substrate,
so that one can convert the results to meaningful physical sizes.
This is the case for systems such as the cellular automata implemented in carbon nanotubes \cite{farstad2015evolving},
as cellular automata are a special case of RBNs.

\subsubsection{Optimal input connectivity}

Temporal density had an optimal input connectivity of roughly 0.44, while temporal parity lay around roughly 0.50 (table \ref{tab:optimal-ic}).
These values differed relatively little across tasks.
This is to be expected, as the reservoir connectivity is the main factor in how large a reservoir perturbation is required \cite{rbn-reservoir},
as discussed in section \ref{section:optimal-perturbance}.

The net decrease in optimal IC from temporal parity 3 to 5 can be explained by the input needing to be remembered for longer in the network.
A lower amount of perturbation results in less overriding of already-stored information in the reservoir.
This conclusion is a bit hairy however,
as there isn't a corresponding decrease in optimal IC for temporal density, which already lies at a low 0.44.
These variations could also be explained by variance the optimal IC sampling process described in subsection \ref{subsection:reservoir_size-input_connectivity:results}.

Finally, these findings can be used to assist in choosing the degree of perturbation in a real-life reservoir computing system,
given a rough idea of what dynamical regime it may exist in (chaotic, critical, stable).
When instrumenting the rat neurons for airplane automation in \cite{demarse2005adaptive},
only two of the sixty available microelectrodes of the MEA were used even though the computational substrate stretched under all of them.
The same two microelectrodes were used for both perturbing the substrate and reading out its state.
This indicates that the required amount of perturbation may depend on how large a part of the reservoir one actually uses,
due to the spatial locality of computation in physical reservoirs.
There might not be a need for additional perturbing of the neurons furthest away from the readout microelectrodes,
as the effect might be minimal at the readout location.
