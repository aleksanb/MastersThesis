\section{Minimum required reservoir size, optimal input connectivity}

As shown in my previous paper \cite{MyPreviousPaper},
there are a plethora of RRC systems with $N=100, K=\{2, 3\}$ that solve the Temporal Parity task for a window size of both 3 and 5,

The reservoirs with $K=3$ were found to have a higher density of accurate reservoirs,
and will therefore be chosen as the connectivity for all the reservoirs we'll be creating here as well.

Presumably the same tasks can be solved with smaller reservoirs,
the question being how small they can be while retaining accurarcy.
As the optimal reservoir size and input connectivity might depend on the task at hand,
we test our reservoirs on both the Temporal Parity \ref{missing} and Temporal Density \ref{missing}
tasks with window sizes of both 3 and 5 (as specified in table \ref{tab:tasks}).

\begin{table}[ht]
  \centering
  \caption{Task parameters. They'll be referred to as TP3, TP5, TD3, and TD5 from now on.}
  \label{tab:tasks}
  \begin{tabular}{ll}
    Task type               & Temporal Parity and Temporal Density \\
    Training dataset length & 4 000                       \\
    Test dataset length     & 200                         \\
    $N$ (window size)       & 3 and 5                     \\
    $t$ (offset)            & 0
  \end{tabular}
\end{table}

To find the optimal reservoir size and input connectivity,
we'll randomly initialize 50 reservoirs for each combination of the reservoir parameters in table \ref{tab:ic-reservoir-parameters}.

For our experiments we'll only be looking at reservoirs with homogenous connectivity of $K=3$.
This was found to be slightly
as this connectivity was found to be 

As well as being more useful due to the homogenous degree of the network (to simulate $\langle K \rangle = 2 $) and
having a higher population accuracy, it reduces the number of parameter combination we have to simulate.

\begin{table}[ht]
    \centering
    \caption{Reservoir parameters for optimal input connectivity}
    \label{tab:ic-reservoir-parameters}
    \begin{tabular}{ll}
        Nodes               & [10..(100, 140)], step size=10 \\
        Connectivity        & 3                              \\
        Input connectivity  & [0..n\_nodes], step size = 5   \\
		Output connectivity & n\_nodes                       \\
        Sample size         & 50
    \end{tabular}
\end{table}

After plotting the resulting accuracies on the previously mentioned tasks,
one should be able to visually identify the minimum required reservoir size for the given task,
as well as what the optimal input connectivity might be, as a function of reservoir size and task.

\label{experiments:1:results}
\subsection{Results}

\begin{figure}[ht]
    \centering
    \caption{Accuracy plots for the smallest reservoirs created for each of the TP3, TP5, TD3, TD5 tasks.}

    \label{fig:accuracy-min-size}
    \resizebox{\textwidth}{!}{
        \subfloat[TP3, N=10]{
            \myboxplot{

\addplot[mark=*, boxplot, boxplot/draw position=0]
table[row sep=\\, y index=0] {
data
0.515 \\
0.465 \\
0.51 \\
0.51 \\
0.55 \\
0.51 \\
0.485 \\
0.485 \\
0.465 \\
0.485 \\
0.47 \\
0.5 \\
0.55 \\
0.49 \\
0.435 \\
0.485 \\
0.51 \\
0.55 \\
0.515 \\
0.535 \\
0.51 \\
0.51 \\
0.485 \\
0.465 \\
0.485 \\
0.575 \\
0.485 \\
0.485 \\
0.465 \\
0.515 \\
0.535 \\
0.465 \\
0.485 \\
0.51 \\
0.485 \\
0.485 \\
0.485 \\
0.49 \\
0.485 \\
0.565 \\
0.465 \\
0.515 \\
0.485 \\
0.485 \\
0.485 \\
0.51 \\
0.485 \\
0.55 \\
0.485 \\
0.485 \\
};

\addplot[mark=*, boxplot, boxplot/draw position=1]
table[row sep=\\, y index=0] {
data
0.785 \\
0.89 \\
0.48 \\
0.495 \\
0.62 \\
0.635 \\
0.735 \\
0.82 \\
0.495 \\
0.65 \\
0.635 \\
0.635 \\
0.73 \\
0.785 \\
0.735 \\
0.815 \\
0.65 \\
0.825 \\
0.69 \\
0.65 \\
0.515 \\
0.665 \\
0.67 \\
0.495 \\
0.495 \\
0.525 \\
0.645 \\
0.72 \\
0.755 \\
0.71 \\
0.635 \\
0.65 \\
0.535 \\
0.78 \\
0.73 \\
0.755 \\
0.775 \\
0.67 \\
0.715 \\
0.495 \\
0.485 \\
0.705 \\
0.66 \\
0.615 \\
0.495 \\
0.495 \\
0.495 \\
0.495 \\
0.87 \\
0.675 \\
};

\addplot[mark=*, boxplot, boxplot/draw position=2]
table[row sep=\\, y index=0] {
data
0.495 \\
0.495 \\
0.495 \\
0.495 \\
0.495 \\
0.495 \\
0.495 \\
0.495 \\
0.495 \\
0.495 \\
0.495 \\
0.495 \\
0.495 \\
0.495 \\
0.495 \\
0.495 \\
0.495 \\
0.495 \\
0.495 \\
0.495 \\
0.495 \\
0.495 \\
0.495 \\
0.495 \\
0.495 \\
0.495 \\
0.495 \\
0.495 \\
0.495 \\
0.495 \\
0.495 \\
0.495 \\
0.495 \\
0.495 \\
0.495 \\
0.495 \\
0.495 \\
0.495 \\
0.495 \\
0.495 \\
0.495 \\
0.495 \\
0.495 \\
0.495 \\
0.495 \\
0.495 \\
0.495 \\
0.495 \\
0.495 \\
0.495 \\
};
}{0.2}{Input connectivity}{21}

        }
        \subfloat[TP5, N=10]{
            \myboxplot{

\addplot[mark=*, boxplot, boxplot/draw position=0]
table[row sep=\\, y index=0] {
data
0.475 \\
0.475 \\
0.475 \\
0.475 \\
0.435 \\
0.475 \\
0.475 \\
0.475 \\
0.475 \\
0.47 \\
0.475 \\
0.475 \\
0.465 \\
0.475 \\
0.475 \\
0.475 \\
0.475 \\
0.475 \\
0.475 \\
0.475 \\
0.475 \\
0.475 \\
0.475 \\
0.485 \\
0.475 \\
0.475 \\
0.475 \\
0.475 \\
0.475 \\
0.435 \\
0.475 \\
0.475 \\
0.475 \\
0.475 \\
0.435 \\
0.475 \\
0.475 \\
0.475 \\
0.475 \\
0.475 \\
0.475 \\
0.465 \\
0.475 \\
0.48 \\
0.475 \\
0.475 \\
0.475 \\
0.475 \\
0.475 \\
0.475 \\
};

\addplot[mark=*, boxplot, boxplot/draw position=1]
table[row sep=\\, y index=0] {
data
0.515 \\
0.54 \\
0.515 \\
0.57 \\
0.47 \\
0.475 \\
0.525 \\
0.45 \\
0.51 \\
0.48 \\
0.43 \\
0.59 \\
0.57 \\
0.58 \\
0.47 \\
0.47 \\
0.57 \\
0.47 \\
0.535 \\
0.545 \\
0.525 \\
0.47 \\
0.56 \\
0.51 \\
0.62 \\
0.445 \\
0.66 \\
0.5 \\
0.47 \\
0.595 \\
0.655 \\
0.62 \\
0.455 \\
0.645 \\
0.59 \\
0.47 \\
0.555 \\
0.53 \\
0.47 \\
0.47 \\
0.46 \\
0.51 \\
0.47 \\
0.47 \\
0.475 \\
0.51 \\
0.635 \\
0.43 \\
0.47 \\
0.43 \\
};

\addplot[mark=*, boxplot, boxplot/draw position=2]
table[row sep=\\, y index=0] {
data
0.47 \\
0.47 \\
0.47 \\
0.475 \\
0.47 \\
0.47 \\
0.47 \\
0.47 \\
0.47 \\
0.47 \\
0.47 \\
0.47 \\
0.47 \\
0.47 \\
0.47 \\
0.47 \\
0.47 \\
0.47 \\
0.47 \\
0.47 \\
0.47 \\
0.47 \\
0.47 \\
0.47 \\
0.47 \\
0.47 \\
0.47 \\
0.47 \\
0.47 \\
0.47 \\
0.47 \\
0.47 \\
0.47 \\
0.47 \\
0.47 \\
0.47 \\
0.47 \\
0.47 \\
0.47 \\
0.47 \\
0.47 \\
0.47 \\
0.47 \\
0.47 \\
0.47 \\
0.47 \\
0.47 \\
0.47 \\
0.47 \\
0.47 \\
};
}{0.2}{Input connectivity}{29}

        }
    }
    \resizebox{\textwidth}{!}{
        \subfloat[TD3, N=5]{
            \myboxplot{

\addplot[mark=*, boxplot, boxplot/draw position=0]
table[row sep=\\, y index=0] {
data
0.51 \\
0.445 \\
0.435 \\
0.51 \\
0.445 \\
0.445 \\
0.445 \\
0.445 \\
0.51 \\
0.6 \\
0.51 \\
0.495 \\
0.595 \\
0.505 \\
0.425 \\
0.505 \\
0.445 \\
0.445 \\
0.51 \\
0.435 \\
0.445 \\
0.445 \\
0.445 \\
0.425 \\
0.51 \\
0.445 \\
0.445 \\
0.425 \\
0.425 \\
0.495 \\
0.41 \\
0.425 \\
0.51 \\
0.445 \\
0.505 \\
0.435 \\
0.51 \\
0.445 \\
0.445 \\
0.435 \\
0.51 \\
0.435 \\
0.425 \\
0.445 \\
0.445 \\
0.445 \\
0.425 \\
0.445 \\
0.41 \\
0.445 \\
};

\addplot[mark=*, boxplot, boxplot/draw position=1]
table[row sep=\\, y index=0] {
data
0.435 \\
0.435 \\
0.755 \\
0.75 \\
0.795 \\
0.435 \\
0.755 \\
0.865 \\
0.805 \\
0.755 \\
0.755 \\
0.745 \\
0.685 \\
0.63 \\
0.825 \\
0.435 \\
0.755 \\
0.755 \\
0.535 \\
0.435 \\
0.705 \\
0.765 \\
0.81 \\
0.615 \\
0.79 \\
0.435 \\
0.755 \\
0.705 \\
0.435 \\
0.675 \\
0.66 \\
0.81 \\
0.555 \\
0.815 \\
0.65 \\
0.715 \\
0.755 \\
0.5 \\
0.755 \\
0.64 \\
0.7 \\
0.585 \\
0.755 \\
0.65 \\
0.6 \\
0.82 \\
0.78 \\
0.755 \\
0.55 \\
0.755 \\
};

\addplot[mark=*, boxplot, boxplot/draw position=2]
table[row sep=\\, y index=0] {
data
0.84 \\
0.675 \\
0.83 \\
0.755 \\
0.665 \\
0.765 \\
0.84 \\
0.845 \\
0.755 \\
0.755 \\
0.68 \\
0.755 \\
0.76 \\
0.755 \\
0.815 \\
0.745 \\
0.685 \\
0.855 \\
0.92 \\
0.855 \\
0.755 \\
0.565 \\
0.71 \\
0.695 \\
0.78 \\
0.755 \\
0.755 \\
0.785 \\
0.71 \\
0.59 \\
0.755 \\
0.755 \\
0.79 \\
0.54 \\
0.6 \\
0.78 \\
0.64 \\
0.75 \\
0.765 \\
0.685 \\
0.68 \\
0.645 \\
0.8 \\
0.74 \\
0.77 \\
0.815 \\
0.755 \\
0.755 \\
0.755 \\
0.85 \\
};

\addplot[mark=*, boxplot, boxplot/draw position=3]
table[row sep=\\, y index=0] {
data
0.72 \\
0.755 \\
0.755 \\
0.755 \\
0.755 \\
0.755 \\
0.755 \\
0.655 \\
0.755 \\
0.765 \\
0.755 \\
0.615 \\
0.755 \\
0.72 \\
0.755 \\
0.72 \\
0.755 \\
0.84 \\
0.755 \\
0.71 \\
0.755 \\
0.755 \\
0.755 \\
0.755 \\
0.755 \\
0.755 \\
0.755 \\
0.755 \\
0.72 \\
0.725 \\
0.865 \\
0.72 \\
0.995 \\
0.755 \\
0.7 \\
0.795 \\
0.71 \\
0.755 \\
0.755 \\
0.755 \\
0.72 \\
0.755 \\
0.755 \\
0.765 \\
0.755 \\
0.755 \\
0.71 \\
0.745 \\
0.72 \\
0.755 \\
};

\addplot[mark=*, boxplot, boxplot/draw position=4]
table[row sep=\\, y index=0] {
data
0.72 \\
0.72 \\
0.755 \\
0.755 \\
0.72 \\
0.755 \\
0.755 \\
0.755 \\
0.755 \\
0.755 \\
0.72 \\
0.755 \\
0.765 \\
0.755 \\
0.755 \\
0.755 \\
0.71 \\
0.755 \\
0.72 \\
0.755 \\
0.755 \\
0.755 \\
0.755 \\
0.72 \\
0.755 \\
0.755 \\
0.72 \\
0.755 \\
0.755 \\
0.755 \\
0.755 \\
0.71 \\
0.755 \\
0.755 \\
0.755 \\
0.755 \\
0.755 \\
0.765 \\
0.755 \\
0.72 \\
0.755 \\
0.755 \\
0.72 \\
0.72 \\
0.755 \\
0.755 \\
0.755 \\
0.755 \\
0.755 \\
0.72 \\
};

\addplot[mark=*, boxplot, boxplot/draw position=5]
table[row sep=\\, y index=0] {
data
0.755 \\
0.755 \\
0.755 \\
0.755 \\
0.755 \\
0.755 \\
0.755 \\
0.755 \\
0.755 \\
0.755 \\
0.755 \\
0.755 \\
0.755 \\
0.755 \\
0.755 \\
0.755 \\
0.755 \\
0.755 \\
0.755 \\
0.755 \\
0.755 \\
0.755 \\
0.755 \\
0.755 \\
0.755 \\
0.755 \\
0.755 \\
0.755 \\
0.755 \\
0.755 \\
0.755 \\
0.755 \\
0.755 \\
0.755 \\
0.755 \\
0.755 \\
0.435 \\
0.755 \\
0.755 \\
0.755 \\
0.755 \\
0.755 \\
0.755 \\
0.755 \\
0.755 \\
0.755 \\
0.755 \\
0.755 \\
0.755 \\
0.755 \\
};
}{1.0}{}{30}

        }
        \subfloat[TD5, N=10]{
            \myboxplot{

\addplot[mark=*, boxplot, boxplot/draw position=0]
table[row sep=\\, y index=0] {
data
0.5 \\
0.5 \\
0.5 \\
0.5 \\
0.5 \\
0.505 \\
0.5 \\
0.5 \\
0.5 \\
0.5 \\
0.5 \\
0.5 \\
0.515 \\
0.5 \\
0.51 \\
0.5 \\
0.5 \\
0.5 \\
0.5 \\
0.5 \\
0.515 \\
0.5 \\
0.5 \\
0.5 \\
0.5 \\
0.5 \\
0.545 \\
0.5 \\
0.5 \\
0.5 \\
0.5 \\
0.5 \\
0.56 \\
0.5 \\
0.5 \\
0.5 \\
0.5 \\
0.5 \\
0.5 \\
0.5 \\
0.5 \\
0.5 \\
0.5 \\
0.51 \\
0.5 \\
0.5 \\
0.5 \\
0.5 \\
0.5 \\
0.535 \\
};

\addplot[mark=*, boxplot, boxplot/draw position=1]
table[row sep=\\, y index=0] {
data
0.695 \\
0.725 \\
0.74 \\
0.695 \\
0.695 \\
0.74 \\
0.8 \\
0.695 \\
0.65 \\
0.835 \\
0.74 \\
0.725 \\
0.715 \\
0.72 \\
0.71 \\
0.83 \\
0.8 \\
0.68 \\
0.735 \\
0.695 \\
0.8 \\
0.8 \\
0.695 \\
0.78 \\
0.74 \\
0.73 \\
0.705 \\
0.67 \\
0.695 \\
0.715 \\
0.685 \\
0.74 \\
0.675 \\
0.695 \\
0.685 \\
0.695 \\
0.725 \\
0.695 \\
0.755 \\
0.815 \\
0.73 \\
0.8 \\
0.705 \\
0.69 \\
0.73 \\
0.725 \\
0.8 \\
0.695 \\
0.775 \\
0.785 \\
};

\addplot[mark=*, boxplot, boxplot/draw position=2]
table[row sep=\\, y index=0] {
data
0.695 \\
0.695 \\
0.695 \\
0.695 \\
0.695 \\
0.695 \\
0.695 \\
0.695 \\
0.695 \\
0.695 \\
0.695 \\
0.695 \\
0.695 \\
0.695 \\
0.695 \\
0.695 \\
0.695 \\
0.695 \\
0.695 \\
0.695 \\
0.695 \\
0.695 \\
0.695 \\
0.695 \\
0.695 \\
0.695 \\
0.695 \\
0.695 \\
0.695 \\
0.695 \\
0.695 \\
0.695 \\
0.695 \\
0.695 \\
0.695 \\
0.695 \\
0.695 \\
0.695 \\
0.695 \\
0.695 \\
0.695 \\
0.695 \\
0.695 \\
0.695 \\
0.695 \\
0.695 \\
0.695 \\
0.695 \\
0.695 \\
0.695 \\
};
}{0.2}{}{14}

        }
    }
\end{figure}

\begin{figure}[ht]
    \centering
    \caption{Accuracy plots for the reservoirs on the 98\% accuracy threshold for each of the TP3, TP5, TD3, TD5 tasks.}

    \label{fig:accuracy-threshold-size}
    \resizebox{\textwidth}{!}{
        \subfloat[TP3, N=20]{
            \myboxplot{

\addplot[mark=*, boxplot, boxplot/draw position=0]
table[row sep=\\, y index=0] {
data
0.485 \\
0.465 \\
0.46 \\
0.485 \\
0.465 \\
0.465 \\
0.485 \\
0.485 \\
0.53 \\
0.485 \\
0.515 \\
0.51 \\
0.51 \\
0.51 \\
0.465 \\
0.515 \\
0.52 \\
0.53 \\
0.53 \\
0.575 \\
0.465 \\
0.465 \\
0.485 \\
0.515 \\
0.485 \\
0.55 \\
0.465 \\
0.51 \\
0.485 \\
0.51 \\
0.515 \\
0.51 \\
0.49 \\
0.54 \\
0.435 \\
0.465 \\
0.485 \\
0.465 \\
0.495 \\
0.485 \\
0.51 \\
0.485 \\
0.465 \\
0.545 \\
0.485 \\
0.485 \\
0.465 \\
0.485 \\
0.55 \\
0.595 \\
};

\addplot[mark=*, boxplot, boxplot/draw position=1]
table[row sep=\\, y index=0] {
data
0.67 \\
0.465 \\
0.52 \\
0.93 \\
0.895 \\
0.68 \\
0.865 \\
0.59 \\
0.675 \\
0.79 \\
0.63 \\
0.595 \\
0.565 \\
0.815 \\
0.745 \\
0.735 \\
0.585 \\
0.485 \\
0.8 \\
0.65 \\
0.69 \\
0.83 \\
0.82 \\
0.555 \\
0.64 \\
0.645 \\
0.46 \\
0.695 \\
0.625 \\
0.785 \\
0.745 \\
0.46 \\
0.62 \\
0.545 \\
0.835 \\
0.57 \\
0.785 \\
0.565 \\
0.555 \\
0.435 \\
0.69 \\
0.645 \\
0.58 \\
0.555 \\
0.585 \\
0.53 \\
0.67 \\
0.525 \\
0.705 \\
0.685 \\
};

\addplot[mark=*, boxplot, boxplot/draw position=2]
table[row sep=\\, y index=0] {
data
0.92 \\
0.995 \\
0.9 \\
0.76 \\
0.945 \\
0.575 \\
0.96 \\
0.925 \\
0.96 \\
0.995 \\
0.61 \\
0.625 \\
0.615 \\
0.78 \\
0.995 \\
0.885 \\
0.785 \\
0.955 \\
0.995 \\
0.735 \\
0.87 \\
0.94 \\
0.81 \\
0.63 \\
0.995 \\
0.465 \\
0.525 \\
0.525 \\
0.97 \\
0.72 \\
0.55 \\
0.94 \\
0.965 \\
0.885 \\
0.665 \\
0.935 \\
0.7 \\
0.94 \\
0.8 \\
0.66 \\
0.815 \\
0.81 \\
0.87 \\
0.77 \\
0.895 \\
0.945 \\
0.845 \\
0.955 \\
0.925 \\
0.765 \\
};

\addplot[mark=*, boxplot, boxplot/draw position=3]
table[row sep=\\, y index=0] {
data
0.495 \\
0.825 \\
0.495 \\
0.995 \\
0.76 \\
0.495 \\
0.995 \\
0.48 \\
0.495 \\
0.695 \\
0.475 \\
0.495 \\
0.505 \\
0.495 \\
0.495 \\
0.62 \\
0.995 \\
0.525 \\
0.84 \\
0.735 \\
0.495 \\
0.73 \\
0.37 \\
0.65 \\
0.495 \\
0.76 \\
0.675 \\
0.73 \\
0.995 \\
0.995 \\
0.495 \\
0.495 \\
0.495 \\
0.655 \\
0.495 \\
0.995 \\
0.655 \\
0.615 \\
0.995 \\
0.88 \\
0.695 \\
0.995 \\
0.495 \\
0.76 \\
0.87 \\
0.675 \\
0.495 \\
0.87 \\
0.58 \\
0.995 \\
};

\addplot[mark=*, boxplot, boxplot/draw position=4]
table[row sep=\\, y index=0] {
data
0.495 \\
0.495 \\
0.495 \\
0.495 \\
0.495 \\
0.495 \\
0.495 \\
0.495 \\
0.495 \\
0.495 \\
0.495 \\
0.495 \\
0.495 \\
0.495 \\
0.495 \\
0.495 \\
0.495 \\
0.495 \\
0.495 \\
0.495 \\
0.495 \\
0.495 \\
0.495 \\
0.495 \\
0.495 \\
0.495 \\
0.495 \\
0.495 \\
0.495 \\
0.495 \\
0.495 \\
0.495 \\
0.495 \\
0.495 \\
0.495 \\
0.495 \\
0.495 \\
0.495 \\
0.495 \\
0.495 \\
0.495 \\
0.495 \\
0.495 \\
0.495 \\
0.495 \\
0.495 \\
0.495 \\
0.495 \\
0.495 \\
0.495 \\
};
}{0.2}{Input connectivity}{21}

        }
        \subfloat[TP5, N=90]{
            \myboxplot{

\addplot[mark=*, boxplot, boxplot/draw position=0]
table[row sep=\\, y index=0] {
data
0.47 \\
0.48 \\
0.515 \\
0.485 \\
0.55 \\
0.505 \\
0.49 \\
0.51 \\
0.53 \\
0.535 \\
0.475 \\
0.53 \\
0.45 \\
0.455 \\
0.43 \\
0.515 \\
0.48 \\
0.56 \\
0.43 \\
0.46 \\
0.48 \\
0.51 \\
0.52 \\
0.5 \\
0.54 \\
0.52 \\
0.465 \\
0.455 \\
0.5 \\
0.475 \\
0.46 \\
0.48 \\
0.48 \\
0.505 \\
0.42 \\
0.49 \\
0.51 \\
0.49 \\
0.55 \\
0.545 \\
0.48 \\
0.455 \\
0.435 \\
0.515 \\
0.485 \\
0.48 \\
0.5 \\
0.47 \\
0.48 \\
0.49 \\
};

\addplot[mark=*, boxplot, boxplot/draw position=1]
table[row sep=\\, y index=0] {
data
0.56 \\
0.41 \\
0.46 \\
0.52 \\
0.45 \\
0.485 \\
0.475 \\
0.545 \\
0.485 \\
0.45 \\
0.55 \\
0.495 \\
0.49 \\
0.475 \\
0.495 \\
0.53 \\
0.5 \\
0.565 \\
0.495 \\
0.56 \\
0.46 \\
0.495 \\
0.51 \\
0.535 \\
0.465 \\
0.485 \\
0.47 \\
0.475 \\
0.53 \\
0.485 \\
0.45 \\
0.5 \\
0.455 \\
0.505 \\
0.54 \\
0.54 \\
0.495 \\
0.455 \\
0.55 \\
0.455 \\
0.45 \\
0.515 \\
0.51 \\
0.515 \\
0.445 \\
0.53 \\
0.535 \\
0.43 \\
0.475 \\
0.49 \\
};

\addplot[mark=*, boxplot, boxplot/draw position=2]
table[row sep=\\, y index=0] {
data
0.47 \\
0.505 \\
0.475 \\
0.495 \\
0.425 \\
0.45 \\
0.43 \\
0.525 \\
0.445 \\
0.455 \\
0.445 \\
0.495 \\
0.485 \\
0.5 \\
0.44 \\
0.485 \\
0.475 \\
0.45 \\
0.47 \\
0.475 \\
0.45 \\
0.445 \\
0.535 \\
0.46 \\
0.455 \\
0.525 \\
0.455 \\
0.5 \\
0.47 \\
0.545 \\
0.51 \\
0.58 \\
0.48 \\
0.43 \\
0.49 \\
0.5 \\
0.495 \\
0.46 \\
0.465 \\
0.505 \\
0.445 \\
0.52 \\
0.47 \\
0.5 \\
0.37 \\
0.52 \\
0.54 \\
0.58 \\
0.54 \\
0.53 \\
};

\addplot[mark=*, boxplot, boxplot/draw position=3]
table[row sep=\\, y index=0] {
data
0.505 \\
0.525 \\
0.455 \\
0.53 \\
0.475 \\
0.4 \\
0.575 \\
0.46 \\
0.47 \\
0.495 \\
0.465 \\
0.5 \\
0.51 \\
0.51 \\
0.52 \\
0.44 \\
0.5 \\
0.525 \\
0.565 \\
0.455 \\
0.51 \\
0.505 \\
0.585 \\
0.485 \\
0.515 \\
0.505 \\
0.51 \\
0.43 \\
0.52 \\
0.59 \\
0.53 \\
0.495 \\
0.52 \\
0.475 \\
0.615 \\
0.58 \\
0.57 \\
0.47 \\
0.47 \\
0.495 \\
0.44 \\
0.585 \\
0.47 \\
0.48 \\
0.455 \\
0.535 \\
0.505 \\
0.45 \\
0.48 \\
0.5 \\
};

\addplot[mark=*, boxplot, boxplot/draw position=4]
table[row sep=\\, y index=0] {
data
0.54 \\
0.715 \\
0.5 \\
0.595 \\
0.595 \\
0.585 \\
0.59 \\
0.525 \\
0.465 \\
0.59 \\
0.48 \\
0.575 \\
0.455 \\
0.54 \\
0.63 \\
0.53 \\
0.605 \\
0.49 \\
0.65 \\
0.515 \\
0.535 \\
0.535 \\
0.52 \\
0.505 \\
0.635 \\
0.52 \\
0.635 \\
0.575 \\
0.61 \\
0.65 \\
0.475 \\
0.495 \\
0.545 \\
0.485 \\
0.595 \\
0.58 \\
0.625 \\
0.51 \\
0.575 \\
0.485 \\
0.545 \\
0.5 \\
0.575 \\
0.485 \\
0.51 \\
0.635 \\
0.465 \\
0.52 \\
0.535 \\
0.61 \\
};

\addplot[mark=*, boxplot, boxplot/draw position=5]
table[row sep=\\, y index=0] {
data
0.58 \\
0.655 \\
0.565 \\
0.595 \\
0.545 \\
0.62 \\
0.55 \\
0.635 \\
0.58 \\
0.49 \\
0.755 \\
0.54 \\
0.575 \\
0.47 \\
0.565 \\
0.53 \\
0.55 \\
0.535 \\
0.54 \\
0.545 \\
0.725 \\
0.53 \\
0.675 \\
0.735 \\
0.575 \\
0.55 \\
0.615 \\
0.64 \\
0.64 \\
0.605 \\
0.575 \\
0.485 \\
0.465 \\
0.61 \\
0.625 \\
0.62 \\
0.41 \\
0.65 \\
0.635 \\
0.585 \\
0.655 \\
0.68 \\
0.575 \\
0.65 \\
0.61 \\
0.575 \\
0.805 \\
0.515 \\
0.59 \\
0.705 \\
};

\addplot[mark=*, boxplot, boxplot/draw position=6]
table[row sep=\\, y index=0] {
data
0.735 \\
0.79 \\
0.81 \\
0.635 \\
0.69 \\
0.66 \\
0.625 \\
0.605 \\
0.765 \\
0.66 \\
0.535 \\
0.45 \\
0.67 \\
0.67 \\
0.79 \\
0.66 \\
0.505 \\
0.555 \\
0.615 \\
0.545 \\
0.86 \\
0.57 \\
0.675 \\
0.475 \\
0.76 \\
0.825 \\
0.535 \\
0.635 \\
0.545 \\
0.695 \\
0.59 \\
0.59 \\
0.655 \\
0.56 \\
0.66 \\
0.505 \\
0.5 \\
0.57 \\
0.58 \\
0.725 \\
0.76 \\
0.71 \\
0.595 \\
0.705 \\
0.615 \\
0.66 \\
0.54 \\
0.605 \\
0.57 \\
0.53 \\
};

\addplot[mark=*, boxplot, boxplot/draw position=7]
table[row sep=\\, y index=0] {
data
0.69 \\
0.695 \\
0.745 \\
0.64 \\
0.545 \\
0.62 \\
0.66 \\
0.9 \\
0.88 \\
0.77 \\
0.885 \\
0.85 \\
0.585 \\
0.585 \\
0.63 \\
0.69 \\
0.75 \\
0.905 \\
0.795 \\
0.71 \\
0.52 \\
0.71 \\
0.51 \\
0.71 \\
0.765 \\
0.65 \\
0.85 \\
0.8 \\
0.75 \\
0.67 \\
0.645 \\
0.84 \\
0.83 \\
0.68 \\
0.7 \\
0.785 \\
0.7 \\
0.915 \\
0.77 \\
0.655 \\
0.665 \\
0.635 \\
0.75 \\
0.685 \\
0.645 \\
0.615 \\
0.675 \\
0.815 \\
0.695 \\
0.79 \\
};

\addplot[mark=*, boxplot, boxplot/draw position=8]
table[row sep=\\, y index=0] {
data
0.84 \\
0.825 \\
0.86 \\
0.825 \\
0.65 \\
0.915 \\
0.73 \\
0.7 \\
0.795 \\
0.465 \\
0.73 \\
0.825 \\
0.665 \\
0.765 \\
0.79 \\
0.765 \\
0.66 \\
0.645 \\
0.615 \\
0.64 \\
0.81 \\
0.64 \\
0.715 \\
0.585 \\
0.8 \\
0.665 \\
0.895 \\
0.605 \\
0.815 \\
0.83 \\
0.66 \\
0.73 \\
0.82 \\
0.755 \\
0.73 \\
0.665 \\
0.67 \\
0.78 \\
0.83 \\
0.735 \\
0.92 \\
0.89 \\
0.78 \\
0.65 \\
0.92 \\
0.635 \\
0.835 \\
0.705 \\
0.875 \\
0.65 \\
};

\addplot[mark=*, boxplot, boxplot/draw position=9]
table[row sep=\\, y index=0] {
data
0.88 \\
0.88 \\
0.69 \\
0.79 \\
0.67 \\
0.735 \\
0.845 \\
0.625 \\
0.715 \\
0.575 \\
0.69 \\
0.82 \\
0.95 \\
0.54 \\
0.66 \\
0.795 \\
0.685 \\
0.635 \\
0.835 \\
0.98 \\
0.735 \\
0.66 \\
0.585 \\
0.695 \\
0.6 \\
0.785 \\
0.765 \\
0.675 \\
0.77 \\
0.615 \\
0.665 \\
0.75 \\
0.73 \\
0.75 \\
0.745 \\
0.705 \\
0.64 \\
0.835 \\
0.89 \\
0.76 \\
0.87 \\
0.825 \\
0.66 \\
0.71 \\
0.83 \\
0.535 \\
0.715 \\
0.615 \\
0.72 \\
0.645 \\
};

\addplot[mark=*, boxplot, boxplot/draw position=10]
table[row sep=\\, y index=0] {
data
0.72 \\
0.745 \\
0.565 \\
0.845 \\
0.835 \\
0.74 \\
0.685 \\
0.79 \\
0.63 \\
0.835 \\
0.515 \\
0.76 \\
0.735 \\
0.755 \\
0.685 \\
0.62 \\
0.885 \\
0.725 \\
0.8 \\
0.695 \\
0.67 \\
0.675 \\
0.935 \\
0.6 \\
0.695 \\
0.76 \\
0.645 \\
0.665 \\
0.745 \\
0.61 \\
0.885 \\
0.945 \\
0.825 \\
0.655 \\
0.745 \\
0.74 \\
0.815 \\
0.71 \\
0.885 \\
0.78 \\
0.775 \\
0.58 \\
0.7 \\
0.68 \\
0.555 \\
0.715 \\
0.765 \\
0.805 \\
0.805 \\
0.83 \\
};

\addplot[mark=*, boxplot, boxplot/draw position=11]
table[row sep=\\, y index=0] {
data
0.68 \\
0.99 \\
0.66 \\
0.895 \\
0.725 \\
0.565 \\
0.69 \\
0.72 \\
0.665 \\
0.77 \\
0.61 \\
0.67 \\
0.72 \\
0.5 \\
0.605 \\
0.935 \\
0.555 \\
0.615 \\
0.745 \\
0.64 \\
0.605 \\
0.74 \\
0.665 \\
0.635 \\
0.76 \\
0.585 \\
0.785 \\
0.475 \\
0.69 \\
0.725 \\
0.7 \\
0.615 \\
0.465 \\
0.645 \\
0.625 \\
0.73 \\
0.6 \\
0.61 \\
0.615 \\
0.515 \\
0.525 \\
0.76 \\
0.4 \\
0.63 \\
0.655 \\
0.795 \\
0.655 \\
0.8 \\
0.585 \\
0.74 \\
};

\addplot[mark=*, boxplot, boxplot/draw position=12]
table[row sep=\\, y index=0] {
data
0.635 \\
0.665 \\
0.455 \\
0.43 \\
0.72 \\
0.69 \\
0.57 \\
0.745 \\
0.815 \\
0.69 \\
0.69 \\
0.555 \\
0.525 \\
0.745 \\
0.585 \\
0.46 \\
0.59 \\
0.45 \\
0.475 \\
0.985 \\
0.65 \\
0.725 \\
0.68 \\
0.43 \\
0.625 \\
0.79 \\
0.76 \\
0.705 \\
0.71 \\
0.57 \\
0.52 \\
0.58 \\
0.605 \\
0.495 \\
0.565 \\
0.575 \\
0.81 \\
0.595 \\
0.575 \\
0.88 \\
0.515 \\
0.495 \\
0.745 \\
0.7 \\
0.615 \\
0.375 \\
0.69 \\
0.665 \\
0.795 \\
0.7 \\
};

\addplot[mark=*, boxplot, boxplot/draw position=13]
table[row sep=\\, y index=0] {
data
0.655 \\
0.45 \\
0.48 \\
0.735 \\
0.5 \\
0.75 \\
0.635 \\
0.545 \\
0.575 \\
0.43 \\
0.755 \\
0.62 \\
0.645 \\
0.47 \\
0.69 \\
0.595 \\
0.58 \\
0.59 \\
0.54 \\
0.465 \\
0.72 \\
0.585 \\
0.74 \\
0.785 \\
0.54 \\
0.46 \\
0.43 \\
0.56 \\
0.64 \\
0.465 \\
0.445 \\
0.465 \\
0.515 \\
0.585 \\
0.6 \\
0.515 \\
0.46 \\
0.435 \\
0.515 \\
0.57 \\
0.61 \\
0.55 \\
0.7 \\
0.47 \\
0.43 \\
0.605 \\
0.56 \\
0.63 \\
0.545 \\
0.705 \\
};

\addplot[mark=*, boxplot, boxplot/draw position=14]
table[row sep=\\, y index=0] {
data
0.515 \\
0.485 \\
0.445 \\
0.445 \\
0.47 \\
0.455 \\
0.56 \\
0.445 \\
0.575 \\
0.43 \\
0.43 \\
0.43 \\
0.43 \\
0.58 \\
0.455 \\
0.43 \\
0.415 \\
0.52 \\
0.445 \\
0.595 \\
0.47 \\
0.455 \\
0.47 \\
0.445 \\
0.455 \\
0.495 \\
0.665 \\
0.63 \\
0.47 \\
0.48 \\
0.465 \\
0.535 \\
0.455 \\
0.43 \\
0.43 \\
0.475 \\
0.46 \\
0.43 \\
0.445 \\
0.43 \\
0.535 \\
0.43 \\
0.58 \\
0.505 \\
0.43 \\
0.575 \\
0.43 \\
0.435 \\
0.695 \\
0.445 \\
};

\addplot[mark=*, boxplot, boxplot/draw position=15]
table[row sep=\\, y index=0] {
data
0.63 \\
0.445 \\
0.455 \\
0.445 \\
0.46 \\
0.43 \\
0.43 \\
0.43 \\
0.575 \\
0.525 \\
0.485 \\
0.445 \\
0.635 \\
0.51 \\
0.515 \\
0.47 \\
0.5 \\
0.43 \\
0.415 \\
0.43 \\
0.45 \\
0.47 \\
0.65 \\
0.43 \\
0.455 \\
0.47 \\
0.485 \\
0.43 \\
0.465 \\
0.47 \\
0.43 \\
0.43 \\
0.47 \\
0.43 \\
0.57 \\
0.505 \\
0.57 \\
0.465 \\
0.445 \\
0.47 \\
0.47 \\
0.43 \\
0.47 \\
0.43 \\
0.43 \\
0.47 \\
0.47 \\
0.47 \\
0.4 \\
0.47 \\
};

\addplot[mark=*, boxplot, boxplot/draw position=16]
table[row sep=\\, y index=0] {
data
0.43 \\
0.6 \\
0.47 \\
0.55 \\
0.47 \\
0.43 \\
0.475 \\
0.47 \\
0.47 \\
0.47 \\
0.47 \\
0.515 \\
0.47 \\
0.47 \\
0.47 \\
0.43 \\
0.47 \\
0.47 \\
0.47 \\
0.47 \\
0.59 \\
0.43 \\
0.47 \\
0.47 \\
0.465 \\
0.47 \\
0.43 \\
0.47 \\
0.42 \\
0.445 \\
0.47 \\
0.47 \\
0.47 \\
0.47 \\
0.47 \\
0.445 \\
0.47 \\
0.47 \\
0.445 \\
0.47 \\
0.43 \\
0.45 \\
0.47 \\
0.47 \\
0.47 \\
0.6 \\
0.47 \\
0.555 \\
0.53 \\
0.47 \\
};

\addplot[mark=*, boxplot, boxplot/draw position=17]
table[row sep=\\, y index=0] {
data
0.47 \\
0.535 \\
0.47 \\
0.47 \\
0.47 \\
0.47 \\
0.47 \\
0.47 \\
0.47 \\
0.47 \\
0.47 \\
0.47 \\
0.47 \\
0.47 \\
0.47 \\
0.47 \\
0.47 \\
0.47 \\
0.47 \\
0.43 \\
0.475 \\
0.5 \\
0.47 \\
0.47 \\
0.47 \\
0.47 \\
0.47 \\
0.47 \\
0.47 \\
0.47 \\
0.47 \\
0.47 \\
0.47 \\
0.47 \\
0.47 \\
0.47 \\
0.47 \\
0.47 \\
0.47 \\
0.47 \\
0.43 \\
0.47 \\
0.47 \\
0.47 \\
0.47 \\
0.47 \\
0.545 \\
0.47 \\
0.47 \\
0.47 \\
};

\addplot[mark=*, boxplot, boxplot/draw position=18]
table[row sep=\\, y index=0] {
data
0.47 \\
0.47 \\
0.47 \\
0.47 \\
0.47 \\
0.47 \\
0.47 \\
0.47 \\
0.47 \\
0.47 \\
0.47 \\
0.47 \\
0.47 \\
0.47 \\
0.47 \\
0.47 \\
0.47 \\
0.47 \\
0.47 \\
0.47 \\
0.47 \\
0.47 \\
0.47 \\
0.47 \\
0.47 \\
0.47 \\
0.47 \\
0.47 \\
0.47 \\
0.47 \\
0.47 \\
0.47 \\
0.47 \\
0.47 \\
0.47 \\
0.47 \\
0.47 \\
0.47 \\
0.47 \\
0.47 \\
0.47 \\
0.47 \\
0.47 \\
0.47 \\
0.47 \\
0.47 \\
0.47 \\
0.47 \\
0.47 \\
0.47 \\
};
}{0.2}{Input connectivity}{29}

        }
    }
    \resizebox{\textwidth}{!}{
        \subfloat[TD3, N=10]{
            \myboxplot{

\addplot[mark=*, boxplot, boxplot/draw position=0]
table[row sep=\\, y index=0] {
data
0.595 \\
0.595 \\
0.57 \\
0.595 \\
0.595 \\
0.595 \\
0.595 \\
0.55 \\
0.595 \\
0.585 \\
0.57 \\
0.595 \\
0.595 \\
0.595 \\
0.545 \\
0.595 \\
0.595 \\
0.595 \\
0.595 \\
0.57 \\
0.595 \\
0.585 \\
0.57 \\
0.595 \\
0.61 \\
0.595 \\
0.595 \\
0.595 \\
0.595 \\
0.595 \\
0.595 \\
0.595 \\
0.615 \\
0.55 \\
0.57 \\
0.595 \\
0.585 \\
0.595 \\
0.585 \\
0.595 \\
0.57 \\
0.595 \\
0.595 \\
0.595 \\
0.595 \\
0.595 \\
0.595 \\
0.55 \\
0.55 \\
0.595 \\
};

\addplot[mark=*, boxplot, boxplot/draw position=1]
table[row sep=\\, y index=0] {
data
0.9 \\
1.0 \\
0.72 \\
0.95 \\
0.9 \\
0.8 \\
0.87 \\
0.94 \\
0.745 \\
0.805 \\
0.875 \\
0.86 \\
0.885 \\
0.8 \\
0.8 \\
0.82 \\
0.91 \\
0.745 \\
0.84 \\
1.0 \\
0.89 \\
0.805 \\
0.9 \\
0.8 \\
0.89 \\
0.815 \\
0.92 \\
0.815 \\
0.81 \\
0.785 \\
0.8 \\
0.855 \\
0.8 \\
0.745 \\
0.745 \\
0.725 \\
0.745 \\
0.805 \\
0.8 \\
0.9 \\
0.9 \\
0.9 \\
1.0 \\
0.99 \\
0.96 \\
0.935 \\
0.87 \\
0.875 \\
0.835 \\
0.845 \\
};

\addplot[mark=*, boxplot, boxplot/draw position=2]
table[row sep=\\, y index=0] {
data
0.745 \\
0.745 \\
0.745 \\
0.745 \\
0.745 \\
0.745 \\
0.745 \\
0.745 \\
0.745 \\
0.745 \\
0.745 \\
0.745 \\
0.745 \\
0.745 \\
0.745 \\
0.745 \\
0.745 \\
0.745 \\
0.745 \\
0.745 \\
0.745 \\
0.745 \\
0.745 \\
0.745 \\
0.745 \\
0.745 \\
0.745 \\
0.745 \\
0.745 \\
0.745 \\
0.745 \\
0.745 \\
0.745 \\
0.745 \\
0.745 \\
0.745 \\
0.745 \\
0.745 \\
0.745 \\
0.745 \\
0.745 \\
0.745 \\
0.745 \\
0.745 \\
0.745 \\
0.745 \\
0.745 \\
0.745 \\
0.745 \\
0.745 \\
};
}{0.2}{}{7}

        }
        \subfloat[TD5, N=55]{
            \myboxplot{

\addplot[mark=*, boxplot, boxplot/draw position=0]
table[row sep=\\, y index=0] {
data
0.505 \\
0.475 \\
0.505 \\
0.475 \\
0.55 \\
0.545 \\
0.51 \\
0.52 \\
0.535 \\
0.595 \\
0.54 \\
0.555 \\
0.485 \\
0.48 \\
0.51 \\
0.54 \\
0.53 \\
0.525 \\
0.495 \\
0.56 \\
0.555 \\
0.48 \\
0.525 \\
0.47 \\
0.535 \\
0.535 \\
0.435 \\
0.535 \\
0.495 \\
0.46 \\
0.495 \\
0.52 \\
0.565 \\
0.535 \\
0.485 \\
0.515 \\
0.52 \\
0.55 \\
0.53 \\
0.535 \\
0.535 \\
0.48 \\
0.49 \\
0.51 \\
0.585 \\
0.485 \\
0.42 \\
0.535 \\
0.47 \\
0.52 \\
};

\addplot[mark=*, boxplot, boxplot/draw position=1]
table[row sep=\\, y index=0] {
data
0.78 \\
0.725 \\
0.665 \\
0.75 \\
0.735 \\
0.705 \\
0.71 \\
0.77 \\
0.755 \\
0.715 \\
0.74 \\
0.745 \\
0.74 \\
0.735 \\
0.895 \\
0.71 \\
0.84 \\
0.77 \\
0.79 \\
0.75 \\
0.82 \\
0.9 \\
0.88 \\
0.725 \\
0.77 \\
0.77 \\
0.7 \\
0.75 \\
0.765 \\
0.815 \\
0.75 \\
0.85 \\
0.805 \\
0.745 \\
0.775 \\
0.705 \\
0.735 \\
0.745 \\
0.655 \\
0.725 \\
0.705 \\
0.94 \\
0.725 \\
0.72 \\
0.79 \\
0.755 \\
0.68 \\
0.775 \\
0.72 \\
0.705 \\
};

\addplot[mark=*, boxplot, boxplot/draw position=2]
table[row sep=\\, y index=0] {
data
0.78 \\
0.865 \\
0.77 \\
0.81 \\
0.795 \\
0.775 \\
0.805 \\
0.805 \\
0.855 \\
0.81 \\
0.855 \\
0.735 \\
0.805 \\
0.79 \\
0.785 \\
0.815 \\
0.76 \\
0.835 \\
0.845 \\
0.785 \\
0.755 \\
0.785 \\
0.77 \\
0.76 \\
0.775 \\
0.895 \\
0.82 \\
0.87 \\
0.895 \\
0.87 \\
0.735 \\
0.795 \\
0.805 \\
0.79 \\
0.83 \\
0.885 \\
0.885 \\
0.775 \\
0.835 \\
0.8 \\
0.79 \\
0.84 \\
0.8 \\
0.79 \\
0.835 \\
0.815 \\
0.87 \\
0.715 \\
0.825 \\
0.785 \\
};

\addplot[mark=*, boxplot, boxplot/draw position=3]
table[row sep=\\, y index=0] {
data
0.82 \\
0.82 \\
0.76 \\
0.825 \\
0.85 \\
0.82 \\
0.815 \\
0.97 \\
0.795 \\
0.79 \\
0.77 \\
0.82 \\
0.795 \\
0.865 \\
0.835 \\
0.86 \\
0.805 \\
0.895 \\
0.91 \\
0.895 \\
0.73 \\
0.855 \\
0.805 \\
0.82 \\
0.87 \\
0.935 \\
0.905 \\
0.895 \\
0.855 \\
0.77 \\
0.805 \\
0.915 \\
0.865 \\
0.75 \\
0.855 \\
0.805 \\
0.865 \\
0.845 \\
0.835 \\
0.845 \\
0.935 \\
0.84 \\
0.805 \\
0.875 \\
0.79 \\
0.845 \\
0.855 \\
0.795 \\
0.86 \\
0.89 \\
};

\addplot[mark=*, boxplot, boxplot/draw position=4]
table[row sep=\\, y index=0] {
data
0.915 \\
0.83 \\
0.85 \\
0.76 \\
0.84 \\
0.86 \\
0.905 \\
0.775 \\
0.85 \\
0.76 \\
0.905 \\
0.95 \\
0.99 \\
0.905 \\
0.87 \\
0.815 \\
0.845 \\
0.875 \\
0.82 \\
0.9 \\
0.835 \\
0.815 \\
0.86 \\
0.805 \\
0.92 \\
0.845 \\
0.95 \\
0.855 \\
0.935 \\
0.85 \\
0.885 \\
0.825 \\
0.91 \\
0.755 \\
0.835 \\
0.885 \\
0.9 \\
0.86 \\
0.8 \\
0.795 \\
0.915 \\
0.875 \\
0.885 \\
0.93 \\
0.805 \\
0.945 \\
0.92 \\
0.89 \\
0.89 \\
0.9 \\
};

\addplot[mark=*, boxplot, boxplot/draw position=5]
table[row sep=\\, y index=0] {
data
0.865 \\
0.915 \\
0.855 \\
0.95 \\
0.9 \\
0.885 \\
0.92 \\
0.9 \\
0.86 \\
0.925 \\
0.855 \\
0.885 \\
0.795 \\
0.9 \\
0.85 \\
0.88 \\
0.78 \\
0.96 \\
0.845 \\
0.83 \\
0.825 \\
0.92 \\
0.91 \\
0.845 \\
0.83 \\
0.95 \\
0.76 \\
0.87 \\
0.91 \\
0.925 \\
0.84 \\
0.77 \\
0.885 \\
0.915 \\
0.92 \\
0.835 \\
0.975 \\
0.89 \\
0.86 \\
0.95 \\
0.865 \\
0.91 \\
0.825 \\
0.905 \\
0.855 \\
0.965 \\
0.955 \\
0.915 \\
0.82 \\
0.89 \\
};

\addplot[mark=*, boxplot, boxplot/draw position=6]
table[row sep=\\, y index=0] {
data
0.88 \\
0.875 \\
0.895 \\
0.945 \\
0.91 \\
0.9 \\
0.85 \\
0.82 \\
0.915 \\
0.85 \\
0.84 \\
0.915 \\
0.895 \\
0.98 \\
0.895 \\
0.915 \\
0.825 \\
0.85 \\
0.86 \\
0.81 \\
0.84 \\
0.885 \\
0.855 \\
0.865 \\
0.93 \\
0.805 \\
0.87 \\
0.855 \\
0.965 \\
0.905 \\
0.845 \\
0.865 \\
0.84 \\
0.805 \\
0.82 \\
0.885 \\
0.855 \\
0.905 \\
0.82 \\
0.935 \\
0.925 \\
0.93 \\
0.855 \\
0.825 \\
0.835 \\
0.835 \\
0.855 \\
0.91 \\
0.7 \\
0.915 \\
};

\addplot[mark=*, boxplot, boxplot/draw position=7]
table[row sep=\\, y index=0] {
data
0.87 \\
0.815 \\
0.82 \\
0.805 \\
0.87 \\
0.845 \\
0.795 \\
0.845 \\
0.775 \\
0.84 \\
0.835 \\
0.925 \\
0.96 \\
0.7 \\
0.79 \\
0.82 \\
0.885 \\
0.865 \\
0.82 \\
0.8 \\
0.815 \\
0.89 \\
0.805 \\
0.81 \\
0.7 \\
0.845 \\
0.75 \\
0.795 \\
0.825 \\
0.76 \\
0.81 \\
0.795 \\
0.87 \\
0.83 \\
0.855 \\
0.94 \\
0.865 \\
0.85 \\
0.83 \\
0.83 \\
0.865 \\
0.865 \\
0.845 \\
0.805 \\
0.84 \\
0.805 \\
0.795 \\
0.915 \\
0.82 \\
0.79 \\
};

\addplot[mark=*, boxplot, boxplot/draw position=8]
table[row sep=\\, y index=0] {
data
0.755 \\
0.7 \\
0.755 \\
0.82 \\
0.82 \\
0.82 \\
0.82 \\
0.825 \\
0.82 \\
0.82 \\
0.895 \\
0.805 \\
0.885 \\
0.815 \\
0.805 \\
0.82 \\
0.82 \\
0.82 \\
0.765 \\
0.795 \\
0.835 \\
0.81 \\
0.835 \\
0.805 \\
0.7 \\
0.775 \\
0.82 \\
0.74 \\
0.84 \\
0.765 \\
0.81 \\
0.865 \\
0.795 \\
0.845 \\
0.755 \\
0.82 \\
0.775 \\
0.765 \\
0.82 \\
0.9 \\
0.7 \\
0.82 \\
0.74 \\
0.82 \\
0.82 \\
0.81 \\
0.81 \\
0.755 \\
0.74 \\
0.82 \\
};

\addplot[mark=*, boxplot, boxplot/draw position=9]
table[row sep=\\, y index=0] {
data
0.765 \\
0.765 \\
0.82 \\
0.675 \\
0.755 \\
0.755 \\
0.72 \\
0.82 \\
0.81 \\
0.755 \\
0.755 \\
0.72 \\
0.765 \\
0.7 \\
0.705 \\
0.82 \\
0.735 \\
0.755 \\
0.75 \\
0.82 \\
0.82 \\
0.7 \\
0.765 \\
0.755 \\
0.82 \\
0.7 \\
0.7 \\
0.7 \\
0.82 \\
0.7 \\
0.82 \\
0.75 \\
0.7 \\
0.7 \\
0.84 \\
0.805 \\
0.82 \\
0.765 \\
0.7 \\
0.755 \\
0.725 \\
0.81 \\
0.835 \\
0.72 \\
0.755 \\
0.7 \\
0.76 \\
0.755 \\
0.84 \\
0.82 \\
};

\addplot[mark=*, boxplot, boxplot/draw position=10]
table[row sep=\\, y index=0] {
data
0.7 \\
0.7 \\
0.7 \\
0.7 \\
0.765 \\
0.765 \\
0.68 \\
0.7 \\
0.755 \\
0.7 \\
0.7 \\
0.7 \\
0.765 \\
0.755 \\
0.68 \\
0.82 \\
0.7 \\
0.7 \\
0.82 \\
0.82 \\
0.7 \\
0.7 \\
0.755 \\
0.82 \\
0.7 \\
0.7 \\
0.765 \\
0.7 \\
0.84 \\
0.82 \\
0.7 \\
0.7 \\
0.7 \\
0.755 \\
0.68 \\
0.755 \\
0.7 \\
0.7 \\
0.7 \\
0.7 \\
0.7 \\
0.68 \\
0.81 \\
0.7 \\
0.7 \\
0.7 \\
0.765 \\
0.7 \\
0.7 \\
0.7 \\
};

\addplot[mark=*, boxplot, boxplot/draw position=11]
table[row sep=\\, y index=0] {
data
0.68 \\
0.68 \\
0.68 \\
0.68 \\
0.68 \\
0.68 \\
0.68 \\
0.68 \\
0.68 \\
0.68 \\
0.68 \\
0.68 \\
0.68 \\
0.68 \\
0.68 \\
0.68 \\
0.68 \\
0.68 \\
0.68 \\
0.68 \\
0.68 \\
0.68 \\
0.68 \\
0.68 \\
0.68 \\
0.68 \\
0.68 \\
0.68 \\
0.68 \\
0.68 \\
0.68 \\
0.68 \\
0.68 \\
0.68 \\
0.68 \\
0.68 \\
0.68 \\
0.68 \\
0.68 \\
0.68 \\
0.68 \\
0.68 \\
0.68 \\
0.68 \\
0.68 \\
0.68 \\
0.68 \\
0.68 \\
0.68 \\
0.68 \\
};
}{0.2}{}{14}

        }
    }
\end{figure}

\begin{figure}[ht]
    \centering
    \caption{Accuracy plots for the largest reservoirs created for each of the TP3, TP5, TD3, TD5 tasks.}

    \label{fig:accuracy-max-size}
    \resizebox{\textwidth}{!}{
        \subfloat[TP3, N=100]{
            \myboxplot{

\addplot[mark=*, boxplot, boxplot/draw position=0]
table[row sep=\\, y index=0] {
data
0.52 \\
0.47 \\
0.49 \\
0.475 \\
0.475 \\
0.49 \\
0.535 \\
0.495 \\
0.495 \\
0.515 \\
0.475 \\
0.5 \\
0.47 \\
0.505 \\
0.48 \\
0.495 \\
0.43 \\
0.495 \\
0.53 \\
0.495 \\
0.47 \\
0.53 \\
0.51 \\
0.475 \\
0.515 \\
0.585 \\
0.545 \\
0.515 \\
0.485 \\
0.535 \\
0.49 \\
0.465 \\
0.5 \\
0.515 \\
0.51 \\
0.485 \\
0.505 \\
0.51 \\
0.49 \\
0.5 \\
0.475 \\
0.485 \\
0.5 \\
0.49 \\
0.535 \\
0.455 \\
0.465 \\
0.475 \\
0.485 \\
0.56 \\
};

\addplot[mark=*, boxplot, boxplot/draw position=1]
table[row sep=\\, y index=0] {
data
0.46 \\
0.645 \\
0.56 \\
0.6 \\
0.47 \\
0.47 \\
0.385 \\
0.45 \\
0.615 \\
0.54 \\
0.545 \\
0.485 \\
0.65 \\
0.52 \\
0.6 \\
0.505 \\
0.585 \\
0.52 \\
0.51 \\
0.44 \\
0.635 \\
0.515 \\
0.585 \\
0.515 \\
0.59 \\
0.52 \\
0.49 \\
0.5 \\
0.53 \\
0.58 \\
0.465 \\
0.59 \\
0.485 \\
0.56 \\
0.45 \\
0.55 \\
0.495 \\
0.52 \\
0.69 \\
0.48 \\
0.5 \\
0.52 \\
0.48 \\
0.55 \\
0.47 \\
0.515 \\
0.545 \\
0.53 \\
0.485 \\
0.605 \\
};

\addplot[mark=*, boxplot, boxplot/draw position=2]
table[row sep=\\, y index=0] {
data
0.535 \\
0.56 \\
0.63 \\
0.43 \\
0.635 \\
0.665 \\
0.58 \\
0.54 \\
0.685 \\
0.865 \\
0.685 \\
0.545 \\
0.6 \\
0.715 \\
0.735 \\
0.735 \\
0.645 \\
0.56 \\
0.52 \\
0.52 \\
0.845 \\
0.495 \\
0.505 \\
0.525 \\
0.5 \\
0.58 \\
0.585 \\
0.745 \\
0.48 \\
0.47 \\
0.7 \\
0.48 \\
0.67 \\
0.685 \\
0.545 \\
0.47 \\
0.595 \\
0.59 \\
0.66 \\
0.545 \\
0.53 \\
0.595 \\
0.605 \\
0.82 \\
0.495 \\
0.465 \\
0.54 \\
0.815 \\
0.71 \\
0.635 \\
};

\addplot[mark=*, boxplot, boxplot/draw position=3]
table[row sep=\\, y index=0] {
data
0.675 \\
0.685 \\
0.785 \\
0.72 \\
0.8 \\
0.64 \\
0.615 \\
0.745 \\
0.605 \\
0.905 \\
0.59 \\
0.985 \\
0.88 \\
0.7 \\
0.925 \\
0.585 \\
0.61 \\
0.89 \\
0.745 \\
0.895 \\
0.63 \\
0.775 \\
0.83 \\
0.75 \\
0.69 \\
0.685 \\
0.73 \\
0.555 \\
0.76 \\
0.58 \\
0.625 \\
0.695 \\
0.54 \\
0.71 \\
0.54 \\
0.76 \\
0.55 \\
0.805 \\
0.855 \\
0.755 \\
0.74 \\
0.71 \\
0.745 \\
0.805 \\
0.53 \\
0.735 \\
0.52 \\
0.79 \\
0.735 \\
0.855 \\
};

\addplot[mark=*, boxplot, boxplot/draw position=4]
table[row sep=\\, y index=0] {
data
0.715 \\
0.795 \\
0.74 \\
0.87 \\
0.59 \\
0.865 \\
0.805 \\
0.83 \\
0.865 \\
0.8 \\
0.805 \\
0.69 \\
0.86 \\
0.875 \\
0.78 \\
0.775 \\
0.805 \\
0.57 \\
0.835 \\
0.68 \\
0.765 \\
0.775 \\
0.75 \\
0.61 \\
0.905 \\
0.875 \\
0.865 \\
0.905 \\
0.885 \\
0.795 \\
0.62 \\
0.97 \\
0.655 \\
0.805 \\
0.735 \\
0.55 \\
0.695 \\
0.75 \\
0.865 \\
0.54 \\
0.845 \\
0.705 \\
0.88 \\
0.88 \\
0.775 \\
0.805 \\
0.975 \\
0.68 \\
0.795 \\
0.81 \\
};

\addplot[mark=*, boxplot, boxplot/draw position=5]
table[row sep=\\, y index=0] {
data
0.835 \\
0.93 \\
0.71 \\
0.865 \\
0.99 \\
0.78 \\
0.83 \\
0.79 \\
0.98 \\
0.89 \\
0.815 \\
0.98 \\
0.87 \\
0.94 \\
0.91 \\
0.88 \\
0.84 \\
0.94 \\
0.74 \\
0.93 \\
0.995 \\
0.66 \\
0.79 \\
0.825 \\
0.77 \\
0.83 \\
0.935 \\
0.74 \\
0.825 \\
0.85 \\
0.815 \\
0.95 \\
0.72 \\
0.925 \\
0.955 \\
0.855 \\
0.945 \\
0.955 \\
0.855 \\
0.83 \\
0.82 \\
0.85 \\
0.87 \\
0.845 \\
0.95 \\
0.955 \\
0.84 \\
0.905 \\
0.97 \\
0.94 \\
};

\addplot[mark=*, boxplot, boxplot/draw position=6]
table[row sep=\\, y index=0] {
data
0.995 \\
0.895 \\
0.935 \\
0.97 \\
0.805 \\
0.99 \\
0.995 \\
0.935 \\
0.98 \\
0.84 \\
0.77 \\
0.91 \\
0.9 \\
0.995 \\
0.96 \\
0.94 \\
0.975 \\
0.985 \\
0.995 \\
0.795 \\
0.905 \\
0.925 \\
0.89 \\
0.95 \\
0.88 \\
0.855 \\
0.995 \\
0.835 \\
0.83 \\
0.87 \\
0.89 \\
0.995 \\
0.94 \\
0.965 \\
0.91 \\
0.99 \\
0.785 \\
0.995 \\
0.77 \\
0.77 \\
0.865 \\
0.975 \\
0.99 \\
0.9 \\
0.975 \\
0.595 \\
0.955 \\
0.925 \\
0.975 \\
0.845 \\
};

\addplot[mark=*, boxplot, boxplot/draw position=7]
table[row sep=\\, y index=0] {
data
0.95 \\
0.855 \\
0.985 \\
0.94 \\
0.96 \\
0.995 \\
0.93 \\
0.845 \\
0.965 \\
0.87 \\
0.935 \\
0.995 \\
0.9 \\
0.995 \\
0.995 \\
0.84 \\
0.97 \\
0.995 \\
0.79 \\
0.975 \\
0.88 \\
0.995 \\
0.985 \\
0.985 \\
0.945 \\
0.935 \\
0.955 \\
0.94 \\
0.995 \\
0.955 \\
0.99 \\
0.975 \\
0.91 \\
0.85 \\
0.97 \\
0.995 \\
0.995 \\
0.875 \\
0.79 \\
0.995 \\
0.98 \\
0.995 \\
0.855 \\
0.995 \\
0.985 \\
0.865 \\
0.985 \\
0.995 \\
0.895 \\
0.945 \\
};

\addplot[mark=*, boxplot, boxplot/draw position=8]
table[row sep=\\, y index=0] {
data
0.995 \\
0.95 \\
0.985 \\
0.995 \\
0.895 \\
0.955 \\
0.955 \\
0.97 \\
0.995 \\
0.995 \\
0.985 \\
0.965 \\
0.995 \\
0.995 \\
0.96 \\
0.965 \\
0.995 \\
0.915 \\
0.995 \\
0.99 \\
0.96 \\
0.995 \\
0.96 \\
0.985 \\
0.985 \\
0.995 \\
0.995 \\
0.98 \\
0.945 \\
0.995 \\
0.995 \\
0.99 \\
0.995 \\
0.995 \\
0.985 \\
0.945 \\
0.97 \\
0.975 \\
0.955 \\
0.995 \\
0.985 \\
0.975 \\
0.995 \\
0.995 \\
0.99 \\
0.985 \\
0.975 \\
0.995 \\
0.97 \\
0.985 \\
};

\addplot[mark=*, boxplot, boxplot/draw position=9]
table[row sep=\\, y index=0] {
data
0.995 \\
0.995 \\
0.995 \\
0.985 \\
0.995 \\
0.995 \\
0.975 \\
0.995 \\
0.94 \\
0.985 \\
0.995 \\
0.995 \\
0.995 \\
0.965 \\
0.995 \\
0.995 \\
0.98 \\
0.995 \\
0.995 \\
0.995 \\
0.995 \\
0.995 \\
0.975 \\
0.98 \\
0.995 \\
0.995 \\
0.97 \\
0.995 \\
0.995 \\
0.985 \\
0.985 \\
0.945 \\
0.895 \\
0.995 \\
0.995 \\
0.985 \\
0.995 \\
0.995 \\
0.995 \\
0.88 \\
0.875 \\
0.995 \\
0.995 \\
0.97 \\
0.995 \\
0.995 \\
0.995 \\
0.995 \\
0.995 \\
0.995 \\
};

\addplot[mark=*, boxplot, boxplot/draw position=10]
table[row sep=\\, y index=0] {
data
0.995 \\
0.995 \\
0.925 \\
0.995 \\
0.995 \\
0.995 \\
0.995 \\
0.955 \\
0.985 \\
0.99 \\
0.995 \\
0.88 \\
0.995 \\
0.995 \\
0.995 \\
0.995 \\
0.995 \\
0.995 \\
0.995 \\
0.945 \\
0.995 \\
0.995 \\
0.995 \\
0.995 \\
0.885 \\
0.995 \\
0.995 \\
0.97 \\
0.995 \\
0.995 \\
0.995 \\
0.995 \\
0.995 \\
0.995 \\
0.995 \\
0.995 \\
0.85 \\
0.995 \\
0.995 \\
0.995 \\
0.995 \\
0.965 \\
0.995 \\
0.995 \\
0.995 \\
0.995 \\
0.995 \\
0.995 \\
0.995 \\
0.995 \\
};

\addplot[mark=*, boxplot, boxplot/draw position=11]
table[row sep=\\, y index=0] {
data
0.955 \\
0.995 \\
0.995 \\
0.995 \\
0.995 \\
0.995 \\
0.995 \\
0.995 \\
0.995 \\
0.995 \\
0.995 \\
0.995 \\
0.995 \\
0.995 \\
0.995 \\
0.995 \\
0.995 \\
0.995 \\
0.995 \\
0.995 \\
0.985 \\
0.995 \\
0.985 \\
0.995 \\
0.985 \\
0.985 \\
0.995 \\
0.995 \\
0.995 \\
0.97 \\
0.995 \\
0.995 \\
0.995 \\
0.995 \\
0.995 \\
0.995 \\
0.995 \\
0.995 \\
0.995 \\
0.995 \\
0.995 \\
0.995 \\
0.895 \\
0.995 \\
0.995 \\
0.995 \\
0.995 \\
0.995 \\
0.99 \\
0.995 \\
};

\addplot[mark=*, boxplot, boxplot/draw position=12]
table[row sep=\\, y index=0] {
data
0.995 \\
0.91 \\
0.995 \\
0.995 \\
0.995 \\
0.995 \\
0.97 \\
0.995 \\
0.995 \\
0.995 \\
0.995 \\
0.995 \\
0.955 \\
0.995 \\
0.995 \\
0.995 \\
0.995 \\
0.995 \\
0.965 \\
0.995 \\
0.995 \\
0.995 \\
0.995 \\
0.995 \\
0.995 \\
0.995 \\
0.995 \\
0.945 \\
0.995 \\
0.995 \\
0.995 \\
0.985 \\
0.995 \\
0.995 \\
0.995 \\
0.995 \\
0.995 \\
0.995 \\
0.995 \\
0.995 \\
0.995 \\
0.995 \\
0.995 \\
0.995 \\
0.995 \\
0.995 \\
0.995 \\
0.995 \\
0.995 \\
0.995 \\
};

\addplot[mark=*, boxplot, boxplot/draw position=13]
table[row sep=\\, y index=0] {
data
0.995 \\
0.995 \\
0.995 \\
0.995 \\
0.995 \\
0.995 \\
0.995 \\
0.995 \\
0.995 \\
0.995 \\
0.905 \\
0.995 \\
0.995 \\
0.995 \\
0.98 \\
0.995 \\
0.995 \\
0.995 \\
0.995 \\
0.995 \\
0.995 \\
0.995 \\
0.995 \\
0.995 \\
0.995 \\
0.995 \\
0.995 \\
0.995 \\
0.995 \\
0.995 \\
0.995 \\
0.995 \\
0.995 \\
0.995 \\
0.995 \\
0.995 \\
0.995 \\
0.955 \\
0.97 \\
0.995 \\
0.98 \\
0.995 \\
0.995 \\
0.995 \\
0.995 \\
0.995 \\
0.995 \\
0.985 \\
0.995 \\
0.995 \\
};

\addplot[mark=*, boxplot, boxplot/draw position=14]
table[row sep=\\, y index=0] {
data
0.995 \\
0.995 \\
0.8 \\
0.995 \\
0.975 \\
0.995 \\
0.995 \\
0.97 \\
0.955 \\
0.995 \\
0.995 \\
0.995 \\
0.995 \\
0.955 \\
0.995 \\
0.995 \\
0.975 \\
0.995 \\
0.995 \\
0.995 \\
0.995 \\
0.995 \\
0.995 \\
0.87 \\
0.995 \\
0.995 \\
0.91 \\
0.815 \\
0.885 \\
0.995 \\
0.985 \\
0.995 \\
0.995 \\
0.995 \\
0.995 \\
0.995 \\
0.995 \\
0.975 \\
0.97 \\
0.995 \\
0.935 \\
0.995 \\
0.995 \\
0.995 \\
0.965 \\
0.96 \\
0.995 \\
0.995 \\
0.995 \\
0.995 \\
};

\addplot[mark=*, boxplot, boxplot/draw position=15]
table[row sep=\\, y index=0] {
data
0.995 \\
0.995 \\
0.995 \\
0.995 \\
0.995 \\
0.73 \\
0.995 \\
0.495 \\
0.995 \\
0.995 \\
0.87 \\
0.995 \\
0.81 \\
0.995 \\
0.995 \\
0.495 \\
0.855 \\
0.94 \\
0.995 \\
0.995 \\
0.87 \\
0.73 \\
0.99 \\
0.995 \\
0.905 \\
0.995 \\
0.995 \\
0.995 \\
0.935 \\
0.995 \\
0.975 \\
0.96 \\
0.965 \\
0.995 \\
0.905 \\
0.995 \\
0.99 \\
0.87 \\
0.74 \\
0.37 \\
0.995 \\
0.37 \\
0.995 \\
0.995 \\
0.995 \\
0.825 \\
0.93 \\
0.995 \\
0.995 \\
0.995 \\
};

\addplot[mark=*, boxplot, boxplot/draw position=16]
table[row sep=\\, y index=0] {
data
0.73 \\
0.855 \\
0.995 \\
0.88 \\
0.905 \\
0.995 \\
0.73 \\
0.995 \\
0.945 \\
0.86 \\
0.855 \\
0.995 \\
0.69 \\
0.995 \\
0.965 \\
0.87 \\
0.88 \\
0.995 \\
0.995 \\
0.995 \\
0.995 \\
0.82 \\
0.995 \\
0.495 \\
0.62 \\
0.875 \\
0.995 \\
0.995 \\
0.73 \\
0.44 \\
0.995 \\
0.995 \\
0.87 \\
0.995 \\
0.995 \\
0.995 \\
0.995 \\
0.93 \\
0.75 \\
0.915 \\
0.635 \\
0.995 \\
0.9 \\
0.76 \\
0.87 \\
0.88 \\
0.935 \\
0.995 \\
0.995 \\
0.995 \\
};

\addplot[mark=*, boxplot, boxplot/draw position=17]
table[row sep=\\, y index=0] {
data
0.995 \\
0.495 \\
0.495 \\
0.915 \\
0.995 \\
0.995 \\
0.875 \\
0.495 \\
0.995 \\
0.635 \\
0.635 \\
0.62 \\
0.905 \\
0.735 \\
0.76 \\
0.995 \\
0.935 \\
0.76 \\
0.73 \\
0.615 \\
0.495 \\
0.995 \\
0.82 \\
0.79 \\
0.48 \\
0.995 \\
0.995 \\
0.76 \\
0.76 \\
0.73 \\
0.495 \\
0.725 \\
0.495 \\
0.9 \\
0.995 \\
0.88 \\
0.665 \\
0.75 \\
0.48 \\
0.495 \\
0.995 \\
0.87 \\
0.995 \\
0.995 \\
0.875 \\
0.615 \\
0.945 \\
0.76 \\
0.495 \\
0.495 \\
};

\addplot[mark=*, boxplot, boxplot/draw position=18]
table[row sep=\\, y index=0] {
data
0.635 \\
0.995 \\
0.875 \\
0.635 \\
0.93 \\
0.495 \\
0.495 \\
0.495 \\
0.76 \\
0.73 \\
0.495 \\
0.855 \\
0.76 \\
0.495 \\
0.795 \\
0.495 \\
0.76 \\
0.495 \\
0.76 \\
0.495 \\
0.495 \\
0.495 \\
0.87 \\
0.495 \\
0.495 \\
0.67 \\
0.525 \\
0.495 \\
0.495 \\
0.76 \\
0.73 \\
0.875 \\
0.56 \\
0.495 \\
0.76 \\
0.495 \\
0.495 \\
0.495 \\
0.495 \\
0.995 \\
0.53 \\
0.73 \\
0.995 \\
0.495 \\
0.8 \\
0.695 \\
0.495 \\
0.495 \\
0.995 \\
0.48 \\
};

\addplot[mark=*, boxplot, boxplot/draw position=19]
table[row sep=\\, y index=0] {
data
0.495 \\
0.495 \\
0.495 \\
0.495 \\
0.495 \\
0.495 \\
0.495 \\
0.495 \\
0.495 \\
0.495 \\
0.995 \\
0.495 \\
0.61 \\
0.495 \\
0.495 \\
0.495 \\
0.87 \\
0.685 \\
0.495 \\
0.495 \\
0.495 \\
0.495 \\
0.495 \\
0.495 \\
0.495 \\
0.495 \\
0.495 \\
0.495 \\
0.495 \\
0.495 \\
0.495 \\
0.495 \\
0.495 \\
0.495 \\
0.495 \\
0.495 \\
0.495 \\
0.495 \\
0.76 \\
0.495 \\
0.495 \\
0.495 \\
0.495 \\
0.495 \\
0.495 \\
0.995 \\
0.495 \\
0.495 \\
0.495 \\
0.495 \\
};

\addplot[mark=*, boxplot, boxplot/draw position=20]
table[row sep=\\, y index=0] {
data
0.495 \\
0.495 \\
0.495 \\
0.495 \\
0.495 \\
0.495 \\
0.495 \\
0.495 \\
0.495 \\
0.495 \\
0.495 \\
0.495 \\
0.495 \\
0.495 \\
0.495 \\
0.495 \\
0.495 \\
0.495 \\
0.495 \\
0.495 \\
0.495 \\
0.495 \\
0.495 \\
0.495 \\
0.495 \\
0.495 \\
0.495 \\
0.495 \\
0.495 \\
0.495 \\
0.495 \\
0.495 \\
0.495 \\
0.495 \\
0.495 \\
0.495 \\
0.495 \\
0.495 \\
0.495 \\
0.495 \\
0.495 \\
0.495 \\
0.495 \\
0.495 \\
0.495 \\
0.495 \\
0.495 \\
0.495 \\
0.495 \\
0.495 \\
};
}{0.2}{Input connectivity}{21}

        }
        \subfloat[TP5, N=140]{
            \myboxplot{

\addplot[mark=*, boxplot, boxplot/draw position=0]
table[row sep=\\, y index=0] {
data
0.53 \\
0.495 \\
0.58 \\
0.495 \\
0.47 \\
0.47 \\
0.515 \\
0.53 \\
0.53 \\
0.52 \\
0.475 \\
0.56 \\
0.485 \\
0.565 \\
0.515 \\
0.485 \\
0.53 \\
0.43 \\
0.515 \\
0.5 \\
0.51 \\
0.425 \\
0.5 \\
0.5 \\
0.465 \\
0.485 \\
0.495 \\
0.45 \\
0.48 \\
0.505 \\
0.51 \\
0.49 \\
0.505 \\
0.55 \\
0.515 \\
0.465 \\
0.525 \\
0.47 \\
0.58 \\
0.53 \\
0.49 \\
0.575 \\
0.495 \\
0.535 \\
0.525 \\
0.525 \\
0.535 \\
0.505 \\
0.5 \\
0.505 \\
};

\addplot[mark=*, boxplot, boxplot/draw position=1]
table[row sep=\\, y index=0] {
data
0.48 \\
0.51 \\
0.57 \\
0.54 \\
0.49 \\
0.52 \\
0.475 \\
0.48 \\
0.54 \\
0.42 \\
0.515 \\
0.5 \\
0.435 \\
0.515 \\
0.565 \\
0.45 \\
0.51 \\
0.435 \\
0.54 \\
0.47 \\
0.43 \\
0.54 \\
0.48 \\
0.5 \\
0.5 \\
0.52 \\
0.555 \\
0.555 \\
0.49 \\
0.545 \\
0.49 \\
0.5 \\
0.48 \\
0.47 \\
0.54 \\
0.51 \\
0.455 \\
0.565 \\
0.46 \\
0.51 \\
0.48 \\
0.515 \\
0.52 \\
0.46 \\
0.525 \\
0.42 \\
0.475 \\
0.48 \\
0.47 \\
0.55 \\
};

\addplot[mark=*, boxplot, boxplot/draw position=2]
table[row sep=\\, y index=0] {
data
0.47 \\
0.435 \\
0.47 \\
0.47 \\
0.46 \\
0.48 \\
0.49 \\
0.54 \\
0.51 \\
0.53 \\
0.45 \\
0.52 \\
0.515 \\
0.475 \\
0.495 \\
0.505 \\
0.495 \\
0.46 \\
0.545 \\
0.52 \\
0.52 \\
0.495 \\
0.5 \\
0.465 \\
0.525 \\
0.46 \\
0.46 \\
0.505 \\
0.48 \\
0.555 \\
0.445 \\
0.48 \\
0.5 \\
0.48 \\
0.435 \\
0.445 \\
0.525 \\
0.49 \\
0.465 \\
0.525 \\
0.465 \\
0.52 \\
0.53 \\
0.47 \\
0.57 \\
0.515 \\
0.435 \\
0.46 \\
0.47 \\
0.49 \\
};

\addplot[mark=*, boxplot, boxplot/draw position=3]
table[row sep=\\, y index=0] {
data
0.45 \\
0.465 \\
0.51 \\
0.49 \\
0.495 \\
0.52 \\
0.47 \\
0.51 \\
0.57 \\
0.49 \\
0.51 \\
0.46 \\
0.5 \\
0.545 \\
0.47 \\
0.515 \\
0.43 \\
0.49 \\
0.47 \\
0.515 \\
0.475 \\
0.45 \\
0.475 \\
0.535 \\
0.52 \\
0.565 \\
0.5 \\
0.545 \\
0.49 \\
0.53 \\
0.555 \\
0.475 \\
0.47 \\
0.47 \\
0.48 \\
0.52 \\
0.575 \\
0.455 \\
0.445 \\
0.49 \\
0.525 \\
0.515 \\
0.49 \\
0.49 \\
0.525 \\
0.485 \\
0.505 \\
0.5 \\
0.57 \\
0.505 \\
};

\addplot[mark=*, boxplot, boxplot/draw position=4]
table[row sep=\\, y index=0] {
data
0.555 \\
0.595 \\
0.515 \\
0.54 \\
0.505 \\
0.51 \\
0.505 \\
0.54 \\
0.54 \\
0.505 \\
0.495 \\
0.455 \\
0.535 \\
0.54 \\
0.59 \\
0.48 \\
0.5 \\
0.465 \\
0.455 \\
0.485 \\
0.47 \\
0.48 \\
0.48 \\
0.51 \\
0.475 \\
0.51 \\
0.465 \\
0.5 \\
0.48 \\
0.61 \\
0.495 \\
0.53 \\
0.525 \\
0.485 \\
0.545 \\
0.485 \\
0.425 \\
0.505 \\
0.535 \\
0.555 \\
0.545 \\
0.51 \\
0.51 \\
0.515 \\
0.515 \\
0.55 \\
0.43 \\
0.595 \\
0.51 \\
0.485 \\
};

\addplot[mark=*, boxplot, boxplot/draw position=5]
table[row sep=\\, y index=0] {
data
0.625 \\
0.495 \\
0.52 \\
0.475 \\
0.68 \\
0.505 \\
0.535 \\
0.51 \\
0.515 \\
0.51 \\
0.605 \\
0.405 \\
0.47 \\
0.56 \\
0.57 \\
0.505 \\
0.57 \\
0.46 \\
0.535 \\
0.555 \\
0.505 \\
0.485 \\
0.565 \\
0.56 \\
0.47 \\
0.57 \\
0.525 \\
0.545 \\
0.54 \\
0.535 \\
0.625 \\
0.51 \\
0.555 \\
0.54 \\
0.535 \\
0.5 \\
0.515 \\
0.485 \\
0.545 \\
0.535 \\
0.545 \\
0.505 \\
0.735 \\
0.615 \\
0.67 \\
0.51 \\
0.62 \\
0.46 \\
0.55 \\
0.535 \\
};

\addplot[mark=*, boxplot, boxplot/draw position=6]
table[row sep=\\, y index=0] {
data
0.615 \\
0.575 \\
0.505 \\
0.505 \\
0.535 \\
0.635 \\
0.48 \\
0.48 \\
0.665 \\
0.54 \\
0.525 \\
0.5 \\
0.59 \\
0.53 \\
0.565 \\
0.53 \\
0.6 \\
0.56 \\
0.51 \\
0.54 \\
0.535 \\
0.565 \\
0.515 \\
0.48 \\
0.59 \\
0.54 \\
0.475 \\
0.735 \\
0.62 \\
0.47 \\
0.545 \\
0.525 \\
0.53 \\
0.46 \\
0.585 \\
0.57 \\
0.55 \\
0.475 \\
0.555 \\
0.565 \\
0.485 \\
0.53 \\
0.595 \\
0.475 \\
0.565 \\
0.505 \\
0.58 \\
0.685 \\
0.605 \\
0.545 \\
};

\addplot[mark=*, boxplot, boxplot/draw position=7]
table[row sep=\\, y index=0] {
data
0.715 \\
0.57 \\
0.59 \\
0.57 \\
0.825 \\
0.575 \\
0.64 \\
0.575 \\
0.495 \\
0.59 \\
0.48 \\
0.55 \\
0.715 \\
0.575 \\
0.635 \\
0.54 \\
0.54 \\
0.57 \\
0.61 \\
0.575 \\
0.58 \\
0.535 \\
0.62 \\
0.705 \\
0.585 \\
0.63 \\
0.7 \\
0.735 \\
0.51 \\
0.555 \\
0.505 \\
0.58 \\
0.57 \\
0.555 \\
0.575 \\
0.625 \\
0.67 \\
0.65 \\
0.59 \\
0.575 \\
0.57 \\
0.75 \\
0.53 \\
0.63 \\
0.64 \\
0.53 \\
0.585 \\
0.635 \\
0.65 \\
0.67 \\
};

\addplot[mark=*, boxplot, boxplot/draw position=8]
table[row sep=\\, y index=0] {
data
0.585 \\
0.62 \\
0.575 \\
0.74 \\
0.72 \\
0.72 \\
0.675 \\
0.57 \\
0.595 \\
0.555 \\
0.54 \\
0.605 \\
0.735 \\
0.615 \\
0.675 \\
0.555 \\
0.615 \\
0.805 \\
0.69 \\
0.635 \\
0.69 \\
0.715 \\
0.555 \\
0.595 \\
0.605 \\
0.485 \\
0.615 \\
0.715 \\
0.63 \\
0.69 \\
0.64 \\
0.685 \\
0.67 \\
0.655 \\
0.65 \\
0.58 \\
0.57 \\
0.645 \\
0.55 \\
0.915 \\
0.775 \\
0.635 \\
0.75 \\
0.485 \\
0.755 \\
0.68 \\
0.64 \\
0.595 \\
0.655 \\
0.61 \\
};

\addplot[mark=*, boxplot, boxplot/draw position=9]
table[row sep=\\, y index=0] {
data
0.595 \\
0.715 \\
0.665 \\
0.67 \\
0.635 \\
0.57 \\
0.77 \\
0.545 \\
0.57 \\
0.6 \\
0.75 \\
0.845 \\
0.63 \\
0.64 \\
0.605 \\
0.655 \\
0.675 \\
0.7 \\
0.7 \\
0.785 \\
0.61 \\
0.525 \\
0.855 \\
0.675 \\
0.775 \\
0.635 \\
0.775 \\
0.695 \\
0.66 \\
0.73 \\
0.705 \\
0.7 \\
0.605 \\
0.655 \\
0.495 \\
0.745 \\
0.53 \\
0.655 \\
0.66 \\
0.575 \\
0.775 \\
0.625 \\
0.6 \\
0.675 \\
0.68 \\
0.67 \\
0.62 \\
0.685 \\
0.565 \\
0.835 \\
};

\addplot[mark=*, boxplot, boxplot/draw position=10]
table[row sep=\\, y index=0] {
data
0.68 \\
0.85 \\
0.7 \\
0.67 \\
0.715 \\
0.68 \\
0.805 \\
0.775 \\
0.785 \\
0.71 \\
0.745 \\
0.65 \\
0.84 \\
0.915 \\
0.635 \\
0.805 \\
0.67 \\
0.515 \\
0.65 \\
0.795 \\
0.51 \\
0.69 \\
0.795 \\
0.605 \\
0.785 \\
0.69 \\
0.945 \\
0.82 \\
0.845 \\
0.785 \\
0.91 \\
0.62 \\
0.78 \\
0.73 \\
0.63 \\
0.825 \\
0.645 \\
0.705 \\
0.655 \\
0.665 \\
0.57 \\
0.76 \\
0.82 \\
0.64 \\
0.775 \\
0.79 \\
0.72 \\
0.7 \\
0.625 \\
0.63 \\
};

\addplot[mark=*, boxplot, boxplot/draw position=11]
table[row sep=\\, y index=0] {
data
0.61 \\
0.81 \\
0.725 \\
0.815 \\
0.705 \\
0.67 \\
0.625 \\
0.795 \\
0.79 \\
0.65 \\
0.83 \\
0.67 \\
0.73 \\
0.835 \\
0.855 \\
0.835 \\
0.64 \\
0.71 \\
0.755 \\
0.775 \\
0.77 \\
0.665 \\
0.855 \\
0.675 \\
0.665 \\
0.815 \\
0.68 \\
0.875 \\
0.98 \\
0.74 \\
0.9 \\
0.55 \\
0.745 \\
0.85 \\
0.495 \\
0.8 \\
0.635 \\
0.81 \\
0.67 \\
0.69 \\
0.835 \\
0.89 \\
0.86 \\
0.98 \\
0.825 \\
0.6 \\
0.76 \\
0.725 \\
0.855 \\
0.735 \\
};

\addplot[mark=*, boxplot, boxplot/draw position=12]
table[row sep=\\, y index=0] {
data
0.915 \\
0.865 \\
0.88 \\
0.595 \\
0.935 \\
0.685 \\
0.835 \\
0.9 \\
0.795 \\
0.735 \\
0.635 \\
0.94 \\
0.875 \\
0.73 \\
0.82 \\
0.85 \\
0.77 \\
0.835 \\
0.9 \\
0.81 \\
0.67 \\
0.65 \\
0.835 \\
0.795 \\
0.805 \\
0.69 \\
0.84 \\
0.86 \\
0.865 \\
0.855 \\
0.705 \\
0.785 \\
0.69 \\
0.805 \\
0.745 \\
0.855 \\
0.85 \\
0.865 \\
0.715 \\
0.76 \\
0.87 \\
0.605 \\
0.815 \\
0.845 \\
0.755 \\
0.835 \\
0.88 \\
0.945 \\
0.735 \\
0.76 \\
};

\addplot[mark=*, boxplot, boxplot/draw position=13]
table[row sep=\\, y index=0] {
data
0.795 \\
0.85 \\
0.715 \\
0.9 \\
0.805 \\
0.82 \\
0.865 \\
0.835 \\
0.75 \\
0.815 \\
0.815 \\
0.95 \\
0.785 \\
0.78 \\
0.73 \\
0.905 \\
0.845 \\
0.605 \\
0.58 \\
0.79 \\
0.895 \\
0.87 \\
0.885 \\
0.795 \\
0.86 \\
0.79 \\
0.665 \\
0.91 \\
0.805 \\
0.825 \\
0.815 \\
0.955 \\
0.665 \\
0.885 \\
0.86 \\
0.87 \\
0.945 \\
0.72 \\
0.76 \\
0.76 \\
0.635 \\
0.74 \\
0.8 \\
0.82 \\
0.785 \\
0.885 \\
0.825 \\
0.895 \\
0.965 \\
0.72 \\
};

\addplot[mark=*, boxplot, boxplot/draw position=14]
table[row sep=\\, y index=0] {
data
0.98 \\
0.92 \\
0.91 \\
0.645 \\
0.895 \\
0.74 \\
0.835 \\
0.815 \\
0.83 \\
0.91 \\
0.76 \\
0.79 \\
0.635 \\
0.885 \\
0.695 \\
0.795 \\
0.685 \\
0.85 \\
0.89 \\
0.73 \\
0.86 \\
0.79 \\
0.965 \\
0.625 \\
0.755 \\
0.79 \\
0.76 \\
0.99 \\
0.56 \\
0.82 \\
0.6 \\
0.87 \\
0.745 \\
0.915 \\
0.66 \\
0.93 \\
0.615 \\
0.895 \\
0.81 \\
0.73 \\
0.88 \\
0.805 \\
0.955 \\
0.94 \\
0.835 \\
0.775 \\
0.84 \\
0.73 \\
0.93 \\
0.775 \\
};

\addplot[mark=*, boxplot, boxplot/draw position=15]
table[row sep=\\, y index=0] {
data
0.635 \\
0.65 \\
0.845 \\
0.88 \\
0.785 \\
0.935 \\
0.875 \\
0.89 \\
0.77 \\
0.66 \\
0.77 \\
0.775 \\
0.89 \\
0.91 \\
0.685 \\
0.805 \\
0.98 \\
0.96 \\
0.83 \\
0.855 \\
0.815 \\
0.965 \\
0.745 \\
0.885 \\
0.965 \\
0.825 \\
0.895 \\
0.94 \\
0.89 \\
0.82 \\
0.735 \\
0.685 \\
0.835 \\
0.78 \\
0.685 \\
0.82 \\
0.865 \\
0.805 \\
0.845 \\
0.865 \\
0.875 \\
0.805 \\
0.825 \\
0.905 \\
0.64 \\
0.825 \\
0.85 \\
0.935 \\
0.685 \\
0.94 \\
};

\addplot[mark=*, boxplot, boxplot/draw position=16]
table[row sep=\\, y index=0] {
data
0.68 \\
0.47 \\
0.845 \\
0.835 \\
0.805 \\
0.895 \\
0.825 \\
0.745 \\
0.515 \\
0.625 \\
0.935 \\
0.89 \\
0.72 \\
0.885 \\
0.665 \\
0.785 \\
0.715 \\
0.835 \\
0.875 \\
0.76 \\
0.585 \\
0.835 \\
0.99 \\
0.88 \\
0.51 \\
0.875 \\
0.815 \\
0.94 \\
0.935 \\
0.755 \\
0.855 \\
0.775 \\
0.565 \\
0.845 \\
0.865 \\
0.99 \\
0.87 \\
0.805 \\
0.615 \\
0.715 \\
0.45 \\
0.68 \\
0.69 \\
0.68 \\
0.835 \\
0.75 \\
0.755 \\
0.715 \\
0.76 \\
0.705 \\
};

\addplot[mark=*, boxplot, boxplot/draw position=17]
table[row sep=\\, y index=0] {
data
0.805 \\
0.75 \\
0.8 \\
0.62 \\
0.865 \\
0.685 \\
0.925 \\
0.565 \\
0.665 \\
0.84 \\
0.92 \\
0.65 \\
0.81 \\
0.735 \\
0.735 \\
0.6 \\
0.7 \\
0.96 \\
0.84 \\
0.805 \\
0.75 \\
0.875 \\
0.745 \\
0.815 \\
0.555 \\
0.69 \\
0.93 \\
0.695 \\
0.955 \\
0.85 \\
0.515 \\
0.555 \\
0.82 \\
0.76 \\
0.7 \\
0.895 \\
0.87 \\
0.7 \\
0.845 \\
0.88 \\
0.68 \\
0.87 \\
0.815 \\
0.92 \\
0.71 \\
0.85 \\
0.78 \\
0.81 \\
0.89 \\
0.77 \\
};

\addplot[mark=*, boxplot, boxplot/draw position=18]
table[row sep=\\, y index=0] {
data
0.735 \\
0.55 \\
0.635 \\
0.635 \\
0.865 \\
0.69 \\
0.88 \\
0.76 \\
0.55 \\
0.65 \\
0.455 \\
0.77 \\
0.685 \\
0.52 \\
0.49 \\
0.73 \\
0.57 \\
0.85 \\
0.83 \\
0.455 \\
0.74 \\
0.845 \\
0.775 \\
0.675 \\
0.64 \\
0.645 \\
0.8 \\
0.485 \\
0.575 \\
0.69 \\
0.58 \\
0.84 \\
0.79 \\
0.65 \\
0.665 \\
0.705 \\
0.725 \\
0.6 \\
0.635 \\
0.905 \\
0.635 \\
0.92 \\
0.625 \\
0.815 \\
0.705 \\
0.82 \\
0.775 \\
0.76 \\
0.88 \\
0.455 \\
};

\addplot[mark=*, boxplot, boxplot/draw position=19]
table[row sep=\\, y index=0] {
data
0.84 \\
0.465 \\
0.56 \\
0.79 \\
0.71 \\
0.815 \\
0.665 \\
0.74 \\
0.655 \\
0.725 \\
0.965 \\
0.43 \\
0.435 \\
0.685 \\
0.845 \\
0.58 \\
0.74 \\
0.67 \\
0.66 \\
0.75 \\
0.54 \\
0.56 \\
0.92 \\
0.415 \\
0.61 \\
0.725 \\
0.47 \\
0.485 \\
0.545 \\
0.78 \\
0.775 \\
0.625 \\
0.71 \\
0.51 \\
0.755 \\
0.835 \\
0.525 \\
0.465 \\
0.665 \\
0.835 \\
0.77 \\
0.715 \\
0.72 \\
0.65 \\
0.59 \\
0.85 \\
0.615 \\
0.755 \\
0.785 \\
0.67 \\
};

\addplot[mark=*, boxplot, boxplot/draw position=20]
table[row sep=\\, y index=0] {
data
0.75 \\
0.585 \\
0.49 \\
0.71 \\
0.645 \\
0.74 \\
0.71 \\
0.585 \\
0.61 \\
0.49 \\
0.675 \\
0.625 \\
0.59 \\
0.735 \\
0.655 \\
0.75 \\
0.465 \\
0.46 \\
0.56 \\
0.475 \\
0.395 \\
0.515 \\
0.485 \\
0.575 \\
0.89 \\
0.785 \\
0.43 \\
0.745 \\
0.53 \\
0.6 \\
0.7 \\
0.47 \\
0.435 \\
0.43 \\
0.67 \\
0.705 \\
0.705 \\
0.885 \\
0.535 \\
0.52 \\
0.54 \\
0.72 \\
0.55 \\
0.625 \\
0.575 \\
0.695 \\
0.46 \\
0.47 \\
0.535 \\
0.73 \\
};

\addplot[mark=*, boxplot, boxplot/draw position=21]
table[row sep=\\, y index=0] {
data
0.6 \\
0.49 \\
0.475 \\
0.58 \\
0.75 \\
0.43 \\
0.705 \\
0.445 \\
0.51 \\
0.465 \\
0.46 \\
0.675 \\
0.455 \\
0.5 \\
0.42 \\
0.605 \\
0.735 \\
0.77 \\
0.435 \\
0.44 \\
0.58 \\
0.49 \\
0.7 \\
0.535 \\
0.45 \\
0.735 \\
0.69 \\
0.7 \\
0.445 \\
0.47 \\
0.7 \\
0.535 \\
0.68 \\
0.57 \\
0.43 \\
0.52 \\
0.435 \\
0.59 \\
0.425 \\
0.43 \\
0.625 \\
0.51 \\
0.675 \\
0.455 \\
0.46 \\
0.595 \\
0.685 \\
0.55 \\
0.53 \\
0.505 \\
};

\addplot[mark=*, boxplot, boxplot/draw position=22]
table[row sep=\\, y index=0] {
data
0.43 \\
0.47 \\
0.645 \\
0.455 \\
0.565 \\
0.47 \\
0.545 \\
0.57 \\
0.605 \\
0.525 \\
0.425 \\
0.505 \\
0.595 \\
0.58 \\
0.59 \\
0.43 \\
0.485 \\
0.61 \\
0.56 \\
0.57 \\
0.425 \\
0.865 \\
0.565 \\
0.4 \\
0.59 \\
0.43 \\
0.475 \\
0.55 \\
0.43 \\
0.525 \\
0.565 \\
0.455 \\
0.47 \\
0.55 \\
0.51 \\
0.375 \\
0.44 \\
0.39 \\
0.515 \\
0.56 \\
0.465 \\
0.655 \\
0.43 \\
0.46 \\
0.585 \\
0.455 \\
0.45 \\
0.465 \\
0.43 \\
0.43 \\
};

\addplot[mark=*, boxplot, boxplot/draw position=23]
table[row sep=\\, y index=0] {
data
0.45 \\
0.66 \\
0.455 \\
0.465 \\
0.46 \\
0.47 \\
0.47 \\
0.435 \\
0.43 \\
0.43 \\
0.61 \\
0.43 \\
0.43 \\
0.465 \\
0.395 \\
0.52 \\
0.435 \\
0.47 \\
0.525 \\
0.43 \\
0.45 \\
0.455 \\
0.405 \\
0.43 \\
0.47 \\
0.585 \\
0.455 \\
0.4 \\
0.59 \\
0.41 \\
0.63 \\
0.76 \\
0.435 \\
0.5 \\
0.5 \\
0.465 \\
0.43 \\
0.43 \\
0.43 \\
0.43 \\
0.47 \\
0.425 \\
0.47 \\
0.43 \\
0.495 \\
0.525 \\
0.42 \\
0.43 \\
0.465 \\
0.465 \\
};

\addplot[mark=*, boxplot, boxplot/draw position=24]
table[row sep=\\, y index=0] {
data
0.43 \\
0.47 \\
0.455 \\
0.47 \\
0.43 \\
0.47 \\
0.43 \\
0.455 \\
0.435 \\
0.47 \\
0.43 \\
0.43 \\
0.43 \\
0.65 \\
0.495 \\
0.455 \\
0.47 \\
0.47 \\
0.46 \\
0.59 \\
0.43 \\
0.43 \\
0.49 \\
0.47 \\
0.655 \\
0.43 \\
0.43 \\
0.445 \\
0.485 \\
0.43 \\
0.43 \\
0.43 \\
0.48 \\
0.5 \\
0.54 \\
0.475 \\
0.475 \\
0.585 \\
0.47 \\
0.545 \\
0.445 \\
0.43 \\
0.475 \\
0.47 \\
0.555 \\
0.47 \\
0.43 \\
0.43 \\
0.43 \\
0.455 \\
};

\addplot[mark=*, boxplot, boxplot/draw position=25]
table[row sep=\\, y index=0] {
data
0.43 \\
0.47 \\
0.43 \\
0.47 \\
0.47 \\
0.43 \\
0.47 \\
0.43 \\
0.47 \\
0.475 \\
0.435 \\
0.43 \\
0.47 \\
0.47 \\
0.47 \\
0.475 \\
0.43 \\
0.43 \\
0.47 \\
0.43 \\
0.47 \\
0.515 \\
0.505 \\
0.43 \\
0.43 \\
0.43 \\
0.47 \\
0.47 \\
0.49 \\
0.43 \\
0.47 \\
0.43 \\
0.43 \\
0.43 \\
0.425 \\
0.43 \\
0.47 \\
0.43 \\
0.47 \\
0.47 \\
0.43 \\
0.475 \\
0.47 \\
0.47 \\
0.47 \\
0.43 \\
0.425 \\
0.47 \\
0.43 \\
0.47 \\
};

\addplot[mark=*, boxplot, boxplot/draw position=26]
table[row sep=\\, y index=0] {
data
0.47 \\
0.47 \\
0.47 \\
0.555 \\
0.47 \\
0.43 \\
0.43 \\
0.47 \\
0.47 \\
0.43 \\
0.43 \\
0.47 \\
0.47 \\
0.46 \\
0.47 \\
0.47 \\
0.47 \\
0.47 \\
0.43 \\
0.47 \\
0.43 \\
0.43 \\
0.505 \\
0.47 \\
0.47 \\
0.615 \\
0.475 \\
0.47 \\
0.47 \\
0.47 \\
0.47 \\
0.47 \\
0.47 \\
0.47 \\
0.47 \\
0.47 \\
0.59 \\
0.43 \\
0.47 \\
0.49 \\
0.47 \\
0.47 \\
0.47 \\
0.47 \\
0.43 \\
0.45 \\
0.43 \\
0.51 \\
0.47 \\
0.47 \\
};

\addplot[mark=*, boxplot, boxplot/draw position=27]
table[row sep=\\, y index=0] {
data
0.47 \\
0.47 \\
0.47 \\
0.47 \\
0.47 \\
0.43 \\
0.47 \\
0.47 \\
0.47 \\
0.47 \\
0.47 \\
0.47 \\
0.47 \\
0.47 \\
0.47 \\
0.47 \\
0.47 \\
0.47 \\
0.47 \\
0.47 \\
0.47 \\
0.47 \\
0.49 \\
0.47 \\
0.47 \\
0.47 \\
0.475 \\
0.47 \\
0.47 \\
0.47 \\
0.47 \\
0.47 \\
0.49 \\
0.47 \\
0.47 \\
0.47 \\
0.47 \\
0.47 \\
0.47 \\
0.47 \\
0.47 \\
0.47 \\
0.47 \\
0.47 \\
0.47 \\
0.47 \\
0.47 \\
0.47 \\
0.47 \\
0.47 \\
};

\addplot[mark=*, boxplot, boxplot/draw position=28]
table[row sep=\\, y index=0] {
data
0.47 \\
0.47 \\
0.47 \\
0.47 \\
0.47 \\
0.47 \\
0.47 \\
0.47 \\
0.47 \\
0.47 \\
0.47 \\
0.47 \\
0.47 \\
0.47 \\
0.47 \\
0.47 \\
0.47 \\
0.47 \\
0.47 \\
0.47 \\
0.47 \\
0.47 \\
0.47 \\
0.47 \\
0.47 \\
0.47 \\
0.47 \\
0.47 \\
0.47 \\
0.47 \\
0.47 \\
0.47 \\
0.47 \\
0.47 \\
0.47 \\
0.47 \\
0.47 \\
0.47 \\
0.47 \\
0.47 \\
0.47 \\
0.47 \\
0.47 \\
0.47 \\
0.47 \\
0.47 \\
0.47 \\
0.47 \\
0.47 \\
0.47 \\
};
}{0.2}{Input connectivity}{29}

        }
    }
    \resizebox{\textwidth}{!}{
        \subfloat[TD3, N=30]{
            \myboxplot{

\addplot[mark=*, boxplot, boxplot/draw position=0]
table[row sep=\\, y index=0] {
data
0.56 \\
0.535 \\
0.505 \\
0.575 \\
0.57 \\
0.57 \\
0.565 \\
0.595 \\
0.515 \\
0.55 \\
0.545 \\
0.575 \\
0.595 \\
0.585 \\
0.465 \\
0.53 \\
0.565 \\
0.595 \\
0.515 \\
0.585 \\
0.57 \\
0.595 \\
0.545 \\
0.595 \\
0.575 \\
0.54 \\
0.61 \\
0.53 \\
0.595 \\
0.595 \\
0.51 \\
0.515 \\
0.595 \\
0.575 \\
0.595 \\
0.555 \\
0.485 \\
0.55 \\
0.59 \\
0.55 \\
0.595 \\
0.5 \\
0.515 \\
0.535 \\
0.49 \\
0.535 \\
0.57 \\
0.54 \\
0.595 \\
0.515 \\
};

\addplot[mark=*, boxplot, boxplot/draw position=1]
table[row sep=\\, y index=0] {
data
0.785 \\
0.785 \\
0.96 \\
0.88 \\
0.99 \\
0.845 \\
0.92 \\
0.895 \\
0.76 \\
0.765 \\
0.785 \\
0.84 \\
0.975 \\
0.94 \\
0.855 \\
0.86 \\
0.895 \\
0.87 \\
0.86 \\
0.79 \\
0.82 \\
0.92 \\
0.965 \\
0.94 \\
0.915 \\
0.89 \\
0.955 \\
0.87 \\
0.935 \\
0.935 \\
0.825 \\
0.845 \\
0.97 \\
0.87 \\
0.855 \\
0.94 \\
0.91 \\
0.995 \\
0.82 \\
0.855 \\
0.675 \\
0.93 \\
0.94 \\
0.97 \\
0.87 \\
0.84 \\
0.92 \\
0.885 \\
0.865 \\
0.94 \\
};

\addplot[mark=*, boxplot, boxplot/draw position=2]
table[row sep=\\, y index=0] {
data
0.735 \\
0.905 \\
1.0 \\
0.9 \\
0.965 \\
0.945 \\
1.0 \\
0.925 \\
0.865 \\
0.85 \\
1.0 \\
1.0 \\
0.975 \\
1.0 \\
0.94 \\
1.0 \\
0.875 \\
0.98 \\
0.785 \\
0.905 \\
0.985 \\
0.945 \\
0.94 \\
0.985 \\
0.89 \\
0.97 \\
0.845 \\
0.965 \\
0.915 \\
0.965 \\
1.0 \\
0.93 \\
0.925 \\
0.87 \\
0.805 \\
1.0 \\
0.99 \\
0.985 \\
0.88 \\
0.87 \\
0.95 \\
1.0 \\
0.97 \\
0.93 \\
0.87 \\
0.845 \\
0.94 \\
0.915 \\
0.945 \\
1.0 \\
};

\addplot[mark=*, boxplot, boxplot/draw position=3]
table[row sep=\\, y index=0] {
data
0.97 \\
0.935 \\
1.0 \\
0.96 \\
0.955 \\
0.975 \\
0.955 \\
0.94 \\
0.98 \\
0.96 \\
0.85 \\
0.915 \\
0.99 \\
1.0 \\
0.97 \\
0.97 \\
0.945 \\
1.0 \\
1.0 \\
1.0 \\
1.0 \\
0.935 \\
1.0 \\
0.915 \\
0.97 \\
1.0 \\
0.9 \\
0.97 \\
0.92 \\
0.91 \\
0.97 \\
1.0 \\
0.935 \\
1.0 \\
1.0 \\
1.0 \\
0.855 \\
0.95 \\
0.925 \\
0.985 \\
0.89 \\
1.0 \\
0.975 \\
0.855 \\
0.985 \\
0.98 \\
0.995 \\
0.82 \\
0.8 \\
0.97 \\
};

\addplot[mark=*, boxplot, boxplot/draw position=4]
table[row sep=\\, y index=0] {
data
1.0 \\
1.0 \\
1.0 \\
1.0 \\
0.875 \\
0.97 \\
0.8 \\
0.94 \\
0.98 \\
0.925 \\
1.0 \\
0.9 \\
0.97 \\
1.0 \\
0.87 \\
0.94 \\
1.0 \\
0.925 \\
1.0 \\
1.0 \\
0.855 \\
1.0 \\
1.0 \\
1.0 \\
0.94 \\
0.915 \\
0.985 \\
0.97 \\
1.0 \\
0.9 \\
0.9 \\
1.0 \\
0.955 \\
0.93 \\
1.0 \\
0.845 \\
0.995 \\
0.905 \\
0.9 \\
0.865 \\
0.955 \\
0.9 \\
0.99 \\
1.0 \\
0.88 \\
0.85 \\
0.975 \\
0.94 \\
0.945 \\
0.935 \\
};

\addplot[mark=*, boxplot, boxplot/draw position=5]
table[row sep=\\, y index=0] {
data
1.0 \\
0.8 \\
1.0 \\
0.8 \\
0.8 \\
0.9 \\
0.8 \\
0.8 \\
0.92 \\
1.0 \\
0.9 \\
0.8 \\
0.8 \\
0.745 \\
0.87 \\
0.8 \\
1.0 \\
0.745 \\
0.855 \\
0.9 \\
0.98 \\
0.8 \\
1.0 \\
0.855 \\
1.0 \\
0.745 \\
1.0 \\
0.8 \\
0.9 \\
0.9 \\
0.745 \\
1.0 \\
0.8 \\
0.9 \\
0.94 \\
0.8 \\
0.8 \\
0.8 \\
0.8 \\
0.9 \\
0.8 \\
0.97 \\
0.845 \\
0.8 \\
0.9 \\
0.8 \\
0.8 \\
1.0 \\
0.9 \\
0.87 \\
};

\addplot[mark=*, boxplot, boxplot/draw position=6]
table[row sep=\\, y index=0] {
data
0.745 \\
0.745 \\
0.745 \\
0.745 \\
0.745 \\
0.745 \\
0.745 \\
0.745 \\
0.745 \\
0.745 \\
0.745 \\
0.745 \\
0.745 \\
0.745 \\
0.745 \\
0.745 \\
0.745 \\
0.745 \\
0.745 \\
0.745 \\
0.745 \\
0.745 \\
0.745 \\
0.745 \\
0.745 \\
0.745 \\
0.745 \\
0.745 \\
0.745 \\
0.745 \\
0.745 \\
0.745 \\
0.745 \\
0.745 \\
0.745 \\
0.745 \\
0.745 \\
0.745 \\
0.745 \\
0.745 \\
0.745 \\
0.745 \\
0.745 \\
0.745 \\
0.745 \\
0.745 \\
0.745 \\
0.745 \\
0.745 \\
0.745 \\
};
}{0.2}{}{7}

        }
        \subfloat[TD5, N=65]{
            \myboxplot{

\addplot[mark=*, boxplot, boxplot/draw position=0]
table[row sep=\\, y index=0] {
data
0.46 \\
0.54 \\
0.455 \\
0.51 \\
0.515 \\
0.58 \\
0.49 \\
0.505 \\
0.5 \\
0.45 \\
0.415 \\
0.51 \\
0.535 \\
0.49 \\
0.495 \\
0.535 \\
0.405 \\
0.48 \\
0.545 \\
0.525 \\
0.53 \\
0.48 \\
0.55 \\
0.505 \\
0.505 \\
0.52 \\
0.5 \\
0.47 \\
0.595 \\
0.505 \\
0.565 \\
0.535 \\
0.445 \\
0.515 \\
0.535 \\
0.58 \\
0.58 \\
0.525 \\
0.56 \\
0.525 \\
0.535 \\
0.515 \\
0.45 \\
0.42 \\
0.605 \\
0.545 \\
0.535 \\
0.52 \\
0.465 \\
0.445 \\
};

\addplot[mark=*, boxplot, boxplot/draw position=1]
table[row sep=\\, y index=0] {
data
0.805 \\
0.685 \\
0.8 \\
0.68 \\
0.87 \\
0.685 \\
0.795 \\
0.685 \\
0.845 \\
0.66 \\
0.835 \\
0.75 \\
0.835 \\
0.605 \\
0.79 \\
0.765 \\
0.755 \\
0.725 \\
0.83 \\
0.59 \\
0.765 \\
0.82 \\
0.765 \\
0.685 \\
0.745 \\
0.73 \\
0.78 \\
0.765 \\
0.735 \\
0.725 \\
0.735 \\
0.675 \\
0.79 \\
0.77 \\
0.755 \\
0.775 \\
0.76 \\
0.74 \\
0.68 \\
0.605 \\
0.805 \\
0.705 \\
0.76 \\
0.745 \\
0.685 \\
0.88 \\
0.785 \\
0.72 \\
0.735 \\
0.79 \\
};

\addplot[mark=*, boxplot, boxplot/draw position=2]
table[row sep=\\, y index=0] {
data
0.925 \\
0.805 \\
0.8 \\
0.77 \\
0.725 \\
0.72 \\
0.795 \\
0.705 \\
0.805 \\
0.755 \\
0.805 \\
0.89 \\
0.835 \\
0.755 \\
0.86 \\
0.825 \\
0.86 \\
0.725 \\
0.865 \\
0.745 \\
0.75 \\
0.77 \\
0.725 \\
0.735 \\
0.79 \\
0.775 \\
0.775 \\
0.845 \\
0.845 \\
0.845 \\
0.835 \\
0.835 \\
0.895 \\
0.765 \\
0.815 \\
0.745 \\
0.785 \\
0.785 \\
0.72 \\
0.795 \\
0.89 \\
0.79 \\
0.79 \\
0.86 \\
0.83 \\
0.83 \\
0.695 \\
0.875 \\
0.81 \\
0.87 \\
};

\addplot[mark=*, boxplot, boxplot/draw position=3]
table[row sep=\\, y index=0] {
data
0.77 \\
0.885 \\
0.88 \\
0.795 \\
0.9 \\
0.755 \\
0.85 \\
0.82 \\
0.87 \\
0.9 \\
0.795 \\
0.79 \\
0.835 \\
0.915 \\
0.77 \\
0.825 \\
0.845 \\
0.84 \\
0.81 \\
0.85 \\
0.9 \\
0.79 \\
0.875 \\
0.85 \\
0.9 \\
0.885 \\
0.78 \\
0.76 \\
0.81 \\
0.875 \\
0.81 \\
0.865 \\
0.905 \\
0.905 \\
0.88 \\
0.865 \\
0.79 \\
0.91 \\
0.91 \\
0.805 \\
0.84 \\
0.85 \\
0.895 \\
0.895 \\
0.835 \\
0.9 \\
0.865 \\
0.88 \\
0.775 \\
0.77 \\
};

\addplot[mark=*, boxplot, boxplot/draw position=4]
table[row sep=\\, y index=0] {
data
0.835 \\
0.855 \\
0.82 \\
0.795 \\
0.845 \\
0.93 \\
0.845 \\
0.9 \\
0.815 \\
0.88 \\
0.935 \\
0.9 \\
0.84 \\
0.835 \\
0.87 \\
0.81 \\
0.82 \\
0.92 \\
0.81 \\
0.82 \\
0.875 \\
0.82 \\
0.895 \\
0.815 \\
0.85 \\
0.91 \\
0.8 \\
0.895 \\
0.87 \\
0.86 \\
0.86 \\
0.875 \\
0.875 \\
0.92 \\
0.88 \\
0.855 \\
0.83 \\
0.8 \\
0.765 \\
0.835 \\
0.85 \\
0.845 \\
0.845 \\
0.855 \\
0.815 \\
0.805 \\
0.79 \\
0.92 \\
0.9 \\
0.96 \\
};

\addplot[mark=*, boxplot, boxplot/draw position=5]
table[row sep=\\, y index=0] {
data
0.99 \\
0.875 \\
0.96 \\
0.85 \\
0.86 \\
0.865 \\
0.885 \\
0.815 \\
0.845 \\
0.915 \\
0.9 \\
0.83 \\
0.915 \\
0.86 \\
0.945 \\
0.945 \\
0.935 \\
0.96 \\
0.815 \\
0.86 \\
0.85 \\
0.87 \\
0.905 \\
0.895 \\
0.915 \\
0.87 \\
0.89 \\
0.93 \\
0.965 \\
0.875 \\
0.85 \\
0.915 \\
0.9 \\
0.92 \\
0.94 \\
0.87 \\
0.93 \\
0.795 \\
0.93 \\
0.88 \\
0.94 \\
0.92 \\
0.855 \\
0.92 \\
0.96 \\
0.8 \\
0.88 \\
0.86 \\
0.91 \\
0.925 \\
};

\addplot[mark=*, boxplot, boxplot/draw position=6]
table[row sep=\\, y index=0] {
data
0.925 \\
0.965 \\
0.87 \\
0.895 \\
0.92 \\
0.9 \\
0.92 \\
0.9 \\
0.9 \\
0.91 \\
0.915 \\
0.82 \\
0.92 \\
0.895 \\
0.91 \\
0.875 \\
0.885 \\
0.89 \\
0.855 \\
0.91 \\
0.855 \\
0.875 \\
0.915 \\
0.895 \\
0.955 \\
0.885 \\
0.91 \\
0.92 \\
0.89 \\
0.84 \\
0.815 \\
0.88 \\
0.81 \\
0.925 \\
0.96 \\
0.835 \\
0.91 \\
0.845 \\
0.885 \\
0.935 \\
0.865 \\
0.965 \\
0.925 \\
0.785 \\
0.91 \\
0.935 \\
0.84 \\
0.88 \\
0.945 \\
0.915 \\
};

\addplot[mark=*, boxplot, boxplot/draw position=7]
table[row sep=\\, y index=0] {
data
0.875 \\
0.895 \\
0.855 \\
0.935 \\
0.965 \\
0.86 \\
0.91 \\
0.965 \\
0.89 \\
0.85 \\
0.835 \\
0.875 \\
0.86 \\
0.825 \\
0.875 \\
0.915 \\
0.835 \\
0.865 \\
0.965 \\
0.945 \\
0.835 \\
0.82 \\
0.9 \\
0.925 \\
0.945 \\
0.86 \\
0.84 \\
0.885 \\
0.935 \\
0.83 \\
0.87 \\
0.825 \\
0.86 \\
0.915 \\
0.885 \\
0.92 \\
0.83 \\
0.93 \\
0.74 \\
0.865 \\
0.84 \\
0.885 \\
0.9 \\
0.805 \\
0.885 \\
0.895 \\
0.895 \\
0.845 \\
0.865 \\
0.935 \\
};

\addplot[mark=*, boxplot, boxplot/draw position=8]
table[row sep=\\, y index=0] {
data
0.82 \\
0.845 \\
0.88 \\
0.905 \\
0.9 \\
0.915 \\
0.765 \\
0.805 \\
0.855 \\
0.88 \\
0.885 \\
0.815 \\
0.835 \\
0.82 \\
0.82 \\
0.835 \\
0.81 \\
0.84 \\
0.76 \\
0.82 \\
0.925 \\
0.885 \\
0.845 \\
0.82 \\
0.845 \\
0.895 \\
0.89 \\
0.81 \\
0.89 \\
0.89 \\
0.82 \\
0.85 \\
0.81 \\
0.805 \\
0.865 \\
0.87 \\
0.885 \\
0.775 \\
0.82 \\
0.915 \\
0.945 \\
0.88 \\
0.82 \\
0.7 \\
0.825 \\
0.875 \\
0.765 \\
0.905 \\
0.92 \\
0.87 \\
};

\addplot[mark=*, boxplot, boxplot/draw position=9]
table[row sep=\\, y index=0] {
data
0.88 \\
0.82 \\
0.82 \\
0.83 \\
0.775 \\
0.76 \\
0.825 \\
0.77 \\
0.815 \\
0.82 \\
0.835 \\
0.82 \\
0.765 \\
0.845 \\
0.795 \\
0.82 \\
0.925 \\
0.82 \\
0.835 \\
0.82 \\
0.83 \\
0.805 \\
0.82 \\
0.82 \\
0.805 \\
0.82 \\
0.775 \\
0.91 \\
0.795 \\
0.835 \\
0.84 \\
0.795 \\
0.82 \\
0.765 \\
0.885 \\
0.88 \\
0.81 \\
0.81 \\
0.815 \\
0.795 \\
0.92 \\
0.7 \\
0.765 \\
0.875 \\
0.835 \\
0.9 \\
0.7 \\
0.815 \\
0.805 \\
0.795 \\
};

\addplot[mark=*, boxplot, boxplot/draw position=10]
table[row sep=\\, y index=0] {
data
0.7 \\
0.825 \\
0.71 \\
0.865 \\
0.82 \\
0.765 \\
0.82 \\
0.81 \\
0.755 \\
0.685 \\
0.755 \\
0.82 \\
0.82 \\
0.83 \\
0.82 \\
0.82 \\
0.88 \\
0.8 \\
0.765 \\
0.805 \\
0.805 \\
0.82 \\
0.845 \\
0.82 \\
0.845 \\
0.73 \\
0.82 \\
0.795 \\
0.82 \\
0.81 \\
0.765 \\
0.755 \\
0.82 \\
0.775 \\
0.82 \\
0.83 \\
0.82 \\
0.83 \\
0.79 \\
0.805 \\
0.82 \\
0.795 \\
0.765 \\
0.765 \\
0.805 \\
0.765 \\
0.755 \\
0.83 \\
0.865 \\
0.765 \\
};

\addplot[mark=*, boxplot, boxplot/draw position=11]
table[row sep=\\, y index=0] {
data
0.73 \\
0.845 \\
0.765 \\
0.8 \\
0.7 \\
0.755 \\
0.82 \\
0.7 \\
0.805 \\
0.775 \\
0.815 \\
0.765 \\
0.795 \\
0.755 \\
0.765 \\
0.835 \\
0.845 \\
0.68 \\
0.755 \\
0.75 \\
0.765 \\
0.765 \\
0.7 \\
0.82 \\
0.755 \\
0.755 \\
0.7 \\
0.72 \\
0.7 \\
0.82 \\
0.82 \\
0.7 \\
0.82 \\
0.755 \\
0.765 \\
0.82 \\
0.7 \\
0.82 \\
0.765 \\
0.7 \\
0.755 \\
0.765 \\
0.82 \\
0.82 \\
0.68 \\
0.82 \\
0.82 \\
0.82 \\
0.815 \\
0.7 \\
};

\addplot[mark=*, boxplot, boxplot/draw position=12]
table[row sep=\\, y index=0] {
data
0.7 \\
0.7 \\
0.755 \\
0.7 \\
0.7 \\
0.68 \\
0.785 \\
0.7 \\
0.7 \\
0.7 \\
0.7 \\
0.765 \\
0.755 \\
0.68 \\
0.68 \\
0.75 \\
0.7 \\
0.7 \\
0.82 \\
0.765 \\
0.7 \\
0.7 \\
0.82 \\
0.755 \\
0.7 \\
0.7 \\
0.7 \\
0.68 \\
0.7 \\
0.82 \\
0.7 \\
0.7 \\
0.7 \\
0.765 \\
0.7 \\
0.765 \\
0.7 \\
0.7 \\
0.7 \\
0.7 \\
0.82 \\
0.68 \\
0.765 \\
0.7 \\
0.7 \\
0.7 \\
0.7 \\
0.7 \\
0.7 \\
0.7 \\
};

\addplot[mark=*, boxplot, boxplot/draw position=13]
table[row sep=\\, y index=0] {
data
0.68 \\
0.68 \\
0.68 \\
0.68 \\
0.68 \\
0.68 \\
0.68 \\
0.68 \\
0.68 \\
0.68 \\
0.68 \\
0.68 \\
0.68 \\
0.68 \\
0.68 \\
0.68 \\
0.68 \\
0.68 \\
0.68 \\
0.68 \\
0.68 \\
0.68 \\
0.68 \\
0.68 \\
0.68 \\
0.68 \\
0.68 \\
0.68 \\
0.68 \\
0.68 \\
0.68 \\
0.68 \\
0.68 \\
0.68 \\
0.68 \\
0.68 \\
0.68 \\
0.68 \\
0.68 \\
0.68 \\
0.68 \\
0.68 \\
0.68 \\
0.68 \\
0.68 \\
0.68 \\
0.68 \\
0.68 \\
0.68 \\
0.68 \\
};
}{0.2}{}{14}

        }
    }
\end{figure}

Results are created from the data behind figures \ref{fig:TP3-IC-1} through \ref{fig:TP5-IC-2},

\begin{table}[ht]
    \centering
    \caption{The smallest reservoir sizes where at least two reservoirs have the required accuracy}
    \label{tab:accuracy-thresholds}
    \begin{tabular}{lllll}
                     & TP3 & TP5 & TD3 & TD5 \\
    90\% accuracy threshold & 15  & 70  & 10  & 30  \\
    98\% accuracy threshold & 20  & 90 & 10  & 55
    \end{tabular}
\end{table}


We calculate the optimal input connectivity for each task as follows:
$$ optimal\_ic^{task} = average(optimal\_ic_{n\_nodes}^{task} / n\_nodes) $$
where $ optimal\_ic_{n\_nodes}^{task} $ is the connectivity which gives the highest number of high-accuracy reservoirs for that task and reservoir size.
Results are created from the data behind figures \ref{fig:TP3-IC-1} through \ref{fig:TP5-IC-2},
and tabulated in \ref{tab:optimal-ic}.

\begin{table}[h]
	\centering
	\caption{Optimal input connectivities as fraction of reservoir size for task-window size combinations }
	\label{tab:optimal-ic}
	\begin{tabular}{lllll}
						 & $T=3$  & $T=5$ \\
        Temporal Parity  & 0.528          & 0.489         \\
        Temporal Density & 0.439          & 0.443
	\end{tabular}
\end{table}

\subsection{Discussion}

As seen in figures \ref{fig:results:tp3-1}–\ref{fig:results:tp3-2},
a reservoir size of XX is sufficient to solve the Temporal Parity 3 task.
In figures \ref{fig:results:tp5-1}–\ref{fig:results:tp5-4},
a reservoir size of XX is sufficient to solve the Temporal Parity 5 task.

Notice for all plots, that the highest population accuracies are found when the input connectivity is roughly half of the reservoir size, or $IC=n\_nodes/2$.
As this holds for both tasks (Temporal Parity 3 and 5), there seems to be no correlation between task difficulty and required input connectivity.
In \cm{thatpaperwiththeoreticalcomputationsforinputconnectivity}, the reservoir connectivity is the main factor in how large a perturbance of the is required.
In my previous paper, \cm{prevpaper}, a larger connectivity resulted in a higher required connectivity.
These findings might therefore not be generalizable to reservoirs with other connectivities.
