\section{Minimum required reservoir size, optimal input connectivity}

In the chapter summarizing my previous paper \ref{background:my-previous-paper},
we saw that there was a plethora of RRC systems with $N=100, K=\{2, 3\}$ that were able to solve TP3 (Temporal Parity with a window size of 3).

With TP5, $N=100, K=3$ was barely enough for a few of the generated reservoirs to reach an accuracy of over 95\%.
In both cases the reservoirs with $K=3$ performed better than their $K=\{1, 2\}$ brethren,
with $K=3$ being requried for TP5.
A homogenous connectivity of $K=3$ will therefore be chosen as the only connectivity looked at in these experiments
It more closely approximates $\langle K \rangle = 2 $, where most critical reservoirs are found,
as well as reducing the size of the parameter space we have to search.

To find the minimum required reservoir size,
we'll be creating reservoirs with the parameters presented in table \ref{tab:ic-reservoir-parameters},
for all four task combinations presented in table \ref{tab:tasks},
namely TP3, TP5, TD3 and TD5 (TP = Temporal Parity, TD = Temporal Density, number equals window size).
Presumably the TP3 task can be solved with a much smaller reservoir,
while TP5 might require a slightly larger reservoir.
As the TD task is computationally less expensive,
we can expect a smaller required reservoir size.

\begin{table}[ht]
    \centering
    \caption{Task parameters. The tasks are explained in detail in chapter \ref{section:tasks}}
    \label{tab:tasks}
    \begin{tabular}{ll}
        Task type               & Temporal Parity and Temporal Density \\
        Training dataset length & 4 000                       \\
        Test dataset length     & 200                         \\
        $N$ (window size)       & 3 and 5                     \\
        $t$ (offset)            & 0
    \end{tabular}
\end{table}

\begin{table}[ht]
    \centering
    \caption{Reservoir parameters for optimal input connectivity}
    \label{tab:ic-reservoir-parameters}
    \begin{tabular}{ll}
        Nodes               & 10 to (100, 140) for TP, 5 to (35, 65) for TD \\
        Node step size      & 10 for TP, 5 for TD \\
        Connectivity        & 3                              \\
        Input connectivity  & [0..n\_nodes], step size = 5   \\
		Output connectivity & n\_nodes                       \\
        Sample size         & 50
    \end{tabular}
\end{table}

For each reservoir size, we'll also iterate over the input connectivities,
to find which input connectivity gives the greatest population accuracy.
In my previous paper \ref{background:my-previous-paper},
the optimal input connectivity seemed to lie at roughly $ 0.5*n\_nodes $,
with the $K=3$ reservoirs having a slight skew to the right.
The Temporal Density task might have different input connectivity requirements to the Temporal Parity task.

We calculate the optimal input connectivity for each task as follows:
\begin{equation} \label{eq:optimal-ic}
optimal\_ic^{task} = average(max\_accuracy\_ic_{n\_nodes}^{task} / n\_nodes)
\end{equation}
where $ max\_accuracy\_ic_{n\_nodes}^{task} $ is the connectivity which gives the highest number of high-accuracy reservoirs for that task and reservoir size.

The resulting accuracies wil be plotted as boxplots, as previously done in \ref{background:my-previous-paper}.
This should allow for visual identification of where the optimal reservoir size lies for each task,
as well as the optimal input connectivity.

\subsection{Results}

The number of reservoir combinations resulting from tables \ref{tab:ic-reservoir-parameters} and \ref{tab:tasks} is quite large.
Therefore only reservoir sizes of $ N=[10...30, 80...100]$ from the accuracy distributions on the Temporal Parity 3 task is presented here (figure \ref{fig:TP3-IC}).
In addition, plots for the reservoirs on the 98\% accuracy threshold for all four tasks is shown in figure \label{fig:accuracy-threshold-size}.
The entire set of plots is available in appendix \ref{app:reservoir_size-input_connectivity}.

Task accuracy thresholds,
defined as the smallest reservoir size where at least two reservoirs have the requried accuracy,
is presented in table \ref{tab:accuracy-thresholds}.

\begin{table}[ht]
    \centering
    \caption{Accuracy thresholds for all four tasks.}
    \label{tab:accuracy-thresholds}
    \begin{tabular}{lllll}
                     & TP3 & TP5 & TD3 & TD5 \\
    90\% accuracy threshold & 15  & 70  & 10  & 30  \\
    98\% accuracy threshold & 20  & 90 & 10  & 55
    \end{tabular}
\end{table}

Task optimal input connectivity (as defined in formula \ref{eq:optimal-ic}) is presented in table \ref{tab:optimal-ic}.

\begin{table}[h]
	\centering
	\caption{Optimal input connectivities as fraction of reservoir size.}
	\label{tab:optimal-ic}
	\begin{tabular}{lllll}
						 & $T=3$  & $T=5$ \\
        Temporal Parity  & 0.528          & 0.489         \\
        Temporal Density & 0.439          & 0.443
	\end{tabular}
\end{table}

\begin{figure}[ht]
    \centering
    \caption{
        Accuracy plots for the required reservoir sizes to reach the 98\% accuracy threshold for each of the four tasks:
        TP3 (Figure \ref{fig:threshold-TP3}), TP5 (Figure \ref{fig:threshold-TP5}), TD3 (Figure \ref{fig:threshold-TD3}) and TD5 (Figure \ref{fig:threshold-TD5}).
        The x-axis for all plots has been set to the largest required reservoir size ($N=90$), to facilitate simple comparison.
    }
    \label{fig:accuracy-threshold-size}
    \resizebox{\textwidth}{!}{
        \subfloat[TP3, N=20]{
            \input{experiments/results/normalized-threshold-plots/boxplot-input_connectivity-N20-K3-S50.tex}
            \label{fig:threshold-TP3}
        }
        \subfloat[TP5, N=90]{
            \input{experiments/results/normalized-threshold-plots/boxplot-input_connectivity-N90-K3-S50.tex}
            \label{fig:threshold-TP5}
        }
    }
    \resizebox{\textwidth}{!}{
        \subfloat[TD3, N=10]{
            \input{experiments/results/normalized-threshold-plots/boxplot-input_connectivity-N10-K3-S50.tex}
            \label{fig:threshold-TD3}
        }
        \subfloat[TD5, N=55]{
            \input{experiments/results/normalized-threshold-plots/boxplot-input_connectivity-N55-K3-S50.tex}
            \label{fig:threshold-TD5}
        }
    }
\end{figure}

\begin{figure*}[ht]
    \centering
    \caption{
        Plots of input connectivity against accuracy on TP3. Reservoir sizes $4[0...70]$ are omitted for brevity.
        Note that the optimal input connectivity tends slightly to the right of the middle for all reservoir sizes.
        The omitted plots are presented in figures \ref{fig:TP3-IC-1} and \ref{fig:TP3-IC-2} in appendix \ref{app:reservoir_size-input_connectivity}.
        }
    \label{fig:TP3-IC}
    \resizebox{\textwidth}{!}{
        \subfloat[N=10]{
            \input{experiments/results/TP3-IO/boxplot-input_connectivity-N10-K3-S50.tex}
        }
        \subfloat[N=80]{
            \input{experiments/results/TP3-IO/boxplot-input_connectivity-N80-K3-S50.tex}
        }
    }
    \resizebox{\textwidth}{!}{
        \subfloat[N=20]{
            \input{experiments/results/TP3-IO/boxplot-input_connectivity-N20-K3-S50.tex}
        }
        \subfloat[N=90]{
            \input{experiments/results/TP3-IO/boxplot-input_connectivity-N90-K3-S50.tex}
        }
    }
    \resizebox{\textwidth}{!}{
        \subfloat[N=30]{
            \input{experiments/results/TP3-IO/boxplot-input_connectivity-N30-K3-S50.tex}
        }
        \subfloat[N=100]{
            \input{experiments/results/TP3-IO/boxplot-input_connectivity-N100-K3-S50.tex}
        }
    }
\end{figure*}

\subsection{Discussion}

As seen in figures \ref{fig:results:tp3-1}–\ref{fig:results:tp3-2},
a reservoir size of XX is sufficient to solve the Temporal Parity 3 task.
In figures \ref{fig:results:tp5-1}–\ref{fig:results:tp5-4},
a reservoir size of XX is sufficient to solve the Temporal Parity 5 task.

Notice for all plots, that the highest population accuracies are found when the input connectivity is roughly half of the reservoir size, or $IC=n\_nodes/2$.
As this holds for both tasks (Temporal Parity 3 and 5), there seems to be no correlation between task difficulty and required input connectivity.
In \cm{thatpaperwiththeoreticalcomputationsforinputconnectivity}, the reservoir connectivity is the main factor in how large a perturbance of the is required.
In my previous paper, \cm{prevpaper}, a larger connectivity resulted in a higher required connectivity.
These findings might therefore not be generalizable to reservoirs with other connectivities.
