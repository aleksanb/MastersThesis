\section{Analyzing reservoir dynamics}
\label{section:reservoir-dynamics}

\subsection{Description}

When analyzing Random Boolean Networks one often does so in aggregate,
looking at the population averages of the attributes of RBNs of a certain construction.
This as there can be large variance in individual RBN dynamics.
One often looks at whether the average network is chaotic, stable, or critical,
and the number of attractors, their size, their basins, and state-space transient times.
One can obtain expected values of the averages of these values if given a networks connectivity and size.

In my pre-thesis project (section \ref{section:pre-thesis-project}),
computational capability (section \ref{section:computational-capability}) is used as an indicator of reservoir performance.
Could there be a deeper link between the computational capability of an RBN used in a RRCs related to the RBNs dynamical properties?
What correlation is there between these dynamical properties (number of attractors and their lenght)
and how effective the RBN is as the basis of an RRC system?
If a relationship is found, does a change in task complexity constrain what values are required in aggregate?
Such findings can aid in the construction of RRCs by guiding the evolution of generative genomes used to create the backing RBN.

The number of attractors and their lengths will be calculated by help of the SAT-solver described in section \ref{section:computing-attractors},
which allows reservoirs with a maximum reservoir size of 26 to be analyzed within reasonable time limits.
The task Temporal Density will be used to investigate the attractor-accuracy relationship,
as $ n\_nodes = 26 $ is on the verge of solving TD5 with 90\% accuracy \ref{tab:accuracy-thresholds}.
Temporal Parity would require too large a reservoir (at least 70).
Parameters for the experiment are shown in table \ref{tab:reservoir-dynamics-parameters}

\begin{table}[ht]
    \centering
    \caption{Parameters for analysis of reservoir dynamics}
    \label{tab:reservoir-dynamics-parameters}
    \begin{tabular}{ll}
        \hline
        \textbf{Parameter} & \textbf{Configuration} \\
        \hline
        \hline
        Task                & Temporal Density 3 and 5  \\
        Nodes               & 26                        \\
        Connectivity        & 3                         \\
        Input connectivity  & $ n\_nodes / 2 $          \\
        Output connectivity & $ n\_nodes $              \\
        Sample size         & 500 \\
        \hline
    \end{tabular}
\end{table}

\subsection{Results}

Figure \ref{fig:attractor-overview} plots the attractor/length distributions for the 500 RBNs generated on the TD3 and TD5 tasks,
as well as the attractor/lengths of the RBNs scoring more than 95\% and 85\% accuracy on their respective tasks.
It is difficult to find theoretical estimates for the means of attractor/lengths for connectivities other than 1 \cite{drossel2005number} and 2 \cite{samuelsson2003superpolynomial}.
Figure \ref{fig:attractor-results}, showing the combined 1000 RBNs, is therefore used to get an intuition of which attractor/length combinations are common for reservoirs with $ K=3, p=0.5, N=26 $.
Table \ref{tab:attractor-values} shows the means of the attractor lengths and number of attractors for each RBN population subset.

\begin{figure*}
    \centering
    \caption{
        Figures \ref{fig:attractor-overview-TD3} and \ref{fig:attractor-results-TD3} show the distribution of the 500 RBNs generated for TD3,
        and the 273 of 500 that had an $ accuracy >= 0.95\% $ on TD3, respectively.
        Figures \ref{fig:attractor-overview-TD5} and \ref{fig:attractor-results-TD5} show the distribution of the 500 RBNs generated for TD5,
        and the 116 of 500 that had an $ accuracy >= 0.85\% $ on TD5, respectively.
    }
    \label{fig:attractor-overview}
    \resizebox{\textwidth}{!}{
        \subfloat[TD3 distribution]{
            \label{fig:attractor-overview-TD3}
            \myheatmap
                {experiments/results/sat-3/all-rbns-heatmap.dat}
                {Number of attractors}
                {Mean attractor length}
                {6}
        }
        \subfloat[TD3, $accuracy >= 95\%$]{
            \label{fig:attractor-results-TD3}
            \myheatmap
                {experiments/results/sat-3/heatmap.dat}
                {Number of attractors}
                {Mean attractor length}
                {6}
        }
    }
    \resizebox{\textwidth}{!}{
        \subfloat[TD5]{
            \label{fig:attractor-overview-TD5}
            \myheatmap
                {experiments/results/sat-5/all-rbns-heatmap.dat}
                {Number of attractors}
                {Mean attractor length}
                {6}
        }
        \subfloat[TD5, $accuracy >= 85\%$]{
            \label{fig:attractor-results-TD5}
            \myheatmap
                {experiments/results/sat-5/heatmap.dat}
                {Number of attractors}
                {Mean attractor length}
                {6}
        }
    }
\end{figure*}

\begin{figure*}
    \centering
    \caption{
        Combined distribution of mean attractor lengths and number of attractors for for all 1000 generated RBNs
        (500 from TD3 \ref{fig:attractor-overview-TD3} and 500 from TD5 \ref{fig:attractor-overview-TD5}).
    }
    \label{fig:attractor-results}
    \resizebox{0.5\textwidth}{!}{
        \myheatmap
            {experiments/results/sat-5/combined-3-5-data.dat}
            {Number of attractors}
            {Mean attractor length}
            {10}
    }
\end{figure*}

\begin{table}[]
    \centering
    \caption{Means and medians of the different RBN subsets' attractor lengths and number of attractors}
    \label{tab:attractor-values}
    \begin{tabular}{llllll}
        \hline
        \hline
        & \textbf{Minimum} & \multicolumn{2}{l}{\textbf{Attractor length}} & \multicolumn{2}{l}{\textbf{Number of attractors}} \\
        & \textbf{Accuracy} & Mean & Median & Mean & Median \\
        \hline
        TD3 & 95\% & 13.90 & 8.67 & 5.97 & 5.00 \\
            & ALL  & 13.45 & 8.50 & 5.98 & 5.00 \\

        TD5 & 85\% & 11.79 & 7.90 & 6.20 & 5.00 \\
            & ALL  & 12.44 & 7.67 & 6.09 & 5.00 \\

        TD3+TD5 & ALL & 12.95 & 8.12 & 6.04 & 5.00 \\
        \hline
    \end{tabular}
\end{table}

\subsection{Discussion}

Comparing plots \ref{fig:attractor-overview-TD3} to \ref{fig:attractor-results-TD3},
and \ref{fig:attractor-overview-TD5} to \ref{fig:attractor-results-TD5},
one notices a trend.
The distribution of accurate reservoirs seems to largely mimic the overall reservoir distribution.
The assumption is confirmed by comparing the average values for each group of RBNs in table \ref{tab:attractor-values}.
There is a minisculine difference in the means and medians of TD3 95\% versus all TD3,
equally so for TD5 85\% versus all TD5.

From this one could conclude that there's no correlation between the number of attractors, their length, and reservoir task performance.
There are simply more RBNs with a mean number of attractors of 5 and mean length of ~8 than other values,
resulting in a higher number of accurate reservoirs located around the same values.
In \cite{rbn-reservoir} the authors note that for Reservoir Computing,
computation cannot depend on the attractors of the system,
due to the continious perturbation.
The system can still enter attractors however,
at which point computation would cease to be productive.
For the RRC systems benched in this paper,
the large amount of perturbation (input connectivities of 50\%) would require attractors to have humongous basins to successfully prevent computation.
Small attractor basins and large transient times could still be indicators of reservoir performance,
as they would allow for more paths through statespace, lessening the chance that a perturbation ends the system in an attractor.

As criticality is an indicator of reservoir performance,
looking into the relationship between criticality and network topology might be interesting.
Finally, one can still use other metrics to evaluating RRC performance before the fact,
such as computational capability (section \ref{section:computational-capability}).
