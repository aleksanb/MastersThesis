\section{Analyzing reservoir dynamics}

\subsection{Description}

When analyzing Random Boolean Networks one often does so in aggregate,
looking at the population averages of the attributes of RBNs of a certain construction.
This as there can be large variance in individual RBN dynamics.
One often looks at whether the average network is chaotic, stable, or critical,
and the number of attractors, their size, their basins, and state-space transient times.
One can obtain expected values of the averages of these values if given a networks connectivity and size.

In my pre-thesis project (section \ref{section:pre-thesis-project}),
computational capability (section \ref{section:computational-capability}) is used as an indicator of reservoir performance.
Could there be a deeper link between the computational capability of an RBN used in a RRCs related to the RBNs dynamical properties?
What correlation is there between these dynamical properties (number of attractors and their lenght)
and how effective the RBN is as the basis of an RRC system?
If a relationship is found, does a change in task complexity constrain what values are required in aggregate?
Such findings can aid in the construction of RRCs by guiding the evolution of generative genomes used to create the backing RBN.

There are a number of ways to calculate a RBNs attractors and their properties.
They generally fall somewhere on the spectrum between exact enumeration (exhaustively searching for attractors from all initial states) and numerical sampling (searching from a subset only).
Numerical sampling is biased for $K\neq2$ \cite{berdahl2009random}, and cannot be used by us.
A brute-force exhaustive search quickly becomes infeasible, as the number of RBN states is exponential in the number of nodes.
The authors home-cooked exhaustive searcher times out for RBNs with more than 15 nodes.
A SAT-Based algorithm for finding attractors in CRBNs (introduced in \cite{dubrova2011sat}) is therefore used,
and allows for an increase from 15 to 26 nodes.
The source code is available online at \cite{dubrova2011sat-online}.

Temporal Density will be used to investigate the attractor-accuracy relationship,
as $ n\_nodes = 26 $ is on the verge of solving TD5 with 90\% accuracy \ref{tab:accuracy-thresholds}.
Temporal Parity would require too large a reservoir (at least 70).
Parameters for the experiment are shown in table \ref{tab:reservoir-dynamics-parameters}

\begin{table}[ht]
    \centering
    \caption{Parameters for analysis of reservoir dynamics}
    \label{tab:reservoir-dynamics-parameters}
    \begin{tabular}{ll}
        \hline
        \textbf{Parameter} & \textbf{Configuration} \\
        \hline
        \hline
        Task                & Temporal Density 3 and 5  \\
        Nodes               & 26                        \\
        Connectivity        & 3                         \\
        Input connectivity  & $ n\_nodes / 2 $          \\
        Output connectivity & $ n\_nodes $              \\
        Sample size         & 500 \\
        \hline
    \end{tabular}
\end{table}

\subsection{Results}

Figure \ref{fig:attractor-overview} plots the attractor/length distributions for the 500 RBNs generated on the TD3 and TD5 tasks,
as well as the attractor/lengths of the RBNs scoring more than 95\% and 85\% accuracy on their respective tasks.
It is difficult to find theoretical estimates for the means of attractor/lengths for connectivities other than 1 \cite{drossel2005number} and 2 \cite{samuelsson2003superpolynomial}.
Figure \ref{fig:attractor-results}, showing the combined 1000 RBNs, is therefore used to get an intuition of which attractor/length combinations are common for reservoirs with $ K=3, p=0.5, N=26 $.
Table \ref{tab:attractor-values} shows the means of the attractor lengths and number of attractors for each RBN population subset.

\begin{figure*}
    \centering
    \caption{
        Figures \ref{fig:attractor-overview-TD3} and \ref{fig:attractor-results-TD3} show the distribution of the 500 RBNs generated for TD3,
        and the 273 of 500 that had an $ accuracy >= 0.95\% $ on TD3, respectively.
        Figures \ref{fig:attractor-overview-TD5} and \ref{fig:attractor-results-TD5} show the distribution of the 500 RBNs generated for TD5,
        and the 116 of 500 that had an $ accuracy >= 0.85\% $ on TD5, respectively.
    }
    \label{fig:attractor-overview}
    \resizebox{\textwidth}{!}{
        \subfloat[TD3 distribution]{
            \label{fig:attractor-overview-TD3}
            \myheatmap
                {experiments/results/sat-3/all-rbns-heatmap.dat}
                {Number of attractors}
                {Mean attractor length}
                {6}
        }
        \subfloat[TD3, $accuracy >= 95\%$]{
            \label{fig:attractor-results-TD3}
            \myheatmap
                {experiments/results/sat-3/heatmap.dat}
                {Number of attractors}
                {Mean attractor length}
                {6}
        }
    }
    \resizebox{\textwidth}{!}{
        \subfloat[TD5]{
            \label{fig:attractor-overview-TD5}
            \myheatmap
                {experiments/results/sat-5/all-rbns-heatmap.dat}
                {Number of attractors}
                {Mean attractor length}
                {6}
        }
        \subfloat[TD5, $accuracy >= 85\%$]{
            \label{fig:attractor-results-TD5}
            \myheatmap
                {experiments/results/sat-5/heatmap.dat}
                {Number of attractors}
                {Mean attractor length}
                {6}
        }
    }
\end{figure*}

\begin{figure*}
    \centering
    \caption{
        Combined distribution of mean attractor lengths and number of attractors for for all 1000 generated RBNs
        (500 from TD3 \ref{fig:attractor-overview-TD3} and 500 from TD5 \ref{fig:attractor-overview-TD5}).
    }
    \label{fig:attractor-results}
    \resizebox{0.5\textwidth}{!}{
        \myheatmap
            {experiments/results/sat-5/combined-3-5-data.dat}
            {Number of attractors}
            {Mean attractor length}
            {10}
    }
\end{figure*}

\begin{table}[]
    \centering
    \caption{Means and medians of the different RBN subsets' attractor lengths and number of attractors}
    \label{tab:attractor-values}
    \begin{tabular}{llllll}
        \hline
        \hline
        & \textbf{Minimum} & \multicolumn{2}{l}{\textbf{Attractor length}} & \multicolumn{2}{l}{\textbf{Number of attractors}} \\
        & \textbf{Accuracy} & Mean & Median & Mean & Median \\
        \hline
        TD3 & 95\% & 13.90 & 8.67 & 5.97 & 5.00 \\
            & ALL  & 13.45 & 8.50 & 5.98 & 5.00 \\

        TD5 & 85\% & 11.79 & 7.90 & 6.20 & 5.00 \\
            & ALL  & 12.44 & 7.67 & 6.09 & 5.00 \\

        TD3+TD5 & ALL & 12.95 & 8.12 & 6.04 & 5.00 \\
        \hline
    \end{tabular}
\end{table}

\subsection{Discussion}

Well, who'da thunk it's such a small difference? \ref{tab:attractor-values}.
Seems like just tighter bounds based on visual.
Table data confirms just tighter bounds aroundd the same values, no apperent connection between the different stuffz.
