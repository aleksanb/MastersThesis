\section{Analyzing reservoir dynamics}

\subsection{Description}

When analyzing Random Boolean Networks one often does so in aggregate,
looking at the population averages of the attributes of RBNs of a certain construction.
This as there can be large variance in individual RBN dynamics.
One often looks at whether the average network is chaotic, stable, or critical,
and the number of attractors, their size, their basins, and state-space transient times.
One can obtain expected values of the averages of these values if given a networks connectivity and size.

In my pre-thesis project (section \ref{section:pre-thesis-project}),
computational capability (section \ref{section:computational-capability}) is used as an indicator of reservoir performance.
Could there be a deeper link between the computational capability of an RBN used in a RRCs related to the RBNs dynamical properties?
What correlation is there between these dynamical properties (number of attractors, their lenght, transient times)
and how effective the RBN is as the basis of an RRC system?
If a relationship is found, does a change in task complexity constrain what values are required in aggregate?
Such findings can aid in the construction of RRCs by guiding the evolution of generative genomes used to create the backing RBN.

There are a number of ways to calculate a RBNs attractors and their properties.
They generally fall somewhere on the spectrum between exact enumeration (exhaustively searching for attractors from all initial states) and numerical sampling (searching from a subset only).
Numerical sampling is biased for $K\neq2$ \cite{berdahl2009random}, and cannot be used by us.
A brute-force exhaustive search quickly becomes infeasible, as the number of RBN states is exponential in the number of nodes.
The authors home-cooked exhaustive searcher times out for RBNs with more than 15 nodes.
A SAT-Based algorithm for finding attractors in CRBNs is therefore used instead (as introduced in \cite{dubrova2011sat}).
It allows an increase from 15 to 25 nodes, with the source being available online at \cite{dubrova2011sat-online}.

Temporal Density will be used to investigate the attractor-accuracy relationship,
as $ n\_nodes = 25 $ is on the verge of solving TD5 with 90\% accuracy,
Temporal Parity would require much larger reservoirs \ref{tab:accuracy-thresholds}.
Parameters for the experiment are shown in table \ref{tab:reservoir-dynamics-parameters}

\begin{table}[ht]
    \centering
    \caption{Parameters for analysis of reservoir dynamics}
    \label{tab:reservoir-dynamics-parameters}
    \begin{tabular}{ll}
        Task                & Temporal Density 3 and 5  \\
        Nodes               & 25                        \\
        Connectivity        & 3                         \\
        Input connectivity  & $ n\_nodes / 2 $          \\
        Output connectivity & $ n\_nodes $              \\
        Sample size         & 500
    \end{tabular}
\end{table}

%\begin{minipage}{\linewidth}
%\begin{lstlisting}[
%    caption=
%        Simplified version of exhaustive-search attractor length script.
%        Executed for all initial states of all generated RRCs.,
%    frame=single,
%    label=lst:attractors]
%current_state = initial_state
%visited_states = set()
%
%loop {
%    visited_states.add(current_state)
%    current_state = get_next_state()
%
%    if current_state in visited_states {
%        attractor_length =
%            len(visited_states)
%            - visited_states.indexOf(current_state)
%        return attractor_length
%    }
%}
%\end{lstlisting}
%\end{minipage}

\subsection{Results}

\begin{figure*}
    \centering
    \resizebox{\textwidth}{!}{
        \myheatmap
            {experiments/results/temporal-density/3/3-600.dat}
            {Number of attractors}
            {Mean attractor length}
        \myheatmap
            {experiments/results/temporal-density/5/5-600.dat}
            {Number of attractors}
            {Mean attractor length}
    }
    \caption{
        600 samples. The left one has 116 with at least 95\% accuracy,
        the right one has 129 over 81\% accuracy.
    }
\end{figure*}

\subsection{Discussion}

LOL PLZ NO
