\section{Analyzing reservoir dynamics}

Analysis of Random Boolean Networks often focuses on their dymanical properties.
These include whether the network is chaotic, stable or critical,
the number of attractors, attractor size, basin size, and transient times.
For a given set of RBN parameters, one can analytically obtain bounds on the statistical averages of these values.

The accuracies of reservoirs for a given set of parameters also follow some distribution.
To find out if there is a correlation between a reservoirs dynamical properties and its accuracy on a task,
the number of attractors, their length, and transient times will be calculated alongside its accuracy on the task.

To find the number of attractors in a Random Boolean Network, there are two choices:
To exhaustively search all trajectories from all initial states untill a cycle is found,
or exploring only a subset of initial states.
For larger networks, an exhaustive search may be computationally infeasible,
with the number of initial states exponential in the size of the RBN.

As shown in paper \cm{that paper},
the means of the number of attractors and their lengths when calculated from a subset of initial states is in fact representative of the actual means for reservoirs with $K=2$.
As our reservoirs have a connectivity of $K = 3$, exhaustive search will have to be conducted instead.
\cm{Fact-check this, there's also that algorithm thats much faster than exhaustive search in that paper you found that time}.
Emprical testing shows that this limits us to reservoirs of a maximum number of 16 nodes,
enough to solve the Temporal Parity task with a window size of 3, but not for 5.
In section \ref{results-of-required-size} we see that the minimum required reservoir size for that task is closer to SOMENUNMBER.
The computationally easier Temporal Density task \ref{task-description-section},
with window sizes of both 3 and 5,
will therefore be used instead.

\todo{Beskrive hovdan gjøre eksperimentene mye mer detaljert, med grafer}

\subsection{Results}

\begin{figure*}
    \centering
    \resizebox{\textwidth}{!}{
        \myheatmap
            {experiments/results/temporal-density/3/3-600.dat}
            {Number of attractors}
            {Mean attractor length}
        \myheatmap
            {experiments/results/temporal-density/5/5-600.dat}
            {Number of attractors}
            {Mean attractor length}
    }
    \caption{
        600 samples. The left one has 116 with at least 95\% accuracy,
        the right one has 129 over 81\% accuracy.
    }
\end{figure*}

\subsection{Discussion}

LOL PLZ NO
