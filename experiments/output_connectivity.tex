\section{Minimum required output connectivity}

How few connections can one have from the reservoir itself to the readout layer,
while staying accurate on the task at hand?
As the reservoir prediction is computed by regressing the states of the network nodes against the expected output bits,
reducing the number of nodes used in the regression can never give a higher accuracy than using all of the nodes.
If there is redundant information in the network, quite a few nodes could be cut off from the readout layer while maintaining the best accuracy for this specific reservoir.
This does require a sufficient amount of linear independence between the remaining nodes of the network.

As shown in the previous section, different problems require different reservoir sizes.
Past a certain reservoir size, all reservoirs have a sufficient accuracy on the given task.
One could postulate that for reservoirs larger than the 95\% accuracy breakpoint,
it suffices to read out a number of nodes equal to the breakpoint.
If this is the case, one can safely use growing reservoirs without having to change the output connections,
given that the reservoir was large enough to solve the given task in the first place.

As shown in section \ref{experiments:1:results}, there is a correlation between input connectivity and accuracy.
For our reservoirs with $K=3$, the optimal input connectivity was found to be $n\_nodes/2$.
Using this information, we can reduce our search space for minimum required output connectivity.
The parameters in table \ref{tab:oc-reservoir-parameters} will be used for simulation.

\begin{table}[ht]
    \centering
    \caption{Reservoir parameters for optimal output connectivity}
    \label{tab:oc-reservoir-parameters}
    \begin{tabular}{ll}
        Nodes               & [10..(100, 140)], step size=10 \\
        Connectivity        & 3                              \\
        Input connectivity  & $ n\_nodes / 2 $               \\
        Output connectivity & [0..n\_nodes], step size=10    \\
        Sample size         & 50
    \end{tabular}
\end{table}

\subsection{Results}

\begin{figure*}[ht]
    \centering
    \resizebox{\textwidth}{!}{
        \subfloat[N=10]{
            \myboxplot{

\addplot[mark=*, boxplot, boxplot/draw position=1]
table[row sep=\\, y index=0] {
data
0.81 \\
0.93 \\
0.7 \\
0.91 \\
0.635 \\
0.68 \\
0.74 \\
0.815 \\
0.505 \\
0.8 \\
0.685 \\
0.855 \\
0.525 \\
0.88 \\
0.505 \\
0.43 \\
0.635 \\
0.505 \\
0.41 \\
0.655 \\
0.91 \\
0.585 \\
0.725 \\
0.91 \\
0.7 \\
0.495 \\
0.59 \\
0.93 \\
0.53 \\
0.84 \\
0.7 \\
0.795 \\
0.67 \\
0.425 \\
0.525 \\
0.545 \\
0.805 \\
0.535 \\
0.79 \\
0.505 \\
0.745 \\
0.775 \\
0.455 \\
0.54 \\
0.525 \\
0.775 \\
0.94 \\
0.685 \\
0.7 \\
0.67 \\
};
}{0.1}{Reservoir subset size}{11}

        }
        \subfloat[N=20]{
            \myboxplot{

\addplot[mark=*, boxplot, boxplot/draw position=1]
table[row sep=\\, y index=0] {
data
0.645 \\
0.89 \\
0.735 \\
0.55 \\
0.765 \\
0.815 \\
0.865 \\
0.825 \\
0.675 \\
0.58 \\
0.565 \\
0.75 \\
0.625 \\
0.935 \\
0.82 \\
0.685 \\
0.725 \\
0.55 \\
0.775 \\
0.785 \\
0.285 \\
0.7 \\
0.52 \\
0.47 \\
0.685 \\
0.755 \\
0.635 \\
0.69 \\
0.605 \\
0.39 \\
0.84 \\
0.77 \\
0.505 \\
0.78 \\
0.51 \\
0.67 \\
0.715 \\
0.65 \\
0.47 \\
0.74 \\
0.66 \\
0.42 \\
0.705 \\
0.725 \\
0.575 \\
0.805 \\
0.7 \\
0.715 \\
0.905 \\
0.76 \\
};

\addplot[mark=*, boxplot, boxplot/draw position=2]
table[row sep=\\, y index=0] {
data
0.91 \\
0.995 \\
0.995 \\
0.585 \\
0.815 \\
0.925 \\
0.865 \\
0.9 \\
0.62 \\
0.995 \\
0.65 \\
0.83 \\
0.76 \\
0.505 \\
0.625 \\
0.93 \\
0.535 \\
0.9 \\
0.9 \\
0.68 \\
0.68 \\
0.725 \\
0.97 \\
0.49 \\
0.525 \\
0.8 \\
0.805 \\
0.93 \\
0.715 \\
0.995 \\
0.9 \\
0.775 \\
0.7 \\
0.51 \\
0.795 \\
0.75 \\
0.765 \\
0.945 \\
0.755 \\
0.905 \\
0.945 \\
0.935 \\
0.995 \\
0.93 \\
0.995 \\
0.57 \\
0.93 \\
0.995 \\
0.535 \\
0.52 \\
};
}{0.1}{Output connectivity}{11}

        }   
    }   
    \resizebox{\textwidth}{!}{
        \subfloat[N=30]{
            \myboxplot{

\addplot[mark=*, boxplot, boxplot/draw position=1]
table[row sep=\\, y index=0] {
data
0.81 \\
0.545 \\
0.64 \\
0.715 \\
0.855 \\
0.79 \\
0.77 \\
0.72 \\
0.955 \\
0.905 \\
0.585 \\
0.63 \\
0.78 \\
0.825 \\
0.46 \\
0.94 \\
0.65 \\
0.675 \\
0.805 \\
0.775 \\
0.495 \\
0.615 \\
0.525 \\
0.55 \\
0.9 \\
0.89 \\
0.775 \\
0.495 \\
0.76 \\
0.855 \\
0.69 \\
0.435 \\
0.675 \\
0.91 \\
0.74 \\
0.84 \\
0.635 \\
0.62 \\
0.84 \\
0.485 \\
0.44 \\
0.925 \\
0.885 \\
0.87 \\
0.775 \\
0.69 \\
0.585 \\
0.9 \\
0.66 \\
0.93 \\
};

\addplot[mark=*, boxplot, boxplot/draw position=2]
table[row sep=\\, y index=0] {
data
0.77 \\
0.51 \\
0.995 \\
0.805 \\
0.655 \\
0.72 \\
0.77 \\
0.87 \\
0.46 \\
0.91 \\
0.73 \\
0.67 \\
0.875 \\
0.86 \\
0.7 \\
0.995 \\
0.995 \\
0.885 \\
0.815 \\
0.785 \\
0.905 \\
0.995 \\
0.82 \\
0.72 \\
0.665 \\
0.995 \\
0.87 \\
0.96 \\
0.795 \\
0.3 \\
0.52 \\
0.83 \\
0.895 \\
0.82 \\
0.895 \\
0.995 \\
0.77 \\
0.995 \\
0.93 \\
0.995 \\
0.925 \\
0.995 \\
0.78 \\
0.955 \\
0.985 \\
0.96 \\
0.56 \\
0.88 \\
0.9 \\
0.745 \\
};

\addplot[mark=*, boxplot, boxplot/draw position=3]
table[row sep=\\, y index=0] {
data
0.965 \\
0.87 \\
0.69 \\
0.995 \\
0.945 \\
0.88 \\
0.995 \\
0.94 \\
0.9 \\
0.995 \\
0.875 \\
0.905 \\
0.94 \\
0.875 \\
0.945 \\
0.82 \\
0.915 \\
0.775 \\
0.615 \\
0.95 \\
0.745 \\
0.995 \\
0.91 \\
0.98 \\
0.995 \\
0.88 \\
0.85 \\
0.805 \\
0.89 \\
0.93 \\
0.875 \\
0.815 \\
0.745 \\
0.815 \\
0.995 \\
0.845 \\
0.915 \\
0.98 \\
0.995 \\
0.995 \\
0.84 \\
0.835 \\
0.815 \\
0.74 \\
0.74 \\
0.7 \\
0.525 \\
0.565 \\
0.78 \\
0.945 \\
};
}{0.1}{Output connectivity}{11}

        }   
        \subfloat[N=40]{
            \myboxplot{

\addplot[mark=*, boxplot, boxplot/draw position=1]
table[row sep=\\, y index=0] {
data
0.615 \\
0.68 \\
0.705 \\
0.9 \\
0.89 \\
0.505 \\
0.935 \\
0.435 \\
0.775 \\
0.7 \\
0.42 \\
0.42 \\
0.61 \\
0.815 \\
0.45 \\
0.605 \\
0.745 \\
0.675 \\
0.55 \\
0.635 \\
0.475 \\
0.625 \\
0.585 \\
0.665 \\
0.7 \\
0.675 \\
0.565 \\
0.75 \\
0.815 \\
0.625 \\
0.82 \\
0.485 \\
0.815 \\
0.7 \\
0.88 \\
0.83 \\
0.63 \\
0.61 \\
0.775 \\
0.5 \\
0.585 \\
0.9 \\
0.625 \\
0.75 \\
0.305 \\
0.62 \\
0.68 \\
0.555 \\
0.815 \\
0.58 \\
};

\addplot[mark=*, boxplot, boxplot/draw position=2]
table[row sep=\\, y index=0] {
data
0.915 \\
0.88 \\
0.965 \\
0.59 \\
0.965 \\
0.87 \\
0.68 \\
0.66 \\
0.735 \\
0.91 \\
0.995 \\
0.735 \\
0.795 \\
0.96 \\
0.49 \\
0.72 \\
0.495 \\
0.48 \\
0.8 \\
0.995 \\
0.81 \\
0.635 \\
0.85 \\
0.945 \\
0.745 \\
0.895 \\
0.98 \\
0.865 \\
0.93 \\
0.8 \\
0.8 \\
0.75 \\
0.965 \\
0.88 \\
0.61 \\
0.855 \\
0.995 \\
0.995 \\
0.755 \\
0.845 \\
0.865 \\
0.705 \\
0.82 \\
0.715 \\
0.92 \\
0.64 \\
0.665 \\
0.72 \\
0.72 \\
0.815 \\
};

\addplot[mark=*, boxplot, boxplot/draw position=3]
table[row sep=\\, y index=0] {
data
0.7 \\
0.785 \\
0.995 \\
0.92 \\
0.925 \\
0.965 \\
0.965 \\
0.995 \\
0.875 \\
0.905 \\
0.6 \\
0.91 \\
0.995 \\
0.94 \\
0.935 \\
0.93 \\
0.755 \\
0.83 \\
0.815 \\
0.995 \\
0.85 \\
0.68 \\
0.68 \\
0.705 \\
0.96 \\
0.99 \\
0.955 \\
0.815 \\
0.995 \\
0.89 \\
0.995 \\
0.995 \\
0.995 \\
0.97 \\
0.995 \\
0.925 \\
0.67 \\
0.97 \\
0.785 \\
0.86 \\
0.905 \\
0.995 \\
0.7 \\
0.955 \\
0.93 \\
0.88 \\
0.835 \\
0.83 \\
0.695 \\
0.965 \\
};

\addplot[mark=*, boxplot, boxplot/draw position=4]
table[row sep=\\, y index=0] {
data
0.935 \\
0.955 \\
0.985 \\
0.77 \\
0.965 \\
0.93 \\
0.91 \\
0.965 \\
0.895 \\
0.81 \\
0.92 \\
0.955 \\
0.995 \\
0.995 \\
0.9 \\
0.945 \\
0.955 \\
0.975 \\
0.875 \\
0.995 \\
0.84 \\
0.895 \\
0.95 \\
0.785 \\
0.965 \\
0.835 \\
0.98 \\
0.97 \\
0.995 \\
0.8 \\
0.995 \\
0.93 \\
0.92 \\
0.65 \\
0.995 \\
0.98 \\
0.985 \\
0.955 \\
0.995 \\
0.81 \\
0.965 \\
0.995 \\
0.94 \\
0.995 \\
0.93 \\
0.985 \\
0.995 \\
0.995 \\
0.885 \\
0.89 \\
};
}{0.1}{Output connectivity}{11}

        }   
    }   
    \resizebox{\textwidth}{!}{
        \subfloat[N=50]{
            \myboxplot{

\addplot[mark=*, boxplot, boxplot/draw position=1]
table[row sep=\\, y index=0] {
data
0.655 \\
0.515 \\
0.72 \\
0.8 \\
0.74 \\
0.7 \\
0.725 \\
0.765 \\
0.72 \\
0.715 \\
0.685 \\
0.555 \\
0.62 \\
0.75 \\
0.56 \\
0.535 \\
0.66 \\
0.615 \\
0.805 \\
0.63 \\
0.93 \\
0.58 \\
0.815 \\
0.91 \\
0.445 \\
0.75 \\
0.665 \\
0.685 \\
0.79 \\
0.7 \\
0.835 \\
0.505 \\
0.67 \\
0.8 \\
0.56 \\
0.645 \\
0.75 \\
0.995 \\
0.525 \\
0.655 \\
0.745 \\
0.72 \\
0.65 \\
0.69 \\
0.525 \\
0.72 \\
0.765 \\
0.66 \\
0.81 \\
0.77 \\
};

\addplot[mark=*, boxplot, boxplot/draw position=2]
table[row sep=\\, y index=0] {
data
0.895 \\
0.84 \\
0.995 \\
0.85 \\
0.685 \\
0.89 \\
0.535 \\
0.695 \\
0.92 \\
0.955 \\
0.915 \\
0.885 \\
0.995 \\
0.955 \\
0.66 \\
0.835 \\
0.95 \\
0.8 \\
0.84 \\
0.285 \\
0.55 \\
0.755 \\
0.895 \\
0.8 \\
0.755 \\
0.6 \\
0.845 \\
0.995 \\
0.715 \\
0.64 \\
0.995 \\
0.59 \\
0.665 \\
0.855 \\
0.97 \\
0.745 \\
0.82 \\
0.815 \\
0.975 \\
0.64 \\
0.935 \\
0.64 \\
0.725 \\
0.995 \\
0.875 \\
0.995 \\
0.965 \\
0.875 \\
0.68 \\
0.89 \\
};

\addplot[mark=*, boxplot, boxplot/draw position=3]
table[row sep=\\, y index=0] {
data
0.73 \\
0.93 \\
0.79 \\
0.88 \\
0.86 \\
0.995 \\
0.735 \\
0.995 \\
0.935 \\
0.945 \\
0.995 \\
0.61 \\
0.995 \\
0.995 \\
0.81 \\
0.785 \\
0.87 \\
0.995 \\
0.6 \\
0.83 \\
0.995 \\
0.685 \\
0.865 \\
0.915 \\
0.93 \\
0.995 \\
0.735 \\
0.675 \\
0.665 \\
0.99 \\
0.875 \\
0.945 \\
0.74 \\
0.91 \\
0.795 \\
0.935 \\
0.995 \\
0.77 \\
0.995 \\
0.75 \\
0.92 \\
0.995 \\
0.755 \\
0.7 \\
0.995 \\
0.995 \\
0.995 \\
0.775 \\
0.685 \\
0.885 \\
};

\addplot[mark=*, boxplot, boxplot/draw position=4]
table[row sep=\\, y index=0] {
data
0.97 \\
0.925 \\
0.895 \\
0.895 \\
0.935 \\
0.995 \\
0.9 \\
0.51 \\
0.995 \\
0.995 \\
0.995 \\
0.88 \\
0.99 \\
0.94 \\
0.995 \\
0.88 \\
0.87 \\
0.995 \\
0.995 \\
0.955 \\
0.995 \\
0.995 \\
0.74 \\
0.74 \\
0.925 \\
0.86 \\
0.785 \\
0.905 \\
0.995 \\
0.99 \\
0.995 \\
0.905 \\
0.995 \\
0.995 \\
0.86 \\
0.92 \\
0.975 \\
0.965 \\
0.91 \\
0.92 \\
0.885 \\
0.905 \\
0.99 \\
0.98 \\
0.995 \\
0.98 \\
0.995 \\
0.995 \\
0.995 \\
0.965 \\
};

\addplot[mark=*, boxplot, boxplot/draw position=5]
table[row sep=\\, y index=0] {
data
0.995 \\
0.995 \\
0.995 \\
0.955 \\
0.945 \\
0.995 \\
0.79 \\
0.995 \\
0.975 \\
0.955 \\
0.98 \\
0.995 \\
0.995 \\
0.995 \\
0.91 \\
0.995 \\
0.905 \\
0.995 \\
0.985 \\
0.995 \\
0.985 \\
0.8 \\
0.995 \\
0.97 \\
0.99 \\
0.995 \\
0.86 \\
0.82 \\
0.995 \\
0.995 \\
0.995 \\
0.995 \\
0.97 \\
0.995 \\
0.985 \\
0.995 \\
0.94 \\
0.995 \\
0.925 \\
0.995 \\
0.845 \\
0.995 \\
0.995 \\
0.945 \\
0.885 \\
0.9 \\
0.98 \\
0.995 \\
0.955 \\
0.995 \\
};
}{0.1}{Reservoir subset size}{11}

        }   
        \subfloat[N=60]{
            \myboxplot{

\addplot[mark=*, boxplot, boxplot/draw position=1]
table[row sep=\\, y index=0] {
data
0.61 \\
0.755 \\
0.8 \\
0.585 \\
0.42 \\
0.525 \\
0.61 \\
0.79 \\
0.475 \\
0.625 \\
0.74 \\
0.695 \\
0.635 \\
0.78 \\
0.755 \\
0.85 \\
0.695 \\
0.685 \\
0.53 \\
0.62 \\
0.745 \\
0.62 \\
0.885 \\
0.925 \\
0.63 \\
0.76 \\
0.69 \\
0.715 \\
0.605 \\
0.915 \\
0.695 \\
0.865 \\
0.715 \\
0.9 \\
0.89 \\
0.395 \\
0.725 \\
0.93 \\
0.53 \\
0.71 \\
0.485 \\
0.71 \\
0.51 \\
0.595 \\
0.765 \\
0.67 \\
0.505 \\
0.91 \\
0.445 \\
0.68 \\
};

\addplot[mark=*, boxplot, boxplot/draw position=2]
table[row sep=\\, y index=0] {
data
0.77 \\
0.97 \\
0.97 \\
0.78 \\
0.75 \\
0.995 \\
0.8 \\
0.69 \\
0.775 \\
0.795 \\
0.83 \\
0.79 \\
0.915 \\
0.695 \\
0.69 \\
0.94 \\
0.74 \\
0.945 \\
0.8 \\
0.925 \\
0.675 \\
0.705 \\
0.96 \\
0.82 \\
0.65 \\
0.875 \\
0.765 \\
0.875 \\
0.805 \\
0.995 \\
0.975 \\
0.95 \\
0.68 \\
0.72 \\
0.995 \\
0.605 \\
0.835 \\
0.66 \\
0.845 \\
0.785 \\
0.855 \\
0.995 \\
0.82 \\
0.98 \\
0.925 \\
0.775 \\
0.945 \\
0.61 \\
0.84 \\
0.935 \\
};

\addplot[mark=*, boxplot, boxplot/draw position=3]
table[row sep=\\, y index=0] {
data
0.86 \\
0.975 \\
0.92 \\
0.74 \\
0.75 \\
0.88 \\
0.93 \\
0.835 \\
0.93 \\
0.705 \\
0.87 \\
0.985 \\
0.96 \\
0.71 \\
0.995 \\
0.915 \\
0.95 \\
0.69 \\
0.84 \\
0.75 \\
0.925 \\
0.995 \\
0.985 \\
0.915 \\
0.79 \\
0.94 \\
0.94 \\
0.805 \\
0.815 \\
0.995 \\
0.935 \\
0.98 \\
0.995 \\
0.735 \\
0.995 \\
0.915 \\
0.985 \\
0.825 \\
0.9 \\
0.975 \\
0.865 \\
0.995 \\
0.82 \\
0.87 \\
0.995 \\
0.995 \\
0.995 \\
0.86 \\
0.95 \\
0.93 \\
};

\addplot[mark=*, boxplot, boxplot/draw position=4]
table[row sep=\\, y index=0] {
data
0.995 \\
0.9 \\
0.995 \\
0.99 \\
0.87 \\
0.88 \\
0.92 \\
0.84 \\
0.995 \\
0.845 \\
0.895 \\
0.995 \\
0.99 \\
0.97 \\
0.98 \\
0.855 \\
0.985 \\
0.995 \\
0.995 \\
0.99 \\
0.895 \\
0.965 \\
0.97 \\
0.89 \\
0.995 \\
0.995 \\
0.96 \\
0.88 \\
0.97 \\
0.9 \\
0.92 \\
0.785 \\
0.865 \\
0.995 \\
0.825 \\
0.995 \\
0.945 \\
0.98 \\
0.975 \\
0.83 \\
0.885 \\
0.945 \\
0.995 \\
0.75 \\
0.995 \\
0.99 \\
0.985 \\
0.98 \\
0.92 \\
0.975 \\
};

\addplot[mark=*, boxplot, boxplot/draw position=5]
table[row sep=\\, y index=0] {
data
0.995 \\
0.995 \\
0.995 \\
0.995 \\
0.97 \\
0.995 \\
0.91 \\
0.995 \\
0.995 \\
0.995 \\
0.975 \\
0.905 \\
0.955 \\
0.885 \\
0.995 \\
0.995 \\
0.99 \\
0.88 \\
0.96 \\
0.995 \\
0.895 \\
0.995 \\
0.965 \\
0.995 \\
0.81 \\
0.995 \\
0.995 \\
0.955 \\
0.995 \\
0.89 \\
0.875 \\
0.995 \\
0.78 \\
0.4 \\
0.97 \\
0.92 \\
0.975 \\
0.985 \\
0.985 \\
0.995 \\
0.845 \\
0.82 \\
0.965 \\
0.995 \\
0.975 \\
0.935 \\
0.92 \\
0.985 \\
0.995 \\
0.995 \\
};

\addplot[mark=*, boxplot, boxplot/draw position=6]
table[row sep=\\, y index=0] {
data
0.995 \\
0.99 \\
0.995 \\
0.935 \\
0.97 \\
0.995 \\
0.995 \\
0.995 \\
0.945 \\
0.995 \\
0.995 \\
0.895 \\
0.995 \\
0.995 \\
0.995 \\
0.995 \\
0.995 \\
0.975 \\
0.995 \\
0.995 \\
0.995 \\
0.995 \\
0.995 \\
0.995 \\
0.995 \\
0.995 \\
0.94 \\
0.995 \\
0.895 \\
0.88 \\
0.96 \\
0.995 \\
0.995 \\
0.975 \\
0.96 \\
0.875 \\
0.995 \\
0.985 \\
0.945 \\
0.995 \\
0.995 \\
0.995 \\
0.995 \\
0.815 \\
0.935 \\
0.995 \\
0.995 \\
0.995 \\
0.7 \\
0.995 \\
};
}{0.1}{Reservoir subset size}{11}

        }   
    }   
    \caption{Plots of output connectivity against accuracy on TP3 - Part 1 of 2}
\end{figure*}

\begin{figure*}[ht]
    \centering
    \resizebox{\textwidth}{!}{
        \subfloat[N=70]{
            \myboxplot{

\addplot[mark=*, boxplot, boxplot/draw position=1]
table[row sep=\\, y index=0] {
data
0.84 \\
0.68 \\
0.995 \\
0.8 \\
0.83 \\
0.51 \\
0.55 \\
0.525 \\
0.695 \\
0.555 \\
0.945 \\
0.625 \\
0.775 \\
0.75 \\
0.895 \\
0.93 \\
0.805 \\
0.725 \\
0.68 \\
0.505 \\
0.53 \\
0.88 \\
0.73 \\
0.71 \\
0.44 \\
0.78 \\
0.87 \\
0.79 \\
0.88 \\
0.54 \\
0.75 \\
0.995 \\
0.615 \\
0.655 \\
0.61 \\
0.365 \\
0.805 \\
0.42 \\
0.82 \\
0.66 \\
0.715 \\
0.43 \\
0.685 \\
0.815 \\
0.86 \\
0.88 \\
0.515 \\
0.535 \\
0.52 \\
0.735 \\
};

\addplot[mark=*, boxplot, boxplot/draw position=2]
table[row sep=\\, y index=0] {
data
0.7 \\
0.995 \\
0.78 \\
0.83 \\
0.915 \\
0.845 \\
0.595 \\
0.725 \\
0.7 \\
0.705 \\
0.94 \\
0.905 \\
0.995 \\
0.535 \\
0.73 \\
0.705 \\
0.88 \\
0.71 \\
0.935 \\
0.925 \\
0.87 \\
0.835 \\
0.695 \\
0.67 \\
0.84 \\
0.935 \\
0.99 \\
0.91 \\
0.8 \\
0.995 \\
0.79 \\
0.925 \\
0.695 \\
0.89 \\
0.84 \\
0.86 \\
0.895 \\
0.97 \\
0.88 \\
0.785 \\
0.86 \\
0.66 \\
0.845 \\
0.845 \\
0.92 \\
0.615 \\
0.785 \\
0.695 \\
0.84 \\
0.905 \\
};

\addplot[mark=*, boxplot, boxplot/draw position=3]
table[row sep=\\, y index=0] {
data
0.98 \\
0.875 \\
0.47 \\
0.91 \\
0.835 \\
0.825 \\
0.995 \\
0.995 \\
0.795 \\
0.705 \\
0.96 \\
0.995 \\
0.985 \\
0.995 \\
0.995 \\
0.995 \\
0.985 \\
0.995 \\
0.965 \\
0.95 \\
0.995 \\
0.88 \\
0.825 \\
0.995 \\
0.965 \\
0.95 \\
0.96 \\
0.995 \\
0.9 \\
0.93 \\
0.82 \\
0.955 \\
0.925 \\
0.66 \\
0.985 \\
0.94 \\
0.945 \\
0.995 \\
0.995 \\
0.93 \\
0.975 \\
0.985 \\
0.765 \\
0.79 \\
0.77 \\
0.95 \\
0.95 \\
0.99 \\
0.87 \\
0.965 \\
};

\addplot[mark=*, boxplot, boxplot/draw position=4]
table[row sep=\\, y index=0] {
data
0.985 \\
0.995 \\
0.895 \\
0.93 \\
0.73 \\
0.925 \\
0.995 \\
0.88 \\
0.94 \\
0.885 \\
0.8 \\
0.995 \\
0.905 \\
0.72 \\
0.92 \\
0.965 \\
0.995 \\
0.96 \\
0.74 \\
0.995 \\
0.945 \\
0.995 \\
0.995 \\
0.93 \\
0.995 \\
0.99 \\
0.925 \\
0.91 \\
0.95 \\
0.97 \\
0.945 \\
0.945 \\
0.995 \\
0.995 \\
0.96 \\
0.94 \\
0.85 \\
0.88 \\
0.945 \\
0.78 \\
0.935 \\
0.915 \\
0.995 \\
0.97 \\
0.665 \\
0.78 \\
0.995 \\
0.995 \\
0.97 \\
0.9 \\
};

\addplot[mark=*, boxplot, boxplot/draw position=5]
table[row sep=\\, y index=0] {
data
0.995 \\
0.825 \\
0.995 \\
0.995 \\
0.995 \\
0.92 \\
0.995 \\
0.995 \\
0.94 \\
0.97 \\
0.96 \\
0.975 \\
0.995 \\
0.98 \\
0.995 \\
0.825 \\
0.885 \\
0.995 \\
0.9 \\
0.93 \\
0.95 \\
0.92 \\
0.95 \\
0.965 \\
0.905 \\
0.94 \\
0.98 \\
0.865 \\
0.885 \\
0.96 \\
0.995 \\
0.925 \\
0.71 \\
0.995 \\
0.985 \\
0.955 \\
0.725 \\
0.95 \\
0.885 \\
0.965 \\
0.995 \\
0.97 \\
0.94 \\
0.995 \\
0.995 \\
0.995 \\
0.8 \\
0.9 \\
0.995 \\
0.98 \\
};

\addplot[mark=*, boxplot, boxplot/draw position=6]
table[row sep=\\, y index=0] {
data
0.9 \\
0.995 \\
0.995 \\
0.995 \\
0.995 \\
0.995 \\
0.95 \\
0.96 \\
0.995 \\
0.995 \\
0.995 \\
0.99 \\
0.995 \\
0.995 \\
0.995 \\
0.995 \\
0.995 \\
0.945 \\
0.87 \\
0.955 \\
0.99 \\
0.89 \\
0.995 \\
0.955 \\
0.995 \\
0.92 \\
0.95 \\
0.995 \\
0.985 \\
0.8 \\
0.995 \\
0.995 \\
0.96 \\
0.895 \\
0.9 \\
0.995 \\
0.93 \\
0.89 \\
0.92 \\
0.91 \\
0.995 \\
0.99 \\
0.845 \\
0.995 \\
0.995 \\
0.86 \\
0.915 \\
0.96 \\
0.695 \\
0.77 \\
};

\addplot[mark=*, boxplot, boxplot/draw position=7]
table[row sep=\\, y index=0] {
data
0.87 \\
0.985 \\
0.86 \\
0.995 \\
0.995 \\
0.995 \\
0.9 \\
0.995 \\
0.995 \\
0.995 \\
0.995 \\
0.995 \\
0.99 \\
0.965 \\
0.985 \\
0.995 \\
0.95 \\
0.995 \\
0.995 \\
0.96 \\
0.995 \\
0.995 \\
0.995 \\
0.995 \\
0.995 \\
0.995 \\
0.995 \\
0.995 \\
0.995 \\
0.91 \\
0.995 \\
0.995 \\
0.995 \\
0.995 \\
0.95 \\
0.995 \\
0.995 \\
0.94 \\
0.995 \\
0.995 \\
0.995 \\
0.995 \\
0.965 \\
0.995 \\
0.995 \\
0.995 \\
0.965 \\
0.955 \\
0.995 \\
0.995 \\
};
}{0.1}{Output connectivity}{11}

        }
        \subfloat[N=80]{
            \myboxplot{

\addplot[mark=*, boxplot, boxplot/draw position=1]
table[row sep=\\, y index=0] {
data
0.685 \\
0.67 \\
0.815 \\
0.765 \\
0.76 \\
0.81 \\
0.69 \\
0.675 \\
0.83 \\
0.8 \\
0.62 \\
0.585 \\
0.49 \\
0.715 \\
0.37 \\
0.84 \\
0.835 \\
0.7 \\
0.675 \\
0.705 \\
0.61 \\
0.9 \\
0.68 \\
0.455 \\
0.375 \\
0.81 \\
0.74 \\
0.83 \\
0.5 \\
0.545 \\
0.585 \\
0.7 \\
0.725 \\
0.995 \\
0.86 \\
0.655 \\
0.525 \\
0.835 \\
0.83 \\
0.705 \\
0.49 \\
0.675 \\
0.85 \\
0.8 \\
0.71 \\
0.87 \\
0.58 \\
0.615 \\
0.815 \\
0.58 \\
};

\addplot[mark=*, boxplot, boxplot/draw position=2]
table[row sep=\\, y index=0] {
data
0.995 \\
0.875 \\
0.825 \\
0.93 \\
0.755 \\
0.855 \\
0.995 \\
0.755 \\
0.935 \\
0.81 \\
0.58 \\
0.705 \\
0.89 \\
0.995 \\
0.87 \\
0.79 \\
0.995 \\
0.595 \\
0.8 \\
0.83 \\
0.825 \\
0.625 \\
0.995 \\
0.995 \\
0.865 \\
0.74 \\
0.89 \\
0.765 \\
0.97 \\
0.875 \\
0.995 \\
0.77 \\
0.76 \\
0.75 \\
0.935 \\
0.935 \\
0.935 \\
0.59 \\
0.905 \\
0.7 \\
0.95 \\
0.835 \\
0.95 \\
0.855 \\
0.81 \\
0.83 \\
0.875 \\
0.735 \\
0.555 \\
0.945 \\
};

\addplot[mark=*, boxplot, boxplot/draw position=3]
table[row sep=\\, y index=0] {
data
0.885 \\
0.975 \\
0.81 \\
0.75 \\
0.82 \\
0.915 \\
0.995 \\
0.445 \\
0.79 \\
0.9 \\
0.925 \\
0.86 \\
0.955 \\
0.955 \\
0.95 \\
0.93 \\
0.9 \\
0.95 \\
0.96 \\
0.81 \\
0.945 \\
0.965 \\
0.955 \\
0.955 \\
0.945 \\
0.87 \\
0.95 \\
0.87 \\
0.925 \\
0.715 \\
0.82 \\
0.67 \\
0.97 \\
0.765 \\
0.9 \\
0.985 \\
0.98 \\
0.88 \\
0.905 \\
0.715 \\
0.995 \\
0.995 \\
0.865 \\
0.925 \\
0.995 \\
0.995 \\
0.995 \\
0.925 \\
0.83 \\
0.855 \\
};

\addplot[mark=*, boxplot, boxplot/draw position=4]
table[row sep=\\, y index=0] {
data
0.935 \\
0.935 \\
0.995 \\
0.785 \\
0.935 \\
0.995 \\
0.995 \\
0.56 \\
0.995 \\
0.975 \\
0.79 \\
0.825 \\
0.95 \\
0.89 \\
0.925 \\
0.815 \\
0.925 \\
0.95 \\
0.99 \\
0.995 \\
0.94 \\
0.995 \\
0.895 \\
0.845 \\
0.98 \\
0.9 \\
0.87 \\
0.81 \\
0.955 \\
0.995 \\
0.995 \\
0.995 \\
0.72 \\
0.915 \\
0.995 \\
0.995 \\
0.955 \\
0.995 \\
0.935 \\
0.905 \\
0.92 \\
0.93 \\
0.97 \\
0.975 \\
0.995 \\
0.935 \\
0.75 \\
0.99 \\
0.885 \\
0.995 \\
};

\addplot[mark=*, boxplot, boxplot/draw position=5]
table[row sep=\\, y index=0] {
data
0.995 \\
0.87 \\
0.995 \\
0.89 \\
0.98 \\
0.995 \\
0.77 \\
0.995 \\
0.995 \\
0.99 \\
0.86 \\
0.97 \\
0.995 \\
0.995 \\
0.995 \\
0.995 \\
0.995 \\
0.995 \\
0.975 \\
0.995 \\
0.72 \\
0.995 \\
0.825 \\
0.995 \\
0.995 \\
0.725 \\
0.995 \\
0.965 \\
0.88 \\
0.995 \\
0.98 \\
0.96 \\
0.965 \\
0.95 \\
0.995 \\
0.995 \\
0.995 \\
0.995 \\
0.995 \\
0.95 \\
0.745 \\
0.93 \\
0.945 \\
0.995 \\
0.995 \\
0.96 \\
0.995 \\
0.985 \\
0.995 \\
0.925 \\
};

\addplot[mark=*, boxplot, boxplot/draw position=6]
table[row sep=\\, y index=0] {
data
0.995 \\
0.99 \\
0.955 \\
0.995 \\
0.97 \\
0.995 \\
0.995 \\
0.99 \\
0.995 \\
0.995 \\
0.995 \\
0.945 \\
0.995 \\
0.995 \\
0.995 \\
0.975 \\
0.985 \\
0.995 \\
0.995 \\
0.995 \\
0.99 \\
0.99 \\
0.86 \\
0.885 \\
0.945 \\
0.92 \\
0.995 \\
0.985 \\
0.995 \\
0.935 \\
0.995 \\
0.975 \\
0.89 \\
0.95 \\
0.945 \\
0.91 \\
0.995 \\
0.995 \\
0.995 \\
0.995 \\
0.985 \\
0.995 \\
0.9 \\
0.995 \\
0.98 \\
0.845 \\
0.99 \\
0.975 \\
0.995 \\
0.95 \\
};

\addplot[mark=*, boxplot, boxplot/draw position=7]
table[row sep=\\, y index=0] {
data
0.995 \\
0.995 \\
0.99 \\
0.975 \\
0.985 \\
0.995 \\
0.995 \\
0.995 \\
0.98 \\
0.985 \\
0.995 \\
0.995 \\
0.995 \\
0.965 \\
0.995 \\
0.995 \\
0.99 \\
0.935 \\
0.995 \\
0.98 \\
0.995 \\
0.77 \\
0.995 \\
0.995 \\
0.945 \\
0.985 \\
0.995 \\
0.98 \\
0.995 \\
0.995 \\
0.995 \\
0.975 \\
0.99 \\
0.95 \\
0.935 \\
0.995 \\
0.95 \\
0.995 \\
0.91 \\
0.995 \\
0.995 \\
0.995 \\
0.995 \\
0.995 \\
0.965 \\
0.945 \\
0.995 \\
0.97 \\
0.98 \\
0.995 \\
};

\addplot[mark=*, boxplot, boxplot/draw position=8]
table[row sep=\\, y index=0] {
data
0.95 \\
0.995 \\
0.995 \\
0.995 \\
0.965 \\
0.95 \\
0.995 \\
0.995 \\
0.97 \\
0.99 \\
0.995 \\
0.965 \\
0.995 \\
0.995 \\
0.995 \\
0.995 \\
0.995 \\
0.965 \\
0.97 \\
0.995 \\
0.99 \\
0.995 \\
0.97 \\
0.995 \\
0.995 \\
0.995 \\
0.995 \\
0.965 \\
0.995 \\
0.995 \\
0.995 \\
0.995 \\
0.965 \\
0.945 \\
0.995 \\
0.995 \\
0.995 \\
0.985 \\
0.995 \\
0.985 \\
0.995 \\
0.995 \\
0.82 \\
0.995 \\
0.995 \\
0.995 \\
0.97 \\
0.955 \\
0.995 \\
0.995 \\
};
}{0.1}{Output connectivity}{11}

        }   
    }   
    \resizebox{\textwidth}{!}{
        \subfloat[N=90]{
            \myboxplot{

\addplot[mark=*, boxplot, boxplot/draw position=1]
table[row sep=\\, y index=0] {
data
0.62 \\
0.7 \\
0.655 \\
0.71 \\
0.7 \\
0.695 \\
0.54 \\
0.55 \\
0.83 \\
0.76 \\
0.555 \\
0.7 \\
0.695 \\
0.635 \\
0.705 \\
0.53 \\
0.71 \\
0.725 \\
0.51 \\
0.615 \\
0.755 \\
0.795 \\
0.675 \\
0.605 \\
0.74 \\
0.545 \\
0.94 \\
0.34 \\
0.715 \\
0.75 \\
0.635 \\
0.63 \\
0.755 \\
0.795 \\
0.71 \\
0.69 \\
0.59 \\
0.995 \\
0.5 \\
0.875 \\
0.91 \\
0.625 \\
0.74 \\
0.865 \\
0.56 \\
0.87 \\
0.545 \\
0.73 \\
0.72 \\
0.775 \\
};

\addplot[mark=*, boxplot, boxplot/draw position=2]
table[row sep=\\, y index=0] {
data
0.91 \\
0.925 \\
0.92 \\
0.775 \\
0.805 \\
0.915 \\
0.795 \\
0.995 \\
0.845 \\
0.93 \\
0.81 \\
0.64 \\
0.665 \\
0.81 \\
0.815 \\
0.97 \\
0.94 \\
0.935 \\
0.995 \\
0.7 \\
0.585 \\
0.845 \\
0.81 \\
0.88 \\
0.8 \\
0.915 \\
0.63 \\
0.535 \\
0.875 \\
0.695 \\
0.95 \\
0.67 \\
0.61 \\
0.575 \\
0.685 \\
0.995 \\
0.83 \\
0.705 \\
0.985 \\
0.79 \\
0.875 \\
0.56 \\
0.955 \\
0.845 \\
0.78 \\
0.89 \\
0.685 \\
0.835 \\
0.61 \\
0.92 \\
};

\addplot[mark=*, boxplot, boxplot/draw position=3]
table[row sep=\\, y index=0] {
data
0.935 \\
0.855 \\
0.995 \\
0.745 \\
0.9 \\
0.68 \\
0.44 \\
0.86 \\
0.965 \\
0.98 \\
0.925 \\
0.995 \\
0.98 \\
0.995 \\
0.995 \\
0.975 \\
0.9 \\
0.9 \\
0.975 \\
0.965 \\
0.885 \\
0.985 \\
0.905 \\
0.935 \\
0.995 \\
0.915 \\
0.955 \\
0.825 \\
0.995 \\
0.785 \\
0.995 \\
0.845 \\
0.95 \\
0.995 \\
0.945 \\
0.995 \\
0.995 \\
0.985 \\
0.995 \\
0.995 \\
0.52 \\
0.84 \\
0.855 \\
0.93 \\
0.995 \\
0.9 \\
0.995 \\
0.96 \\
0.79 \\
0.89 \\
};

\addplot[mark=*, boxplot, boxplot/draw position=4]
table[row sep=\\, y index=0] {
data
0.99 \\
0.82 \\
0.85 \\
0.93 \\
0.875 \\
0.94 \\
0.975 \\
0.96 \\
0.92 \\
0.995 \\
0.995 \\
0.995 \\
0.87 \\
0.995 \\
0.93 \\
0.995 \\
0.985 \\
0.99 \\
0.905 \\
0.995 \\
0.995 \\
0.995 \\
0.995 \\
0.995 \\
0.97 \\
0.97 \\
0.91 \\
0.995 \\
0.925 \\
0.99 \\
0.95 \\
0.935 \\
0.98 \\
0.965 \\
0.975 \\
0.985 \\
0.95 \\
0.755 \\
0.97 \\
0.975 \\
0.95 \\
0.995 \\
0.89 \\
0.9 \\
0.99 \\
0.995 \\
0.97 \\
0.93 \\
0.995 \\
0.995 \\
};

\addplot[mark=*, boxplot, boxplot/draw position=5]
table[row sep=\\, y index=0] {
data
0.995 \\
0.995 \\
0.995 \\
0.91 \\
0.995 \\
0.995 \\
0.995 \\
0.995 \\
0.995 \\
0.855 \\
0.875 \\
0.95 \\
0.78 \\
0.825 \\
0.945 \\
0.995 \\
0.975 \\
0.985 \\
0.995 \\
0.95 \\
0.995 \\
0.89 \\
0.995 \\
0.995 \\
0.995 \\
0.995 \\
0.995 \\
0.97 \\
0.995 \\
0.985 \\
0.995 \\
0.91 \\
0.995 \\
0.995 \\
0.995 \\
0.795 \\
0.995 \\
0.995 \\
0.97 \\
0.995 \\
0.955 \\
0.995 \\
0.98 \\
0.955 \\
0.95 \\
0.94 \\
0.995 \\
0.995 \\
0.755 \\
0.995 \\
};

\addplot[mark=*, boxplot, boxplot/draw position=6]
table[row sep=\\, y index=0] {
data
0.965 \\
0.995 \\
0.995 \\
0.995 \\
0.995 \\
0.97 \\
0.985 \\
0.99 \\
0.995 \\
0.935 \\
0.995 \\
0.995 \\
0.995 \\
0.92 \\
0.995 \\
0.86 \\
0.995 \\
0.96 \\
0.995 \\
0.995 \\
0.985 \\
0.995 \\
0.955 \\
0.995 \\
0.995 \\
0.985 \\
0.985 \\
0.99 \\
0.945 \\
0.995 \\
0.975 \\
0.995 \\
0.995 \\
0.995 \\
0.995 \\
0.915 \\
0.995 \\
0.99 \\
0.92 \\
0.985 \\
0.795 \\
0.995 \\
0.96 \\
0.96 \\
0.995 \\
0.96 \\
0.955 \\
0.99 \\
0.995 \\
0.94 \\
};

\addplot[mark=*, boxplot, boxplot/draw position=7]
table[row sep=\\, y index=0] {
data
0.985 \\
0.995 \\
0.985 \\
0.995 \\
0.995 \\
0.99 \\
0.995 \\
0.995 \\
0.995 \\
0.995 \\
0.98 \\
0.995 \\
0.93 \\
0.935 \\
0.94 \\
0.995 \\
0.995 \\
0.965 \\
0.955 \\
0.945 \\
0.995 \\
0.95 \\
0.95 \\
0.995 \\
0.995 \\
0.995 \\
0.985 \\
0.995 \\
0.925 \\
0.985 \\
0.97 \\
0.995 \\
0.995 \\
0.995 \\
0.995 \\
0.995 \\
0.985 \\
0.59 \\
0.985 \\
0.995 \\
0.94 \\
0.995 \\
0.995 \\
0.995 \\
0.975 \\
0.995 \\
0.995 \\
0.995 \\
0.97 \\
0.96 \\
};

\addplot[mark=*, boxplot, boxplot/draw position=8]
table[row sep=\\, y index=0] {
data
0.995 \\
0.995 \\
0.885 \\
0.995 \\
0.985 \\
0.945 \\
0.995 \\
0.995 \\
0.995 \\
0.995 \\
0.995 \\
0.99 \\
0.995 \\
0.995 \\
0.995 \\
0.995 \\
0.995 \\
0.995 \\
0.97 \\
0.945 \\
0.995 \\
0.995 \\
0.995 \\
0.995 \\
0.995 \\
0.995 \\
0.96 \\
0.975 \\
0.965 \\
0.995 \\
0.99 \\
0.995 \\
0.975 \\
0.995 \\
0.995 \\
0.995 \\
0.995 \\
0.995 \\
0.99 \\
0.995 \\
0.985 \\
0.985 \\
0.995 \\
0.99 \\
0.995 \\
0.995 \\
0.995 \\
0.995 \\
0.985 \\
0.995 \\
};

\addplot[mark=*, boxplot, boxplot/draw position=9]
table[row sep=\\, y index=0] {
data
0.995 \\
0.995 \\
0.995 \\
0.995 \\
0.995 \\
0.985 \\
0.995 \\
0.93 \\
0.995 \\
0.995 \\
0.99 \\
0.995 \\
0.995 \\
0.995 \\
0.995 \\
0.995 \\
0.94 \\
0.995 \\
0.995 \\
0.995 \\
0.995 \\
0.985 \\
0.995 \\
0.99 \\
0.995 \\
0.995 \\
0.995 \\
0.995 \\
0.995 \\
0.995 \\
0.995 \\
0.995 \\
0.925 \\
0.995 \\
0.995 \\
0.995 \\
0.99 \\
0.995 \\
0.995 \\
0.995 \\
0.995 \\
0.995 \\
0.995 \\
0.995 \\
0.995 \\
0.995 \\
0.995 \\
0.98 \\
0.995 \\
0.995 \\
};
}{0.1}{Reservoir subset size}{11}

        }   
        \subfloat[N=100]{
            \myboxplot{

\addplot[mark=*, boxplot, boxplot/draw position=1]
table[row sep=\\, y index=0] {
data
0.745 \\
0.375 \\
0.495 \\
0.935 \\
0.68 \\
0.76 \\
0.695 \\
0.64 \\
0.67 \\
0.595 \\
0.955 \\
0.755 \\
0.5 \\
0.65 \\
0.815 \\
0.73 \\
0.63 \\
0.695 \\
0.645 \\
0.695 \\
0.485 \\
0.765 \\
0.745 \\
0.665 \\
0.72 \\
0.61 \\
0.66 \\
0.585 \\
0.755 \\
0.78 \\
0.815 \\
0.475 \\
0.51 \\
0.995 \\
0.505 \\
0.995 \\
0.59 \\
0.58 \\
0.62 \\
0.725 \\
0.73 \\
0.74 \\
0.865 \\
0.665 \\
0.87 \\
0.68 \\
0.55 \\
0.77 \\
0.595 \\
0.62 \\
};

\addplot[mark=*, boxplot, boxplot/draw position=2]
table[row sep=\\, y index=0] {
data
0.93 \\
0.7 \\
0.775 \\
0.745 \\
0.84 \\
0.925 \\
0.72 \\
0.88 \\
0.58 \\
0.585 \\
0.945 \\
0.935 \\
0.975 \\
0.925 \\
0.735 \\
0.995 \\
0.88 \\
0.48 \\
0.655 \\
0.75 \\
0.66 \\
0.51 \\
0.925 \\
0.725 \\
0.85 \\
0.655 \\
0.96 \\
0.955 \\
0.785 \\
0.885 \\
0.775 \\
0.77 \\
0.97 \\
0.995 \\
0.75 \\
0.995 \\
0.89 \\
0.78 \\
0.87 \\
0.89 \\
0.775 \\
0.795 \\
0.995 \\
0.745 \\
0.95 \\
0.865 \\
0.78 \\
0.995 \\
0.995 \\
0.77 \\
};

\addplot[mark=*, boxplot, boxplot/draw position=3]
table[row sep=\\, y index=0] {
data
0.935 \\
0.97 \\
0.85 \\
0.96 \\
0.825 \\
0.75 \\
0.84 \\
0.94 \\
0.875 \\
0.99 \\
0.95 \\
0.88 \\
0.895 \\
0.995 \\
0.87 \\
0.955 \\
0.94 \\
0.72 \\
0.785 \\
0.995 \\
0.94 \\
0.935 \\
0.775 \\
0.78 \\
0.89 \\
0.92 \\
0.83 \\
0.775 \\
0.955 \\
0.875 \\
0.975 \\
0.795 \\
0.985 \\
0.995 \\
0.995 \\
0.995 \\
0.705 \\
0.97 \\
0.855 \\
0.995 \\
0.91 \\
0.975 \\
0.8 \\
0.995 \\
0.905 \\
0.83 \\
0.965 \\
0.995 \\
0.9 \\
0.995 \\
};

\addplot[mark=*, boxplot, boxplot/draw position=4]
table[row sep=\\, y index=0] {
data
0.995 \\
0.85 \\
0.995 \\
0.98 \\
0.88 \\
0.995 \\
0.995 \\
0.975 \\
0.96 \\
0.87 \\
0.995 \\
0.93 \\
0.995 \\
0.975 \\
0.815 \\
0.87 \\
0.795 \\
0.86 \\
0.995 \\
0.995 \\
0.995 \\
0.915 \\
0.805 \\
0.95 \\
0.995 \\
0.99 \\
0.89 \\
0.98 \\
0.995 \\
0.895 \\
0.935 \\
0.93 \\
0.845 \\
0.97 \\
0.955 \\
0.815 \\
0.945 \\
0.985 \\
0.95 \\
0.96 \\
0.93 \\
0.995 \\
0.985 \\
0.995 \\
0.98 \\
0.91 \\
0.965 \\
0.995 \\
0.825 \\
0.995 \\
};

\addplot[mark=*, boxplot, boxplot/draw position=5]
table[row sep=\\, y index=0] {
data
0.965 \\
0.99 \\
0.995 \\
0.875 \\
0.915 \\
0.7 \\
0.995 \\
0.985 \\
0.935 \\
0.995 \\
0.945 \\
0.995 \\
0.995 \\
0.96 \\
0.98 \\
0.995 \\
0.985 \\
0.955 \\
0.995 \\
0.995 \\
0.995 \\
0.995 \\
0.98 \\
0.935 \\
0.995 \\
0.965 \\
0.935 \\
0.895 \\
0.995 \\
0.995 \\
0.985 \\
0.985 \\
0.995 \\
0.985 \\
0.78 \\
0.995 \\
0.995 \\
0.995 \\
0.98 \\
0.98 \\
0.89 \\
0.995 \\
0.995 \\
0.84 \\
0.99 \\
0.995 \\
0.98 \\
0.985 \\
0.855 \\
0.815 \\
};

\addplot[mark=*, boxplot, boxplot/draw position=6]
table[row sep=\\, y index=0] {
data
0.995 \\
0.995 \\
0.955 \\
0.995 \\
0.995 \\
0.825 \\
0.9 \\
0.95 \\
0.97 \\
0.99 \\
0.995 \\
0.995 \\
0.995 \\
0.995 \\
0.995 \\
0.995 \\
0.975 \\
0.98 \\
0.9 \\
0.995 \\
0.91 \\
0.86 \\
0.995 \\
0.995 \\
0.995 \\
0.995 \\
0.98 \\
0.91 \\
0.945 \\
0.985 \\
0.93 \\
0.905 \\
0.92 \\
0.995 \\
0.935 \\
0.995 \\
0.995 \\
0.995 \\
0.96 \\
0.995 \\
0.995 \\
0.98 \\
0.965 \\
0.8 \\
0.995 \\
0.995 \\
0.995 \\
0.99 \\
0.945 \\
0.995 \\
};

\addplot[mark=*, boxplot, boxplot/draw position=7]
table[row sep=\\, y index=0] {
data
0.99 \\
0.995 \\
0.995 \\
0.97 \\
0.995 \\
0.995 \\
0.82 \\
0.995 \\
0.995 \\
0.995 \\
0.995 \\
0.995 \\
0.995 \\
0.995 \\
0.995 \\
0.995 \\
0.995 \\
0.99 \\
0.98 \\
0.855 \\
0.995 \\
0.995 \\
0.99 \\
0.98 \\
0.995 \\
0.98 \\
0.995 \\
0.995 \\
0.99 \\
0.995 \\
0.99 \\
0.975 \\
0.995 \\
0.96 \\
0.995 \\
0.99 \\
0.89 \\
0.99 \\
0.98 \\
0.995 \\
0.985 \\
0.985 \\
0.975 \\
0.995 \\
0.915 \\
0.995 \\
0.995 \\
0.995 \\
0.965 \\
0.995 \\
};

\addplot[mark=*, boxplot, boxplot/draw position=8]
table[row sep=\\, y index=0] {
data
0.995 \\
0.995 \\
0.97 \\
0.995 \\
0.995 \\
0.995 \\
0.995 \\
0.995 \\
0.995 \\
0.995 \\
0.94 \\
0.995 \\
0.995 \\
0.975 \\
0.925 \\
0.995 \\
0.995 \\
0.995 \\
0.995 \\
0.98 \\
0.985 \\
0.955 \\
0.995 \\
0.995 \\
0.995 \\
0.95 \\
0.995 \\
0.995 \\
0.995 \\
0.975 \\
0.97 \\
0.995 \\
0.995 \\
0.94 \\
0.99 \\
0.995 \\
0.995 \\
0.985 \\
0.995 \\
0.995 \\
0.995 \\
0.995 \\
0.995 \\
0.995 \\
0.93 \\
0.995 \\
0.995 \\
0.995 \\
0.995 \\
0.99 \\
};

\addplot[mark=*, boxplot, boxplot/draw position=9]
table[row sep=\\, y index=0] {
data
0.95 \\
0.995 \\
0.995 \\
0.995 \\
0.995 \\
0.995 \\
0.995 \\
0.995 \\
0.995 \\
0.995 \\
0.995 \\
0.995 \\
0.995 \\
0.995 \\
0.995 \\
0.995 \\
0.995 \\
0.995 \\
0.995 \\
0.885 \\
0.995 \\
0.995 \\
0.995 \\
0.995 \\
0.99 \\
0.995 \\
0.995 \\
0.995 \\
0.945 \\
0.995 \\
0.995 \\
0.995 \\
0.995 \\
0.995 \\
0.995 \\
0.995 \\
0.995 \\
0.995 \\
0.97 \\
0.995 \\
0.985 \\
0.995 \\
0.995 \\
0.995 \\
0.995 \\
0.99 \\
0.995 \\
0.995 \\
0.995 \\
0.995 \\
};

\addplot[mark=*, boxplot, boxplot/draw position=10]
table[row sep=\\, y index=0] {
data
0.995 \\
0.995 \\
0.995 \\
0.995 \\
0.995 \\
0.995 \\
0.995 \\
0.965 \\
0.995 \\
0.84 \\
0.995 \\
0.99 \\
0.995 \\
0.975 \\
0.995 \\
0.995 \\
0.995 \\
0.995 \\
0.995 \\
0.995 \\
0.965 \\
0.995 \\
0.995 \\
0.995 \\
0.995 \\
0.995 \\
0.995 \\
0.985 \\
0.995 \\
0.995 \\
0.995 \\
0.995 \\
0.995 \\
0.995 \\
0.995 \\
0.985 \\
0.995 \\
0.995 \\
0.995 \\
0.995 \\
0.995 \\
0.995 \\
0.995 \\
0.97 \\
0.99 \\
0.995 \\
0.945 \\
0.995 \\
0.995 \\
0.995 \\
};
}{0.1}{Reservoir subset size}{11}

        }   
    }   
    \caption{Plots of output connectivity against accuracy on TP3 - Part 1 of 2}
\end{figure*}

%% BRekkku
%% BRekkku
%% BRekkku
%% BRekkku

\begin{figure*}[ht]
    \centering
    \resizebox{\textwidth}{!}{
        \subfloat[N=30]{
            \myboxplot{

\addplot[mark=*, boxplot, boxplot/draw position=1]
table[row sep=\\, y index=0] {
data
0.54 \\
0.555 \\
0.53 \\
0.585 \\
0.575 \\
0.555 \\
0.59 \\
0.435 \\
0.53 \\
0.49 \\
0.52 \\
0.55 \\
0.625 \\
0.535 \\
0.525 \\
0.525 \\
0.53 \\
0.47 \\
0.54 \\
0.61 \\
0.545 \\
0.545 \\
0.625 \\
0.515 \\
0.535 \\
0.59 \\
0.565 \\
0.53 \\
0.585 \\
0.6 \\
0.525 \\
0.595 \\
0.525 \\
0.52 \\
0.555 \\
0.51 \\
0.56 \\
0.55 \\
0.49 \\
0.58 \\
0.555 \\
0.495 \\
0.535 \\
0.56 \\
0.535 \\
0.515 \\
0.65 \\
0.52 \\
0.575 \\
0.515 \\
};

\addplot[mark=*, boxplot, boxplot/draw position=2]
table[row sep=\\, y index=0] {
data
0.615 \\
0.575 \\
0.525 \\
0.7 \\
0.665 \\
0.615 \\
0.545 \\
0.51 \\
0.525 \\
0.52 \\
0.49 \\
0.54 \\
0.74 \\
0.505 \\
0.58 \\
0.555 \\
0.7 \\
0.545 \\
0.58 \\
0.56 \\
0.525 \\
0.62 \\
0.655 \\
0.53 \\
0.535 \\
0.635 \\
0.66 \\
0.615 \\
0.61 \\
0.715 \\
0.525 \\
0.645 \\
0.545 \\
0.65 \\
0.5 \\
0.59 \\
0.53 \\
0.635 \\
0.54 \\
0.635 \\
0.565 \\
0.515 \\
0.675 \\
0.57 \\
0.535 \\
0.61 \\
0.675 \\
0.51 \\
0.55 \\
0.615 \\
};

\addplot[mark=*, boxplot, boxplot/draw position=3]
table[row sep=\\, y index=0] {
data
0.71 \\
0.565 \\
0.505 \\
0.705 \\
0.66 \\
0.615 \\
0.55 \\
0.55 \\
0.495 \\
0.505 \\
0.57 \\
0.56 \\
0.81 \\
0.54 \\
0.58 \\
0.6 \\
0.695 \\
0.585 \\
0.65 \\
0.575 \\
0.55 \\
0.555 \\
0.62 \\
0.63 \\
0.51 \\
0.68 \\
0.65 \\
0.56 \\
0.595 \\
0.705 \\
0.755 \\
0.67 \\
0.53 \\
0.655 \\
0.48 \\
0.55 \\
0.495 \\
0.56 \\
0.575 \\
0.635 \\
0.65 \\
0.465 \\
0.715 \\
0.57 \\
0.535 \\
0.575 \\
0.71 \\
0.55 \\
0.53 \\
0.58 \\
};
}{0.1}{Reservoir subset size}{15}

        }
        \subfloat[N=40]{
            \myboxplot{

\addplot[mark=*, boxplot, boxplot/draw position=1]
table[row sep=\\, y index=0] {
data
0.495 \\
0.52 \\
0.575 \\
0.62 \\
0.62 \\
0.525 \\
0.58 \\
0.61 \\
0.515 \\
0.59 \\
0.58 \\
0.55 \\
0.56 \\
0.475 \\
0.585 \\
0.63 \\
0.64 \\
0.535 \\
0.54 \\
0.525 \\
0.53 \\
0.58 \\
0.545 \\
0.615 \\
0.595 \\
0.545 \\
0.535 \\
0.53 \\
0.585 \\
0.585 \\
0.535 \\
0.545 \\
0.52 \\
0.485 \\
0.545 \\
0.53 \\
0.55 \\
0.67 \\
0.5 \\
0.49 \\
0.625 \\
0.565 \\
0.56 \\
0.66 \\
0.495 \\
0.51 \\
0.505 \\
0.54 \\
0.51 \\
0.62 \\
};

\addplot[mark=*, boxplot, boxplot/draw position=2]
table[row sep=\\, y index=0] {
data
0.605 \\
0.565 \\
0.58 \\
0.525 \\
0.59 \\
0.575 \\
0.57 \\
0.595 \\
0.62 \\
0.465 \\
0.52 \\
0.635 \\
0.505 \\
0.66 \\
0.52 \\
0.545 \\
0.51 \\
0.65 \\
0.62 \\
0.505 \\
0.585 \\
0.56 \\
0.6 \\
0.53 \\
0.655 \\
0.51 \\
0.675 \\
0.515 \\
0.625 \\
0.505 \\
0.585 \\
0.695 \\
0.56 \\
0.52 \\
0.525 \\
0.62 \\
0.505 \\
0.525 \\
0.66 \\
0.53 \\
0.56 \\
0.475 \\
0.63 \\
0.555 \\
0.59 \\
0.575 \\
0.575 \\
0.6 \\
0.535 \\
0.54 \\
};

\addplot[mark=*, boxplot, boxplot/draw position=3]
table[row sep=\\, y index=0] {
data
0.68 \\
0.53 \\
0.585 \\
0.51 \\
0.595 \\
0.55 \\
0.575 \\
0.775 \\
0.635 \\
0.505 \\
0.485 \\
0.67 \\
0.43 \\
0.65 \\
0.565 \\
0.515 \\
0.485 \\
0.655 \\
0.64 \\
0.525 \\
0.575 \\
0.57 \\
0.61 \\
0.53 \\
0.645 \\
0.61 \\
0.62 \\
0.525 \\
0.685 \\
0.55 \\
0.56 \\
0.715 \\
0.58 \\
0.515 \\
0.53 \\
0.675 \\
0.755 \\
0.54 \\
0.695 \\
0.59 \\
0.525 \\
0.535 \\
0.675 \\
0.555 \\
0.5 \\
0.52 \\
0.71 \\
0.53 \\
0.685 \\
0.565 \\
};

\addplot[mark=*, boxplot, boxplot/draw position=4]
table[row sep=\\, y index=0] {
data
0.685 \\
0.69 \\
0.525 \\
0.505 \\
0.675 \\
0.555 \\
0.725 \\
0.61 \\
0.565 \\
0.57 \\
0.645 \\
0.555 \\
0.72 \\
0.715 \\
0.575 \\
0.645 \\
0.67 \\
0.59 \\
0.57 \\
0.755 \\
0.59 \\
0.625 \\
0.62 \\
0.62 \\
0.625 \\
0.69 \\
0.625 \\
0.55 \\
0.59 \\
0.505 \\
0.45 \\
0.515 \\
0.7 \\
0.835 \\
0.575 \\
0.575 \\
0.535 \\
0.595 \\
0.715 \\
0.65 \\
0.7 \\
0.585 \\
0.585 \\
0.615 \\
0.695 \\
0.525 \\
0.695 \\
0.615 \\
0.585 \\
0.58 \\
};
}{0.1}{Output connectivity}{15}

        }
    }
    \resizebox{\textwidth}{!}{
        \subfloat[N=50]{
            \myboxplot{

\addplot[mark=*, boxplot, boxplot/draw position=1]
table[row sep=\\, y index=0] {
data
0.55 \\
0.66 \\
0.525 \\
0.58 \\
0.565 \\
0.625 \\
0.605 \\
0.565 \\
0.515 \\
0.645 \\
0.555 \\
0.64 \\
0.56 \\
0.49 \\
0.63 \\
0.66 \\
0.53 \\
0.505 \\
0.475 \\
0.51 \\
0.525 \\
0.55 \\
0.715 \\
0.56 \\
0.55 \\
0.595 \\
0.58 \\
0.675 \\
0.61 \\
0.57 \\
0.55 \\
0.54 \\
0.525 \\
0.535 \\
0.535 \\
0.575 \\
0.55 \\
0.655 \\
0.525 \\
0.63 \\
0.63 \\
0.53 \\
0.5 \\
0.565 \\
0.58 \\
0.525 \\
0.58 \\
0.61 \\
0.595 \\
0.52 \\
};

\addplot[mark=*, boxplot, boxplot/draw position=2]
table[row sep=\\, y index=0] {
data
0.575 \\
0.665 \\
0.535 \\
0.705 \\
0.535 \\
0.63 \\
0.645 \\
0.56 \\
0.64 \\
0.67 \\
0.57 \\
0.63 \\
0.685 \\
0.465 \\
0.565 \\
0.625 \\
0.53 \\
0.565 \\
0.455 \\
0.51 \\
0.525 \\
0.52 \\
0.675 \\
0.545 \\
0.58 \\
0.59 \\
0.565 \\
0.765 \\
0.73 \\
0.625 \\
0.585 \\
0.53 \\
0.545 \\
0.63 \\
0.525 \\
0.575 \\
0.565 \\
0.665 \\
0.495 \\
0.63 \\
0.67 \\
0.545 \\
0.495 \\
0.645 \\
0.585 \\
0.54 \\
0.6 \\
0.56 \\
0.59 \\
0.65 \\
};

\addplot[mark=*, boxplot, boxplot/draw position=3]
table[row sep=\\, y index=0] {
data
0.57 \\
0.535 \\
0.77 \\
0.505 \\
0.535 \\
0.755 \\
0.63 \\
0.685 \\
0.53 \\
0.575 \\
0.51 \\
0.625 \\
0.69 \\
0.61 \\
0.595 \\
0.645 \\
0.715 \\
0.605 \\
0.695 \\
0.655 \\
0.67 \\
0.57 \\
0.6 \\
0.635 \\
0.54 \\
0.76 \\
0.545 \\
0.59 \\
0.74 \\
0.58 \\
0.67 \\
0.59 \\
0.59 \\
0.535 \\
0.61 \\
0.64 \\
0.6 \\
0.565 \\
0.55 \\
0.535 \\
0.565 \\
0.565 \\
0.625 \\
0.555 \\
0.545 \\
0.545 \\
0.63 \\
0.735 \\
0.655 \\
0.575 \\
};

\addplot[mark=*, boxplot, boxplot/draw position=4]
table[row sep=\\, y index=0] {
data
0.595 \\
0.625 \\
0.6 \\
0.795 \\
0.715 \\
0.63 \\
0.575 \\
0.6 \\
0.61 \\
0.64 \\
0.6 \\
0.725 \\
0.605 \\
0.645 \\
0.545 \\
0.68 \\
0.71 \\
0.59 \\
0.49 \\
0.79 \\
0.655 \\
0.59 \\
0.735 \\
0.675 \\
0.645 \\
0.62 \\
0.685 \\
0.735 \\
0.54 \\
0.66 \\
0.555 \\
0.7 \\
0.525 \\
0.565 \\
0.72 \\
0.785 \\
0.8 \\
0.59 \\
0.725 \\
0.855 \\
0.515 \\
0.725 \\
0.545 \\
0.695 \\
0.685 \\
0.555 \\
0.635 \\
0.56 \\
0.56 \\
0.66 \\
};

\addplot[mark=*, boxplot, boxplot/draw position=5]
table[row sep=\\, y index=0] {
data
0.63 \\
0.55 \\
0.65 \\
0.8 \\
0.71 \\
0.73 \\
0.6 \\
0.595 \\
0.7 \\
0.66 \\
0.615 \\
0.735 \\
0.695 \\
0.76 \\
0.59 \\
0.725 \\
0.77 \\
0.585 \\
0.565 \\
0.795 \\
0.695 \\
0.58 \\
0.81 \\
0.68 \\
0.605 \\
0.64 \\
0.75 \\
0.755 \\
0.62 \\
0.69 \\
0.54 \\
0.68 \\
0.525 \\
0.645 \\
0.695 \\
0.795 \\
0.865 \\
0.57 \\
0.71 \\
0.83 \\
0.51 \\
0.715 \\
0.555 \\
0.7 \\
0.635 \\
0.565 \\
0.62 \\
0.555 \\
0.585 \\
0.68 \\
};
}{0.1}{Reservoir subset size}{15}

        }
        \subfloat[N=60]{
            \myboxplot{

\addplot[mark=*, boxplot, boxplot/draw position=1]
table[row sep=\\, y index=0] {
data
0.59 \\
0.535 \\
0.495 \\
0.565 \\
0.56 \\
0.525 \\
0.525 \\
0.65 \\
0.55 \\
0.53 \\
0.515 \\
0.595 \\
0.58 \\
0.645 \\
0.61 \\
0.58 \\
0.6 \\
0.53 \\
0.545 \\
0.6 \\
0.535 \\
0.53 \\
0.605 \\
0.67 \\
0.54 \\
0.515 \\
0.61 \\
0.545 \\
0.53 \\
0.61 \\
0.655 \\
0.495 \\
0.525 \\
0.575 \\
0.535 \\
0.6 \\
0.555 \\
0.485 \\
0.57 \\
0.535 \\
0.5 \\
0.605 \\
0.575 \\
0.54 \\
0.5 \\
0.525 \\
0.535 \\
0.55 \\
0.52 \\
0.495 \\
};

\addplot[mark=*, boxplot, boxplot/draw position=2]
table[row sep=\\, y index=0] {
data
0.605 \\
0.595 \\
0.58 \\
0.64 \\
0.45 \\
0.54 \\
0.58 \\
0.58 \\
0.545 \\
0.58 \\
0.51 \\
0.535 \\
0.53 \\
0.49 \\
0.6 \\
0.505 \\
0.445 \\
0.48 \\
0.525 \\
0.63 \\
0.565 \\
0.58 \\
0.62 \\
0.51 \\
0.695 \\
0.62 \\
0.605 \\
0.585 \\
0.625 \\
0.54 \\
0.65 \\
0.58 \\
0.62 \\
0.655 \\
0.555 \\
0.5 \\
0.58 \\
0.635 \\
0.615 \\
0.61 \\
0.635 \\
0.67 \\
0.54 \\
0.505 \\
0.695 \\
0.525 \\
0.645 \\
0.53 \\
0.54 \\
0.685 \\
};

\addplot[mark=*, boxplot, boxplot/draw position=3]
table[row sep=\\, y index=0] {
data
0.64 \\
0.555 \\
0.6 \\
0.685 \\
0.425 \\
0.6 \\
0.745 \\
0.6 \\
0.62 \\
0.595 \\
0.62 \\
0.535 \\
0.53 \\
0.535 \\
0.64 \\
0.475 \\
0.485 \\
0.555 \\
0.63 \\
0.605 \\
0.56 \\
0.6 \\
0.655 \\
0.5 \\
0.715 \\
0.69 \\
0.58 \\
0.53 \\
0.665 \\
0.585 \\
0.65 \\
0.79 \\
0.665 \\
0.665 \\
0.57 \\
0.585 \\
0.575 \\
0.695 \\
0.635 \\
0.66 \\
0.625 \\
0.665 \\
0.51 \\
0.545 \\
0.71 \\
0.525 \\
0.61 \\
0.585 \\
0.62 \\
0.66 \\
};

\addplot[mark=*, boxplot, boxplot/draw position=4]
table[row sep=\\, y index=0] {
data
0.615 \\
0.77 \\
0.67 \\
0.51 \\
0.565 \\
0.59 \\
0.495 \\
0.575 \\
0.87 \\
0.62 \\
0.695 \\
0.62 \\
0.755 \\
0.735 \\
0.75 \\
0.855 \\
0.725 \\
0.55 \\
0.62 \\
0.65 \\
0.765 \\
0.605 \\
0.63 \\
0.735 \\
0.59 \\
0.515 \\
0.74 \\
0.58 \\
0.715 \\
0.65 \\
0.6 \\
0.605 \\
0.64 \\
0.725 \\
0.675 \\
0.515 \\
0.75 \\
0.765 \\
0.615 \\
0.615 \\
0.655 \\
0.79 \\
0.585 \\
0.7 \\
0.585 \\
0.695 \\
0.635 \\
0.635 \\
0.72 \\
0.555 \\
};

\addplot[mark=*, boxplot, boxplot/draw position=5]
table[row sep=\\, y index=0] {
data
0.61 \\
0.555 \\
0.49 \\
0.66 \\
0.655 \\
0.63 \\
0.635 \\
0.555 \\
0.655 \\
0.665 \\
0.555 \\
0.665 \\
0.8 \\
0.525 \\
0.49 \\
0.69 \\
0.73 \\
0.69 \\
0.745 \\
0.68 \\
0.615 \\
0.64 \\
0.575 \\
0.865 \\
0.73 \\
0.635 \\
0.605 \\
0.545 \\
0.77 \\
0.72 \\
0.545 \\
0.525 \\
0.66 \\
0.59 \\
0.81 \\
0.575 \\
0.795 \\
0.605 \\
0.67 \\
0.675 \\
0.51 \\
0.535 \\
0.65 \\
0.615 \\
0.58 \\
0.74 \\
0.65 \\
0.515 \\
0.53 \\
0.73 \\
};

\addplot[mark=*, boxplot, boxplot/draw position=6]
table[row sep=\\, y index=0] {
data
0.6 \\
0.59 \\
0.49 \\
0.65 \\
0.895 \\
0.635 \\
0.665 \\
0.555 \\
0.695 \\
0.665 \\
0.57 \\
0.675 \\
0.775 \\
0.555 \\
0.59 \\
0.7 \\
0.755 \\
0.71 \\
0.76 \\
0.685 \\
0.66 \\
0.655 \\
0.605 \\
0.87 \\
0.795 \\
0.685 \\
0.595 \\
0.68 \\
0.77 \\
0.71 \\
0.565 \\
0.53 \\
0.705 \\
0.625 \\
0.795 \\
0.665 \\
0.795 \\
0.6 \\
0.62 \\
0.675 \\
0.505 \\
0.525 \\
0.74 \\
0.68 \\
0.47 \\
0.75 \\
0.645 \\
0.68 \\
0.505 \\
0.72 \\
};
}{0.1}{Output connectivity}{15}

        }
    }
    \resizebox{\textwidth}{!}{
        \subfloat[N=70]{
            \myboxplot{

\addplot[mark=*, boxplot, boxplot/draw position=1]
table[row sep=\\, y index=0] {
data
0.625 \\
0.545 \\
0.495 \\
0.485 \\
0.615 \\
0.535 \\
0.525 \\
0.495 \\
0.585 \\
0.565 \\
0.535 \\
0.565 \\
0.53 \\
0.63 \\
0.565 \\
0.54 \\
0.485 \\
0.465 \\
0.54 \\
0.5 \\
0.485 \\
0.63 \\
0.545 \\
0.54 \\
0.53 \\
0.55 \\
0.5 \\
0.56 \\
0.465 \\
0.54 \\
0.575 \\
0.505 \\
0.53 \\
0.505 \\
0.54 \\
0.515 \\
0.585 \\
0.525 \\
0.455 \\
0.525 \\
0.585 \\
0.6 \\
0.57 \\
0.635 \\
0.57 \\
0.675 \\
0.49 \\
0.525 \\
0.585 \\
0.525 \\
};

\addplot[mark=*, boxplot, boxplot/draw position=2]
table[row sep=\\, y index=0] {
data
0.645 \\
0.575 \\
0.605 \\
0.63 \\
0.585 \\
0.77 \\
0.62 \\
0.615 \\
0.565 \\
0.555 \\
0.565 \\
0.545 \\
0.575 \\
0.595 \\
0.63 \\
0.575 \\
0.595 \\
0.615 \\
0.545 \\
0.45 \\
0.54 \\
0.57 \\
0.465 \\
0.62 \\
0.485 \\
0.595 \\
0.535 \\
0.595 \\
0.565 \\
0.58 \\
0.57 \\
0.635 \\
0.58 \\
0.595 \\
0.58 \\
0.565 \\
0.665 \\
0.55 \\
0.525 \\
0.725 \\
0.485 \\
0.625 \\
0.53 \\
0.57 \\
0.63 \\
0.54 \\
0.62 \\
0.615 \\
0.565 \\
0.49 \\
};

\addplot[mark=*, boxplot, boxplot/draw position=3]
table[row sep=\\, y index=0] {
data
0.58 \\
0.56 \\
0.445 \\
0.55 \\
0.63 \\
0.645 \\
0.605 \\
0.475 \\
0.64 \\
0.46 \\
0.655 \\
0.54 \\
0.56 \\
0.645 \\
0.74 \\
0.555 \\
0.61 \\
0.575 \\
0.545 \\
0.705 \\
0.755 \\
0.71 \\
0.57 \\
0.485 \\
0.56 \\
0.665 \\
0.62 \\
0.645 \\
0.61 \\
0.54 \\
0.69 \\
0.615 \\
0.665 \\
0.595 \\
0.615 \\
0.64 \\
0.61 \\
0.73 \\
0.645 \\
0.515 \\
0.61 \\
0.585 \\
0.605 \\
0.615 \\
0.615 \\
0.55 \\
0.855 \\
0.525 \\
0.56 \\
0.585 \\
};

\addplot[mark=*, boxplot, boxplot/draw position=4]
table[row sep=\\, y index=0] {
data
0.88 \\
0.565 \\
0.615 \\
0.6 \\
0.535 \\
0.745 \\
0.78 \\
0.63 \\
0.51 \\
0.625 \\
0.57 \\
0.62 \\
0.67 \\
0.665 \\
0.805 \\
0.53 \\
0.595 \\
0.65 \\
0.59 \\
0.525 \\
0.565 \\
0.595 \\
0.65 \\
0.67 \\
0.55 \\
0.665 \\
0.525 \\
0.67 \\
0.615 \\
0.615 \\
0.56 \\
0.69 \\
0.695 \\
0.63 \\
0.615 \\
0.625 \\
0.805 \\
0.565 \\
0.615 \\
0.715 \\
0.53 \\
0.7 \\
0.6 \\
0.45 \\
0.715 \\
0.635 \\
0.655 \\
0.66 \\
0.67 \\
0.43 \\
};

\addplot[mark=*, boxplot, boxplot/draw position=5]
table[row sep=\\, y index=0] {
data
0.8 \\
0.71 \\
0.555 \\
0.745 \\
0.605 \\
0.84 \\
0.755 \\
0.75 \\
0.72 \\
0.665 \\
0.62 \\
0.625 \\
0.565 \\
0.74 \\
0.75 \\
0.575 \\
0.56 \\
0.51 \\
0.715 \\
0.66 \\
0.675 \\
0.59 \\
0.705 \\
0.47 \\
0.71 \\
0.74 \\
0.615 \\
0.645 \\
0.465 \\
0.67 \\
0.645 \\
0.625 \\
0.62 \\
0.56 \\
0.665 \\
0.66 \\
0.65 \\
0.52 \\
0.6 \\
0.71 \\
0.66 \\
0.63 \\
0.645 \\
0.62 \\
0.655 \\
0.69 \\
0.66 \\
0.585 \\
0.58 \\
0.775 \\
};

\addplot[mark=*, boxplot, boxplot/draw position=6]
table[row sep=\\, y index=0] {
data
0.87 \\
0.7 \\
0.71 \\
0.75 \\
0.61 \\
0.85 \\
0.795 \\
0.75 \\
0.755 \\
0.695 \\
0.555 \\
0.635 \\
0.59 \\
0.745 \\
0.785 \\
0.575 \\
0.66 \\
0.555 \\
0.69 \\
0.66 \\
0.68 \\
0.55 \\
0.77 \\
0.47 \\
0.71 \\
0.745 \\
0.615 \\
0.61 \\
0.505 \\
0.7 \\
0.665 \\
0.795 \\
0.6 \\
0.555 \\
0.665 \\
0.77 \\
0.58 \\
0.53 \\
0.63 \\
0.86 \\
0.68 \\
0.615 \\
0.73 \\
0.6 \\
0.675 \\
0.745 \\
0.69 \\
0.67 \\
0.565 \\
0.765 \\
};

\addplot[mark=*, boxplot, boxplot/draw position=7]
table[row sep=\\, y index=0] {
data
0.75 \\
0.735 \\
0.565 \\
0.81 \\
0.705 \\
0.74 \\
0.7 \\
0.785 \\
0.56 \\
0.815 \\
0.815 \\
0.785 \\
0.765 \\
0.73 \\
0.815 \\
0.635 \\
0.64 \\
0.805 \\
0.665 \\
0.685 \\
0.72 \\
0.765 \\
0.585 \\
0.725 \\
0.635 \\
0.715 \\
0.74 \\
0.59 \\
0.78 \\
0.81 \\
0.515 \\
0.85 \\
0.67 \\
0.735 \\
0.68 \\
0.615 \\
0.8 \\
0.755 \\
0.55 \\
0.65 \\
0.69 \\
0.725 \\
0.75 \\
0.825 \\
0.665 \\
0.735 \\
0.59 \\
0.825 \\
0.805 \\
0.71 \\
};
}{0.1}{Output connectivity}{15}

        }
        \subfloat[N=80]{
            \myboxplot{

\addplot[mark=*, boxplot, boxplot/draw position=1]
table[row sep=\\, y index=0] {
data
0.565 \\
0.505 \\
0.51 \\
0.52 \\
0.53 \\
0.59 \\
0.55 \\
0.535 \\
0.525 \\
0.51 \\
0.575 \\
0.63 \\
0.58 \\
0.515 \\
0.575 \\
0.545 \\
0.465 \\
0.52 \\
0.495 \\
0.49 \\
0.6 \\
0.525 \\
0.495 \\
0.58 \\
0.57 \\
0.485 \\
0.57 \\
0.5 \\
0.645 \\
0.57 \\
0.435 \\
0.525 \\
0.62 \\
0.645 \\
0.6 \\
0.555 \\
0.58 \\
0.52 \\
0.565 \\
0.52 \\
0.51 \\
0.515 \\
0.51 \\
0.535 \\
0.525 \\
0.575 \\
0.56 \\
0.625 \\
0.545 \\
0.59 \\
};

\addplot[mark=*, boxplot, boxplot/draw position=2]
table[row sep=\\, y index=0] {
data
0.605 \\
0.5 \\
0.47 \\
0.735 \\
0.55 \\
0.64 \\
0.53 \\
0.57 \\
0.55 \\
0.685 \\
0.56 \\
0.58 \\
0.58 \\
0.56 \\
0.515 \\
0.61 \\
0.435 \\
0.655 \\
0.56 \\
0.55 \\
0.57 \\
0.69 \\
0.675 \\
0.575 \\
0.575 \\
0.51 \\
0.585 \\
0.465 \\
0.645 \\
0.67 \\
0.55 \\
0.47 \\
0.635 \\
0.605 \\
0.6 \\
0.61 \\
0.68 \\
0.525 \\
0.605 \\
0.55 \\
0.615 \\
0.585 \\
0.55 \\
0.57 \\
0.555 \\
0.63 \\
0.61 \\
0.685 \\
0.56 \\
0.605 \\
};

\addplot[mark=*, boxplot, boxplot/draw position=3]
table[row sep=\\, y index=0] {
data
0.62 \\
0.535 \\
0.665 \\
0.685 \\
0.56 \\
0.645 \\
0.65 \\
0.615 \\
0.57 \\
0.54 \\
0.65 \\
0.53 \\
0.605 \\
0.7 \\
0.58 \\
0.555 \\
0.58 \\
0.74 \\
0.56 \\
0.47 \\
0.6 \\
0.6 \\
0.575 \\
0.585 \\
0.68 \\
0.57 \\
0.515 \\
0.735 \\
0.535 \\
0.62 \\
0.665 \\
0.7 \\
0.6 \\
0.535 \\
0.66 \\
0.5 \\
0.565 \\
0.655 \\
0.575 \\
0.575 \\
0.62 \\
0.59 \\
0.46 \\
0.63 \\
0.555 \\
0.72 \\
0.56 \\
0.59 \\
0.555 \\
0.555 \\
};

\addplot[mark=*, boxplot, boxplot/draw position=4]
table[row sep=\\, y index=0] {
data
0.72 \\
0.68 \\
0.71 \\
0.6 \\
0.6 \\
0.625 \\
0.62 \\
0.64 \\
0.7 \\
0.545 \\
0.555 \\
0.625 \\
0.635 \\
0.71 \\
0.675 \\
0.625 \\
0.68 \\
0.52 \\
0.535 \\
0.6 \\
0.6 \\
0.655 \\
0.815 \\
0.575 \\
0.605 \\
0.785 \\
0.79 \\
0.575 \\
0.7 \\
0.68 \\
0.565 \\
0.59 \\
0.565 \\
0.655 \\
0.705 \\
0.615 \\
0.585 \\
0.67 \\
0.595 \\
0.555 \\
0.64 \\
0.55 \\
0.76 \\
0.825 \\
0.51 \\
0.56 \\
0.665 \\
0.53 \\
0.745 \\
0.71 \\
};

\addplot[mark=*, boxplot, boxplot/draw position=5]
table[row sep=\\, y index=0] {
data
0.74 \\
0.61 \\
0.765 \\
0.62 \\
0.76 \\
0.67 \\
0.6 \\
0.74 \\
0.74 \\
0.59 \\
0.57 \\
0.615 \\
0.67 \\
0.755 \\
0.685 \\
0.63 \\
0.67 \\
0.51 \\
0.53 \\
0.64 \\
0.61 \\
0.62 \\
0.83 \\
0.635 \\
0.59 \\
0.795 \\
0.87 \\
0.645 \\
0.725 \\
0.69 \\
0.615 \\
0.61 \\
0.565 \\
0.65 \\
0.695 \\
0.645 \\
0.605 \\
0.725 \\
0.62 \\
0.625 \\
0.665 \\
0.62 \\
0.795 \\
0.835 \\
0.465 \\
0.635 \\
0.615 \\
0.56 \\
0.785 \\
0.845 \\
};

\addplot[mark=*, boxplot, boxplot/draw position=6]
table[row sep=\\, y index=0] {
data
0.64 \\
0.635 \\
0.725 \\
0.725 \\
0.76 \\
0.78 \\
0.725 \\
0.86 \\
0.605 \\
0.74 \\
0.495 \\
0.775 \\
0.725 \\
0.55 \\
0.64 \\
0.625 \\
0.645 \\
0.77 \\
0.685 \\
0.71 \\
0.65 \\
0.675 \\
0.735 \\
0.65 \\
0.625 \\
0.545 \\
0.72 \\
0.615 \\
0.605 \\
0.65 \\
0.595 \\
0.695 \\
0.715 \\
0.63 \\
0.825 \\
0.8 \\
0.635 \\
0.81 \\
0.705 \\
0.765 \\
0.77 \\
0.705 \\
0.735 \\
0.78 \\
0.74 \\
0.59 \\
0.585 \\
0.745 \\
0.845 \\
0.59 \\
};

\addplot[mark=*, boxplot, boxplot/draw position=7]
table[row sep=\\, y index=0] {
data
0.8 \\
0.645 \\
0.5 \\
0.78 \\
0.585 \\
0.635 \\
0.81 \\
0.845 \\
0.7 \\
0.755 \\
0.78 \\
0.605 \\
0.755 \\
0.79 \\
0.825 \\
0.835 \\
0.665 \\
0.73 \\
0.625 \\
0.91 \\
0.61 \\
0.695 \\
0.66 \\
0.595 \\
0.575 \\
0.725 \\
0.66 \\
0.54 \\
0.76 \\
0.835 \\
0.585 \\
0.875 \\
0.795 \\
0.71 \\
0.555 \\
0.675 \\
0.65 \\
0.825 \\
0.585 \\
0.84 \\
0.805 \\
0.695 \\
0.715 \\
0.75 \\
0.81 \\
0.635 \\
0.715 \\
0.535 \\
0.525 \\
0.68 \\
};

\addplot[mark=*, boxplot, boxplot/draw position=8]
table[row sep=\\, y index=0] {
data
0.845 \\
0.685 \\
0.725 \\
0.785 \\
0.845 \\
0.805 \\
0.805 \\
0.89 \\
0.58 \\
0.77 \\
0.475 \\
0.78 \\
0.94 \\
0.6 \\
0.67 \\
0.62 \\
0.74 \\
0.775 \\
0.64 \\
0.735 \\
0.655 \\
0.72 \\
0.805 \\
0.675 \\
0.66 \\
0.64 \\
0.735 \\
0.67 \\
0.635 \\
0.675 \\
0.665 \\
0.745 \\
0.745 \\
0.685 \\
0.9 \\
0.775 \\
0.725 \\
0.84 \\
0.745 \\
0.845 \\
0.795 \\
0.75 \\
0.76 \\
0.84 \\
0.795 \\
0.655 \\
0.72 \\
0.8 \\
0.845 \\
0.77 \\
};
}{0.1}{Output connectivity}{15}

        }
    }
    \caption{Plots of output connectivity against accuracy on TP5 - Part 1 of 2}
\end{figure*}

\begin{figure*}[ht]
    \centering
    \resizebox{\textwidth}{!}{
        \subfloat[N=90]{
            \myboxplot{

\addplot[mark=*, boxplot, boxplot/draw position=1]
table[row sep=\\, y index=0] {
data
0.545 \\
0.53 \\
0.6 \\
0.52 \\
0.5 \\
0.495 \\
0.58 \\
0.565 \\
0.575 \\
0.575 \\
0.49 \\
0.59 \\
0.59 \\
0.555 \\
0.605 \\
0.505 \\
0.61 \\
0.52 \\
0.54 \\
0.54 \\
0.58 \\
0.56 \\
0.57 \\
0.51 \\
0.595 \\
0.48 \\
0.505 \\
0.615 \\
0.51 \\
0.535 \\
0.55 \\
0.515 \\
0.53 \\
0.58 \\
0.51 \\
0.495 \\
0.54 \\
0.55 \\
0.545 \\
0.62 \\
0.525 \\
0.605 \\
0.62 \\
0.535 \\
0.555 \\
0.525 \\
0.44 \\
0.585 \\
0.54 \\
0.53 \\
};

\addplot[mark=*, boxplot, boxplot/draw position=2]
table[row sep=\\, y index=0] {
data
0.615 \\
0.51 \\
0.575 \\
0.57 \\
0.525 \\
0.615 \\
0.62 \\
0.51 \\
0.555 \\
0.59 \\
0.63 \\
0.635 \\
0.475 \\
0.63 \\
0.52 \\
0.62 \\
0.565 \\
0.54 \\
0.54 \\
0.56 \\
0.605 \\
0.62 \\
0.5 \\
0.605 \\
0.525 \\
0.59 \\
0.72 \\
0.525 \\
0.515 \\
0.54 \\
0.63 \\
0.53 \\
0.58 \\
0.605 \\
0.535 \\
0.57 \\
0.54 \\
0.46 \\
0.545 \\
0.485 \\
0.53 \\
0.51 \\
0.605 \\
0.53 \\
0.595 \\
0.515 \\
0.52 \\
0.525 \\
0.575 \\
0.565 \\
};

\addplot[mark=*, boxplot, boxplot/draw position=3]
table[row sep=\\, y index=0] {
data
0.66 \\
0.49 \\
0.635 \\
0.625 \\
0.61 \\
0.64 \\
0.635 \\
0.54 \\
0.54 \\
0.565 \\
0.55 \\
0.625 \\
0.52 \\
0.675 \\
0.525 \\
0.63 \\
0.5 \\
0.545 \\
0.56 \\
0.485 \\
0.635 \\
0.675 \\
0.545 \\
0.59 \\
0.47 \\
0.72 \\
0.72 \\
0.53 \\
0.495 \\
0.67 \\
0.57 \\
0.565 \\
0.69 \\
0.605 \\
0.575 \\
0.68 \\
0.65 \\
0.435 \\
0.575 \\
0.545 \\
0.59 \\
0.505 \\
0.635 \\
0.53 \\
0.695 \\
0.51 \\
0.575 \\
0.545 \\
0.56 \\
0.79 \\
};

\addplot[mark=*, boxplot, boxplot/draw position=4]
table[row sep=\\, y index=0] {
data
0.72 \\
0.585 \\
0.585 \\
0.57 \\
0.555 \\
0.615 \\
0.57 \\
0.535 \\
0.58 \\
0.695 \\
0.505 \\
0.6 \\
0.6 \\
0.585 \\
0.54 \\
0.615 \\
0.565 \\
0.775 \\
0.74 \\
0.69 \\
0.765 \\
0.605 \\
0.655 \\
0.46 \\
0.61 \\
0.595 \\
0.61 \\
0.625 \\
0.615 \\
0.7 \\
0.565 \\
0.59 \\
0.55 \\
0.68 \\
0.765 \\
0.66 \\
0.69 \\
0.57 \\
0.615 \\
0.59 \\
0.57 \\
0.62 \\
0.64 \\
0.6 \\
0.755 \\
0.72 \\
0.62 \\
0.525 \\
0.81 \\
0.655 \\
};

\addplot[mark=*, boxplot, boxplot/draw position=5]
table[row sep=\\, y index=0] {
data
0.81 \\
0.61 \\
0.615 \\
0.575 \\
0.59 \\
0.635 \\
0.595 \\
0.73 \\
0.57 \\
0.71 \\
0.555 \\
0.59 \\
0.635 \\
0.595 \\
0.555 \\
0.595 \\
0.615 \\
0.815 \\
0.755 \\
0.67 \\
0.795 \\
0.675 \\
0.645 \\
0.525 \\
0.68 \\
0.535 \\
0.625 \\
0.7 \\
0.6 \\
0.675 \\
0.59 \\
0.62 \\
0.56 \\
0.75 \\
0.815 \\
0.655 \\
0.805 \\
0.605 \\
0.58 \\
0.595 \\
0.62 \\
0.665 \\
0.63 \\
0.6 \\
0.75 \\
0.73 \\
0.645 \\
0.525 \\
0.845 \\
0.65 \\
};

\addplot[mark=*, boxplot, boxplot/draw position=6]
table[row sep=\\, y index=0] {
data
0.8 \\
0.72 \\
0.595 \\
0.67 \\
0.6 \\
0.86 \\
0.645 \\
0.75 \\
0.565 \\
0.61 \\
0.67 \\
0.615 \\
0.68 \\
0.765 \\
0.67 \\
0.665 \\
0.755 \\
0.745 \\
0.865 \\
0.59 \\
0.855 \\
0.945 \\
0.62 \\
0.665 \\
0.61 \\
0.635 \\
0.865 \\
0.655 \\
0.53 \\
0.6 \\
0.615 \\
0.485 \\
0.68 \\
0.54 \\
0.54 \\
0.66 \\
0.68 \\
0.825 \\
0.65 \\
0.795 \\
0.68 \\
0.735 \\
0.745 \\
0.655 \\
0.845 \\
0.67 \\
0.55 \\
0.655 \\
0.67 \\
0.77 \\
};

\addplot[mark=*, boxplot, boxplot/draw position=7]
table[row sep=\\, y index=0] {
data
0.74 \\
0.825 \\
0.67 \\
0.49 \\
0.545 \\
0.75 \\
0.615 \\
0.82 \\
0.675 \\
0.785 \\
0.555 \\
0.67 \\
0.565 \\
0.93 \\
0.66 \\
0.565 \\
0.66 \\
0.505 \\
0.65 \\
0.56 \\
0.815 \\
0.815 \\
0.55 \\
0.79 \\
0.93 \\
0.64 \\
0.725 \\
0.69 \\
0.785 \\
0.65 \\
0.66 \\
0.745 \\
0.68 \\
0.645 \\
0.545 \\
0.725 \\
0.745 \\
0.575 \\
0.8 \\
0.66 \\
0.86 \\
0.74 \\
0.695 \\
0.73 \\
0.81 \\
0.85 \\
0.64 \\
0.615 \\
0.74 \\
0.685 \\
};

\addplot[mark=*, boxplot, boxplot/draw position=8]
table[row sep=\\, y index=0] {
data
0.84 \\
0.825 \\
0.83 \\
0.71 \\
0.735 \\
0.66 \\
0.845 \\
0.79 \\
0.685 \\
0.795 \\
0.91 \\
0.76 \\
0.735 \\
0.845 \\
0.725 \\
0.735 \\
0.615 \\
0.74 \\
0.75 \\
0.815 \\
0.685 \\
0.81 \\
0.75 \\
0.72 \\
0.74 \\
0.65 \\
0.815 \\
0.86 \\
0.705 \\
0.675 \\
0.635 \\
0.705 \\
0.565 \\
0.79 \\
0.86 \\
0.63 \\
0.565 \\
0.76 \\
0.515 \\
0.625 \\
0.545 \\
0.835 \\
0.735 \\
0.7 \\
0.775 \\
0.7 \\
0.625 \\
0.885 \\
0.635 \\
0.515 \\
};

\addplot[mark=*, boxplot, boxplot/draw position=9]
table[row sep=\\, y index=0] {
data
0.82 \\
0.81 \\
0.83 \\
0.78 \\
0.765 \\
0.655 \\
0.845 \\
0.79 \\
0.68 \\
0.805 \\
0.9 \\
0.755 \\
0.765 \\
0.845 \\
0.74 \\
0.735 \\
0.765 \\
0.775 \\
0.855 \\
0.815 \\
0.665 \\
0.81 \\
0.75 \\
0.745 \\
0.895 \\
0.61 \\
0.84 \\
0.835 \\
0.71 \\
0.685 \\
0.66 \\
0.76 \\
0.68 \\
0.785 \\
0.85 \\
0.645 \\
0.61 \\
0.77 \\
0.595 \\
0.595 \\
0.565 \\
0.865 \\
0.72 \\
0.645 \\
0.77 \\
0.73 \\
0.615 \\
0.885 \\
0.67 \\
0.515 \\
};
}{0.1}{Reservoir subset size}{15}

        }
        \subfloat[N=100]{
            \myboxplot{

\addplot[mark=*, boxplot, boxplot/draw position=1]
table[row sep=\\, y index=0] {
data
0.605 \\
0.57 \\
0.54 \\
0.595 \\
0.575 \\
0.53 \\
0.555 \\
0.61 \\
0.55 \\
0.51 \\
0.525 \\
0.505 \\
0.565 \\
0.595 \\
0.605 \\
0.59 \\
0.555 \\
0.53 \\
0.525 \\
0.55 \\
0.57 \\
0.69 \\
0.535 \\
0.605 \\
0.595 \\
0.545 \\
0.5 \\
0.505 \\
0.66 \\
0.495 \\
0.59 \\
0.52 \\
0.55 \\
0.585 \\
0.51 \\
0.655 \\
0.53 \\
0.61 \\
0.56 \\
0.44 \\
0.525 \\
0.555 \\
0.565 \\
0.56 \\
0.52 \\
0.56 \\
0.705 \\
0.495 \\
0.545 \\
0.63 \\
};

\addplot[mark=*, boxplot, boxplot/draw position=2]
table[row sep=\\, y index=0] {
data
0.595 \\
0.57 \\
0.6 \\
0.595 \\
0.545 \\
0.525 \\
0.625 \\
0.635 \\
0.645 \\
0.575 \\
0.56 \\
0.485 \\
0.62 \\
0.615 \\
0.6 \\
0.625 \\
0.5 \\
0.565 \\
0.52 \\
0.545 \\
0.615 \\
0.725 \\
0.605 \\
0.59 \\
0.675 \\
0.535 \\
0.6 \\
0.52 \\
0.67 \\
0.56 \\
0.595 \\
0.61 \\
0.665 \\
0.625 \\
0.595 \\
0.675 \\
0.525 \\
0.555 \\
0.635 \\
0.515 \\
0.52 \\
0.58 \\
0.575 \\
0.625 \\
0.665 \\
0.61 \\
0.72 \\
0.585 \\
0.61 \\
0.575 \\
};

\addplot[mark=*, boxplot, boxplot/draw position=3]
table[row sep=\\, y index=0] {
data
0.645 \\
0.52 \\
0.52 \\
0.535 \\
0.615 \\
0.585 \\
0.615 \\
0.635 \\
0.655 \\
0.665 \\
0.605 \\
0.485 \\
0.63 \\
0.565 \\
0.615 \\
0.695 \\
0.49 \\
0.545 \\
0.555 \\
0.525 \\
0.67 \\
0.7 \\
0.56 \\
0.69 \\
0.54 \\
0.635 \\
0.54 \\
0.625 \\
0.585 \\
0.56 \\
0.55 \\
0.68 \\
0.625 \\
0.535 \\
0.65 \\
0.72 \\
0.615 \\
0.535 \\
0.565 \\
0.625 \\
0.625 \\
0.685 \\
0.59 \\
0.575 \\
0.56 \\
0.57 \\
0.495 \\
0.46 \\
0.65 \\
0.705 \\
};

\addplot[mark=*, boxplot, boxplot/draw position=4]
table[row sep=\\, y index=0] {
data
0.78 \\
0.55 \\
0.545 \\
0.715 \\
0.555 \\
0.765 \\
0.695 \\
0.585 \\
0.68 \\
0.6 \\
0.66 \\
0.64 \\
0.695 \\
0.64 \\
0.635 \\
0.675 \\
0.63 \\
0.725 \\
0.61 \\
0.74 \\
0.635 \\
0.535 \\
0.665 \\
0.61 \\
0.64 \\
0.655 \\
0.665 \\
0.665 \\
0.63 \\
0.675 \\
0.685 \\
0.745 \\
0.635 \\
0.705 \\
0.64 \\
0.73 \\
0.655 \\
0.64 \\
0.595 \\
0.655 \\
0.62 \\
0.525 \\
0.66 \\
0.585 \\
0.79 \\
0.695 \\
0.545 \\
0.655 \\
0.52 \\
0.545 \\
};

\addplot[mark=*, boxplot, boxplot/draw position=5]
table[row sep=\\, y index=0] {
data
0.65 \\
0.605 \\
0.715 \\
0.625 \\
0.725 \\
0.65 \\
0.63 \\
0.655 \\
0.625 \\
0.7 \\
0.655 \\
0.66 \\
0.705 \\
0.58 \\
0.605 \\
0.62 \\
0.535 \\
0.695 \\
0.775 \\
0.625 \\
0.615 \\
0.62 \\
0.7 \\
0.61 \\
0.775 \\
0.63 \\
0.72 \\
0.665 \\
0.745 \\
0.855 \\
0.805 \\
0.54 \\
0.695 \\
0.605 \\
0.645 \\
0.7 \\
0.59 \\
0.7 \\
0.675 \\
0.555 \\
0.66 \\
0.73 \\
0.785 \\
0.665 \\
0.685 \\
0.71 \\
0.74 \\
0.565 \\
0.675 \\
0.565 \\
};

\addplot[mark=*, boxplot, boxplot/draw position=6]
table[row sep=\\, y index=0] {
data
0.63 \\
0.625 \\
0.725 \\
0.675 \\
0.75 \\
0.675 \\
0.685 \\
0.665 \\
0.64 \\
0.64 \\
0.81 \\
0.72 \\
0.72 \\
0.615 \\
0.6 \\
0.735 \\
0.67 \\
0.685 \\
0.735 \\
0.645 \\
0.66 \\
0.65 \\
0.725 \\
0.705 \\
0.78 \\
0.715 \\
0.72 \\
0.685 \\
0.77 \\
0.865 \\
0.8 \\
0.575 \\
0.675 \\
0.605 \\
0.685 \\
0.81 \\
0.595 \\
0.735 \\
0.63 \\
0.53 \\
0.65 \\
0.755 \\
0.845 \\
0.645 \\
0.685 \\
0.74 \\
0.73 \\
0.515 \\
0.645 \\
0.58 \\
};

\addplot[mark=*, boxplot, boxplot/draw position=7]
table[row sep=\\, y index=0] {
data
0.66 \\
0.835 \\
0.755 \\
0.7 \\
0.575 \\
0.695 \\
0.575 \\
0.835 \\
0.865 \\
0.75 \\
0.74 \\
0.545 \\
0.715 \\
0.81 \\
0.795 \\
0.675 \\
0.645 \\
0.505 \\
0.87 \\
0.985 \\
0.65 \\
0.71 \\
0.68 \\
0.77 \\
0.82 \\
0.76 \\
0.585 \\
0.55 \\
0.59 \\
0.635 \\
0.79 \\
0.575 \\
0.64 \\
0.565 \\
0.635 \\
0.695 \\
0.71 \\
0.72 \\
0.725 \\
0.75 \\
0.78 \\
0.575 \\
0.705 \\
0.66 \\
0.845 \\
0.585 \\
0.74 \\
0.78 \\
0.765 \\
0.78 \\
};

\addplot[mark=*, boxplot, boxplot/draw position=8]
table[row sep=\\, y index=0] {
data
0.71 \\
0.84 \\
0.74 \\
0.72 \\
0.575 \\
0.74 \\
0.57 \\
0.835 \\
0.88 \\
0.745 \\
0.785 \\
0.605 \\
0.76 \\
0.815 \\
0.785 \\
0.715 \\
0.645 \\
0.51 \\
0.865 \\
0.99 \\
0.705 \\
0.735 \\
0.72 \\
0.785 \\
0.83 \\
0.835 \\
0.605 \\
0.605 \\
0.605 \\
0.69 \\
0.79 \\
0.655 \\
0.635 \\
0.565 \\
0.65 \\
0.67 \\
0.705 \\
0.705 \\
0.725 \\
0.77 \\
0.8 \\
0.605 \\
0.73 \\
0.735 \\
0.83 \\
0.6 \\
0.76 \\
0.855 \\
0.765 \\
0.77 \\
};

\addplot[mark=*, boxplot, boxplot/draw position=9]
table[row sep=\\, y index=0] {
data
0.825 \\
0.765 \\
0.745 \\
0.655 \\
0.52 \\
0.85 \\
0.99 \\
0.73 \\
0.75 \\
0.7 \\
0.81 \\
0.85 \\
0.83 \\
0.655 \\
0.605 \\
0.61 \\
0.675 \\
0.81 \\
0.635 \\
0.67 \\
0.565 \\
0.625 \\
0.75 \\
0.73 \\
0.705 \\
0.735 \\
0.775 \\
0.805 \\
0.605 \\
0.7 \\
0.715 \\
0.825 \\
0.58 \\
0.825 \\
0.9 \\
0.765 \\
0.81 \\
0.8 \\
0.78 \\
0.855 \\
0.725 \\
0.8 \\
0.84 \\
0.665 \\
0.645 \\
0.87 \\
0.925 \\
0.845 \\
0.735 \\
0.705 \\
};

\addplot[mark=*, boxplot, boxplot/draw position=10]
table[row sep=\\, y index=0] {
data
0.725 \\
0.635 \\
0.815 \\
0.64 \\
0.88 \\
0.855 \\
0.67 \\
0.865 \\
0.625 \\
0.84 \\
0.825 \\
0.905 \\
0.85 \\
0.82 \\
0.85 \\
0.82 \\
0.7 \\
0.705 \\
0.655 \\
0.77 \\
0.905 \\
0.795 \\
0.84 \\
0.705 \\
0.715 \\
0.99 \\
0.75 \\
0.7 \\
0.755 \\
0.675 \\
0.825 \\
0.855 \\
0.72 \\
0.77 \\
0.765 \\
0.76 \\
0.65 \\
0.77 \\
0.755 \\
0.77 \\
0.795 \\
0.81 \\
0.675 \\
0.75 \\
0.885 \\
0.825 \\
0.81 \\
0.535 \\
0.72 \\
0.705 \\
};
}{0.1}{Reservoir subset size}{15}

        }
    }
    \resizebox{\textwidth}{!}{
        \subfloat[N=110]{
            \myboxplot{

\addplot[mark=*, boxplot, boxplot/draw position=1]
table[row sep=\\, y index=0] {
data
0.635 \\
0.51 \\
0.6 \\
0.495 \\
0.595 \\
0.49 \\
0.54 \\
0.585 \\
0.54 \\
0.48 \\
0.505 \\
0.65 \\
0.56 \\
0.585 \\
0.56 \\
0.62 \\
0.555 \\
0.52 \\
0.54 \\
0.56 \\
0.49 \\
0.61 \\
0.535 \\
0.505 \\
0.52 \\
0.475 \\
0.55 \\
0.525 \\
0.58 \\
0.61 \\
0.515 \\
0.565 \\
0.53 \\
0.55 \\
0.5 \\
0.505 \\
0.57 \\
0.585 \\
0.545 \\
0.465 \\
0.525 \\
0.535 \\
0.56 \\
0.505 \\
0.505 \\
0.51 \\
0.505 \\
0.54 \\
0.585 \\
0.55 \\
};

\addplot[mark=*, boxplot, boxplot/draw position=2]
table[row sep=\\, y index=0] {
data
0.655 \\
0.53 \\
0.615 \\
0.54 \\
0.615 \\
0.58 \\
0.565 \\
0.605 \\
0.565 \\
0.485 \\
0.585 \\
0.65 \\
0.55 \\
0.565 \\
0.52 \\
0.665 \\
0.54 \\
0.515 \\
0.67 \\
0.585 \\
0.49 \\
0.605 \\
0.66 \\
0.52 \\
0.555 \\
0.5 \\
0.6 \\
0.52 \\
0.68 \\
0.59 \\
0.515 \\
0.62 \\
0.515 \\
0.59 \\
0.6 \\
0.595 \\
0.61 \\
0.59 \\
0.64 \\
0.555 \\
0.55 \\
0.55 \\
0.575 \\
0.505 \\
0.515 \\
0.625 \\
0.565 \\
0.6 \\
0.685 \\
0.615 \\
};

\addplot[mark=*, boxplot, boxplot/draw position=3]
table[row sep=\\, y index=0] {
data
0.58 \\
0.505 \\
0.625 \\
0.6 \\
0.505 \\
0.725 \\
0.57 \\
0.755 \\
0.515 \\
0.5 \\
0.6 \\
0.6 \\
0.515 \\
0.765 \\
0.645 \\
0.6 \\
0.675 \\
0.545 \\
0.57 \\
0.64 \\
0.565 \\
0.785 \\
0.59 \\
0.705 \\
0.575 \\
0.525 \\
0.65 \\
0.57 \\
0.64 \\
0.4 \\
0.555 \\
0.705 \\
0.57 \\
0.535 \\
0.6 \\
0.59 \\
0.545 \\
0.56 \\
0.625 \\
0.595 \\
0.625 \\
0.745 \\
0.68 \\
0.675 \\
0.635 \\
0.565 \\
0.61 \\
0.615 \\
0.54 \\
0.58 \\
};

\addplot[mark=*, boxplot, boxplot/draw position=4]
table[row sep=\\, y index=0] {
data
0.645 \\
0.675 \\
0.56 \\
0.725 \\
0.545 \\
0.755 \\
0.665 \\
0.75 \\
0.515 \\
0.5 \\
0.52 \\
0.63 \\
0.585 \\
0.79 \\
0.635 \\
0.57 \\
0.795 \\
0.605 \\
0.775 \\
0.655 \\
0.59 \\
0.945 \\
0.595 \\
0.74 \\
0.585 \\
0.57 \\
0.715 \\
0.605 \\
0.625 \\
0.535 \\
0.575 \\
0.73 \\
0.575 \\
0.49 \\
0.605 \\
0.56 \\
0.575 \\
0.555 \\
0.65 \\
0.605 \\
0.665 \\
0.73 \\
0.705 \\
0.685 \\
0.665 \\
0.645 \\
0.51 \\
0.59 \\
0.62 \\
0.635 \\
};

\addplot[mark=*, boxplot, boxplot/draw position=5]
table[row sep=\\, y index=0] {
data
0.65 \\
0.61 \\
0.77 \\
0.81 \\
0.755 \\
0.665 \\
0.73 \\
0.67 \\
0.715 \\
0.775 \\
0.845 \\
0.505 \\
0.86 \\
0.685 \\
0.68 \\
0.635 \\
0.625 \\
0.645 \\
0.73 \\
0.685 \\
0.67 \\
0.745 \\
0.59 \\
0.76 \\
0.605 \\
0.7 \\
0.65 \\
0.61 \\
0.75 \\
0.605 \\
0.6 \\
0.66 \\
0.655 \\
0.775 \\
0.58 \\
0.615 \\
0.815 \\
0.71 \\
0.72 \\
0.77 \\
0.695 \\
0.8 \\
0.675 \\
0.88 \\
0.625 \\
0.58 \\
0.745 \\
0.625 \\
0.805 \\
0.795 \\
};

\addplot[mark=*, boxplot, boxplot/draw position=6]
table[row sep=\\, y index=0] {
data
0.66 \\
0.645 \\
0.685 \\
0.705 \\
0.625 \\
0.655 \\
0.515 \\
0.735 \\
0.73 \\
0.64 \\
0.845 \\
0.725 \\
0.72 \\
0.575 \\
0.72 \\
0.815 \\
0.655 \\
0.64 \\
0.765 \\
0.725 \\
0.705 \\
0.705 \\
0.72 \\
0.7 \\
0.64 \\
0.72 \\
0.585 \\
0.735 \\
0.9 \\
0.73 \\
0.675 \\
0.7 \\
0.56 \\
0.67 \\
0.715 \\
0.655 \\
0.685 \\
0.725 \\
0.58 \\
0.705 \\
0.67 \\
0.885 \\
0.675 \\
0.795 \\
0.725 \\
0.775 \\
0.61 \\
0.64 \\
0.525 \\
0.625 \\
};

\addplot[mark=*, boxplot, boxplot/draw position=7]
table[row sep=\\, y index=0] {
data
0.66 \\
0.655 \\
0.73 \\
0.695 \\
0.63 \\
0.775 \\
0.78 \\
0.485 \\
0.54 \\
0.645 \\
0.69 \\
0.85 \\
0.815 \\
0.79 \\
0.76 \\
0.77 \\
0.74 \\
0.785 \\
0.66 \\
0.895 \\
0.835 \\
0.63 \\
0.79 \\
0.755 \\
0.795 \\
0.52 \\
0.79 \\
0.64 \\
0.515 \\
0.625 \\
0.64 \\
0.7 \\
0.645 \\
0.61 \\
0.71 \\
0.71 \\
0.825 \\
0.725 \\
0.595 \\
0.82 \\
0.76 \\
0.835 \\
0.895 \\
0.605 \\
0.61 \\
0.65 \\
0.715 \\
0.66 \\
0.81 \\
0.845 \\
};

\addplot[mark=*, boxplot, boxplot/draw position=8]
table[row sep=\\, y index=0] {
data
0.67 \\
0.705 \\
0.75 \\
0.71 \\
0.595 \\
0.835 \\
0.79 \\
0.55 \\
0.595 \\
0.71 \\
0.8 \\
0.905 \\
0.81 \\
0.815 \\
0.73 \\
0.79 \\
0.75 \\
0.81 \\
0.66 \\
0.895 \\
0.845 \\
0.645 \\
0.78 \\
0.855 \\
0.825 \\
0.565 \\
0.835 \\
0.69 \\
0.525 \\
0.575 \\
0.64 \\
0.7 \\
0.7 \\
0.62 \\
0.715 \\
0.74 \\
0.84 \\
0.715 \\
0.6 \\
0.835 \\
0.75 \\
0.825 \\
0.895 \\
0.805 \\
0.65 \\
0.675 \\
0.8 \\
0.685 \\
0.835 \\
0.85 \\
};

\addplot[mark=*, boxplot, boxplot/draw position=9]
table[row sep=\\, y index=0] {
data
0.805 \\
0.665 \\
0.795 \\
0.86 \\
0.885 \\
0.81 \\
0.76 \\
0.595 \\
0.835 \\
0.61 \\
0.835 \\
0.725 \\
0.71 \\
0.545 \\
0.695 \\
0.795 \\
0.58 \\
0.765 \\
0.745 \\
0.68 \\
0.56 \\
0.68 \\
0.775 \\
0.795 \\
0.64 \\
0.57 \\
0.76 \\
0.725 \\
0.81 \\
0.655 \\
0.865 \\
0.595 \\
0.615 \\
0.635 \\
0.735 \\
0.745 \\
0.88 \\
0.89 \\
0.87 \\
0.99 \\
0.805 \\
0.925 \\
0.84 \\
0.775 \\
0.64 \\
0.605 \\
0.915 \\
0.725 \\
0.79 \\
0.755 \\
};

\addplot[mark=*, boxplot, boxplot/draw position=10]
table[row sep=\\, y index=0] {
data
0.785 \\
0.725 \\
0.61 \\
0.88 \\
0.82 \\
0.64 \\
0.805 \\
0.665 \\
0.73 \\
0.92 \\
0.805 \\
0.92 \\
0.84 \\
0.695 \\
0.715 \\
0.875 \\
0.725 \\
0.73 \\
0.595 \\
0.81 \\
0.86 \\
0.82 \\
0.79 \\
0.775 \\
0.775 \\
0.79 \\
0.885 \\
0.83 \\
0.72 \\
0.755 \\
0.765 \\
0.71 \\
0.885 \\
0.675 \\
0.72 \\
0.705 \\
0.705 \\
0.885 \\
0.9 \\
0.65 \\
0.72 \\
0.685 \\
0.775 \\
0.68 \\
0.72 \\
0.685 \\
0.91 \\
0.85 \\
0.83 \\
0.73 \\
};

\addplot[mark=*, boxplot, boxplot/draw position=11]
table[row sep=\\, y index=0] {
data
0.79 \\
0.63 \\
0.855 \\
0.86 \\
0.915 \\
0.855 \\
0.85 \\
0.635 \\
0.865 \\
0.62 \\
0.85 \\
0.755 \\
0.79 \\
0.605 \\
0.78 \\
0.85 \\
0.59 \\
0.8 \\
0.77 \\
0.75 \\
0.56 \\
0.715 \\
0.78 \\
0.83 \\
0.695 \\
0.56 \\
0.795 \\
0.735 \\
0.88 \\
0.75 \\
0.895 \\
0.745 \\
0.62 \\
0.745 \\
0.775 \\
0.875 \\
0.89 \\
0.92 \\
0.87 \\
0.99 \\
0.83 \\
0.935 \\
0.88 \\
0.72 \\
0.71 \\
0.695 \\
0.92 \\
0.79 \\
0.815 \\
0.78 \\
};
}{0.1}{Reservoir subset size}{15}

        }
        \subfloat[N=120]{
            \myboxplot{

\addplot[mark=*, boxplot, boxplot/draw position=1]
table[row sep=\\, y index=0] {
data
0.66 \\
0.51 \\
0.55 \\
0.545 \\
0.525 \\
0.515 \\
0.58 \\
0.575 \\
0.54 \\
0.595 \\
0.52 \\
0.595 \\
0.525 \\
0.55 \\
0.575 \\
0.6 \\
0.59 \\
0.515 \\
0.56 \\
0.675 \\
0.51 \\
0.52 \\
0.5 \\
0.615 \\
0.615 \\
0.515 \\
0.525 \\
0.51 \\
0.585 \\
0.505 \\
0.495 \\
0.575 \\
0.6 \\
0.505 \\
0.46 \\
0.59 \\
0.545 \\
0.55 \\
0.535 \\
0.575 \\
0.605 \\
0.525 \\
0.52 \\
0.53 \\
0.63 \\
0.58 \\
0.545 \\
0.61 \\
0.51 \\
0.505 \\
};

\addplot[mark=*, boxplot, boxplot/draw position=2]
table[row sep=\\, y index=0] {
data
0.525 \\
0.58 \\
0.665 \\
0.52 \\
0.6 \\
0.585 \\
0.485 \\
0.555 \\
0.555 \\
0.6 \\
0.625 \\
0.62 \\
0.66 \\
0.6 \\
0.56 \\
0.545 \\
0.615 \\
0.565 \\
0.685 \\
0.73 \\
0.58 \\
0.595 \\
0.575 \\
0.655 \\
0.51 \\
0.56 \\
0.53 \\
0.725 \\
0.54 \\
0.625 \\
0.585 \\
0.575 \\
0.525 \\
0.67 \\
0.59 \\
0.675 \\
0.68 \\
0.57 \\
0.57 \\
0.545 \\
0.67 \\
0.55 \\
0.505 \\
0.66 \\
0.585 \\
0.52 \\
0.515 \\
0.595 \\
0.705 \\
0.535 \\
};

\addplot[mark=*, boxplot, boxplot/draw position=3]
table[row sep=\\, y index=0] {
data
0.6 \\
0.555 \\
0.66 \\
0.53 \\
0.59 \\
0.58 \\
0.49 \\
0.595 \\
0.64 \\
0.68 \\
0.67 \\
0.625 \\
0.71 \\
0.62 \\
0.595 \\
0.58 \\
0.55 \\
0.625 \\
0.67 \\
0.87 \\
0.55 \\
0.595 \\
0.51 \\
0.73 \\
0.565 \\
0.58 \\
0.55 \\
0.735 \\
0.53 \\
0.63 \\
0.62 \\
0.585 \\
0.525 \\
0.85 \\
0.53 \\
0.675 \\
0.74 \\
0.63 \\
0.575 \\
0.52 \\
0.72 \\
0.66 \\
0.53 \\
0.645 \\
0.545 \\
0.54 \\
0.575 \\
0.625 \\
0.775 \\
0.615 \\
};

\addplot[mark=*, boxplot, boxplot/draw position=4]
table[row sep=\\, y index=0] {
data
0.61 \\
0.6 \\
0.565 \\
0.72 \\
0.805 \\
0.64 \\
0.515 \\
0.59 \\
0.785 \\
0.595 \\
0.64 \\
0.635 \\
0.69 \\
0.705 \\
0.64 \\
0.635 \\
0.83 \\
0.605 \\
0.46 \\
0.765 \\
0.65 \\
0.74 \\
0.55 \\
0.645 \\
0.745 \\
0.595 \\
0.605 \\
0.595 \\
0.76 \\
0.59 \\
0.605 \\
0.59 \\
0.61 \\
0.655 \\
0.59 \\
0.715 \\
0.625 \\
0.59 \\
0.71 \\
0.48 \\
0.65 \\
0.535 \\
0.69 \\
0.57 \\
0.575 \\
0.685 \\
0.52 \\
0.665 \\
0.595 \\
0.59 \\
};

\addplot[mark=*, boxplot, boxplot/draw position=5]
table[row sep=\\, y index=0] {
data
0.88 \\
0.65 \\
0.5 \\
0.635 \\
0.615 \\
0.57 \\
0.805 \\
0.485 \\
0.725 \\
0.62 \\
0.66 \\
0.64 \\
0.605 \\
0.555 \\
0.765 \\
0.63 \\
0.835 \\
0.62 \\
0.775 \\
0.635 \\
0.65 \\
0.6 \\
0.64 \\
0.75 \\
0.73 \\
0.655 \\
0.78 \\
0.86 \\
0.75 \\
0.875 \\
0.72 \\
0.59 \\
0.64 \\
0.84 \\
0.75 \\
0.775 \\
0.83 \\
0.74 \\
0.66 \\
0.62 \\
0.575 \\
0.585 \\
0.72 \\
0.655 \\
0.57 \\
0.635 \\
0.815 \\
0.565 \\
0.565 \\
0.55 \\
};

\addplot[mark=*, boxplot, boxplot/draw position=6]
table[row sep=\\, y index=0] {
data
0.895 \\
0.67 \\
0.535 \\
0.65 \\
0.595 \\
0.6 \\
0.83 \\
0.53 \\
0.73 \\
0.625 \\
0.66 \\
0.675 \\
0.69 \\
0.64 \\
0.795 \\
0.645 \\
0.875 \\
0.63 \\
0.77 \\
0.69 \\
0.73 \\
0.61 \\
0.68 \\
0.735 \\
0.73 \\
0.635 \\
0.78 \\
0.865 \\
0.73 \\
0.875 \\
0.77 \\
0.59 \\
0.7 \\
0.835 \\
0.845 \\
0.8 \\
0.805 \\
0.72 \\
0.765 \\
0.635 \\
0.705 \\
0.64 \\
0.755 \\
0.66 \\
0.67 \\
0.65 \\
0.815 \\
0.725 \\
0.62 \\
0.6 \\
};

\addplot[mark=*, boxplot, boxplot/draw position=7]
table[row sep=\\, y index=0] {
data
0.665 \\
0.695 \\
0.745 \\
0.77 \\
0.58 \\
0.76 \\
0.73 \\
0.64 \\
0.7 \\
0.7 \\
0.775 \\
0.735 \\
0.59 \\
0.705 \\
0.675 \\
0.735 \\
0.565 \\
0.745 \\
0.72 \\
0.775 \\
0.705 \\
0.8 \\
0.695 \\
0.73 \\
0.66 \\
0.775 \\
0.69 \\
0.655 \\
0.71 \\
0.69 \\
0.775 \\
0.66 \\
0.51 \\
0.635 \\
0.58 \\
0.725 \\
0.705 \\
0.69 \\
0.59 \\
0.55 \\
0.75 \\
0.575 \\
0.45 \\
0.715 \\
0.635 \\
0.525 \\
0.505 \\
0.765 \\
0.64 \\
0.565 \\
};

\addplot[mark=*, boxplot, boxplot/draw position=8]
table[row sep=\\, y index=0] {
data
0.68 \\
0.73 \\
0.775 \\
0.565 \\
0.58 \\
0.89 \\
0.715 \\
0.88 \\
0.8 \\
0.75 \\
0.805 \\
0.795 \\
0.585 \\
0.69 \\
0.61 \\
0.89 \\
0.775 \\
0.745 \\
0.915 \\
0.54 \\
0.83 \\
0.68 \\
0.835 \\
0.67 \\
0.72 \\
0.915 \\
0.9 \\
0.715 \\
0.645 \\
0.84 \\
0.565 \\
0.68 \\
0.69 \\
0.65 \\
0.81 \\
0.765 \\
0.875 \\
0.78 \\
0.62 \\
0.675 \\
0.815 \\
0.645 \\
0.66 \\
0.87 \\
0.71 \\
0.735 \\
0.7 \\
0.875 \\
0.74 \\
0.83 \\
};

\addplot[mark=*, boxplot, boxplot/draw position=9]
table[row sep=\\, y index=0] {
data
0.66 \\
0.81 \\
0.69 \\
0.795 \\
0.875 \\
0.76 \\
0.635 \\
0.81 \\
0.83 \\
0.64 \\
0.69 \\
0.735 \\
0.645 \\
0.79 \\
0.885 \\
0.88 \\
0.775 \\
0.725 \\
0.595 \\
0.625 \\
0.71 \\
0.71 \\
0.77 \\
0.735 \\
0.66 \\
0.685 \\
0.705 \\
0.745 \\
0.805 \\
0.77 \\
0.73 \\
0.79 \\
0.695 \\
0.86 \\
0.875 \\
0.775 \\
0.755 \\
0.72 \\
0.785 \\
0.69 \\
0.83 \\
0.81 \\
0.755 \\
0.77 \\
0.675 \\
0.715 \\
0.68 \\
0.72 \\
0.825 \\
0.685 \\
};

\addplot[mark=*, boxplot, boxplot/draw position=10]
table[row sep=\\, y index=0] {
data
0.775 \\
0.75 \\
0.755 \\
0.61 \\
0.705 \\
0.89 \\
0.82 \\
0.945 \\
0.82 \\
0.775 \\
0.805 \\
0.835 \\
0.59 \\
0.7 \\
0.67 \\
0.905 \\
0.83 \\
0.84 \\
0.93 \\
0.515 \\
0.85 \\
0.705 \\
0.78 \\
0.71 \\
0.775 \\
0.925 \\
0.9 \\
0.77 \\
0.705 \\
0.85 \\
0.595 \\
0.695 \\
0.81 \\
0.705 \\
0.845 \\
0.8 \\
0.88 \\
0.79 \\
0.64 \\
0.745 \\
0.74 \\
0.615 \\
0.905 \\
0.89 \\
0.735 \\
0.785 \\
0.81 \\
0.885 \\
0.745 \\
0.905 \\
};

\addplot[mark=*, boxplot, boxplot/draw position=11]
table[row sep=\\, y index=0] {
data
0.835 \\
0.745 \\
0.99 \\
0.84 \\
0.74 \\
0.85 \\
0.73 \\
0.765 \\
0.835 \\
0.8 \\
0.85 \\
0.8 \\
0.895 \\
0.765 \\
0.91 \\
0.845 \\
0.815 \\
0.765 \\
0.665 \\
0.635 \\
0.675 \\
0.76 \\
0.7 \\
0.71 \\
0.805 \\
0.64 \\
0.875 \\
0.565 \\
0.7 \\
0.81 \\
0.865 \\
0.845 \\
0.63 \\
0.93 \\
0.7 \\
0.99 \\
0.705 \\
0.745 \\
0.85 \\
0.895 \\
0.855 \\
0.885 \\
0.8 \\
0.935 \\
0.85 \\
0.84 \\
0.78 \\
0.76 \\
0.8 \\
0.69 \\
};

\addplot[mark=*, boxplot, boxplot/draw position=12]
table[row sep=\\, y index=0] {
data
0.84 \\
0.76 \\
0.99 \\
0.955 \\
0.72 \\
0.85 \\
0.78 \\
0.8 \\
0.865 \\
0.78 \\
0.85 \\
0.93 \\
0.895 \\
0.775 \\
0.92 \\
0.85 \\
0.84 \\
0.775 \\
0.655 \\
0.635 \\
0.675 \\
0.765 \\
0.705 \\
0.67 \\
0.81 \\
0.655 \\
0.885 \\
0.61 \\
0.735 \\
0.795 \\
0.845 \\
0.84 \\
0.625 \\
0.93 \\
0.725 \\
0.99 \\
0.73 \\
0.77 \\
0.875 \\
0.895 \\
0.87 \\
0.915 \\
0.82 \\
0.935 \\
0.87 \\
0.88 \\
0.805 \\
0.795 \\
0.815 \\
0.7 \\
};
}{0.1}{Output connectivity}{15}

        }
    }
    \resizebox{\textwidth}{!}{
        \subfloat[N=130]{
            \myboxplot{

\addplot[mark=*, boxplot, boxplot/draw position=1]
table[row sep=\\, y index=0] {
data
0.505 \\
0.51 \\
0.54 \\
0.465 \\
0.64 \\
0.525 \\
0.525 \\
0.575 \\
0.555 \\
0.54 \\
0.575 \\
0.515 \\
0.525 \\
0.505 \\
0.54 \\
0.535 \\
0.535 \\
0.54 \\
0.605 \\
0.555 \\
0.49 \\
0.555 \\
0.605 \\
0.505 \\
0.585 \\
0.505 \\
0.53 \\
0.56 \\
0.61 \\
0.535 \\
0.64 \\
0.545 \\
0.54 \\
0.575 \\
0.595 \\
0.55 \\
0.56 \\
0.47 \\
0.56 \\
0.59 \\
0.48 \\
0.53 \\
0.51 \\
0.635 \\
0.525 \\
0.545 \\
0.53 \\
0.585 \\
0.495 \\
0.47 \\
};

\addplot[mark=*, boxplot, boxplot/draw position=2]
table[row sep=\\, y index=0] {
data
0.72 \\
0.535 \\
0.59 \\
0.615 \\
0.575 \\
0.565 \\
0.595 \\
0.61 \\
0.455 \\
0.555 \\
0.585 \\
0.59 \\
0.56 \\
0.545 \\
0.605 \\
0.495 \\
0.545 \\
0.79 \\
0.505 \\
0.615 \\
0.56 \\
0.57 \\
0.72 \\
0.56 \\
0.51 \\
0.58 \\
0.525 \\
0.675 \\
0.65 \\
0.63 \\
0.64 \\
0.545 \\
0.61 \\
0.515 \\
0.58 \\
0.575 \\
0.62 \\
0.56 \\
0.57 \\
0.55 \\
0.55 \\
0.52 \\
0.51 \\
0.6 \\
0.51 \\
0.65 \\
0.585 \\
0.68 \\
0.535 \\
0.555 \\
};

\addplot[mark=*, boxplot, boxplot/draw position=3]
table[row sep=\\, y index=0] {
data
0.62 \\
0.48 \\
0.625 \\
0.625 \\
0.69 \\
0.52 \\
0.67 \\
0.525 \\
0.585 \\
0.6 \\
0.55 \\
0.645 \\
0.685 \\
0.585 \\
0.6 \\
0.66 \\
0.67 \\
0.535 \\
0.635 \\
0.63 \\
0.685 \\
0.55 \\
0.505 \\
0.675 \\
0.595 \\
0.56 \\
0.575 \\
0.55 \\
0.545 \\
0.71 \\
0.575 \\
0.555 \\
0.71 \\
0.855 \\
0.73 \\
0.49 \\
0.51 \\
0.645 \\
0.705 \\
0.615 \\
0.685 \\
0.555 \\
0.585 \\
0.55 \\
0.675 \\
0.58 \\
0.575 \\
0.61 \\
0.56 \\
0.5 \\
};

\addplot[mark=*, boxplot, boxplot/draw position=4]
table[row sep=\\, y index=0] {
data
0.805 \\
0.585 \\
0.64 \\
0.6 \\
0.555 \\
0.595 \\
0.595 \\
0.655 \\
0.655 \\
0.68 \\
0.63 \\
0.685 \\
0.605 \\
0.57 \\
0.685 \\
0.66 \\
0.615 \\
0.84 \\
0.565 \\
0.78 \\
0.62 \\
0.64 \\
0.845 \\
0.71 \\
0.535 \\
0.625 \\
0.68 \\
0.71 \\
0.745 \\
0.68 \\
0.675 \\
0.65 \\
0.695 \\
0.585 \\
0.62 \\
0.585 \\
0.71 \\
0.68 \\
0.51 \\
0.655 \\
0.675 \\
0.53 \\
0.615 \\
0.76 \\
0.51 \\
0.71 \\
0.54 \\
0.65 \\
0.52 \\
0.7 \\
};

\addplot[mark=*, boxplot, boxplot/draw position=5]
table[row sep=\\, y index=0] {
data
0.645 \\
0.52 \\
0.7 \\
0.605 \\
0.625 \\
0.7 \\
0.585 \\
0.72 \\
0.67 \\
0.705 \\
0.56 \\
0.645 \\
0.655 \\
0.73 \\
0.645 \\
0.49 \\
0.635 \\
0.59 \\
0.705 \\
0.665 \\
0.58 \\
0.775 \\
0.64 \\
0.57 \\
0.645 \\
0.8 \\
0.635 \\
0.72 \\
0.615 \\
0.78 \\
0.65 \\
0.785 \\
0.525 \\
0.55 \\
0.79 \\
0.715 \\
0.805 \\
0.735 \\
0.715 \\
0.66 \\
0.745 \\
0.86 \\
0.71 \\
0.715 \\
0.59 \\
0.73 \\
0.69 \\
0.575 \\
0.66 \\
0.71 \\
};

\addplot[mark=*, boxplot, boxplot/draw position=6]
table[row sep=\\, y index=0] {
data
0.625 \\
0.49 \\
0.71 \\
0.615 \\
0.63 \\
0.755 \\
0.575 \\
0.76 \\
0.73 \\
0.71 \\
0.625 \\
0.645 \\
0.675 \\
0.785 \\
0.67 \\
0.475 \\
0.68 \\
0.58 \\
0.72 \\
0.71 \\
0.65 \\
0.795 \\
0.72 \\
0.66 \\
0.71 \\
0.82 \\
0.62 \\
0.72 \\
0.625 \\
0.805 \\
0.63 \\
0.775 \\
0.515 \\
0.54 \\
0.74 \\
0.76 \\
0.89 \\
0.745 \\
0.71 \\
0.745 \\
0.8 \\
0.87 \\
0.75 \\
0.7 \\
0.685 \\
0.645 \\
0.675 \\
0.58 \\
0.66 \\
0.73 \\
};

\addplot[mark=*, boxplot, boxplot/draw position=7]
table[row sep=\\, y index=0] {
data
0.575 \\
0.64 \\
0.74 \\
0.63 \\
0.86 \\
0.72 \\
0.76 \\
0.815 \\
0.78 \\
0.675 \\
0.655 \\
0.87 \\
0.905 \\
0.705 \\
0.72 \\
0.775 \\
0.765 \\
0.805 \\
0.895 \\
0.735 \\
0.65 \\
0.57 \\
0.58 \\
0.835 \\
0.675 \\
0.785 \\
0.765 \\
0.66 \\
0.655 \\
0.635 \\
0.64 \\
0.865 \\
0.775 \\
0.64 \\
0.58 \\
0.575 \\
0.75 \\
0.81 \\
0.63 \\
0.715 \\
0.845 \\
0.735 \\
0.675 \\
0.86 \\
0.705 \\
0.66 \\
0.96 \\
0.605 \\
0.73 \\
0.7 \\
};

\addplot[mark=*, boxplot, boxplot/draw position=8]
table[row sep=\\, y index=0] {
data
0.65 \\
0.685 \\
0.595 \\
0.845 \\
0.8 \\
0.595 \\
0.735 \\
0.77 \\
0.67 \\
0.67 \\
0.845 \\
0.78 \\
0.765 \\
0.57 \\
0.615 \\
0.715 \\
0.745 \\
0.92 \\
0.735 \\
0.68 \\
0.84 \\
0.655 \\
0.705 \\
0.595 \\
0.685 \\
0.615 \\
0.72 \\
0.83 \\
0.855 \\
0.92 \\
0.665 \\
0.87 \\
0.79 \\
0.61 \\
0.705 \\
0.735 \\
0.51 \\
0.72 \\
0.63 \\
0.775 \\
0.8 \\
0.805 \\
0.91 \\
0.815 \\
0.7 \\
0.82 \\
0.645 \\
0.495 \\
0.855 \\
0.695 \\
};

\addplot[mark=*, boxplot, boxplot/draw position=9]
table[row sep=\\, y index=0] {
data
0.77 \\
0.88 \\
0.725 \\
0.775 \\
0.88 \\
0.59 \\
0.845 \\
0.795 \\
0.76 \\
0.755 \\
0.79 \\
0.875 \\
0.825 \\
0.72 \\
0.705 \\
0.6 \\
0.705 \\
0.72 \\
0.76 \\
0.69 \\
0.58 \\
0.75 \\
0.69 \\
0.665 \\
0.59 \\
0.71 \\
0.775 \\
0.64 \\
0.87 \\
0.705 \\
0.805 \\
0.68 \\
0.805 \\
0.74 \\
0.69 \\
0.575 \\
0.825 \\
0.815 \\
0.78 \\
0.845 \\
0.66 \\
0.745 \\
0.67 \\
0.62 \\
0.78 \\
0.705 \\
0.78 \\
0.8 \\
0.935 \\
0.84 \\
};

\addplot[mark=*, boxplot, boxplot/draw position=10]
table[row sep=\\, y index=0] {
data
0.835 \\
0.93 \\
0.9 \\
0.825 \\
0.88 \\
0.705 \\
0.87 \\
0.74 \\
0.85 \\
0.69 \\
0.78 \\
0.82 \\
0.965 \\
0.68 \\
0.89 \\
0.835 \\
0.795 \\
0.795 \\
0.865 \\
0.7 \\
0.94 \\
0.575 \\
0.57 \\
0.87 \\
0.77 \\
0.665 \\
0.77 \\
0.76 \\
0.91 \\
0.755 \\
0.72 \\
0.745 \\
0.8 \\
0.785 \\
0.89 \\
0.715 \\
0.915 \\
0.76 \\
0.59 \\
0.865 \\
0.845 \\
0.825 \\
0.685 \\
0.81 \\
0.835 \\
0.785 \\
0.875 \\
0.775 \\
0.545 \\
0.85 \\
};

\addplot[mark=*, boxplot, boxplot/draw position=11]
table[row sep=\\, y index=0] {
data
0.74 \\
0.89 \\
0.8 \\
0.795 \\
0.895 \\
0.675 \\
0.86 \\
0.71 \\
0.84 \\
0.75 \\
0.88 \\
0.89 \\
0.85 \\
0.73 \\
0.745 \\
0.735 \\
0.755 \\
0.715 \\
0.79 \\
0.765 \\
0.6 \\
0.905 \\
0.685 \\
0.67 \\
0.75 \\
0.77 \\
0.81 \\
0.71 \\
0.885 \\
0.74 \\
0.795 \\
0.67 \\
0.885 \\
0.745 \\
0.705 \\
0.62 \\
0.84 \\
0.835 \\
0.79 \\
0.86 \\
0.72 \\
0.745 \\
0.66 \\
0.62 \\
0.775 \\
0.695 \\
0.795 \\
0.88 \\
0.93 \\
0.86 \\
};

\addplot[mark=*, boxplot, boxplot/draw position=12]
table[row sep=\\, y index=0] {
data
0.885 \\
0.89 \\
0.91 \\
0.84 \\
0.9 \\
0.655 \\
0.885 \\
0.8 \\
0.84 \\
0.71 \\
0.8 \\
0.83 \\
0.955 \\
0.695 \\
0.91 \\
0.85 \\
0.82 \\
0.87 \\
0.845 \\
0.81 \\
0.945 \\
0.605 \\
0.605 \\
0.86 \\
0.795 \\
0.705 \\
0.9 \\
0.74 \\
0.95 \\
0.81 \\
0.75 \\
0.765 \\
0.825 \\
0.805 \\
0.925 \\
0.755 \\
0.945 \\
0.82 \\
0.91 \\
0.87 \\
0.88 \\
0.845 \\
0.685 \\
0.815 \\
0.85 \\
0.79 \\
0.885 \\
0.765 \\
0.56 \\
0.845 \\
};

\addplot[mark=*, boxplot, boxplot/draw position=13]
table[row sep=\\, y index=0] {
data
0.74 \\
0.73 \\
0.865 \\
0.76 \\
0.89 \\
0.66 \\
0.82 \\
0.96 \\
0.86 \\
0.685 \\
0.89 \\
0.815 \\
0.93 \\
0.53 \\
0.745 \\
0.895 \\
0.825 \\
0.71 \\
0.825 \\
0.675 \\
0.645 \\
0.605 \\
0.835 \\
0.83 \\
0.885 \\
0.845 \\
0.81 \\
0.8 \\
0.89 \\
0.76 \\
0.87 \\
0.865 \\
0.87 \\
0.51 \\
0.73 \\
0.88 \\
0.785 \\
0.755 \\
0.685 \\
0.975 \\
0.87 \\
0.71 \\
0.875 \\
0.86 \\
0.865 \\
0.59 \\
0.77 \\
0.765 \\
0.775 \\
0.765 \\
};
}{0.1}{Reservoir subset size}{15}

        }
        \subfloat[N=140]{
            \myboxplot{

\addplot[mark=*, boxplot, boxplot/draw position=1]
table[row sep=\\, y index=0] {
data
0.525 \\
0.55 \\
0.495 \\
0.565 \\
0.53 \\
0.63 \\
0.495 \\
0.495 \\
0.56 \\
0.585 \\
0.465 \\
0.68 \\
0.465 \\
0.43 \\
0.58 \\
0.495 \\
0.435 \\
0.53 \\
0.56 \\
0.55 \\
0.5 \\
0.545 \\
0.6 \\
0.51 \\
0.53 \\
0.495 \\
0.45 \\
0.57 \\
0.515 \\
0.5 \\
0.44 \\
0.535 \\
0.565 \\
0.485 \\
0.575 \\
0.51 \\
0.65 \\
0.595 \\
0.6 \\
0.57 \\
0.54 \\
0.55 \\
0.595 \\
0.51 \\
0.65 \\
0.505 \\
0.645 \\
0.545 \\
0.545 \\
0.585 \\
};

\addplot[mark=*, boxplot, boxplot/draw position=2]
table[row sep=\\, y index=0] {
data
0.515 \\
0.63 \\
0.54 \\
0.55 \\
0.495 \\
0.49 \\
0.495 \\
0.52 \\
0.515 \\
0.53 \\
0.575 \\
0.66 \\
0.475 \\
0.595 \\
0.56 \\
0.73 \\
0.55 \\
0.54 \\
0.59 \\
0.595 \\
0.755 \\
0.7 \\
0.67 \\
0.53 \\
0.57 \\
0.645 \\
0.57 \\
0.59 \\
0.515 \\
0.63 \\
0.53 \\
0.53 \\
0.74 \\
0.55 \\
0.425 \\
0.655 \\
0.78 \\
0.745 \\
0.59 \\
0.545 \\
0.54 \\
0.55 \\
0.615 \\
0.66 \\
0.59 \\
0.625 \\
0.535 \\
0.59 \\
0.6 \\
0.53 \\
};

\addplot[mark=*, boxplot, boxplot/draw position=3]
table[row sep=\\, y index=0] {
data
0.625 \\
0.585 \\
0.695 \\
0.69 \\
0.58 \\
0.505 \\
0.59 \\
0.6 \\
0.735 \\
0.54 \\
0.56 \\
0.595 \\
0.585 \\
0.675 \\
0.465 \\
0.61 \\
0.62 \\
0.515 \\
0.625 \\
0.68 \\
0.535 \\
0.625 \\
0.54 \\
0.605 \\
0.565 \\
0.595 \\
0.595 \\
0.715 \\
0.605 \\
0.64 \\
0.74 \\
0.45 \\
0.515 \\
0.63 \\
0.555 \\
0.565 \\
0.525 \\
0.645 \\
0.705 \\
0.605 \\
0.685 \\
0.65 \\
0.55 \\
0.6 \\
0.595 \\
0.56 \\
0.695 \\
0.69 \\
0.58 \\
0.605 \\
};

\addplot[mark=*, boxplot, boxplot/draw position=4]
table[row sep=\\, y index=0] {
data
0.59 \\
0.705 \\
0.565 \\
0.58 \\
0.585 \\
0.63 \\
0.49 \\
0.67 \\
0.575 \\
0.595 \\
0.71 \\
0.715 \\
0.54 \\
0.64 \\
0.615 \\
0.81 \\
0.695 \\
0.62 \\
0.615 \\
0.835 \\
0.755 \\
0.785 \\
0.65 \\
0.53 \\
0.535 \\
0.73 \\
0.76 \\
0.545 \\
0.52 \\
0.665 \\
0.65 \\
0.54 \\
0.815 \\
0.615 \\
0.585 \\
0.64 \\
0.78 \\
0.78 \\
0.605 \\
0.7 \\
0.65 \\
0.54 \\
0.67 \\
0.725 \\
0.665 \\
0.54 \\
0.65 \\
0.595 \\
0.66 \\
0.59 \\
};

\addplot[mark=*, boxplot, boxplot/draw position=5]
table[row sep=\\, y index=0] {
data
0.605 \\
0.61 \\
0.695 \\
0.765 \\
0.6 \\
0.505 \\
0.605 \\
0.655 \\
0.76 \\
0.74 \\
0.585 \\
0.58 \\
0.63 \\
0.67 \\
0.565 \\
0.68 \\
0.63 \\
0.555 \\
0.66 \\
0.675 \\
0.6 \\
0.75 \\
0.565 \\
0.69 \\
0.555 \\
0.615 \\
0.56 \\
0.76 \\
0.68 \\
0.635 \\
0.785 \\
0.77 \\
0.555 \\
0.675 \\
0.59 \\
0.655 \\
0.705 \\
0.63 \\
0.715 \\
0.61 \\
0.725 \\
0.71 \\
0.615 \\
0.655 \\
0.655 \\
0.715 \\
0.695 \\
0.78 \\
0.65 \\
0.67 \\
};

\addplot[mark=*, boxplot, boxplot/draw position=6]
table[row sep=\\, y index=0] {
data
0.655 \\
0.59 \\
0.855 \\
0.62 \\
0.735 \\
0.65 \\
0.5 \\
0.88 \\
0.65 \\
0.755 \\
0.54 \\
0.735 \\
0.77 \\
0.9 \\
0.71 \\
0.66 \\
0.62 \\
0.6 \\
0.645 \\
0.725 \\
0.69 \\
0.65 \\
0.97 \\
0.68 \\
0.88 \\
0.895 \\
0.89 \\
0.72 \\
0.805 \\
0.735 \\
0.715 \\
0.46 \\
0.675 \\
0.615 \\
0.63 \\
0.62 \\
0.695 \\
0.535 \\
0.695 \\
0.665 \\
0.565 \\
0.885 \\
0.71 \\
0.525 \\
0.67 \\
0.77 \\
0.81 \\
0.68 \\
0.6 \\
0.66 \\
};

\addplot[mark=*, boxplot, boxplot/draw position=7]
table[row sep=\\, y index=0] {
data
0.665 \\
0.61 \\
0.675 \\
0.765 \\
0.73 \\
0.73 \\
0.8 \\
0.665 \\
0.83 \\
0.665 \\
0.57 \\
0.91 \\
0.605 \\
0.64 \\
0.855 \\
0.71 \\
0.795 \\
0.76 \\
0.845 \\
0.675 \\
0.71 \\
0.6 \\
0.83 \\
0.685 \\
0.76 \\
0.65 \\
0.615 \\
0.65 \\
0.84 \\
0.59 \\
0.62 \\
0.535 \\
0.835 \\
0.615 \\
0.835 \\
0.5 \\
0.75 \\
0.785 \\
0.81 \\
0.535 \\
0.585 \\
0.755 \\
0.77 \\
0.82 \\
0.68 \\
0.81 \\
0.825 \\
0.655 \\
0.67 \\
0.875 \\
};

\addplot[mark=*, boxplot, boxplot/draw position=8]
table[row sep=\\, y index=0] {
data
0.785 \\
0.645 \\
0.84 \\
0.59 \\
0.765 \\
0.615 \\
0.66 \\
0.675 \\
0.875 \\
0.69 \\
0.625 \\
0.54 \\
0.55 \\
0.635 \\
0.6 \\
0.825 \\
0.73 \\
0.64 \\
0.675 \\
0.86 \\
0.745 \\
0.865 \\
0.65 \\
0.67 \\
0.64 \\
0.77 \\
0.72 \\
0.745 \\
0.795 \\
0.52 \\
0.805 \\
0.91 \\
0.76 \\
0.68 \\
0.7 \\
0.935 \\
0.785 \\
0.91 \\
0.825 \\
0.715 \\
0.815 \\
0.64 \\
0.745 \\
0.655 \\
0.74 \\
0.545 \\
0.68 \\
0.83 \\
0.81 \\
0.675 \\
};

\addplot[mark=*, boxplot, boxplot/draw position=9]
table[row sep=\\, y index=0] {
data
0.88 \\
0.935 \\
0.745 \\
0.64 \\
0.67 \\
0.83 \\
0.825 \\
0.615 \\
0.66 \\
0.81 \\
0.625 \\
0.83 \\
0.9 \\
0.66 \\
0.79 \\
0.755 \\
0.835 \\
0.835 \\
0.715 \\
0.71 \\
0.665 \\
0.665 \\
0.585 \\
0.715 \\
0.765 \\
0.74 \\
0.485 \\
0.705 \\
0.76 \\
0.86 \\
0.745 \\
0.68 \\
0.765 \\
0.885 \\
0.94 \\
0.8 \\
0.555 \\
0.61 \\
0.575 \\
0.76 \\
0.665 \\
0.895 \\
0.825 \\
0.94 \\
0.795 \\
0.84 \\
0.67 \\
0.695 \\
0.87 \\
0.855 \\
};

\addplot[mark=*, boxplot, boxplot/draw position=10]
table[row sep=\\, y index=0] {
data
0.825 \\
0.665 \\
0.865 \\
0.58 \\
0.78 \\
0.66 \\
0.7 \\
0.725 \\
0.88 \\
0.72 \\
0.715 \\
0.595 \\
0.6 \\
0.755 \\
0.565 \\
0.905 \\
0.86 \\
0.685 \\
0.695 \\
0.85 \\
0.805 \\
0.895 \\
0.715 \\
0.695 \\
0.675 \\
0.785 \\
0.855 \\
0.785 \\
0.81 \\
0.54 \\
0.805 \\
0.95 \\
0.825 \\
0.68 \\
0.69 \\
0.955 \\
0.835 \\
0.93 \\
0.8 \\
0.76 \\
0.83 \\
0.85 \\
0.765 \\
0.675 \\
0.795 \\
0.58 \\
0.645 \\
0.89 \\
0.835 \\
0.735 \\
};

\addplot[mark=*, boxplot, boxplot/draw position=11]
table[row sep=\\, y index=0] {
data
0.885 \\
0.95 \\
0.725 \\
0.72 \\
0.74 \\
0.865 \\
0.805 \\
0.62 \\
0.67 \\
0.92 \\
0.71 \\
0.82 \\
0.915 \\
0.71 \\
0.815 \\
0.755 \\
0.85 \\
0.865 \\
0.8 \\
0.74 \\
0.715 \\
0.715 \\
0.555 \\
0.715 \\
0.83 \\
0.675 \\
0.49 \\
0.72 \\
0.805 \\
0.865 \\
0.805 \\
0.68 \\
0.76 \\
0.89 \\
0.915 \\
0.8 \\
0.6 \\
0.565 \\
0.625 \\
0.755 \\
0.705 \\
0.895 \\
0.84 \\
0.945 \\
0.905 \\
0.84 \\
0.66 \\
0.79 \\
0.915 \\
0.91 \\
};

\addplot[mark=*, boxplot, boxplot/draw position=12]
table[row sep=\\, y index=0] {
data
0.735 \\
0.88 \\
0.675 \\
0.805 \\
0.79 \\
0.69 \\
0.91 \\
0.635 \\
0.76 \\
0.855 \\
0.77 \\
0.635 \\
0.905 \\
0.935 \\
0.85 \\
0.755 \\
0.795 \\
0.84 \\
0.885 \\
0.87 \\
0.805 \\
0.76 \\
0.99 \\
0.73 \\
0.68 \\
0.845 \\
0.92 \\
0.655 \\
0.83 \\
0.795 \\
0.73 \\
0.79 \\
0.715 \\
0.78 \\
0.875 \\
0.875 \\
0.835 \\
0.86 \\
0.735 \\
0.89 \\
0.825 \\
0.86 \\
0.835 \\
0.64 \\
0.825 \\
0.63 \\
0.695 \\
0.84 \\
0.71 \\
0.825 \\
};

\addplot[mark=*, boxplot, boxplot/draw position=13]
table[row sep=\\, y index=0] {
data
0.875 \\
0.835 \\
0.7 \\
0.75 \\
0.72 \\
0.815 \\
0.845 \\
0.715 \\
0.725 \\
0.86 \\
0.865 \\
0.815 \\
0.77 \\
0.875 \\
0.68 \\
0.935 \\
0.87 \\
0.865 \\
0.88 \\
0.885 \\
0.895 \\
0.795 \\
0.85 \\
0.84 \\
0.785 \\
0.97 \\
0.795 \\
0.655 \\
0.785 \\
0.85 \\
0.84 \\
0.84 \\
0.665 \\
0.805 \\
0.865 \\
0.88 \\
0.945 \\
0.885 \\
0.91 \\
0.725 \\
0.815 \\
0.835 \\
0.645 \\
0.64 \\
0.935 \\
0.805 \\
0.69 \\
0.825 \\
0.695 \\
0.79 \\
};

\addplot[mark=*, boxplot, boxplot/draw position=14]
table[row sep=\\, y index=0] {
data
0.99 \\
0.84 \\
0.925 \\
0.915 \\
0.845 \\
0.775 \\
0.88 \\
0.925 \\
0.8 \\
0.835 \\
0.91 \\
0.955 \\
0.75 \\
0.79 \\
0.95 \\
0.73 \\
0.635 \\
0.88 \\
0.765 \\
0.865 \\
0.86 \\
0.85 \\
0.69 \\
0.75 \\
0.905 \\
0.93 \\
0.645 \\
0.735 \\
0.925 \\
0.805 \\
0.97 \\
0.725 \\
0.69 \\
0.8 \\
0.925 \\
0.925 \\
0.715 \\
0.82 \\
0.745 \\
0.75 \\
0.91 \\
0.89 \\
0.675 \\
0.805 \\
0.835 \\
0.845 \\
0.74 \\
0.865 \\
0.925 \\
0.95 \\
};
}{0.1}{Reservoir subset size}{15}

        }
    }
    \caption{Plots of output connectivity against accuracy on TP5 - Part 2 of 2}
\end{figure*}

\subsection{Discussion}

LOL PLZ NO
