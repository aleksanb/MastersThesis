\section{Minimum required output connectivity}

How few connections can one have from the reservoir itself to the readout layer,
while staying accurate on the task at hand?
As the reservoir prediction is computed by regressing the states of the network nodes against the expected output bits,
reducing the number of nodes used in the regression can never give a higher accuracy than using all of the nodes.
If there is redundant information in the network, quite a few nodes could be cut off from the readout layer while maintaining the best accuracy for this specific reservoir.
This does require a sufficient amount of linear independence between the remaining nodes of the network.

As shown in the previous section, different problems require different reservoir sizes.
Past a certain reservoir size, all reservoirs have a sufficient accuracy on the given task.
One could postulate that for reservoirs larger than the 95\% accuracy breakpoint,
it suffices to read out a number of nodes equal to the breakpoint.
If this is the case, one can safely use growing reservoirs without having to change the output connections,
given that the reservoir was large enough to solve the given task in the first place.

As shown in section \ref{experiments:1:results}, there is a correlation between input connectivity and accuracy.
For our reservoirs with $K=3$, the optimal input connectivity was found to be $n\_nodes/2$.
Using this information, we can reduce our search space for minimum required output connectivity.
The parameters in table \ref{tab:oc-reservoir-parameters} will be used for simulation.

\begin{table}[ht]
    \centering
    \caption{Reservoir parameters for optimal output connectivity}
    \label{tab:oc-reservoir-parameters}
    \begin{tabular}{ll}
        Nodes               & [10..(100, 140)], step size=10 \\
        Connectivity        & 3                              \\
        Input connectivity  & $ n\_nodes / 2 $               \\
        Output connectivity & [0..n\_nodes], step size=10    \\
        Sample size         & 50
    \end{tabular}
\end{table}

\subsection{Results}

\begin{figure*}[ht]
    \centering
    \resizebox{\textwidth}{!}{
        \subfloat[N=10]{
            \input{experiments/results/TP3-OC/boxplot-output_connectivity-N10-K3-S50.tex}
        }
        \subfloat[N=20]{
            \input{experiments/results/TP3-OC/boxplot-output_connectivity-N20-K3-S50.tex}
        }   
    }   
    \resizebox{\textwidth}{!}{
        \subfloat[N=30]{
            \input{experiments/results/TP3-OC/boxplot-output_connectivity-N30-K3-S50.tex}
        }   
        \subfloat[N=40]{
            \input{experiments/results/TP3-OC/boxplot-output_connectivity-N40-K3-S50.tex}
        }   
    }   
    \resizebox{\textwidth}{!}{
        \subfloat[N=50]{
            \input{experiments/results/TP3-OC/boxplot-output_connectivity-N50-K3-S50.tex}
        }   
        \subfloat[N=60]{
            \input{experiments/results/TP3-OC/boxplot-output_connectivity-N60-K3-S50.tex}
        }   
    }   
    \caption{Plots of output connectivity against accuracy on TP3 - Part 1 of 2}
\end{figure*}

\begin{figure*}[ht]
    \centering
    \resizebox{\textwidth}{!}{
        \subfloat[N=70]{
            \input{experiments/results/TP3-OC/boxplot-output_connectivity-N70-K3-S50.tex}
        }
        \subfloat[N=80]{
            \input{experiments/results/TP3-OC/boxplot-output_connectivity-N80-K3-S50.tex}
        }   
    }   
    \resizebox{\textwidth}{!}{
        \subfloat[N=90]{
            \input{experiments/results/TP3-OC/boxplot-output_connectivity-N90-K3-S50.tex}
        }   
        \subfloat[N=100]{
            \input{experiments/results/TP3-OC/boxplot-output_connectivity-N100-K3-S50.tex}
        }   
    }   
    \caption{Plots of output connectivity against accuracy on TP3 - Part 1 of 2}
\end{figure*}

%% BRekkku
%% BRekkku
%% BRekkku
%% BRekkku

\begin{figure*}[ht]
    \centering
    \resizebox{\textwidth}{!}{
        \subfloat[N=30]{
            \input{experiments/results/TP5-OC/boxplot-N30-K3-S50-output_connectivity.tex}
        }
        \subfloat[N=40]{
            \input{experiments/results/TP5-OC/boxplot-N40-K3-S50-output_connectivity.tex}
        }
    }
    \resizebox{\textwidth}{!}{
        \subfloat[N=50]{
            \input{experiments/results/TP5-OC/boxplot-N50-K3-S50-output_connectivity.tex}
        }
        \subfloat[N=60]{
            \input{experiments/results/TP5-OC/boxplot-N60-K3-S50-output_connectivity.tex}
        }
    }
    \resizebox{\textwidth}{!}{
        \subfloat[N=70]{
            \input{experiments/results/TP5-OC/boxplot-N70-K3-S50-output_connectivity.tex}
        }
        \subfloat[N=80]{
            \input{experiments/results/TP5-OC/boxplot-N80-K3-S50-output_connectivity.tex}
        }
    }
    \caption{Plots of output connectivity against accuracy on TP5 - Part 1 of 2}
\end{figure*}

\begin{figure*}[ht]
    \centering
    \resizebox{\textwidth}{!}{
        \subfloat[N=90]{
            \input{experiments/results/TP5-OC/boxplot-N90-K3-S50-output_connectivity.tex}
        }
        \subfloat[N=100]{
            \input{experiments/results/TP5-OC/boxplot-N100-K3-S50-output_connectivity.tex}
        }
    }
    \resizebox{\textwidth}{!}{
        \subfloat[N=110]{
            \input{experiments/results/TP5-OC/boxplot-N110-K3-S50-output_connectivity.tex}
        }
        \subfloat[N=120]{
            \input{experiments/results/TP5-OC/boxplot-N120-K3-S50-output_connectivity.tex}
        }
    }
    \resizebox{\textwidth}{!}{
        \subfloat[N=130]{
            \input{experiments/results/TP5-OC/boxplot-N130-K3-S50-output_connectivity.tex}
        }
        \subfloat[N=140]{
            \input{experiments/results/TP5-OC/boxplot-N140-K3-S50-output_connectivity.tex}
        }
    }
    \caption{Plots of output connectivity against accuracy on TP5 - Part 2 of 2}
\end{figure*}

\subsection{Discussion}

LOL PLZ NO
