\chapter{Results}

\section{Required reservoir size, optimal input connectivity}

We wish to accomplish two things: What is the minimum required reservoir size that is still able to complete a chosen task with close to 100\% accuracy, as well as what are the optimal input connectivities for these reservoirs.

%\begin{figure*}[ht]
%    \centering
%    \resizebox{\textwidth}{!}{
%        \subfloat[N=10]{
%            \myboxplot{

\addplot[mark=*, boxplot, boxplot/draw position=1]
table[row sep=\\, y index=0] {
data
0.515 \\
0.515 \\
0.515 \\
0.515 \\
0.515 \\
0.515 \\
0.515 \\
0.515 \\
0.515 \\
0.515 \\
0.515 \\
0.515 \\
0.515 \\
0.515 \\
0.515 \\
0.515 \\
0.515 \\
0.515 \\
0.515 \\
0.515 \\
0.515 \\
0.515 \\
0.515 \\
0.515 \\
0.515 \\
0.515 \\
0.515 \\
0.515 \\
0.515 \\
0.515 \\
};
}{{0.1}}

%        }
%        \subfloat[N=20]{
%            \myboxplot{

\addplot[mark=*, boxplot, boxplot/draw position=1]
table[row sep=\\, y index=0] {
data
0.91 \\
0.86 \\
0.785 \\
0.83 \\
0.775 \\
0.665 \\
0.845 \\
0.995 \\
0.91 \\
0.595 \\
0.92 \\
0.995 \\
0.635 \\
0.98 \\
0.865 \\
0.68 \\
0.745 \\
0.89 \\
0.86 \\
0.845 \\
0.69 \\
0.91 \\
0.995 \\
0.73 \\
0.85 \\
0.82 \\
0.51 \\
0.695 \\
0.78 \\
0.515 \\
};

\addplot[mark=*, boxplot, boxplot/draw position=2]
table[row sep=\\, y index=0] {
data
0.515 \\
0.515 \\
0.515 \\
0.515 \\
0.515 \\
0.515 \\
0.515 \\
0.515 \\
0.515 \\
0.515 \\
0.515 \\
0.515 \\
0.515 \\
0.515 \\
0.515 \\
0.515 \\
0.515 \\
0.515 \\
0.515 \\
0.515 \\
0.515 \\
0.515 \\
0.515 \\
0.515 \\
0.515 \\
0.515 \\
0.515 \\
0.515 \\
0.515 \\
0.515 \\
};
}{{0.1}}

%        }
%        \subfloat[N=30]{
%            \myboxplot{

\addplot[mark=*, boxplot, boxplot/draw position=1]
table[row sep=\\, y index=0] {
data
0.73 \\
0.8 \\
0.78 \\
0.715 \\
0.995 \\
0.995 \\
0.775 \\
0.68 \\
0.51 \\
0.66 \\
0.68 \\
0.745 \\
0.725 \\
0.825 \\
0.865 \\
0.69 \\
0.81 \\
0.925 \\
0.785 \\
0.74 \\
0.82 \\
0.755 \\
0.715 \\
0.765 \\
0.49 \\
0.635 \\
0.815 \\
0.785 \\
0.68 \\
0.69 \\
};

\addplot[mark=*, boxplot, boxplot/draw position=3]
table[row sep=\\, y index=0] {
data
0.515 \\
0.515 \\
0.515 \\
0.515 \\
0.515 \\
0.515 \\
0.515 \\
0.515 \\
0.515 \\
0.515 \\
0.515 \\
0.515 \\
0.515 \\
0.515 \\
0.515 \\
0.515 \\
0.515 \\
0.515 \\
0.515 \\
0.515 \\
0.515 \\
0.515 \\
0.515 \\
0.515 \\
0.515 \\
0.515 \\
0.515 \\
0.515 \\
0.515 \\
0.515 \\
};

\addplot[mark=*, boxplot, boxplot/draw position=2]
table[row sep=\\, y index=0] {
data
0.905 \\
0.605 \\
0.635 \\
0.82 \\
0.815 \\
0.6 \\
0.95 \\
0.83 \\
0.515 \\
0.86 \\
0.515 \\
0.995 \\
0.53 \\
0.82 \\
0.635 \\
0.775 \\
0.91 \\
0.95 \\
0.88 \\
0.995 \\
0.565 \\
0.89 \\
0.77 \\
0.515 \\
0.665 \\
0.805 \\
0.515 \\
0.995 \\
0.805 \\
0.905 \\
};
}{{0.1}}

%        }
%    }
%    \resizebox{\textwidth}{!}{
%        \subfloat[N=40]{
%            \myboxplot{

\addplot[mark=*, boxplot, boxplot/draw position=1]
table[row sep=\\, y index=0] {
data
0.745 \\
0.865 \\
0.695 \\
0.525 \\
0.885 \\
0.66 \\
0.505 \\
0.9 \\
0.56 \\
0.8 \\
0.65 \\
0.815 \\
0.93 \\
0.825 \\
0.68 \\
0.565 \\
0.63 \\
0.585 \\
0.7 \\
0.745 \\
0.55 \\
0.63 \\
0.75 \\
0.52 \\
0.635 \\
0.855 \\
0.755 \\
0.865 \\
0.655 \\
0.515 \\
};

\addplot[mark=*, boxplot, boxplot/draw position=3]
table[row sep=\\, y index=0] {
data
0.515 \\
0.83 \\
0.69 \\
0.995 \\
0.48 \\
0.77 \\
0.995 \\
0.515 \\
0.515 \\
0.69 \\
0.78 \\
0.995 \\
0.865 \\
0.77 \\
0.665 \\
0.995 \\
0.905 \\
0.825 \\
0.93 \\
0.82 \\
0.82 \\
0.995 \\
0.72 \\
0.53 \\
0.585 \\
0.69 \\
0.885 \\
0.945 \\
0.895 \\
0.515 \\
};

\addplot[mark=*, boxplot, boxplot/draw position=2]
table[row sep=\\, y index=0] {
data
0.995 \\
0.955 \\
0.945 \\
0.96 \\
0.88 \\
0.995 \\
0.995 \\
0.945 \\
0.995 \\
0.935 \\
0.945 \\
0.905 \\
0.835 \\
0.635 \\
0.925 \\
0.995 \\
0.935 \\
0.875 \\
0.995 \\
0.5 \\
0.875 \\
0.98 \\
0.775 \\
0.93 \\
0.995 \\
0.87 \\
0.995 \\
0.99 \\
0.975 \\
0.995 \\
};

\addplot[mark=*, boxplot, boxplot/draw position=4]
table[row sep=\\, y index=0] {
data
0.515 \\
0.515 \\
0.515 \\
0.515 \\
0.515 \\
0.515 \\
0.515 \\
0.515 \\
0.515 \\
0.515 \\
0.515 \\
0.515 \\
0.515 \\
0.515 \\
0.515 \\
0.515 \\
0.515 \\
0.515 \\
0.515 \\
0.515 \\
0.515 \\
0.515 \\
0.515 \\
0.515 \\
0.515 \\
0.515 \\
0.515 \\
0.515 \\
0.515 \\
0.515 \\
};
}{{0.1}}

%        }
%        \subfloat[N=50]{
%            \myboxplot{

\addplot[mark=*, boxplot, boxplot/draw position=1]
table[row sep=\\, y index=0] {
data
0.965 \\
0.675 \\
0.49 \\
0.735 \\
0.59 \\
0.67 \\
0.78 \\
0.805 \\
0.74 \\
0.63 \\
0.605 \\
0.83 \\
0.605 \\
0.85 \\
0.71 \\
0.695 \\
0.52 \\
0.655 \\
0.68 \\
0.64 \\
0.58 \\
0.69 \\
0.72 \\
0.63 \\
0.84 \\
0.625 \\
0.89 \\
0.635 \\
0.74 \\
0.68 \\
};

\addplot[mark=*, boxplot, boxplot/draw position=3]
table[row sep=\\, y index=0] {
data
0.8 \\
0.79 \\
0.995 \\
0.985 \\
0.995 \\
0.995 \\
0.82 \\
0.98 \\
0.97 \\
0.96 \\
0.995 \\
0.96 \\
0.995 \\
0.69 \\
0.995 \\
0.995 \\
0.995 \\
0.995 \\
0.995 \\
0.995 \\
0.995 \\
0.76 \\
0.515 \\
0.93 \\
0.795 \\
0.995 \\
0.995 \\
0.995 \\
0.515 \\
0.96 \\
};

\addplot[mark=*, boxplot, boxplot/draw position=2]
table[row sep=\\, y index=0] {
data
0.76 \\
0.76 \\
0.89 \\
0.95 \\
0.885 \\
0.995 \\
0.945 \\
0.99 \\
0.94 \\
0.87 \\
0.995 \\
0.995 \\
0.845 \\
0.995 \\
0.97 \\
0.88 \\
0.86 \\
0.795 \\
0.98 \\
0.995 \\
0.975 \\
0.87 \\
0.73 \\
0.975 \\
0.85 \\
0.72 \\
0.965 \\
0.87 \\
0.775 \\
0.97 \\
};

\addplot[mark=*, boxplot, boxplot/draw position=5]
table[row sep=\\, y index=0] {
data
0.515 \\
0.515 \\
0.515 \\
0.515 \\
0.515 \\
0.515 \\
0.515 \\
0.515 \\
0.515 \\
0.515 \\
0.515 \\
0.515 \\
0.515 \\
0.515 \\
0.515 \\
0.515 \\
0.515 \\
0.515 \\
0.515 \\
0.515 \\
0.515 \\
0.515 \\
0.515 \\
0.515 \\
0.515 \\
0.515 \\
0.515 \\
0.515 \\
0.515 \\
0.515 \\
};

\addplot[mark=*, boxplot, boxplot/draw position=4]
table[row sep=\\, y index=0] {
data
0.995 \\
0.995 \\
0.925 \\
0.92 \\
0.995 \\
0.615 \\
0.65 \\
0.515 \\
0.495 \\
0.67 \\
0.515 \\
0.69 \\
0.515 \\
0.615 \\
0.84 \\
0.995 \\
0.515 \\
0.515 \\
0.805 \\
0.615 \\
0.555 \\
0.89 \\
0.995 \\
0.995 \\
0.995 \\
0.515 \\
0.995 \\
0.79 \\
0.82 \\
0.655 \\
};
}{{0.1}}

%        }
%        \subfloat[N=60]{
%            \myboxplot{

\addplot[mark=*, boxplot, boxplot/draw position=1]
table[row sep=\\, y index=0] {
data
0.695 \\
0.8 \\
0.595 \\
0.645 \\
0.65 \\
0.57 \\
0.755 \\
0.76 \\
0.535 \\
0.89 \\
0.54 \\
0.765 \\
0.595 \\
0.73 \\
0.745 \\
0.605 \\
0.72 \\
0.74 \\
0.53 \\
0.565 \\
0.555 \\
0.595 \\
0.54 \\
0.8 \\
0.64 \\
0.535 \\
0.485 \\
0.76 \\
0.75 \\
0.8 \\
};

\addplot[mark=*, boxplot, boxplot/draw position=2]
table[row sep=\\, y index=0] {
data
0.74 \\
0.96 \\
0.84 \\
0.895 \\
0.96 \\
0.77 \\
0.97 \\
0.995 \\
0.895 \\
0.935 \\
0.95 \\
0.64 \\
0.835 \\
0.855 \\
0.995 \\
0.995 \\
0.985 \\
0.995 \\
0.955 \\
0.705 \\
0.555 \\
0.875 \\
0.87 \\
0.97 \\
0.975 \\
0.995 \\
0.905 \\
0.955 \\
0.885 \\
0.99 \\
};

\addplot[mark=*, boxplot, boxplot/draw position=3]
table[row sep=\\, y index=0] {
data
0.995 \\
0.995 \\
0.965 \\
0.995 \\
0.99 \\
0.865 \\
0.995 \\
0.995 \\
0.905 \\
0.995 \\
0.83 \\
0.99 \\
0.915 \\
0.995 \\
0.995 \\
0.995 \\
0.995 \\
0.995 \\
0.995 \\
0.95 \\
0.935 \\
0.995 \\
0.995 \\
0.995 \\
0.995 \\
0.995 \\
0.73 \\
0.995 \\
0.995 \\
0.82 \\
};

\addplot[mark=*, boxplot, boxplot/draw position=5]
table[row sep=\\, y index=0] {
data
0.515 \\
0.69 \\
0.69 \\
0.995 \\
0.82 \\
0.83 \\
0.905 \\
0.5 \\
0.995 \\
0.82 \\
0.67 \\
0.625 \\
0.925 \\
0.465 \\
0.815 \\
0.515 \\
0.515 \\
0.82 \\
0.885 \\
0.995 \\
0.995 \\
0.515 \\
0.515 \\
0.83 \\
0.515 \\
0.82 \\
0.665 \\
0.59 \\
0.695 \\
0.515 \\
};

\addplot[mark=*, boxplot, boxplot/draw position=4]
table[row sep=\\, y index=0] {
data
0.935 \\
0.995 \\
0.995 \\
0.93 \\
0.995 \\
0.995 \\
0.94 \\
0.995 \\
0.905 \\
0.97 \\
0.995 \\
0.995 \\
0.995 \\
0.995 \\
0.995 \\
0.89 \\
0.995 \\
0.91 \\
0.995 \\
0.625 \\
0.69 \\
0.935 \\
0.9 \\
0.965 \\
0.945 \\
0.995 \\
0.995 \\
0.84 \\
0.995 \\
0.965 \\
};

\addplot[mark=*, boxplot, boxplot/draw position=6]
table[row sep=\\, y index=0] {
data
0.515 \\
0.515 \\
0.515 \\
0.515 \\
0.515 \\
0.515 \\
0.515 \\
0.515 \\
0.515 \\
0.515 \\
0.515 \\
0.515 \\
0.515 \\
0.515 \\
0.515 \\
0.515 \\
0.515 \\
0.515 \\
0.515 \\
0.515 \\
0.515 \\
0.515 \\
0.515 \\
0.515 \\
0.515 \\
0.515 \\
0.515 \\
0.515 \\
0.515 \\
0.515 \\
};
}{{0.1}}

%        }
%    }
%    \resizebox{\textwidth}{!}{
%        \subfloat[N=70]{
%            \myboxplot{

\addplot[mark=*, boxplot, boxplot/draw position=1]
table[row sep=\\, y index=0] {
data
0.765 \\
0.585 \\
0.66 \\
0.925 \\
0.65 \\
0.725 \\
0.77 \\
0.84 \\
0.78 \\
0.53 \\
0.655 \\
0.485 \\
0.805 \\
0.635 \\
0.57 \\
0.74 \\
0.82 \\
0.7 \\
0.995 \\
0.6 \\
0.585 \\
0.55 \\
0.65 \\
0.79 \\
0.52 \\
0.695 \\
0.59 \\
0.53 \\
0.625 \\
0.725 \\
};

\addplot[mark=*, boxplot, boxplot/draw position=2]
table[row sep=\\, y index=0] {
data
0.95 \\
0.91 \\
0.92 \\
0.79 \\
0.91 \\
0.775 \\
0.975 \\
0.975 \\
0.87 \\
0.935 \\
0.95 \\
0.84 \\
0.645 \\
0.97 \\
0.89 \\
0.745 \\
0.995 \\
0.75 \\
0.97 \\
0.92 \\
0.93 \\
0.76 \\
0.87 \\
0.94 \\
0.93 \\
0.745 \\
0.695 \\
0.88 \\
0.72 \\
0.985 \\
};

\addplot[mark=*, boxplot, boxplot/draw position=3]
table[row sep=\\, y index=0] {
data
0.99 \\
0.925 \\
0.945 \\
0.995 \\
0.995 \\
0.995 \\
0.995 \\
0.995 \\
0.995 \\
0.99 \\
0.965 \\
0.95 \\
0.775 \\
0.995 \\
0.99 \\
0.975 \\
0.995 \\
0.935 \\
0.995 \\
0.995 \\
0.925 \\
0.995 \\
0.875 \\
0.87 \\
0.995 \\
0.995 \\
0.935 \\
0.995 \\
0.84 \\
0.99 \\
};

\addplot[mark=*, boxplot, boxplot/draw position=5]
table[row sep=\\, y index=0] {
data
0.995 \\
0.97 \\
0.995 \\
0.995 \\
0.995 \\
0.995 \\
0.995 \\
0.995 \\
0.995 \\
0.76 \\
0.905 \\
0.995 \\
0.995 \\
0.995 \\
0.995 \\
0.995 \\
0.9 \\
0.945 \\
0.995 \\
0.765 \\
0.995 \\
0.73 \\
0.995 \\
0.995 \\
0.44 \\
0.89 \\
0.995 \\
0.67 \\
0.995 \\
0.995 \\
};

\addplot[mark=*, boxplot, boxplot/draw position=4]
table[row sep=\\, y index=0] {
data
0.995 \\
0.995 \\
0.98 \\
0.995 \\
0.995 \\
0.995 \\
0.995 \\
0.995 \\
0.995 \\
0.995 \\
0.995 \\
0.995 \\
0.995 \\
0.99 \\
0.975 \\
0.995 \\
0.995 \\
0.995 \\
0.99 \\
0.955 \\
0.985 \\
0.995 \\
0.995 \\
0.98 \\
0.995 \\
0.995 \\
0.995 \\
0.995 \\
0.675 \\
0.995 \\
};

\addplot[mark=*, boxplot, boxplot/draw position=6]
table[row sep=\\, y index=0] {
data
0.995 \\
0.515 \\
0.515 \\
0.515 \\
0.69 \\
0.995 \\
0.995 \\
0.515 \\
0.515 \\
0.82 \\
0.805 \\
0.515 \\
0.515 \\
0.515 \\
0.56 \\
0.995 \\
0.69 \\
0.515 \\
0.74 \\
0.515 \\
0.93 \\
0.515 \\
0.705 \\
0.515 \\
0.82 \\
0.505 \\
0.515 \\
0.625 \\
0.995 \\
0.765 \\
};

\addplot[mark=*, boxplot, boxplot/draw position=7]
table[row sep=\\, y index=0] {
data
0.515 \\
0.515 \\
0.515 \\
0.515 \\
0.515 \\
0.515 \\
0.515 \\
0.515 \\
0.515 \\
0.515 \\
0.515 \\
0.515 \\
0.515 \\
0.515 \\
0.515 \\
0.515 \\
0.515 \\
0.515 \\
0.515 \\
0.515 \\
0.515 \\
0.515 \\
0.515 \\
0.515 \\
0.515 \\
0.515 \\
0.515 \\
0.515 \\
0.515 \\
0.515 \\
};
}{{0.1}}

%        }
%        \subfloat[N=80]{
%            \myboxplot{

\addplot[mark=*, boxplot, boxplot/draw position=1]
table[row sep=\\, y index=0] {
data
0.655 \\
0.65 \\
0.73 \\
0.715 \\
0.675 \\
0.67 \\
0.62 \\
0.59 \\
0.46 \\
0.5 \\
0.615 \\
0.71 \\
0.53 \\
0.585 \\
0.655 \\
0.53 \\
0.59 \\
0.76 \\
0.68 \\
0.655 \\
0.64 \\
0.54 \\
0.525 \\
0.54 \\
0.67 \\
0.6 \\
0.735 \\
0.5 \\
0.585 \\
0.75 \\
};

\addplot[mark=*, boxplot, boxplot/draw position=2]
table[row sep=\\, y index=0] {
data
0.885 \\
0.97 \\
0.965 \\
0.945 \\
0.805 \\
0.965 \\
0.98 \\
0.79 \\
0.84 \\
0.875 \\
0.775 \\
0.95 \\
0.95 \\
0.875 \\
0.775 \\
0.68 \\
0.685 \\
0.88 \\
0.77 \\
0.695 \\
0.885 \\
0.83 \\
0.99 \\
0.88 \\
0.855 \\
0.995 \\
0.79 \\
0.595 \\
0.79 \\
0.93 \\
};

\addplot[mark=*, boxplot, boxplot/draw position=3]
table[row sep=\\, y index=0] {
data
0.985 \\
0.975 \\
0.97 \\
0.95 \\
0.93 \\
0.85 \\
0.985 \\
0.96 \\
0.995 \\
0.99 \\
0.975 \\
0.92 \\
0.98 \\
0.94 \\
0.975 \\
0.865 \\
0.8 \\
0.995 \\
0.995 \\
0.99 \\
0.995 \\
0.965 \\
0.995 \\
0.995 \\
0.86 \\
0.99 \\
0.935 \\
0.9 \\
0.995 \\
0.955 \\
};

\addplot[mark=*, boxplot, boxplot/draw position=5]
table[row sep=\\, y index=0] {
data
0.995 \\
0.995 \\
0.995 \\
0.995 \\
0.995 \\
0.995 \\
0.995 \\
0.995 \\
0.98 \\
0.995 \\
0.995 \\
0.995 \\
0.995 \\
0.995 \\
0.97 \\
0.995 \\
0.945 \\
0.965 \\
0.995 \\
0.925 \\
0.995 \\
0.995 \\
0.995 \\
0.98 \\
0.995 \\
0.995 \\
0.995 \\
0.995 \\
0.995 \\
0.91 \\
};

\addplot[mark=*, boxplot, boxplot/draw position=4]
table[row sep=\\, y index=0] {
data
0.995 \\
0.995 \\
0.99 \\
0.935 \\
0.995 \\
0.945 \\
0.995 \\
0.995 \\
0.995 \\
0.995 \\
0.96 \\
0.995 \\
0.995 \\
0.935 \\
0.995 \\
0.995 \\
0.98 \\
0.995 \\
0.995 \\
0.99 \\
0.995 \\
0.925 \\
0.995 \\
0.995 \\
0.995 \\
0.995 \\
0.96 \\
0.97 \\
0.945 \\
0.98 \\
};

\addplot[mark=*, boxplot, boxplot/draw position=6]
table[row sep=\\, y index=0] {
data
0.855 \\
0.9 \\
0.995 \\
0.87 \\
0.995 \\
0.995 \\
0.995 \\
0.69 \\
0.635 \\
0.845 \\
0.905 \\
0.82 \\
0.995 \\
0.995 \\
0.825 \\
0.96 \\
0.82 \\
0.995 \\
0.87 \\
0.93 \\
0.515 \\
0.955 \\
0.995 \\
0.725 \\
0.91 \\
0.995 \\
0.755 \\
0.995 \\
0.705 \\
0.905 \\
};

\addplot[mark=*, boxplot, boxplot/draw position=7]
table[row sep=\\, y index=0] {
data
0.515 \\
0.69 \\
0.515 \\
0.69 \\
0.515 \\
0.84 \\
0.85 \\
0.995 \\
0.515 \\
0.69 \\
0.82 \\
0.515 \\
0.995 \\
0.85 \\
0.75 \\
0.515 \\
0.585 \\
0.88 \\
0.905 \\
0.845 \\
0.755 \\
0.495 \\
0.615 \\
0.82 \\
0.775 \\
0.69 \\
0.775 \\
0.515 \\
0.515 \\
0.605 \\
};

\addplot[mark=*, boxplot, boxplot/draw position=8]
table[row sep=\\, y index=0] {
data
0.515 \\
0.515 \\
0.515 \\
0.515 \\
0.515 \\
0.515 \\
0.515 \\
0.515 \\
0.515 \\
0.515 \\
0.515 \\
0.515 \\
0.515 \\
0.515 \\
0.515 \\
0.515 \\
0.515 \\
0.515 \\
0.515 \\
0.515 \\
0.515 \\
0.515 \\
0.515 \\
0.515 \\
0.515 \\
0.515 \\
0.515 \\
0.515 \\
0.515 \\
0.515 \\
};
}{{0.1}}

%        }
%        \subfloat[N=90]{
%            \myboxplot{

\addplot[mark=*, boxplot, boxplot/draw position=1]
table[row sep=\\, y index=0] {
data
0.725 \\
0.655 \\
0.825 \\
0.47 \\
0.69 \\
0.645 \\
0.565 \\
0.535 \\
0.76 \\
0.78 \\
0.485 \\
0.515 \\
0.51 \\
0.565 \\
0.55 \\
0.64 \\
0.57 \\
0.515 \\
0.6 \\
0.515 \\
0.74 \\
0.77 \\
0.495 \\
0.505 \\
0.675 \\
0.51 \\
0.665 \\
0.645 \\
0.63 \\
0.61 \\
};

\addplot[mark=*, boxplot, boxplot/draw position=2]
table[row sep=\\, y index=0] {
data
0.92 \\
0.805 \\
0.635 \\
0.99 \\
0.89 \\
0.805 \\
0.935 \\
0.765 \\
0.835 \\
0.98 \\
0.74 \\
0.88 \\
0.76 \\
0.85 \\
0.83 \\
0.935 \\
0.93 \\
0.855 \\
0.975 \\
0.74 \\
0.795 \\
0.755 \\
0.92 \\
0.825 \\
0.775 \\
0.865 \\
0.72 \\
0.665 \\
0.725 \\
0.705 \\
};

\addplot[mark=*, boxplot, boxplot/draw position=3]
table[row sep=\\, y index=0] {
data
0.995 \\
0.995 \\
0.935 \\
0.995 \\
0.83 \\
0.98 \\
0.995 \\
0.935 \\
0.995 \\
0.99 \\
0.85 \\
0.995 \\
0.97 \\
0.98 \\
0.965 \\
0.985 \\
0.99 \\
0.97 \\
0.995 \\
0.795 \\
0.995 \\
0.795 \\
0.945 \\
0.97 \\
0.99 \\
0.985 \\
0.94 \\
0.805 \\
0.995 \\
0.765 \\
};

\addplot[mark=*, boxplot, boxplot/draw position=5]
table[row sep=\\, y index=0] {
data
0.97 \\
0.995 \\
0.995 \\
0.955 \\
0.995 \\
0.995 \\
0.995 \\
0.995 \\
0.995 \\
0.995 \\
0.995 \\
0.995 \\
0.995 \\
0.995 \\
0.995 \\
0.995 \\
0.995 \\
0.995 \\
0.995 \\
0.995 \\
0.995 \\
0.995 \\
0.995 \\
0.995 \\
0.995 \\
0.845 \\
0.99 \\
0.995 \\
0.995 \\
0.985 \\
};

\addplot[mark=*, boxplot, boxplot/draw position=4]
table[row sep=\\, y index=0] {
data
0.995 \\
0.995 \\
0.945 \\
0.995 \\
0.995 \\
0.995 \\
0.995 \\
0.99 \\
0.94 \\
0.97 \\
0.995 \\
0.995 \\
0.995 \\
0.995 \\
0.995 \\
0.995 \\
0.995 \\
0.965 \\
0.98 \\
0.995 \\
0.995 \\
0.995 \\
0.995 \\
0.995 \\
0.995 \\
0.995 \\
0.995 \\
0.995 \\
0.995 \\
0.995 \\
};

\addplot[mark=*, boxplot, boxplot/draw position=6]
table[row sep=\\, y index=0] {
data
0.995 \\
0.955 \\
0.995 \\
0.995 \\
0.995 \\
0.995 \\
0.995 \\
0.995 \\
0.995 \\
0.995 \\
0.995 \\
0.995 \\
0.965 \\
0.995 \\
0.995 \\
0.995 \\
0.995 \\
0.995 \\
0.69 \\
0.97 \\
0.995 \\
0.995 \\
0.995 \\
0.995 \\
0.995 \\
0.995 \\
0.96 \\
0.995 \\
0.995 \\
0.995 \\
};

\addplot[mark=*, boxplot, boxplot/draw position=7]
table[row sep=\\, y index=0] {
data
0.995 \\
0.995 \\
0.755 \\
0.995 \\
0.995 \\
0.995 \\
0.795 \\
0.725 \\
0.995 \\
0.995 \\
0.995 \\
0.995 \\
0.925 \\
0.845 \\
0.995 \\
0.995 \\
0.93 \\
0.995 \\
0.895 \\
0.82 \\
0.92 \\
0.92 \\
0.995 \\
0.905 \\
0.995 \\
0.995 \\
0.995 \\
0.69 \\
0.895 \\
0.995 \\
};

\addplot[mark=*, boxplot, boxplot/draw position=9]
table[row sep=\\, y index=0] {
data
0.515 \\
0.515 \\
0.515 \\
0.515 \\
0.515 \\
0.515 \\
0.515 \\
0.515 \\
0.515 \\
0.515 \\
0.515 \\
0.515 \\
0.515 \\
0.515 \\
0.515 \\
0.515 \\
0.515 \\
0.515 \\
0.515 \\
0.515 \\
0.515 \\
0.515 \\
0.515 \\
0.515 \\
0.515 \\
0.515 \\
0.515 \\
0.515 \\
0.515 \\
0.515 \\
};

\addplot[mark=*, boxplot, boxplot/draw position=8]
table[row sep=\\, y index=0] {
data
0.515 \\
0.515 \\
0.69 \\
0.87 \\
0.515 \\
0.515 \\
0.995 \\
0.515 \\
0.5 \\
0.995 \\
0.605 \\
0.515 \\
0.995 \\
0.515 \\
0.515 \\
0.515 \\
0.605 \\
0.515 \\
0.995 \\
0.515 \\
0.515 \\
0.995 \\
0.515 \\
0.615 \\
0.515 \\
0.82 \\
0.69 \\
0.515 \\
0.995 \\
0.625 \\
};
}{{0.1}}

%        }
%    }
%    \resizebox{\textwidth}{!}{
%        \subfloat[N=100]{
%            \myboxplot{

\addplot[mark=*, boxplot, boxplot/draw position=1]
table[row sep=\\, y index=0] {
data
0.525 \\
0.545 \\
0.5 \\
0.525 \\
0.565 \\
0.625 \\
0.56 \\
0.445 \\
0.49 \\
0.685 \\
0.79 \\
0.875 \\
0.51 \\
0.57 \\
0.49 \\
0.67 \\
0.56 \\
0.565 \\
0.59 \\
0.67 \\
0.59 \\
0.67 \\
0.58 \\
0.535 \\
0.52 \\
0.515 \\
0.815 \\
0.535 \\
0.59 \\
0.64 \\
};

\addplot[mark=*, boxplot, boxplot/draw position=2]
table[row sep=\\, y index=0] {
data
0.675 \\
0.57 \\
0.775 \\
0.71 \\
0.77 \\
0.715 \\
0.905 \\
0.62 \\
0.855 \\
0.825 \\
0.905 \\
0.795 \\
0.745 \\
0.66 \\
0.765 \\
0.75 \\
0.825 \\
0.9 \\
0.945 \\
0.71 \\
0.58 \\
0.7 \\
0.94 \\
0.83 \\
0.795 \\
0.695 \\
0.94 \\
0.81 \\
0.68 \\
0.505 \\
};

\addplot[mark=*, boxplot, boxplot/draw position=3]
table[row sep=\\, y index=0] {
data
0.995 \\
0.735 \\
0.99 \\
0.83 \\
0.71 \\
0.935 \\
0.875 \\
0.925 \\
0.94 \\
0.94 \\
0.935 \\
0.965 \\
0.855 \\
0.92 \\
0.73 \\
0.995 \\
0.995 \\
0.98 \\
0.995 \\
0.82 \\
0.825 \\
0.965 \\
0.975 \\
0.845 \\
0.985 \\
0.9 \\
0.985 \\
0.99 \\
0.85 \\
0.94 \\
};

\addplot[mark=*, boxplot, boxplot/draw position=5]
table[row sep=\\, y index=0] {
data
0.995 \\
0.995 \\
0.995 \\
0.995 \\
0.995 \\
0.985 \\
0.99 \\
0.995 \\
0.995 \\
0.995 \\
0.995 \\
0.995 \\
0.995 \\
0.995 \\
0.995 \\
0.995 \\
0.995 \\
0.995 \\
0.995 \\
0.995 \\
0.995 \\
0.995 \\
0.995 \\
0.995 \\
0.995 \\
0.995 \\
0.85 \\
0.995 \\
0.905 \\
0.995 \\
};

\addplot[mark=*, boxplot, boxplot/draw position=4]
table[row sep=\\, y index=0] {
data
0.96 \\
0.985 \\
0.995 \\
0.995 \\
0.98 \\
0.995 \\
0.945 \\
0.995 \\
0.975 \\
0.995 \\
0.935 \\
0.96 \\
0.995 \\
0.995 \\
0.955 \\
0.975 \\
0.99 \\
0.995 \\
1.0 \\
0.94 \\
0.885 \\
0.995 \\
0.92 \\
0.995 \\
0.96 \\
0.995 \\
0.99 \\
0.995 \\
0.995 \\
0.995 \\
};

\addplot[mark=*, boxplot, boxplot/draw position=6]
table[row sep=\\, y index=0] {
data
0.995 \\
0.995 \\
0.995 \\
0.995 \\
0.995 \\
0.995 \\
0.995 \\
0.995 \\
0.975 \\
0.975 \\
0.995 \\
0.995 \\
0.995 \\
0.995 \\
0.995 \\
0.995 \\
0.955 \\
0.995 \\
0.995 \\
0.995 \\
0.995 \\
0.995 \\
0.995 \\
0.995 \\
0.965 \\
0.995 \\
0.995 \\
0.995 \\
0.92 \\
0.995 \\
};

\addplot[mark=*, boxplot, boxplot/draw position=7]
table[row sep=\\, y index=0] {
data
0.995 \\
0.995 \\
0.89 \\
0.995 \\
0.995 \\
0.985 \\
0.995 \\
0.95 \\
0.995 \\
0.995 \\
0.995 \\
0.995 \\
0.995 \\
0.995 \\
0.89 \\
0.995 \\
0.995 \\
0.995 \\
0.995 \\
0.995 \\
0.96 \\
0.985 \\
0.995 \\
0.995 \\
0.995 \\
0.975 \\
0.995 \\
0.995 \\
0.995 \\
0.995 \\
};

\addplot[mark=*, boxplot, boxplot/draw position=9]
table[row sep=\\, y index=0] {
data
0.995 \\
0.515 \\
0.635 \\
0.515 \\
0.995 \\
0.69 \\
0.82 \\
0.515 \\
0.6 \\
0.515 \\
0.82 \\
0.515 \\
0.73 \\
0.78 \\
0.515 \\
0.995 \\
0.72 \\
0.515 \\
0.89 \\
0.69 \\
0.515 \\
0.515 \\
0.515 \\
0.515 \\
0.515 \\
0.69 \\
0.905 \\
0.69 \\
0.515 \\
0.905 \\
};

\addplot[mark=*, boxplot, boxplot/draw position=8]
table[row sep=\\, y index=0] {
data
0.995 \\
0.995 \\
0.995 \\
0.515 \\
0.995 \\
0.995 \\
0.995 \\
0.89 \\
0.515 \\
0.79 \\
0.91 \\
0.82 \\
0.995 \\
0.895 \\
0.995 \\
0.995 \\
0.995 \\
0.82 \\
0.56 \\
0.995 \\
0.82 \\
0.995 \\
0.995 \\
0.995 \\
0.995 \\
0.895 \\
0.69 \\
0.935 \\
0.995 \\
0.91 \\
};

\addplot[mark=*, boxplot, boxplot/draw position=10]
table[row sep=\\, y index=0] {
data
0.515 \\
0.515 \\
0.515 \\
0.515 \\
0.515 \\
0.515 \\
0.515 \\
0.515 \\
0.515 \\
0.515 \\
0.515 \\
0.515 \\
0.515 \\
0.515 \\
0.515 \\
0.515 \\
0.515 \\
0.515 \\
0.515 \\
0.515 \\
0.515 \\
0.515 \\
0.515 \\
0.515 \\
0.515 \\
0.515 \\
0.515 \\
0.515 \\
0.515 \\
0.515 \\
};
}{{0.1}}

%        }
%    }
%\end{figure*}

\subsection{Temporal parity with t=3}

We solve for both problems at the same time by creating reservoirs with parameters as specified in table \ref{tab:results:reservoir-parameters-task1}.
The result can be seen in figures \ref{fig:results:task1-ic1} and \ref{fig:results:task1-ic2}.

\begin{table}[ht]
    \centering
    \caption{Reservoir parameters for task 1}
    \label{tab:results:reservoir-parameters-task1}
    \begin{tabular}{ll}
        Connectivity        & 3                         \\
        Nodes               & 10 to 50                  \\
        Input connectivity  & 0 to Nodes (10 step size) \\
        Output connectivity & Nodes                     \\
        Sample size         & 30
    \end{tabular}
\end{table}

\begin{figure*}[ht]
    \centering
    \resizebox{\textwidth}{!}{
        \subfloat[N=10]{
            \myboxplot{

\addplot[mark=*, boxplot, boxplot/draw position=0]
table[row sep=\\, y index=0] {
data
0.535 \\
0.495 \\
0.535 \\
0.465 \\
0.535 \\
0.5 \\
0.5 \\
0.5 \\
0.505 \\
0.555 \\
0.505 \\
0.535 \\
0.535 \\
0.5 \\
0.505 \\
0.48 \\
0.505 \\
0.595 \\
0.535 \\
0.535 \\
0.495 \\
0.535 \\
0.535 \\
0.5 \\
0.505 \\
0.495 \\
0.535 \\
0.535 \\
0.515 \\
0.505 \\
};

\addplot[mark=*, boxplot, boxplot/draw position=1]
table[row sep=\\, y index=0] {
data
0.82 \\
0.695 \\
0.51 \\
0.67 \\
0.535 \\
0.59 \\
0.535 \\
1.0 \\
0.45 \\
0.695 \\
0.705 \\
0.715 \\
0.7 \\
0.83 \\
0.63 \\
0.685 \\
0.745 \\
0.74 \\
0.535 \\
0.61 \\
0.57 \\
0.635 \\
0.48 \\
0.715 \\
0.49 \\
0.635 \\
0.725 \\
0.725 \\
0.535 \\
0.475 \\
};

\addplot[mark=*, boxplot, boxplot/draw position=2]
table[row sep=\\, y index=0] {
data
0.535 \\
0.535 \\
0.535 \\
0.535 \\
0.535 \\
0.535 \\
0.535 \\
0.535 \\
0.535 \\
0.535 \\
0.535 \\
0.535 \\
0.535 \\
0.535 \\
0.535 \\
0.535 \\
0.535 \\
0.535 \\
0.535 \\
0.535 \\
0.535 \\
0.535 \\
0.535 \\
0.535 \\
0.535 \\
0.535 \\
0.535 \\
0.535 \\
0.535 \\
0.535 \\
};
}{0.2}{Input connectivity}

        }
        \subfloat[N=15]{
            \myboxplot{

\addplot[mark=*, boxplot, boxplot/draw position=3]
table[row sep=\\, y index=0] {
data
0.535 \\
0.535 \\
0.535 \\
0.535 \\
0.535 \\
0.535 \\
0.535 \\
0.535 \\
0.535 \\
0.535 \\
0.535 \\
0.535 \\
0.535 \\
0.535 \\
0.535 \\
0.535 \\
0.535 \\
0.535 \\
0.535 \\
0.535 \\
0.535 \\
0.535 \\
0.535 \\
0.535 \\
0.535 \\
0.535 \\
0.535 \\
0.535 \\
0.535 \\
0.535 \\
};

\addplot[mark=*, boxplot, boxplot/draw position=0]
table[row sep=\\, y index=0] {
data
0.535 \\
0.555 \\
0.485 \\
0.515 \\
0.495 \\
0.52 \\
0.46 \\
0.46 \\
0.46 \\
0.44 \\
0.505 \\
0.565 \\
0.515 \\
0.47 \\
0.46 \\
0.535 \\
0.495 \\
0.535 \\
0.43 \\
0.51 \\
0.495 \\
0.535 \\
0.525 \\
0.515 \\
0.51 \\
0.585 \\
0.505 \\
0.495 \\
0.505 \\
0.485 \\
};

\addplot[mark=*, boxplot, boxplot/draw position=1]
table[row sep=\\, y index=0] {
data
0.555 \\
0.56 \\
0.615 \\
0.7 \\
0.765 \\
0.805 \\
0.735 \\
0.87 \\
0.61 \\
0.795 \\
0.87 \\
0.685 \\
0.79 \\
0.675 \\
0.645 \\
0.79 \\
0.705 \\
0.51 \\
0.55 \\
0.725 \\
0.86 \\
0.84 \\
0.645 \\
0.55 \\
0.845 \\
0.965 \\
0.765 \\
0.76 \\
0.595 \\
0.75 \\
};

\addplot[mark=*, boxplot, boxplot/draw position=2]
table[row sep=\\, y index=0] {
data
0.66 \\
0.535 \\
0.54 \\
0.775 \\
0.715 \\
0.535 \\
0.63 \\
0.605 \\
0.74 \\
0.505 \\
0.83 \\
0.64 \\
0.675 \\
0.545 \\
0.535 \\
0.48 \\
0.76 \\
0.875 \\
0.505 \\
0.45 \\
0.45 \\
0.78 \\
0.42 \\
0.535 \\
0.505 \\
0.865 \\
0.625 \\
0.78 \\
0.54 \\
0.84 \\
};
}{0.2}{Input connectivity}{10}

        }
    }
    \resizebox{\textwidth}{!}{
        \subfloat[N=20]{
            \myboxplot{

\addplot[mark=*, boxplot, boxplot/draw position=3]
table[row sep=\\, y index=0] {
data
0.66 \\
0.535 \\
0.505 \\
0.72 \\
0.68 \\
0.63 \\
0.825 \\
0.925 \\
0.725 \\
0.725 \\
0.505 \\
0.675 \\
0.765 \\
0.78 \\
0.505 \\
0.475 \\
0.535 \\
0.505 \\
0.795 \\
0.78 \\
0.72 \\
0.505 \\
0.655 \\
0.505 \\
0.385 \\
0.45 \\
0.505 \\
0.65 \\
0.505 \\
0.535 \\
};

\addplot[mark=*, boxplot, boxplot/draw position=0]
table[row sep=\\, y index=0] {
data
0.505 \\
0.535 \\
0.535 \\
0.51 \\
0.505 \\
0.545 \\
0.535 \\
0.435 \\
0.535 \\
0.56 \\
0.485 \\
0.5 \\
0.535 \\
0.5 \\
0.43 \\
0.46 \\
0.505 \\
0.515 \\
0.52 \\
0.535 \\
0.5 \\
0.535 \\
0.535 \\
0.515 \\
0.475 \\
0.565 \\
0.485 \\
0.46 \\
0.495 \\
0.535 \\
};

\addplot[mark=*, boxplot, boxplot/draw position=4]
table[row sep=\\, y index=0] {
data
0.535 \\
0.535 \\
0.535 \\
0.535 \\
0.535 \\
0.535 \\
0.535 \\
0.535 \\
0.535 \\
0.535 \\
0.535 \\
0.535 \\
0.535 \\
0.535 \\
0.535 \\
0.535 \\
0.535 \\
0.535 \\
0.535 \\
0.535 \\
0.535 \\
0.535 \\
0.535 \\
0.535 \\
0.535 \\
0.535 \\
0.535 \\
0.535 \\
0.535 \\
0.535 \\
};

\addplot[mark=*, boxplot, boxplot/draw position=1]
table[row sep=\\, y index=0] {
data
0.675 \\
0.67 \\
0.845 \\
0.9 \\
0.785 \\
0.83 \\
0.69 \\
0.875 \\
0.845 \\
0.715 \\
0.68 \\
0.67 \\
0.445 \\
0.545 \\
0.61 \\
0.57 \\
0.52 \\
0.665 \\
0.7 \\
0.585 \\
0.845 \\
0.735 \\
0.735 \\
0.67 \\
0.51 \\
0.675 \\
0.52 \\
0.93 \\
0.535 \\
0.9 \\
};

\addplot[mark=*, boxplot, boxplot/draw position=2]
table[row sep=\\, y index=0] {
data
0.895 \\
0.725 \\
0.695 \\
1.0 \\
0.71 \\
0.75 \\
0.72 \\
0.83 \\
0.85 \\
0.905 \\
0.925 \\
0.93 \\
0.585 \\
0.845 \\
0.89 \\
1.0 \\
0.82 \\
0.765 \\
0.795 \\
0.725 \\
0.79 \\
1.0 \\
0.91 \\
0.755 \\
0.79 \\
0.695 \\
0.81 \\
0.485 \\
0.885 \\
0.86 \\
};
}{0.2}{Input connectivity}

        }
        \subfloat[N=25]{
            \myboxplot{

\addplot[mark=*, boxplot, boxplot/draw position=5]
table[row sep=\\, y index=0] {
data
0.535 \\
0.535 \\
0.535 \\
0.535 \\
0.535 \\
0.535 \\
0.535 \\
0.535 \\
0.535 \\
0.535 \\
0.535 \\
0.535 \\
0.535 \\
0.535 \\
0.535 \\
0.535 \\
0.535 \\
0.535 \\
0.535 \\
0.535 \\
0.535 \\
0.535 \\
0.535 \\
0.535 \\
0.535 \\
0.535 \\
0.535 \\
0.535 \\
0.535 \\
0.535 \\
};

\addplot[mark=*, boxplot, boxplot/draw position=3]
table[row sep=\\, y index=0] {
data
0.74 \\
0.8 \\
0.845 \\
0.85 \\
0.71 \\
0.725 \\
0.905 \\
0.655 \\
1.0 \\
0.91 \\
1.0 \\
1.0 \\
0.575 \\
0.715 \\
0.865 \\
0.89 \\
0.655 \\
0.69 \\
0.45 \\
0.68 \\
1.0 \\
0.78 \\
0.935 \\
0.975 \\
0.93 \\
0.785 \\
0.92 \\
0.815 \\
0.84 \\
0.655 \\
};

\addplot[mark=*, boxplot, boxplot/draw position=2]
table[row sep=\\, y index=0] {
data
0.985 \\
0.79 \\
0.94 \\
0.995 \\
0.975 \\
0.87 \\
0.74 \\
0.775 \\
0.66 \\
0.65 \\
0.675 \\
0.965 \\
0.625 \\
0.695 \\
0.815 \\
0.78 \\
0.875 \\
0.745 \\
0.73 \\
0.98 \\
0.73 \\
0.45 \\
0.94 \\
0.505 \\
0.805 \\
1.0 \\
1.0 \\
0.385 \\
0.98 \\
0.8 \\
};

\addplot[mark=*, boxplot, boxplot/draw position=0]
table[row sep=\\, y index=0] {
data
0.505 \\
0.52 \\
0.515 \\
0.495 \\
0.535 \\
0.535 \\
0.475 \\
0.495 \\
0.5 \\
0.45 \\
0.535 \\
0.56 \\
0.46 \\
0.52 \\
0.555 \\
0.495 \\
0.515 \\
0.52 \\
0.495 \\
0.51 \\
0.465 \\
0.52 \\
0.535 \\
0.45 \\
0.5 \\
0.52 \\
0.505 \\
0.495 \\
0.49 \\
0.46 \\
};

\addplot[mark=*, boxplot, boxplot/draw position=1]
table[row sep=\\, y index=0] {
data
0.7 \\
0.515 \\
0.76 \\
0.615 \\
0.705 \\
0.535 \\
0.51 \\
0.655 \\
0.48 \\
0.56 \\
0.555 \\
0.73 \\
0.64 \\
0.615 \\
0.485 \\
0.79 \\
0.61 \\
0.635 \\
0.5 \\
0.905 \\
0.55 \\
0.505 \\
0.485 \\
0.735 \\
0.55 \\
0.725 \\
0.35 \\
0.785 \\
0.91 \\
0.725 \\
};

\addplot[mark=*, boxplot, boxplot/draw position=4]
table[row sep=\\, y index=0] {
data
0.795 \\
0.75 \\
0.47 \\
0.505 \\
0.705 \\
0.485 \\
0.775 \\
0.61 \\
0.535 \\
0.505 \\
0.78 \\
0.505 \\
0.51 \\
0.505 \\
0.505 \\
0.42 \\
0.505 \\
0.555 \\
0.45 \\
0.47 \\
0.625 \\
0.505 \\
0.96 \\
0.785 \\
0.505 \\
0.505 \\
0.535 \\
0.505 \\
0.85 \\
0.505 \\
};
}{0.2}{Input connectivity}{10}

        }
    }
    \resizebox{\textwidth}{!}{
        \subfloat[N=30]{
            \myboxplot{

\addplot[mark=*, boxplot, boxplot/draw position=5]
table[row sep=\\, y index=0] {
data
0.725 \\
0.745 \\
0.675 \\
0.505 \\
0.555 \\
0.485 \\
0.505 \\
0.745 \\
0.505 \\
0.535 \\
0.855 \\
0.75 \\
0.505 \\
0.535 \\
0.535 \\
1.0 \\
0.845 \\
1.0 \\
0.505 \\
0.565 \\
0.78 \\
0.505 \\
0.505 \\
0.635 \\
0.725 \\
0.845 \\
0.78 \\
0.505 \\
0.505 \\
0.505 \\
};

\addplot[mark=*, boxplot, boxplot/draw position=3]
table[row sep=\\, y index=0] {
data
0.94 \\
0.85 \\
0.975 \\
0.89 \\
0.775 \\
0.945 \\
0.73 \\
0.71 \\
0.765 \\
1.0 \\
1.0 \\
0.84 \\
0.845 \\
0.89 \\
0.88 \\
0.655 \\
0.96 \\
0.655 \\
0.965 \\
0.83 \\
0.85 \\
1.0 \\
0.8 \\
0.915 \\
0.79 \\
0.87 \\
0.98 \\
0.45 \\
1.0 \\
0.705 \\
};

\addplot[mark=*, boxplot, boxplot/draw position=2]
table[row sep=\\, y index=0] {
data
0.925 \\
0.88 \\
0.945 \\
0.685 \\
0.78 \\
0.995 \\
0.92 \\
0.725 \\
0.865 \\
0.8 \\
0.81 \\
0.885 \\
0.7 \\
0.945 \\
0.96 \\
0.88 \\
0.91 \\
0.99 \\
0.875 \\
0.64 \\
0.965 \\
0.635 \\
0.91 \\
0.96 \\
0.87 \\
0.9 \\
0.685 \\
0.915 \\
0.87 \\
0.765 \\
};

\addplot[mark=*, boxplot, boxplot/draw position=6]
table[row sep=\\, y index=0] {
data
0.535 \\
0.535 \\
0.535 \\
0.535 \\
0.535 \\
0.535 \\
0.535 \\
0.535 \\
0.535 \\
0.535 \\
0.535 \\
0.535 \\
0.535 \\
0.535 \\
0.535 \\
0.535 \\
0.535 \\
0.535 \\
0.535 \\
0.535 \\
0.535 \\
0.535 \\
0.535 \\
0.535 \\
0.535 \\
0.535 \\
0.535 \\
0.535 \\
0.535 \\
0.535 \\
};

\addplot[mark=*, boxplot, boxplot/draw position=0]
table[row sep=\\, y index=0] {
data
0.4 \\
0.48 \\
0.54 \\
0.47 \\
0.485 \\
0.52 \\
0.46 \\
0.45 \\
0.48 \\
0.53 \\
0.535 \\
0.52 \\
0.455 \\
0.465 \\
0.46 \\
0.495 \\
0.53 \\
0.51 \\
0.515 \\
0.44 \\
0.495 \\
0.46 \\
0.505 \\
0.535 \\
0.46 \\
0.515 \\
0.555 \\
0.495 \\
0.455 \\
0.51 \\
};

\addplot[mark=*, boxplot, boxplot/draw position=1]
table[row sep=\\, y index=0] {
data
0.595 \\
0.8 \\
0.865 \\
0.505 \\
0.525 \\
0.455 \\
0.66 \\
0.5 \\
0.915 \\
0.52 \\
0.62 \\
0.895 \\
0.525 \\
0.68 \\
0.505 \\
0.875 \\
0.61 \\
0.57 \\
0.63 \\
0.485 \\
0.53 \\
0.605 \\
0.695 \\
0.565 \\
0.55 \\
0.71 \\
0.705 \\
0.73 \\
0.575 \\
0.54 \\
};

\addplot[mark=*, boxplot, boxplot/draw position=4]
table[row sep=\\, y index=0] {
data
0.995 \\
0.715 \\
0.505 \\
0.78 \\
0.715 \\
0.945 \\
0.81 \\
0.515 \\
0.66 \\
0.78 \\
1.0 \\
0.895 \\
0.775 \\
0.83 \\
1.0 \\
0.915 \\
1.0 \\
1.0 \\
0.505 \\
0.405 \\
0.77 \\
1.0 \\
0.845 \\
0.505 \\
0.725 \\
1.0 \\
1.0 \\
0.725 \\
0.725 \\
0.725 \\
};
}{0.2}{Input connectivity}

        }
        \subfloat[N=35]{
            \myboxplot{

\addplot[mark=*, boxplot, boxplot/draw position=5]
table[row sep=\\, y index=0] {
data
0.505 \\
0.505 \\
0.705 \\
0.74 \\
0.725 \\
0.655 \\
1.0 \\
0.78 \\
0.505 \\
0.65 \\
0.95 \\
1.0 \\
0.465 \\
0.62 \\
0.795 \\
0.79 \\
0.79 \\
0.78 \\
0.895 \\
0.66 \\
0.785 \\
1.0 \\
1.0 \\
0.705 \\
0.645 \\
0.865 \\
0.55 \\
0.845 \\
0.505 \\
0.385 \\
};

\addplot[mark=*, boxplot, boxplot/draw position=3]
table[row sep=\\, y index=0] {
data
0.92 \\
0.705 \\
0.925 \\
0.735 \\
0.87 \\
0.685 \\
0.87 \\
0.965 \\
0.85 \\
0.94 \\
0.99 \\
0.885 \\
0.775 \\
0.9 \\
1.0 \\
0.88 \\
0.805 \\
0.79 \\
1.0 \\
0.76 \\
0.935 \\
0.98 \\
0.94 \\
0.815 \\
0.555 \\
0.685 \\
0.895 \\
0.83 \\
0.89 \\
0.96 \\
};

\addplot[mark=*, boxplot, boxplot/draw position=2]
table[row sep=\\, y index=0] {
data
0.685 \\
0.695 \\
0.79 \\
0.915 \\
0.62 \\
0.5 \\
0.605 \\
0.75 \\
0.945 \\
0.665 \\
0.68 \\
0.89 \\
0.705 \\
0.875 \\
0.805 \\
0.895 \\
0.635 \\
0.78 \\
0.815 \\
0.97 \\
0.745 \\
0.525 \\
0.505 \\
0.725 \\
0.55 \\
0.575 \\
0.965 \\
0.775 \\
0.705 \\
0.84 \\
};

\addplot[mark=*, boxplot, boxplot/draw position=6]
table[row sep=\\, y index=0] {
data
0.58 \\
0.505 \\
0.61 \\
1.0 \\
0.505 \\
0.45 \\
0.45 \\
0.755 \\
0.535 \\
0.485 \\
0.45 \\
0.46 \\
0.505 \\
0.505 \\
0.505 \\
0.505 \\
0.505 \\
0.88 \\
0.635 \\
1.0 \\
0.78 \\
1.0 \\
0.66 \\
0.72 \\
0.625 \\
0.505 \\
0.505 \\
0.505 \\
1.0 \\
0.865 \\
};

\addplot[mark=*, boxplot, boxplot/draw position=7]
table[row sep=\\, y index=0] {
data
0.535 \\
0.535 \\
0.535 \\
0.535 \\
0.535 \\
0.535 \\
0.535 \\
0.535 \\
0.535 \\
0.535 \\
0.535 \\
0.535 \\
0.535 \\
0.535 \\
0.535 \\
0.535 \\
0.535 \\
0.535 \\
0.535 \\
0.535 \\
0.535 \\
0.535 \\
0.535 \\
0.535 \\
0.535 \\
0.535 \\
0.535 \\
0.535 \\
0.535 \\
0.535 \\
};

\addplot[mark=*, boxplot, boxplot/draw position=0]
table[row sep=\\, y index=0] {
data
0.535 \\
0.58 \\
0.48 \\
0.555 \\
0.505 \\
0.5 \\
0.535 \\
0.48 \\
0.455 \\
0.46 \\
0.46 \\
0.495 \\
0.48 \\
0.48 \\
0.47 \\
0.525 \\
0.53 \\
0.505 \\
0.535 \\
0.515 \\
0.47 \\
0.495 \\
0.48 \\
0.535 \\
0.47 \\
0.535 \\
0.445 \\
0.535 \\
0.525 \\
0.55 \\
};

\addplot[mark=*, boxplot, boxplot/draw position=1]
table[row sep=\\, y index=0] {
data
0.595 \\
0.505 \\
0.535 \\
0.495 \\
0.48 \\
0.565 \\
0.78 \\
0.67 \\
0.46 \\
0.59 \\
0.625 \\
0.575 \\
0.54 \\
0.99 \\
0.595 \\
0.445 \\
0.495 \\
0.58 \\
0.545 \\
0.54 \\
0.675 \\
0.625 \\
0.905 \\
0.7 \\
0.725 \\
0.565 \\
0.51 \\
0.685 \\
0.545 \\
0.675 \\
};

\addplot[mark=*, boxplot, boxplot/draw position=4]
table[row sep=\\, y index=0] {
data
0.97 \\
0.985 \\
0.925 \\
1.0 \\
0.85 \\
1.0 \\
0.945 \\
1.0 \\
0.795 \\
1.0 \\
0.84 \\
1.0 \\
1.0 \\
0.83 \\
0.94 \\
0.88 \\
1.0 \\
0.85 \\
1.0 \\
0.795 \\
0.985 \\
1.0 \\
0.88 \\
0.71 \\
0.88 \\
1.0 \\
1.0 \\
0.665 \\
0.765 \\
0.865 \\
};
}{0.2}{Input connectivity}

        }
    }
    \label{fig:results:task1-ic1}
    \caption{Plots for task 1 - part 1}
\end{figure*}

\begin{figure*}[ht]
    \centering
    \resizebox{\textwidth}{!}{
        \subfloat[N=40]{
            \myboxplot{

\addplot[mark=*, boxplot, boxplot/draw position=5]
table[row sep=\\, y index=0] {
data
0.905 \\
1.0 \\
0.685 \\
1.0 \\
0.865 \\
0.845 \\
1.0 \\
1.0 \\
1.0 \\
0.85 \\
0.97 \\
0.935 \\
1.0 \\
1.0 \\
0.9 \\
0.94 \\
0.97 \\
0.975 \\
0.95 \\
1.0 \\
0.88 \\
0.75 \\
0.95 \\
0.735 \\
0.65 \\
0.925 \\
1.0 \\
0.955 \\
0.905 \\
0.985 \\
};

\addplot[mark=*, boxplot, boxplot/draw position=3]
table[row sep=\\, y index=0] {
data
0.58 \\
0.93 \\
1.0 \\
0.945 \\
0.905 \\
0.87 \\
0.545 \\
0.96 \\
0.91 \\
0.655 \\
0.855 \\
0.895 \\
0.93 \\
1.0 \\
1.0 \\
0.89 \\
0.95 \\
0.72 \\
0.76 \\
0.985 \\
0.675 \\
1.0 \\
0.855 \\
1.0 \\
0.955 \\
0.985 \\
0.845 \\
0.735 \\
0.97 \\
0.96 \\
};

\addplot[mark=*, boxplot, boxplot/draw position=2]
table[row sep=\\, y index=0] {
data
0.94 \\
0.605 \\
0.72 \\
0.77 \\
0.71 \\
0.875 \\
0.735 \\
0.55 \\
0.835 \\
0.675 \\
0.585 \\
0.6 \\
0.84 \\
0.695 \\
0.85 \\
0.93 \\
0.91 \\
0.95 \\
0.775 \\
0.685 \\
0.84 \\
0.975 \\
0.94 \\
0.715 \\
0.825 \\
0.685 \\
0.935 \\
0.84 \\
0.79 \\
0.87 \\
};

\addplot[mark=*, boxplot, boxplot/draw position=6]
table[row sep=\\, y index=0] {
data
0.765 \\
0.825 \\
0.735 \\
0.83 \\
0.505 \\
0.505 \\
0.775 \\
0.725 \\
0.51 \\
0.895 \\
0.505 \\
0.85 \\
1.0 \\
0.91 \\
0.75 \\
0.91 \\
0.73 \\
1.0 \\
0.725 \\
1.0 \\
1.0 \\
0.875 \\
0.52 \\
0.81 \\
0.505 \\
0.65 \\
0.505 \\
0.75 \\
0.555 \\
0.505 \\
};

\addplot[mark=*, boxplot, boxplot/draw position=7]
table[row sep=\\, y index=0] {
data
0.535 \\
0.85 \\
0.505 \\
0.505 \\
0.535 \\
0.385 \\
0.505 \\
0.505 \\
0.505 \\
0.535 \\
0.885 \\
0.535 \\
0.785 \\
0.505 \\
0.75 \\
0.505 \\
0.45 \\
0.725 \\
1.0 \\
0.505 \\
0.535 \\
0.635 \\
0.655 \\
0.685 \\
1.0 \\
0.505 \\
0.505 \\
0.505 \\
0.505 \\
0.66 \\
};

\addplot[mark=*, boxplot, boxplot/draw position=0]
table[row sep=\\, y index=0] {
data
0.49 \\
0.535 \\
0.49 \\
0.535 \\
0.45 \\
0.535 \\
0.575 \\
0.535 \\
0.47 \\
0.535 \\
0.46 \\
0.545 \\
0.485 \\
0.485 \\
0.405 \\
0.515 \\
0.585 \\
0.52 \\
0.565 \\
0.53 \\
0.435 \\
0.49 \\
0.515 \\
0.46 \\
0.48 \\
0.43 \\
0.49 \\
0.475 \\
0.435 \\
0.485 \\
};

\addplot[mark=*, boxplot, boxplot/draw position=1]
table[row sep=\\, y index=0] {
data
0.61 \\
0.765 \\
0.665 \\
0.77 \\
0.515 \\
0.57 \\
0.63 \\
0.465 \\
0.5 \\
0.55 \\
0.5 \\
0.635 \\
0.63 \\
0.525 \\
0.515 \\
0.875 \\
0.595 \\
0.51 \\
0.52 \\
0.7 \\
0.53 \\
0.725 \\
0.585 \\
0.58 \\
0.735 \\
0.53 \\
0.735 \\
0.745 \\
0.55 \\
0.8 \\
};

\addplot[mark=*, boxplot, boxplot/draw position=8]
table[row sep=\\, y index=0] {
data
0.535 \\
0.535 \\
0.535 \\
0.535 \\
0.535 \\
0.535 \\
0.535 \\
0.535 \\
0.535 \\
0.535 \\
0.535 \\
0.535 \\
0.535 \\
0.535 \\
0.535 \\
0.535 \\
0.535 \\
0.535 \\
0.535 \\
0.535 \\
0.535 \\
0.535 \\
0.535 \\
0.535 \\
0.535 \\
0.535 \\
0.535 \\
0.535 \\
0.535 \\
0.535 \\
};

\addplot[mark=*, boxplot, boxplot/draw position=4]
table[row sep=\\, y index=0] {
data
0.89 \\
0.835 \\
0.705 \\
0.815 \\
1.0 \\
1.0 \\
1.0 \\
1.0 \\
0.95 \\
0.925 \\
0.97 \\
0.925 \\
0.885 \\
0.895 \\
0.885 \\
0.965 \\
0.99 \\
0.97 \\
0.915 \\
0.825 \\
0.445 \\
0.755 \\
0.88 \\
0.91 \\
0.905 \\
0.96 \\
1.0 \\
0.985 \\
0.81 \\
0.865 \\
};
}{0.2}{Input connectivity}

        }
        \subfloat[N=45]{
            \myboxplot{

\addplot[mark=*, boxplot, boxplot/draw position=5]
table[row sep=\\, y index=0] {
data
1.0 \\
1.0 \\
0.955 \\
0.845 \\
1.0 \\
1.0 \\
1.0 \\
0.97 \\
0.975 \\
0.825 \\
0.38 \\
0.875 \\
1.0 \\
0.975 \\
1.0 \\
0.865 \\
0.99 \\
1.0 \\
0.7 \\
0.72 \\
0.95 \\
0.97 \\
1.0 \\
1.0 \\
1.0 \\
0.91 \\
0.99 \\
1.0 \\
1.0 \\
0.86 \\
};

\addplot[mark=*, boxplot, boxplot/draw position=9]
table[row sep=\\, y index=0] {
data
0.535 \\
0.535 \\
0.535 \\
0.535 \\
0.535 \\
0.535 \\
0.535 \\
0.535 \\
0.535 \\
0.535 \\
0.535 \\
0.535 \\
0.535 \\
0.535 \\
0.535 \\
0.535 \\
0.535 \\
0.535 \\
0.535 \\
0.535 \\
0.535 \\
0.535 \\
0.535 \\
0.535 \\
0.535 \\
0.535 \\
0.535 \\
0.535 \\
0.535 \\
0.535 \\
};

\addplot[mark=*, boxplot, boxplot/draw position=3]
table[row sep=\\, y index=0] {
data
0.675 \\
0.675 \\
0.935 \\
0.905 \\
0.96 \\
0.945 \\
0.93 \\
0.86 \\
0.86 \\
0.96 \\
0.9 \\
0.63 \\
0.885 \\
0.825 \\
1.0 \\
0.815 \\
0.945 \\
0.915 \\
0.915 \\
0.615 \\
1.0 \\
0.94 \\
0.895 \\
0.9 \\
0.885 \\
0.85 \\
0.835 \\
0.845 \\
0.715 \\
0.94 \\
};

\addplot[mark=*, boxplot, boxplot/draw position=2]
table[row sep=\\, y index=0] {
data
0.865 \\
0.635 \\
0.66 \\
0.9 \\
0.87 \\
0.56 \\
0.905 \\
0.53 \\
0.805 \\
0.685 \\
0.945 \\
0.695 \\
0.75 \\
0.51 \\
0.685 \\
0.635 \\
0.56 \\
0.875 \\
0.595 \\
0.78 \\
0.785 \\
0.635 \\
0.6 \\
0.87 \\
0.855 \\
0.685 \\
0.76 \\
0.875 \\
0.795 \\
0.735 \\
};

\addplot[mark=*, boxplot, boxplot/draw position=6]
table[row sep=\\, y index=0] {
data
1.0 \\
1.0 \\
1.0 \\
0.95 \\
0.78 \\
0.63 \\
0.96 \\
0.805 \\
1.0 \\
1.0 \\
1.0 \\
1.0 \\
0.96 \\
1.0 \\
0.66 \\
0.83 \\
0.755 \\
0.725 \\
0.925 \\
0.725 \\
0.915 \\
1.0 \\
0.835 \\
0.965 \\
0.785 \\
1.0 \\
1.0 \\
1.0 \\
1.0 \\
0.78 \\
};

\addplot[mark=*, boxplot, boxplot/draw position=7]
table[row sep=\\, y index=0] {
data
0.755 \\
0.785 \\
0.62 \\
1.0 \\
0.45 \\
0.88 \\
0.635 \\
0.79 \\
0.75 \\
0.505 \\
0.505 \\
0.78 \\
0.75 \\
0.775 \\
0.725 \\
0.73 \\
0.93 \\
0.845 \\
1.0 \\
1.0 \\
0.86 \\
0.505 \\
0.555 \\
0.88 \\
0.505 \\
0.65 \\
1.0 \\
0.51 \\
0.505 \\
0.935 \\
};

\addplot[mark=*, boxplot, boxplot/draw position=0]
table[row sep=\\, y index=0] {
data
0.52 \\
0.505 \\
0.5 \\
0.58 \\
0.49 \\
0.48 \\
0.455 \\
0.495 \\
0.535 \\
0.485 \\
0.485 \\
0.445 \\
0.48 \\
0.52 \\
0.47 \\
0.51 \\
0.535 \\
0.535 \\
0.53 \\
0.46 \\
0.5 \\
0.495 \\
0.535 \\
0.515 \\
0.445 \\
0.47 \\
0.51 \\
0.48 \\
0.535 \\
0.465 \\
};

\addplot[mark=*, boxplot, boxplot/draw position=1]
table[row sep=\\, y index=0] {
data
0.53 \\
0.55 \\
0.535 \\
0.45 \\
0.5 \\
0.525 \\
0.9 \\
0.56 \\
0.48 \\
0.67 \\
0.945 \\
0.505 \\
0.63 \\
0.55 \\
0.605 \\
0.59 \\
0.53 \\
0.575 \\
0.465 \\
0.515 \\
0.56 \\
0.49 \\
0.555 \\
0.625 \\
0.82 \\
0.57 \\
0.48 \\
0.595 \\
0.72 \\
0.545 \\
};

\addplot[mark=*, boxplot, boxplot/draw position=8]
table[row sep=\\, y index=0] {
data
0.505 \\
0.505 \\
0.4 \\
0.505 \\
0.535 \\
0.725 \\
0.635 \\
0.535 \\
0.505 \\
0.505 \\
0.78 \\
0.505 \\
0.57 \\
0.79 \\
0.66 \\
0.505 \\
0.63 \\
0.7 \\
0.725 \\
0.505 \\
0.505 \\
0.535 \\
0.475 \\
0.505 \\
0.505 \\
0.505 \\
0.685 \\
0.495 \\
0.505 \\
0.685 \\
};

\addplot[mark=*, boxplot, boxplot/draw position=4]
table[row sep=\\, y index=0] {
data
0.975 \\
0.86 \\
0.86 \\
0.825 \\
0.92 \\
0.695 \\
1.0 \\
0.945 \\
0.935 \\
0.925 \\
0.975 \\
0.97 \\
1.0 \\
0.92 \\
1.0 \\
1.0 \\
0.995 \\
0.965 \\
0.94 \\
0.945 \\
0.795 \\
0.84 \\
1.0 \\
0.89 \\
1.0 \\
0.99 \\
0.905 \\
0.985 \\
0.885 \\
0.945 \\
};
}{0.2}{Input connectivity}{10}

        }
    }
    \resizebox{0.5\textwidth}{!}{
        \subfloat[N=50]{
            \myboxplot{

\addplot[mark=*, boxplot, boxplot/draw position=5]
table[row sep=\\, y index=0] {
data
1.0 \\
1.0 \\
0.995 \\
0.705 \\
1.0 \\
0.8 \\
0.94 \\
0.86 \\
0.765 \\
0.81 \\
0.45 \\
1.0 \\
0.97 \\
0.95 \\
0.885 \\
0.965 \\
0.97 \\
0.975 \\
1.0 \\
1.0 \\
0.995 \\
0.985 \\
1.0 \\
0.89 \\
0.96 \\
0.975 \\
1.0 \\
1.0 \\
0.87 \\
0.975 \\
};

\addplot[mark=*, boxplot, boxplot/draw position=9]
table[row sep=\\, y index=0] {
data
0.785 \\
0.505 \\
0.505 \\
0.505 \\
1.0 \\
0.505 \\
0.505 \\
0.505 \\
0.505 \\
0.91 \\
0.505 \\
0.535 \\
0.52 \\
0.73 \\
0.505 \\
0.85 \\
0.66 \\
0.505 \\
0.505 \\
0.725 \\
0.45 \\
0.505 \\
0.725 \\
0.505 \\
0.785 \\
0.63 \\
0.505 \\
0.505 \\
0.505 \\
0.505 \\
};

\addplot[mark=*, boxplot, boxplot/draw position=3]
table[row sep=\\, y index=0] {
data
0.77 \\
0.82 \\
0.775 \\
0.61 \\
0.695 \\
0.91 \\
0.88 \\
0.85 \\
0.995 \\
0.745 \\
0.975 \\
0.85 \\
0.835 \\
0.94 \\
0.925 \\
0.67 \\
0.745 \\
0.8 \\
0.815 \\
0.97 \\
0.71 \\
0.685 \\
0.955 \\
0.975 \\
0.805 \\
0.69 \\
0.815 \\
0.68 \\
0.93 \\
0.68 \\
};

\addplot[mark=*, boxplot, boxplot/draw position=2]
table[row sep=\\, y index=0] {
data
0.815 \\
0.955 \\
0.865 \\
0.84 \\
0.565 \\
0.765 \\
0.65 \\
0.565 \\
0.815 \\
0.83 \\
0.8 \\
0.775 \\
0.775 \\
0.735 \\
0.52 \\
0.67 \\
0.57 \\
0.785 \\
0.5 \\
0.77 \\
0.76 \\
0.815 \\
0.9 \\
0.855 \\
0.915 \\
0.775 \\
0.6 \\
0.665 \\
0.705 \\
0.54 \\
};

\addplot[mark=*, boxplot, boxplot/draw position=6]
table[row sep=\\, y index=0] {
data
0.63 \\
0.725 \\
0.99 \\
0.905 \\
0.965 \\
1.0 \\
1.0 \\
0.865 \\
1.0 \\
0.965 \\
0.935 \\
0.975 \\
0.85 \\
0.77 \\
1.0 \\
1.0 \\
1.0 \\
0.685 \\
0.965 \\
0.935 \\
1.0 \\
0.98 \\
1.0 \\
0.87 \\
1.0 \\
1.0 \\
0.98 \\
0.925 \\
1.0 \\
0.925 \\
};

\addplot[mark=*, boxplot, boxplot/draw position=10]
table[row sep=\\, y index=0] {
data
0.535 \\
0.535 \\
0.535 \\
0.535 \\
0.535 \\
0.535 \\
0.535 \\
0.535 \\
0.535 \\
0.535 \\
0.535 \\
0.535 \\
0.535 \\
0.535 \\
0.535 \\
0.535 \\
0.535 \\
0.535 \\
0.535 \\
0.535 \\
0.535 \\
0.535 \\
0.535 \\
0.535 \\
0.535 \\
0.535 \\
0.535 \\
0.535 \\
0.535 \\
0.535 \\
};

\addplot[mark=*, boxplot, boxplot/draw position=7]
table[row sep=\\, y index=0] {
data
0.85 \\
0.915 \\
0.505 \\
1.0 \\
0.835 \\
0.605 \\
1.0 \\
0.915 \\
1.0 \\
1.0 \\
1.0 \\
0.845 \\
0.92 \\
1.0 \\
0.79 \\
1.0 \\
0.785 \\
0.715 \\
1.0 \\
0.88 \\
1.0 \\
1.0 \\
0.945 \\
1.0 \\
1.0 \\
0.845 \\
0.75 \\
0.545 \\
0.505 \\
0.97 \\
};

\addplot[mark=*, boxplot, boxplot/draw position=0]
table[row sep=\\, y index=0] {
data
0.55 \\
0.475 \\
0.535 \\
0.465 \\
0.49 \\
0.515 \\
0.51 \\
0.48 \\
0.525 \\
0.49 \\
0.455 \\
0.495 \\
0.445 \\
0.45 \\
0.46 \\
0.46 \\
0.47 \\
0.46 \\
0.49 \\
0.515 \\
0.505 \\
0.555 \\
0.495 \\
0.515 \\
0.51 \\
0.52 \\
0.55 \\
0.475 \\
0.5 \\
0.47 \\
};

\addplot[mark=*, boxplot, boxplot/draw position=1]
table[row sep=\\, y index=0] {
data
0.48 \\
0.505 \\
0.55 \\
0.65 \\
0.68 \\
0.63 \\
0.52 \\
0.6 \\
0.51 \\
0.585 \\
0.52 \\
0.48 \\
0.54 \\
0.575 \\
0.585 \\
0.65 \\
0.555 \\
0.635 \\
0.49 \\
0.565 \\
0.655 \\
0.525 \\
0.59 \\
0.495 \\
0.585 \\
0.585 \\
0.5 \\
0.445 \\
0.565 \\
0.68 \\
};

\addplot[mark=*, boxplot, boxplot/draw position=8]
table[row sep=\\, y index=0] {
data
0.78 \\
0.505 \\
0.505 \\
0.505 \\
0.85 \\
0.505 \\
0.505 \\
0.505 \\
0.535 \\
0.57 \\
0.49 \\
0.515 \\
0.72 \\
0.715 \\
0.65 \\
1.0 \\
0.94 \\
0.505 \\
0.915 \\
1.0 \\
0.985 \\
0.505 \\
0.605 \\
0.505 \\
0.505 \\
0.765 \\
0.505 \\
0.675 \\
0.505 \\
1.0 \\
};

\addplot[mark=*, boxplot, boxplot/draw position=4]
table[row sep=\\, y index=0] {
data
0.97 \\
1.0 \\
0.695 \\
0.915 \\
0.965 \\
1.0 \\
0.985 \\
1.0 \\
1.0 \\
0.925 \\
0.995 \\
0.965 \\
0.85 \\
0.965 \\
1.0 \\
0.775 \\
1.0 \\
1.0 \\
0.94 \\
0.815 \\
0.79 \\
0.9 \\
1.0 \\
1.0 \\
0.975 \\
0.92 \\
1.0 \\
0.875 \\
0.88 \\
0.865 \\
};
}{0.2}{Input connectivity}{10}

        }
    }
    \label{fig:results:task1-ic2}
    \caption{Plots for task 1 - part 2}
\end{figure*}

We see that even for the smallest reservoir combination (N=10, IC=5) there are reservoirs able to solve the task with an accuracy of 100\%, even though they are outliers compared to the rest of the distribution.
All larger reservoirs are able to solve the task, with the mean accuracy of the distribution increasing as the size of the reservoir increases.

The optimal input connectivity can be identified visually as being roughly $IC = Nodes / 2$ (the x axis is scaled from 0 to Nodes).
This indicates that larger networks need a larger degree of perturbance to be able to accurately compute the result,
but that the optimal connectivity for all these reservoirs should lie at one half of the number of nodes seems rather convenient.
\todo{Should change x-axis to be consistently to 50 for more correct comparisons}

\subsection{Temporal parity with t=5}

A much larger network is required before the required accuracy is achieved on this task.
At N=80 there is an outlier able to achieve 100\% accuracy, but consistent 100\% accuracy isn't achieved before N=100, then at a connectivity of 55.

Observe that for each reservoir size, the accuracy distribution is the greatest for input connectivities close to $Nodes / 2$.
Again the x axis has been scaled to a max value of the current reservoirs number of nodes.

\begin{figure*}[ht]
    \centering
    \resizebox{\textwidth}{!}{
        \subfloat[N=10]{
            \myboxplot{

\addplot[mark=*, boxplot, boxplot/draw position=0]
table[row sep=\\, y index=0] {
data
0.485 \\
0.5 \\
0.5 \\
0.5 \\
0.515 \\
0.5 \\
0.435 \\
0.5 \\
0.52 \\
0.5 \\
0.5 \\
0.51 \\
0.47 \\
0.5 \\
0.475 \\
0.48 \\
0.5 \\
0.5 \\
0.5 \\
0.5 \\
0.5 \\
0.5 \\
0.5 \\
0.51 \\
0.5 \\
0.485 \\
0.515 \\
0.5 \\
0.5 \\
0.5 \\
};

\addplot[mark=*, boxplot, boxplot/draw position=1]
table[row sep=\\, y index=0] {
data
0.445 \\
0.575 \\
0.51 \\
0.445 \\
0.48 \\
0.465 \\
0.55 \\
0.5 \\
0.51 \\
0.44 \\
0.51 \\
0.445 \\
0.455 \\
0.475 \\
0.475 \\
0.42 \\
0.575 \\
0.56 \\
0.445 \\
0.465 \\
0.445 \\
0.645 \\
0.5 \\
0.63 \\
0.51 \\
0.445 \\
0.45 \\
0.62 \\
0.52 \\
0.445 \\
};

\addplot[mark=*, boxplot, boxplot/draw position=2]
table[row sep=\\, y index=0] {
data
0.5 \\
0.5 \\
0.5 \\
0.5 \\
0.5 \\
0.5 \\
0.5 \\
0.5 \\
0.5 \\
0.5 \\
0.5 \\
0.5 \\
0.5 \\
0.5 \\
0.5 \\
0.5 \\
0.5 \\
0.5 \\
0.5 \\
0.5 \\
0.5 \\
0.5 \\
0.5 \\
0.5 \\
0.5 \\
0.5 \\
0.5 \\
0.5 \\
0.5 \\
0.5 \\
};
}{0.2}{Input connectivity}{20}

        }
        \subfloat[N=15]{
            \myboxplot{

\addplot[mark=*, boxplot, boxplot/draw position=3]
table[row sep=\\, y index=0] {
data
0.5 \\
0.5 \\
0.5 \\
0.5 \\
0.5 \\
0.5 \\
0.5 \\
0.5 \\
0.5 \\
0.5 \\
0.5 \\
0.5 \\
0.5 \\
0.5 \\
0.5 \\
0.5 \\
0.5 \\
0.5 \\
0.5 \\
0.5 \\
0.5 \\
0.5 \\
0.5 \\
0.5 \\
0.5 \\
0.5 \\
0.5 \\
0.5 \\
0.5 \\
0.5 \\
};

\addplot[mark=*, boxplot, boxplot/draw position=0]
table[row sep=\\, y index=0] {
data
0.5 \\
0.5 \\
0.49 \\
0.495 \\
0.505 \\
0.5 \\
0.515 \\
0.5 \\
0.48 \\
0.5 \\
0.5 \\
0.52 \\
0.525 \\
0.5 \\
0.5 \\
0.5 \\
0.5 \\
0.5 \\
0.52 \\
0.455 \\
0.5 \\
0.515 \\
0.51 \\
0.49 \\
0.5 \\
0.52 \\
0.51 \\
0.5 \\
0.5 \\
0.52 \\
};

\addplot[mark=*, boxplot, boxplot/draw position=1]
table[row sep=\\, y index=0] {
data
0.645 \\
0.5 \\
0.685 \\
0.585 \\
0.47 \\
0.65 \\
0.565 \\
0.42 \\
0.59 \\
0.63 \\
0.675 \\
0.53 \\
0.455 \\
0.485 \\
0.56 \\
0.445 \\
0.515 \\
0.5 \\
0.48 \\
0.44 \\
0.57 \\
0.55 \\
0.725 \\
0.635 \\
0.61 \\
0.54 \\
0.43 \\
0.54 \\
0.57 \\
0.565 \\
};

\addplot[mark=*, boxplot, boxplot/draw position=2]
table[row sep=\\, y index=0] {
data
0.545 \\
0.5 \\
0.65 \\
0.445 \\
0.65 \\
0.44 \\
0.625 \\
0.445 \\
0.445 \\
0.5 \\
0.445 \\
0.61 \\
0.475 \\
0.445 \\
0.545 \\
0.475 \\
0.445 \\
0.65 \\
0.445 \\
0.5 \\
0.53 \\
0.44 \\
0.605 \\
0.445 \\
0.445 \\
0.525 \\
0.49 \\
0.51 \\
0.61 \\
0.47 \\
};
}{0.2}{Input connectivity}

        }
    }
    \resizebox{\textwidth}{!}{
        \subfloat[N=20]{
            \myboxplot{

\addplot[mark=*, boxplot, boxplot/draw position=3]
table[row sep=\\, y index=0] {
data
0.455 \\
0.44 \\
0.51 \\
0.64 \\
0.445 \\
0.45 \\
0.44 \\
0.485 \\
0.56 \\
0.47 \\
0.545 \\
0.5 \\
0.56 \\
0.445 \\
0.445 \\
0.44 \\
0.445 \\
0.575 \\
0.44 \\
0.44 \\
0.51 \\
0.45 \\
0.46 \\
0.445 \\
0.44 \\
0.44 \\
0.445 \\
0.605 \\
0.5 \\
0.465 \\
};

\addplot[mark=*, boxplot, boxplot/draw position=0]
table[row sep=\\, y index=0] {
data
0.5 \\
0.5 \\
0.475 \\
0.5 \\
0.485 \\
0.5 \\
0.5 \\
0.475 \\
0.525 \\
0.5 \\
0.52 \\
0.5 \\
0.495 \\
0.5 \\
0.515 \\
0.5 \\
0.515 \\
0.51 \\
0.475 \\
0.5 \\
0.515 \\
0.5 \\
0.5 \\
0.5 \\
0.48 \\
0.495 \\
0.5 \\
0.5 \\
0.5 \\
0.48 \\
};

\addplot[mark=*, boxplot, boxplot/draw position=4]
table[row sep=\\, y index=0] {
data
0.5 \\
0.5 \\
0.5 \\
0.5 \\
0.5 \\
0.5 \\
0.5 \\
0.5 \\
0.5 \\
0.5 \\
0.5 \\
0.5 \\
0.5 \\
0.5 \\
0.5 \\
0.5 \\
0.5 \\
0.5 \\
0.5 \\
0.5 \\
0.5 \\
0.5 \\
0.5 \\
0.5 \\
0.5 \\
0.5 \\
0.5 \\
0.5 \\
0.5 \\
0.5 \\
};

\addplot[mark=*, boxplot, boxplot/draw position=1]
table[row sep=\\, y index=0] {
data
0.505 \\
0.655 \\
0.5 \\
0.61 \\
0.435 \\
0.63 \\
0.49 \\
0.56 \\
0.52 \\
0.55 \\
0.495 \\
0.51 \\
0.585 \\
0.435 \\
0.52 \\
0.595 \\
0.59 \\
0.55 \\
0.59 \\
0.445 \\
0.455 \\
0.615 \\
0.53 \\
0.53 \\
0.53 \\
0.565 \\
0.505 \\
0.475 \\
0.525 \\
0.49 \\
};

\addplot[mark=*, boxplot, boxplot/draw position=2]
table[row sep=\\, y index=0] {
data
0.62 \\
0.5 \\
0.63 \\
0.485 \\
0.47 \\
0.635 \\
0.455 \\
0.44 \\
0.54 \\
0.63 \\
0.56 \\
0.535 \\
0.61 \\
0.585 \\
0.515 \\
0.6 \\
0.62 \\
0.595 \\
0.485 \\
0.44 \\
0.59 \\
0.505 \\
0.595 \\
0.595 \\
0.585 \\
0.44 \\
0.445 \\
0.57 \\
0.605 \\
0.51 \\
};
}{0.2}{Input connectivity}{20}

        }
        \subfloat[N=25]{
            \myboxplot{

\addplot[mark=*, boxplot, boxplot/draw position=5]
table[row sep=\\, y index=0] {
data
0.5 \\
0.5 \\
0.5 \\
0.5 \\
0.5 \\
0.5 \\
0.5 \\
0.5 \\
0.5 \\
0.5 \\
0.5 \\
0.5 \\
0.5 \\
0.5 \\
0.5 \\
0.5 \\
0.5 \\
0.5 \\
0.5 \\
0.5 \\
0.5 \\
0.5 \\
0.5 \\
0.5 \\
0.5 \\
0.5 \\
0.5 \\
0.5 \\
0.5 \\
0.5 \\
};

\addplot[mark=*, boxplot, boxplot/draw position=3]
table[row sep=\\, y index=0] {
data
0.635 \\
0.595 \\
0.605 \\
0.445 \\
0.445 \\
0.575 \\
0.655 \\
0.6 \\
0.555 \\
0.665 \\
0.575 \\
0.445 \\
0.53 \\
0.58 \\
0.65 \\
0.575 \\
0.54 \\
0.405 \\
0.45 \\
0.74 \\
0.44 \\
0.465 \\
0.55 \\
0.445 \\
0.635 \\
0.44 \\
0.44 \\
0.68 \\
0.65 \\
0.51 \\
};

\addplot[mark=*, boxplot, boxplot/draw position=2]
table[row sep=\\, y index=0] {
data
0.665 \\
0.55 \\
0.585 \\
0.46 \\
0.415 \\
0.44 \\
0.545 \\
0.485 \\
0.595 \\
0.605 \\
0.62 \\
0.445 \\
0.58 \\
0.51 \\
0.465 \\
0.49 \\
0.65 \\
0.615 \\
0.51 \\
0.525 \\
0.455 \\
0.635 \\
0.55 \\
0.59 \\
0.595 \\
0.51 \\
0.635 \\
0.51 \\
0.465 \\
0.53 \\
};

\addplot[mark=*, boxplot, boxplot/draw position=0]
table[row sep=\\, y index=0] {
data
0.5 \\
0.48 \\
0.5 \\
0.485 \\
0.485 \\
0.5 \\
0.545 \\
0.505 \\
0.53 \\
0.51 \\
0.465 \\
0.5 \\
0.465 \\
0.5 \\
0.5 \\
0.5 \\
0.5 \\
0.485 \\
0.48 \\
0.5 \\
0.545 \\
0.49 \\
0.53 \\
0.53 \\
0.54 \\
0.5 \\
0.5 \\
0.49 \\
0.51 \\
0.515 \\
};

\addplot[mark=*, boxplot, boxplot/draw position=1]
table[row sep=\\, y index=0] {
data
0.605 \\
0.515 \\
0.515 \\
0.545 \\
0.49 \\
0.42 \\
0.685 \\
0.48 \\
0.51 \\
0.545 \\
0.54 \\
0.485 \\
0.415 \\
0.47 \\
0.44 \\
0.5 \\
0.48 \\
0.565 \\
0.475 \\
0.54 \\
0.505 \\
0.465 \\
0.505 \\
0.535 \\
0.48 \\
0.515 \\
0.53 \\
0.53 \\
0.57 \\
0.57 \\
};

\addplot[mark=*, boxplot, boxplot/draw position=4]
table[row sep=\\, y index=0] {
data
0.445 \\
0.445 \\
0.445 \\
0.445 \\
0.445 \\
0.445 \\
0.515 \\
0.445 \\
0.53 \\
0.425 \\
0.44 \\
0.44 \\
0.51 \\
0.44 \\
0.46 \\
0.565 \\
0.44 \\
0.59 \\
0.45 \\
0.445 \\
0.5 \\
0.445 \\
0.505 \\
0.445 \\
0.445 \\
0.445 \\
0.445 \\
0.445 \\
0.52 \\
0.5 \\
};
}{0.2}{Input connectivity}{20}

        }
    }
    \resizebox{\textwidth}{!}{
        \subfloat[N=30]{
            \myboxplot{

\addplot[mark=*, boxplot, boxplot/draw position=5]
table[row sep=\\, y index=0] {
data
0.445 \\
0.445 \\
0.445 \\
0.445 \\
0.445 \\
0.445 \\
0.44 \\
0.475 \\
0.51 \\
0.47 \\
0.445 \\
0.545 \\
0.565 \\
0.445 \\
0.5 \\
0.445 \\
0.445 \\
0.445 \\
0.445 \\
0.54 \\
0.445 \\
0.445 \\
0.535 \\
0.445 \\
0.445 \\
0.445 \\
0.45 \\
0.445 \\
0.445 \\
0.445 \\
};

\addplot[mark=*, boxplot, boxplot/draw position=3]
table[row sep=\\, y index=0] {
data
0.6 \\
0.675 \\
0.445 \\
0.83 \\
0.44 \\
0.695 \\
0.445 \\
0.705 \\
0.5 \\
0.445 \\
0.63 \\
0.615 \\
0.68 \\
0.475 \\
0.535 \\
0.65 \\
0.665 \\
0.545 \\
0.575 \\
0.645 \\
0.74 \\
0.74 \\
0.455 \\
0.47 \\
0.635 \\
0.585 \\
0.64 \\
0.6 \\
0.56 \\
0.525 \\
};

\addplot[mark=*, boxplot, boxplot/draw position=2]
table[row sep=\\, y index=0] {
data
0.53 \\
0.565 \\
0.47 \\
0.51 \\
0.54 \\
0.51 \\
0.595 \\
0.535 \\
0.625 \\
0.515 \\
0.715 \\
0.71 \\
0.585 \\
0.47 \\
0.515 \\
0.465 \\
0.495 \\
0.49 \\
0.56 \\
0.56 \\
0.715 \\
0.48 \\
0.475 \\
0.515 \\
0.63 \\
0.5 \\
0.505 \\
0.64 \\
0.565 \\
0.595 \\
};

\addplot[mark=*, boxplot, boxplot/draw position=6]
table[row sep=\\, y index=0] {
data
0.5 \\
0.5 \\
0.5 \\
0.5 \\
0.5 \\
0.5 \\
0.5 \\
0.5 \\
0.5 \\
0.5 \\
0.5 \\
0.5 \\
0.5 \\
0.5 \\
0.5 \\
0.5 \\
0.5 \\
0.5 \\
0.5 \\
0.5 \\
0.5 \\
0.5 \\
0.5 \\
0.5 \\
0.5 \\
0.5 \\
0.5 \\
0.5 \\
0.5 \\
0.5 \\
};

\addplot[mark=*, boxplot, boxplot/draw position=0]
table[row sep=\\, y index=0] {
data
0.48 \\
0.5 \\
0.485 \\
0.5 \\
0.545 \\
0.525 \\
0.53 \\
0.52 \\
0.545 \\
0.515 \\
0.465 \\
0.475 \\
0.545 \\
0.515 \\
0.515 \\
0.535 \\
0.5 \\
0.455 \\
0.495 \\
0.51 \\
0.52 \\
0.53 \\
0.5 \\
0.5 \\
0.515 \\
0.485 \\
0.485 \\
0.5 \\
0.45 \\
0.505 \\
};

\addplot[mark=*, boxplot, boxplot/draw position=1]
table[row sep=\\, y index=0] {
data
0.465 \\
0.465 \\
0.49 \\
0.4 \\
0.505 \\
0.485 \\
0.46 \\
0.43 \\
0.49 \\
0.495 \\
0.45 \\
0.525 \\
0.6 \\
0.505 \\
0.49 \\
0.5 \\
0.5 \\
0.475 \\
0.46 \\
0.49 \\
0.555 \\
0.46 \\
0.535 \\
0.435 \\
0.51 \\
0.57 \\
0.505 \\
0.54 \\
0.485 \\
0.535 \\
};

\addplot[mark=*, boxplot, boxplot/draw position=4]
table[row sep=\\, y index=0] {
data
0.755 \\
0.445 \\
0.485 \\
0.485 \\
0.475 \\
0.47 \\
0.435 \\
0.45 \\
0.455 \\
0.425 \\
0.58 \\
0.51 \\
0.525 \\
0.555 \\
0.525 \\
0.545 \\
0.46 \\
0.44 \\
0.465 \\
0.735 \\
0.705 \\
0.44 \\
0.44 \\
0.445 \\
0.44 \\
0.45 \\
0.605 \\
0.68 \\
0.535 \\
0.44 \\
};
}{0.2}{Input connectivity}{20}

        }
        \subfloat[N=35]{
            \myboxplot{

\addplot[mark=*, boxplot, boxplot/draw position=5]
table[row sep=\\, y index=0] {
data
0.445 \\
0.445 \\
0.525 \\
0.475 \\
0.555 \\
0.405 \\
0.555 \\
0.46 \\
0.5 \\
0.445 \\
0.425 \\
0.545 \\
0.575 \\
0.445 \\
0.45 \\
0.445 \\
0.42 \\
0.555 \\
0.425 \\
0.44 \\
0.495 \\
0.46 \\
0.445 \\
0.555 \\
0.455 \\
0.575 \\
0.46 \\
0.485 \\
0.595 \\
0.565 \\
};

\addplot[mark=*, boxplot, boxplot/draw position=3]
table[row sep=\\, y index=0] {
data
0.665 \\
0.62 \\
0.785 \\
0.715 \\
0.57 \\
0.615 \\
0.56 \\
0.615 \\
0.615 \\
0.605 \\
0.49 \\
0.62 \\
0.605 \\
0.615 \\
0.615 \\
0.44 \\
0.7 \\
0.505 \\
0.56 \\
0.445 \\
0.425 \\
0.475 \\
0.675 \\
0.68 \\
0.695 \\
0.6 \\
0.455 \\
0.43 \\
0.515 \\
0.535 \\
};

\addplot[mark=*, boxplot, boxplot/draw position=2]
table[row sep=\\, y index=0] {
data
0.49 \\
0.53 \\
0.59 \\
0.52 \\
0.675 \\
0.59 \\
0.59 \\
0.6 \\
0.68 \\
0.475 \\
0.52 \\
0.685 \\
0.785 \\
0.595 \\
0.645 \\
0.58 \\
0.465 \\
0.495 \\
0.56 \\
0.645 \\
0.525 \\
0.51 \\
0.695 \\
0.56 \\
0.54 \\
0.5 \\
0.535 \\
0.51 \\
0.56 \\
0.5 \\
};

\addplot[mark=*, boxplot, boxplot/draw position=6]
table[row sep=\\, y index=0] {
data
0.445 \\
0.51 \\
0.44 \\
0.445 \\
0.445 \\
0.445 \\
0.44 \\
0.5 \\
0.57 \\
0.445 \\
0.445 \\
0.44 \\
0.445 \\
0.44 \\
0.445 \\
0.445 \\
0.48 \\
0.445 \\
0.445 \\
0.445 \\
0.5 \\
0.44 \\
0.44 \\
0.445 \\
0.445 \\
0.455 \\
0.515 \\
0.445 \\
0.445 \\
0.445 \\
};

\addplot[mark=*, boxplot, boxplot/draw position=7]
table[row sep=\\, y index=0] {
data
0.5 \\
0.5 \\
0.5 \\
0.5 \\
0.5 \\
0.5 \\
0.5 \\
0.5 \\
0.5 \\
0.5 \\
0.5 \\
0.5 \\
0.5 \\
0.5 \\
0.5 \\
0.5 \\
0.5 \\
0.5 \\
0.5 \\
0.5 \\
0.5 \\
0.5 \\
0.5 \\
0.5 \\
0.5 \\
0.5 \\
0.5 \\
0.5 \\
0.5 \\
0.5 \\
};

\addplot[mark=*, boxplot, boxplot/draw position=0]
table[row sep=\\, y index=0] {
data
0.5 \\
0.51 \\
0.485 \\
0.415 \\
0.52 \\
0.44 \\
0.5 \\
0.47 \\
0.5 \\
0.5 \\
0.475 \\
0.43 \\
0.5 \\
0.485 \\
0.475 \\
0.495 \\
0.48 \\
0.54 \\
0.465 \\
0.495 \\
0.49 \\
0.495 \\
0.52 \\
0.525 \\
0.435 \\
0.5 \\
0.535 \\
0.5 \\
0.535 \\
0.57 \\
};

\addplot[mark=*, boxplot, boxplot/draw position=1]
table[row sep=\\, y index=0] {
data
0.475 \\
0.49 \\
0.495 \\
0.455 \\
0.47 \\
0.45 \\
0.53 \\
0.435 \\
0.475 \\
0.49 \\
0.51 \\
0.425 \\
0.615 \\
0.545 \\
0.46 \\
0.505 \\
0.42 \\
0.475 \\
0.475 \\
0.44 \\
0.485 \\
0.515 \\
0.51 \\
0.535 \\
0.445 \\
0.435 \\
0.49 \\
0.53 \\
0.555 \\
0.595 \\
};

\addplot[mark=*, boxplot, boxplot/draw position=4]
table[row sep=\\, y index=0] {
data
0.505 \\
0.5 \\
0.44 \\
0.845 \\
0.44 \\
0.505 \\
0.51 \\
0.56 \\
0.545 \\
0.665 \\
0.495 \\
0.475 \\
0.775 \\
0.435 \\
0.635 \\
0.59 \\
0.51 \\
0.44 \\
0.555 \\
0.6 \\
0.605 \\
0.565 \\
0.42 \\
0.68 \\
0.69 \\
0.515 \\
0.465 \\
0.5 \\
0.445 \\
0.59 \\
};
}{0.2}{Input connectivity}{20}

        }
    }
    \caption{Task 2 - Part 1}
\end{figure*}

\begin{figure*}[ht]
    \centering
    \resizebox{\textwidth}{!}{
        \subfloat[N=40]{
            \myboxplot{

\addplot[mark=*, boxplot, boxplot/draw position=5]
table[row sep=\\, y index=0] {
data
0.415 \\
0.57 \\
0.43 \\
0.605 \\
0.725 \\
0.445 \\
0.72 \\
0.66 \\
0.6 \\
0.6 \\
0.515 \\
0.43 \\
0.45 \\
0.605 \\
0.44 \\
0.665 \\
0.51 \\
0.545 \\
0.59 \\
0.675 \\
0.54 \\
0.535 \\
0.575 \\
0.8 \\
0.44 \\
0.575 \\
0.445 \\
0.525 \\
0.635 \\
0.425 \\
};

\addplot[mark=*, boxplot, boxplot/draw position=3]
table[row sep=\\, y index=0] {
data
0.585 \\
0.665 \\
0.6 \\
0.495 \\
0.465 \\
0.565 \\
0.47 \\
0.535 \\
0.505 \\
0.715 \\
0.66 \\
0.645 \\
0.63 \\
0.65 \\
0.625 \\
0.605 \\
0.5 \\
0.62 \\
0.62 \\
0.46 \\
0.61 \\
0.6 \\
0.56 \\
0.62 \\
0.5 \\
0.74 \\
0.58 \\
0.7 \\
0.595 \\
0.685 \\
};

\addplot[mark=*, boxplot, boxplot/draw position=2]
table[row sep=\\, y index=0] {
data
0.475 \\
0.51 \\
0.52 \\
0.465 \\
0.535 \\
0.565 \\
0.57 \\
0.61 \\
0.64 \\
0.535 \\
0.49 \\
0.49 \\
0.6 \\
0.51 \\
0.565 \\
0.55 \\
0.56 \\
0.535 \\
0.435 \\
0.45 \\
0.485 \\
0.525 \\
0.51 \\
0.635 \\
0.63 \\
0.48 \\
0.505 \\
0.65 \\
0.45 \\
0.51 \\
};

\addplot[mark=*, boxplot, boxplot/draw position=6]
table[row sep=\\, y index=0] {
data
0.445 \\
0.44 \\
0.44 \\
0.44 \\
0.445 \\
0.505 \\
0.585 \\
0.445 \\
0.49 \\
0.44 \\
0.445 \\
0.445 \\
0.5 \\
0.445 \\
0.735 \\
0.5 \\
0.445 \\
0.44 \\
0.525 \\
0.44 \\
0.465 \\
0.47 \\
0.69 \\
0.445 \\
0.495 \\
0.44 \\
0.445 \\
0.44 \\
0.44 \\
0.445 \\
};

\addplot[mark=*, boxplot, boxplot/draw position=7]
table[row sep=\\, y index=0] {
data
0.54 \\
0.44 \\
0.445 \\
0.51 \\
0.445 \\
0.44 \\
0.465 \\
0.445 \\
0.44 \\
0.47 \\
0.445 \\
0.445 \\
0.445 \\
0.445 \\
0.51 \\
0.445 \\
0.51 \\
0.445 \\
0.51 \\
0.5 \\
0.44 \\
0.44 \\
0.545 \\
0.5 \\
0.445 \\
0.445 \\
0.445 \\
0.445 \\
0.51 \\
0.445 \\
};

\addplot[mark=*, boxplot, boxplot/draw position=0]
table[row sep=\\, y index=0] {
data
0.505 \\
0.5 \\
0.45 \\
0.515 \\
0.515 \\
0.465 \\
0.5 \\
0.54 \\
0.475 \\
0.46 \\
0.495 \\
0.53 \\
0.485 \\
0.445 \\
0.515 \\
0.48 \\
0.495 \\
0.535 \\
0.5 \\
0.5 \\
0.495 \\
0.58 \\
0.5 \\
0.485 \\
0.475 \\
0.445 \\
0.5 \\
0.515 \\
0.515 \\
0.53 \\
};

\addplot[mark=*, boxplot, boxplot/draw position=1]
table[row sep=\\, y index=0] {
data
0.495 \\
0.45 \\
0.43 \\
0.505 \\
0.49 \\
0.61 \\
0.455 \\
0.475 \\
0.505 \\
0.46 \\
0.5 \\
0.535 \\
0.45 \\
0.425 \\
0.52 \\
0.515 \\
0.52 \\
0.465 \\
0.49 \\
0.545 \\
0.515 \\
0.535 \\
0.48 \\
0.53 \\
0.52 \\
0.415 \\
0.565 \\
0.495 \\
0.475 \\
0.435 \\
};

\addplot[mark=*, boxplot, boxplot/draw position=8]
table[row sep=\\, y index=0] {
data
0.5 \\
0.5 \\
0.5 \\
0.5 \\
0.5 \\
0.5 \\
0.5 \\
0.5 \\
0.5 \\
0.5 \\
0.5 \\
0.5 \\
0.5 \\
0.5 \\
0.5 \\
0.5 \\
0.5 \\
0.5 \\
0.5 \\
0.5 \\
0.5 \\
0.5 \\
0.5 \\
0.5 \\
0.5 \\
0.5 \\
0.5 \\
0.5 \\
0.5 \\
0.5 \\
};

\addplot[mark=*, boxplot, boxplot/draw position=4]
table[row sep=\\, y index=0] {
data
0.65 \\
0.54 \\
0.75 \\
0.485 \\
0.65 \\
0.48 \\
0.53 \\
0.57 \\
0.63 \\
0.725 \\
0.645 \\
0.6 \\
0.785 \\
0.58 \\
0.665 \\
0.61 \\
0.68 \\
0.665 \\
0.635 \\
0.66 \\
0.63 \\
0.67 \\
0.815 \\
0.55 \\
0.74 \\
0.54 \\
0.585 \\
0.525 \\
0.5 \\
0.69 \\
};
}{0.2}{Input connectivity}

        }
        \subfloat[N=45]{
            \myboxplot{

\addplot[mark=*, boxplot, boxplot/draw position=5]
table[row sep=\\, y index=0] {
data
0.575 \\
0.445 \\
0.405 \\
0.56 \\
0.6 \\
0.41 \\
0.525 \\
0.48 \\
0.76 \\
0.515 \\
0.48 \\
0.43 \\
0.695 \\
0.53 \\
0.605 \\
0.695 \\
0.53 \\
0.68 \\
0.635 \\
0.535 \\
0.555 \\
0.61 \\
0.585 \\
0.65 \\
0.59 \\
0.725 \\
0.535 \\
0.73 \\
0.585 \\
0.66 \\
};

\addplot[mark=*, boxplot, boxplot/draw position=9]
table[row sep=\\, y index=0] {
data
0.5 \\
0.5 \\
0.5 \\
0.5 \\
0.5 \\
0.5 \\
0.5 \\
0.5 \\
0.5 \\
0.5 \\
0.5 \\
0.5 \\
0.5 \\
0.5 \\
0.5 \\
0.5 \\
0.5 \\
0.5 \\
0.5 \\
0.5 \\
0.5 \\
0.5 \\
0.5 \\
0.5 \\
0.5 \\
0.5 \\
0.5 \\
0.5 \\
0.5 \\
0.5 \\
};

\addplot[mark=*, boxplot, boxplot/draw position=3]
table[row sep=\\, y index=0] {
data
0.47 \\
0.685 \\
0.55 \\
0.725 \\
0.605 \\
0.63 \\
0.66 \\
0.53 \\
0.845 \\
0.63 \\
0.615 \\
0.495 \\
0.6 \\
0.545 \\
0.485 \\
0.585 \\
0.465 \\
0.545 \\
0.465 \\
0.655 \\
0.515 \\
0.555 \\
0.655 \\
0.54 \\
0.63 \\
0.62 \\
0.615 \\
0.54 \\
0.68 \\
0.495 \\
};

\addplot[mark=*, boxplot, boxplot/draw position=2]
table[row sep=\\, y index=0] {
data
0.45 \\
0.53 \\
0.45 \\
0.52 \\
0.54 \\
0.505 \\
0.585 \\
0.545 \\
0.48 \\
0.585 \\
0.585 \\
0.585 \\
0.575 \\
0.465 \\
0.5 \\
0.52 \\
0.56 \\
0.465 \\
0.59 \\
0.505 \\
0.485 \\
0.555 \\
0.575 \\
0.44 \\
0.52 \\
0.45 \\
0.525 \\
0.675 \\
0.51 \\
0.39 \\
};

\addplot[mark=*, boxplot, boxplot/draw position=6]
table[row sep=\\, y index=0] {
data
0.445 \\
0.545 \\
0.43 \\
0.62 \\
0.615 \\
0.43 \\
0.51 \\
0.495 \\
0.645 \\
0.595 \\
0.405 \\
0.445 \\
0.71 \\
0.445 \\
0.63 \\
0.55 \\
0.615 \\
0.68 \\
0.765 \\
0.585 \\
0.475 \\
0.635 \\
0.41 \\
0.595 \\
0.44 \\
0.44 \\
0.44 \\
0.445 \\
0.465 \\
0.445 \\
};

\addplot[mark=*, boxplot, boxplot/draw position=7]
table[row sep=\\, y index=0] {
data
0.465 \\
0.445 \\
0.495 \\
0.425 \\
0.445 \\
0.445 \\
0.545 \\
0.405 \\
0.485 \\
0.445 \\
0.475 \\
0.595 \\
0.645 \\
0.44 \\
0.59 \\
0.44 \\
0.595 \\
0.425 \\
0.49 \\
0.535 \\
0.425 \\
0.44 \\
0.44 \\
0.445 \\
0.45 \\
0.645 \\
0.445 \\
0.445 \\
0.44 \\
0.47 \\
};

\addplot[mark=*, boxplot, boxplot/draw position=0]
table[row sep=\\, y index=0] {
data
0.525 \\
0.525 \\
0.47 \\
0.5 \\
0.5 \\
0.445 \\
0.515 \\
0.525 \\
0.545 \\
0.535 \\
0.57 \\
0.56 \\
0.52 \\
0.485 \\
0.55 \\
0.485 \\
0.51 \\
0.525 \\
0.485 \\
0.53 \\
0.515 \\
0.51 \\
0.495 \\
0.495 \\
0.495 \\
0.515 \\
0.53 \\
0.5 \\
0.465 \\
0.5 \\
};

\addplot[mark=*, boxplot, boxplot/draw position=1]
table[row sep=\\, y index=0] {
data
0.54 \\
0.51 \\
0.505 \\
0.45 \\
0.475 \\
0.5 \\
0.51 \\
0.475 \\
0.435 \\
0.56 \\
0.465 \\
0.475 \\
0.515 \\
0.465 \\
0.455 \\
0.505 \\
0.47 \\
0.46 \\
0.535 \\
0.485 \\
0.585 \\
0.45 \\
0.455 \\
0.455 \\
0.52 \\
0.51 \\
0.515 \\
0.405 \\
0.495 \\
0.51 \\
};

\addplot[mark=*, boxplot, boxplot/draw position=8]
table[row sep=\\, y index=0] {
data
0.445 \\
0.5 \\
0.445 \\
0.44 \\
0.445 \\
0.445 \\
0.545 \\
0.44 \\
0.445 \\
0.445 \\
0.5 \\
0.5 \\
0.445 \\
0.44 \\
0.595 \\
0.565 \\
0.445 \\
0.645 \\
0.505 \\
0.47 \\
0.44 \\
0.445 \\
0.445 \\
0.445 \\
0.445 \\
0.445 \\
0.445 \\
0.445 \\
0.545 \\
0.44 \\
};

\addplot[mark=*, boxplot, boxplot/draw position=4]
table[row sep=\\, y index=0] {
data
0.66 \\
0.525 \\
0.81 \\
0.635 \\
0.5 \\
0.675 \\
0.645 \\
0.6 \\
0.535 \\
0.415 \\
0.6 \\
0.62 \\
0.555 \\
0.69 \\
0.47 \\
0.49 \\
0.585 \\
0.615 \\
0.38 \\
0.785 \\
0.615 \\
0.49 \\
0.45 \\
0.65 \\
0.445 \\
0.485 \\
0.505 \\
0.775 \\
0.525 \\
0.55 \\
};
}{0.2}{Input connectivity}{20}

        }
    }
    \resizebox{\textwidth}{!}{
        \subfloat[N=50]{
            \myboxplot{

\addplot[mark=*, boxplot, boxplot/draw position=5]
table[row sep=\\, y index=0] {
data
0.4 \\
0.675 \\
0.81 \\
0.59 \\
0.74 \\
0.54 \\
0.59 \\
0.435 \\
0.68 \\
0.88 \\
0.535 \\
0.555 \\
0.485 \\
0.68 \\
0.63 \\
0.83 \\
0.485 \\
0.665 \\
0.64 \\
0.48 \\
0.48 \\
0.57 \\
0.59 \\
0.46 \\
0.585 \\
0.745 \\
0.665 \\
0.815 \\
0.71 \\
0.555 \\
};

\addplot[mark=*, boxplot, boxplot/draw position=9]
table[row sep=\\, y index=0] {
data
0.445 \\
0.445 \\
0.445 \\
0.64 \\
0.605 \\
0.5 \\
0.445 \\
0.445 \\
0.445 \\
0.445 \\
0.44 \\
0.44 \\
0.595 \\
0.445 \\
0.5 \\
0.445 \\
0.445 \\
0.445 \\
0.445 \\
0.445 \\
0.445 \\
0.445 \\
0.445 \\
0.44 \\
0.5 \\
0.5 \\
0.5 \\
0.445 \\
0.445 \\
0.465 \\
};

\addplot[mark=*, boxplot, boxplot/draw position=3]
table[row sep=\\, y index=0] {
data
0.545 \\
0.645 \\
0.545 \\
0.575 \\
0.505 \\
0.555 \\
0.5 \\
0.465 \\
0.515 \\
0.595 \\
0.59 \\
0.56 \\
0.625 \\
0.54 \\
0.59 \\
0.505 \\
0.54 \\
0.63 \\
0.52 \\
0.625 \\
0.515 \\
0.68 \\
0.655 \\
0.655 \\
0.665 \\
0.585 \\
0.435 \\
0.625 \\
0.535 \\
0.585 \\
};

\addplot[mark=*, boxplot, boxplot/draw position=2]
table[row sep=\\, y index=0] {
data
0.455 \\
0.5 \\
0.55 \\
0.475 \\
0.575 \\
0.6 \\
0.495 \\
0.455 \\
0.505 \\
0.58 \\
0.545 \\
0.52 \\
0.62 \\
0.495 \\
0.55 \\
0.595 \\
0.52 \\
0.51 \\
0.535 \\
0.57 \\
0.5 \\
0.505 \\
0.585 \\
0.49 \\
0.62 \\
0.525 \\
0.48 \\
0.51 \\
0.465 \\
0.54 \\
};

\addplot[mark=*, boxplot, boxplot/draw position=6]
table[row sep=\\, y index=0] {
data
0.56 \\
0.465 \\
0.505 \\
0.44 \\
0.795 \\
0.595 \\
0.51 \\
0.68 \\
0.81 \\
0.55 \\
0.49 \\
0.515 \\
0.445 \\
0.625 \\
0.74 \\
0.61 \\
0.705 \\
0.72 \\
0.82 \\
0.42 \\
0.54 \\
0.425 \\
0.74 \\
0.52 \\
0.405 \\
0.55 \\
0.63 \\
0.65 \\
0.615 \\
0.415 \\
};

\addplot[mark=*, boxplot, boxplot/draw position=10]
table[row sep=\\, y index=0] {
data
0.5 \\
0.5 \\
0.5 \\
0.5 \\
0.5 \\
0.5 \\
0.5 \\
0.5 \\
0.5 \\
0.5 \\
0.5 \\
0.5 \\
0.5 \\
0.5 \\
0.5 \\
0.5 \\
0.5 \\
0.5 \\
0.5 \\
0.5 \\
0.5 \\
0.5 \\
0.5 \\
0.5 \\
0.5 \\
0.5 \\
0.5 \\
0.5 \\
0.5 \\
0.5 \\
};

\addplot[mark=*, boxplot, boxplot/draw position=7]
table[row sep=\\, y index=0] {
data
0.505 \\
0.445 \\
0.6 \\
0.54 \\
0.75 \\
0.56 \\
0.44 \\
0.435 \\
0.685 \\
0.485 \\
0.495 \\
0.56 \\
0.435 \\
0.525 \\
0.435 \\
0.445 \\
0.44 \\
0.665 \\
0.44 \\
0.56 \\
0.45 \\
0.44 \\
0.715 \\
0.49 \\
0.435 \\
0.44 \\
0.5 \\
0.685 \\
0.53 \\
0.64 \\
};

\addplot[mark=*, boxplot, boxplot/draw position=0]
table[row sep=\\, y index=0] {
data
0.45 \\
0.515 \\
0.47 \\
0.485 \\
0.49 \\
0.495 \\
0.49 \\
0.47 \\
0.48 \\
0.52 \\
0.475 \\
0.495 \\
0.495 \\
0.465 \\
0.505 \\
0.52 \\
0.485 \\
0.53 \\
0.5 \\
0.5 \\
0.495 \\
0.5 \\
0.55 \\
0.5 \\
0.5 \\
0.465 \\
0.51 \\
0.525 \\
0.515 \\
0.53 \\
};

\addplot[mark=*, boxplot, boxplot/draw position=1]
table[row sep=\\, y index=0] {
data
0.54 \\
0.5 \\
0.47 \\
0.455 \\
0.54 \\
0.475 \\
0.52 \\
0.465 \\
0.445 \\
0.495 \\
0.51 \\
0.495 \\
0.57 \\
0.535 \\
0.495 \\
0.43 \\
0.46 \\
0.49 \\
0.465 \\
0.49 \\
0.495 \\
0.51 \\
0.495 \\
0.48 \\
0.44 \\
0.52 \\
0.52 \\
0.5 \\
0.545 \\
0.445 \\
};

\addplot[mark=*, boxplot, boxplot/draw position=8]
table[row sep=\\, y index=0] {
data
0.44 \\
0.44 \\
0.445 \\
0.44 \\
0.445 \\
0.45 \\
0.445 \\
0.445 \\
0.545 \\
0.44 \\
0.445 \\
0.51 \\
0.68 \\
0.475 \\
0.455 \\
0.445 \\
0.52 \\
0.48 \\
0.51 \\
0.5 \\
0.42 \\
0.595 \\
0.54 \\
0.445 \\
0.445 \\
0.49 \\
0.47 \\
0.44 \\
0.445 \\
0.425 \\
};

\addplot[mark=*, boxplot, boxplot/draw position=4]
table[row sep=\\, y index=0] {
data
0.665 \\
0.8 \\
0.635 \\
0.645 \\
0.65 \\
0.71 \\
0.49 \\
0.61 \\
0.58 \\
0.685 \\
0.54 \\
0.61 \\
0.605 \\
0.595 \\
0.585 \\
0.67 \\
0.645 \\
0.66 \\
0.485 \\
0.645 \\
0.725 \\
0.585 \\
0.55 \\
0.58 \\
0.515 \\
0.495 \\
0.645 \\
0.735 \\
0.605 \\
0.505 \\
};
}{0.2}{Input connectivity}

        }
        \subfloat[N=55]{
            \myboxplot{

\addplot[mark=*, boxplot, boxplot/draw position=5]
table[row sep=\\, y index=0] {
data
0.765 \\
0.62 \\
0.49 \\
0.755 \\
0.625 \\
0.705 \\
0.68 \\
0.685 \\
0.65 \\
0.835 \\
0.74 \\
0.475 \\
0.715 \\
0.56 \\
0.73 \\
0.675 \\
0.53 \\
0.655 \\
0.44 \\
0.675 \\
0.69 \\
0.545 \\
0.82 \\
0.57 \\
0.63 \\
0.685 \\
0.64 \\
0.45 \\
0.775 \\
0.67 \\
};

\addplot[mark=*, boxplot, boxplot/draw position=9]
table[row sep=\\, y index=0] {
data
0.445 \\
0.51 \\
0.445 \\
0.58 \\
0.44 \\
0.445 \\
0.44 \\
0.45 \\
0.44 \\
0.44 \\
0.6 \\
0.48 \\
0.445 \\
0.44 \\
0.445 \\
0.445 \\
0.445 \\
0.44 \\
0.445 \\
0.43 \\
0.445 \\
0.44 \\
0.585 \\
0.445 \\
0.44 \\
0.445 \\
0.44 \\
0.61 \\
0.44 \\
0.5 \\
};

\addplot[mark=*, boxplot, boxplot/draw position=3]
table[row sep=\\, y index=0] {
data
0.565 \\
0.53 \\
0.47 \\
0.72 \\
0.425 \\
0.55 \\
0.56 \\
0.51 \\
0.68 \\
0.605 \\
0.495 \\
0.565 \\
0.61 \\
0.495 \\
0.555 \\
0.585 \\
0.515 \\
0.65 \\
0.525 \\
0.645 \\
0.535 \\
0.57 \\
0.46 \\
0.605 \\
0.53 \\
0.605 \\
0.615 \\
0.52 \\
0.635 \\
0.495 \\
};

\addplot[mark=*, boxplot, boxplot/draw position=2]
table[row sep=\\, y index=0] {
data
0.495 \\
0.525 \\
0.475 \\
0.54 \\
0.49 \\
0.48 \\
0.465 \\
0.52 \\
0.48 \\
0.45 \\
0.54 \\
0.55 \\
0.53 \\
0.52 \\
0.57 \\
0.615 \\
0.465 \\
0.47 \\
0.425 \\
0.49 \\
0.435 \\
0.545 \\
0.55 \\
0.545 \\
0.505 \\
0.475 \\
0.455 \\
0.54 \\
0.455 \\
0.49 \\
};

\addplot[mark=*, boxplot, boxplot/draw position=11]
table[row sep=\\, y index=0] {
data
0.5 \\
0.5 \\
0.5 \\
0.5 \\
0.5 \\
0.5 \\
0.5 \\
0.5 \\
0.5 \\
0.5 \\
0.5 \\
0.5 \\
0.5 \\
0.5 \\
0.5 \\
0.5 \\
0.5 \\
0.5 \\
0.5 \\
0.5 \\
0.5 \\
0.5 \\
0.5 \\
0.5 \\
0.5 \\
0.5 \\
0.5 \\
0.5 \\
0.5 \\
0.5 \\
};

\addplot[mark=*, boxplot, boxplot/draw position=6]
table[row sep=\\, y index=0] {
data
0.44 \\
0.705 \\
0.685 \\
0.74 \\
0.685 \\
0.765 \\
0.565 \\
0.575 \\
0.705 \\
0.755 \\
0.59 \\
0.815 \\
0.55 \\
0.68 \\
0.76 \\
0.63 \\
0.675 \\
0.655 \\
0.735 \\
0.55 \\
0.425 \\
0.69 \\
0.655 \\
0.72 \\
0.455 \\
0.82 \\
0.67 \\
0.735 \\
0.565 \\
0.79 \\
};

\addplot[mark=*, boxplot, boxplot/draw position=10]
table[row sep=\\, y index=0] {
data
0.5 \\
0.445 \\
0.445 \\
0.445 \\
0.445 \\
0.445 \\
0.445 \\
0.44 \\
0.545 \\
0.445 \\
0.445 \\
0.505 \\
0.545 \\
0.575 \\
0.485 \\
0.445 \\
0.445 \\
0.445 \\
0.445 \\
0.445 \\
0.445 \\
0.49 \\
0.445 \\
0.5 \\
0.445 \\
0.445 \\
0.445 \\
0.495 \\
0.445 \\
0.445 \\
};

\addplot[mark=*, boxplot, boxplot/draw position=7]
table[row sep=\\, y index=0] {
data
0.6 \\
0.445 \\
0.81 \\
0.49 \\
0.57 \\
0.535 \\
0.715 \\
0.58 \\
0.645 \\
0.6 \\
0.445 \\
0.63 \\
0.515 \\
0.445 \\
0.43 \\
0.485 \\
0.55 \\
0.465 \\
0.605 \\
0.5 \\
0.5 \\
0.49 \\
0.515 \\
0.535 \\
0.67 \\
0.565 \\
0.55 \\
0.675 \\
0.555 \\
0.45 \\
};

\addplot[mark=*, boxplot, boxplot/draw position=0]
table[row sep=\\, y index=0] {
data
0.51 \\
0.53 \\
0.47 \\
0.495 \\
0.475 \\
0.52 \\
0.565 \\
0.565 \\
0.5 \\
0.46 \\
0.515 \\
0.515 \\
0.485 \\
0.51 \\
0.5 \\
0.47 \\
0.5 \\
0.515 \\
0.465 \\
0.495 \\
0.485 \\
0.48 \\
0.445 \\
0.505 \\
0.47 \\
0.45 \\
0.47 \\
0.55 \\
0.515 \\
0.5 \\
};

\addplot[mark=*, boxplot, boxplot/draw position=1]
table[row sep=\\, y index=0] {
data
0.445 \\
0.52 \\
0.5 \\
0.515 \\
0.415 \\
0.505 \\
0.515 \\
0.455 \\
0.46 \\
0.455 \\
0.48 \\
0.44 \\
0.575 \\
0.465 \\
0.555 \\
0.455 \\
0.495 \\
0.49 \\
0.43 \\
0.505 \\
0.46 \\
0.54 \\
0.51 \\
0.45 \\
0.425 \\
0.58 \\
0.495 \\
0.455 \\
0.495 \\
0.455 \\
};

\addplot[mark=*, boxplot, boxplot/draw position=8]
table[row sep=\\, y index=0] {
data
0.495 \\
0.49 \\
0.405 \\
0.5 \\
0.645 \\
0.44 \\
0.44 \\
0.46 \\
0.445 \\
0.565 \\
0.45 \\
0.44 \\
0.44 \\
0.59 \\
0.53 \\
0.445 \\
0.44 \\
0.54 \\
0.62 \\
0.61 \\
0.425 \\
0.575 \\
0.44 \\
0.505 \\
0.785 \\
0.43 \\
0.53 \\
0.465 \\
0.44 \\
0.665 \\
};

\addplot[mark=*, boxplot, boxplot/draw position=4]
table[row sep=\\, y index=0] {
data
0.7 \\
0.57 \\
0.625 \\
0.565 \\
0.705 \\
0.6 \\
0.55 \\
0.515 \\
0.695 \\
0.535 \\
0.62 \\
0.72 \\
0.625 \\
0.575 \\
0.595 \\
0.615 \\
0.59 \\
0.67 \\
0.72 \\
0.495 \\
0.66 \\
0.585 \\
0.615 \\
0.545 \\
0.59 \\
0.48 \\
0.605 \\
0.69 \\
0.625 \\
0.565 \\
};
}{0.2}{Input connectivity}

        }
    }
    \resizebox{\textwidth}{!}{
        \subfloat[N=60]{
            \myboxplot{

\addplot[mark=*, boxplot, boxplot/draw position=5]
table[row sep=\\, y index=0] {
data
0.48 \\
0.64 \\
0.675 \\
0.68 \\
0.59 \\
0.71 \\
0.645 \\
0.59 \\
0.58 \\
0.78 \\
0.55 \\
0.69 \\
0.515 \\
0.755 \\
0.555 \\
0.585 \\
0.815 \\
0.655 \\
0.62 \\
0.63 \\
0.52 \\
0.65 \\
0.835 \\
0.73 \\
0.6 \\
0.615 \\
0.54 \\
0.625 \\
0.61 \\
0.585 \\
};

\addplot[mark=*, boxplot, boxplot/draw position=9]
table[row sep=\\, y index=0] {
data
0.445 \\
0.435 \\
0.43 \\
0.44 \\
0.44 \\
0.575 \\
0.425 \\
0.445 \\
0.445 \\
0.42 \\
0.535 \\
0.445 \\
0.445 \\
0.47 \\
0.445 \\
0.51 \\
0.44 \\
0.45 \\
0.655 \\
0.595 \\
0.43 \\
0.34 \\
0.55 \\
0.43 \\
0.44 \\
0.625 \\
0.535 \\
0.44 \\
0.645 \\
0.535 \\
};

\addplot[mark=*, boxplot, boxplot/draw position=3]
table[row sep=\\, y index=0] {
data
0.53 \\
0.505 \\
0.455 \\
0.715 \\
0.46 \\
0.53 \\
0.455 \\
0.55 \\
0.485 \\
0.505 \\
0.5 \\
0.53 \\
0.55 \\
0.49 \\
0.56 \\
0.485 \\
0.565 \\
0.55 \\
0.535 \\
0.605 \\
0.58 \\
0.545 \\
0.545 \\
0.65 \\
0.72 \\
0.575 \\
0.56 \\
0.525 \\
0.855 \\
0.525 \\
};

\addplot[mark=*, boxplot, boxplot/draw position=2]
table[row sep=\\, y index=0] {
data
0.43 \\
0.49 \\
0.455 \\
0.46 \\
0.515 \\
0.545 \\
0.57 \\
0.525 \\
0.4 \\
0.465 \\
0.525 \\
0.5 \\
0.49 \\
0.535 \\
0.505 \\
0.685 \\
0.54 \\
0.485 \\
0.53 \\
0.48 \\
0.525 \\
0.455 \\
0.42 \\
0.48 \\
0.46 \\
0.57 \\
0.53 \\
0.53 \\
0.44 \\
0.5 \\
};

\addplot[mark=*, boxplot, boxplot/draw position=11]
table[row sep=\\, y index=0] {
data
0.445 \\
0.445 \\
0.445 \\
0.44 \\
0.44 \\
0.51 \\
0.445 \\
0.5 \\
0.445 \\
0.445 \\
0.445 \\
0.445 \\
0.5 \\
0.445 \\
0.445 \\
0.44 \\
0.56 \\
0.445 \\
0.445 \\
0.445 \\
0.5 \\
0.445 \\
0.445 \\
0.445 \\
0.445 \\
0.445 \\
0.445 \\
0.445 \\
0.5 \\
0.445 \\
};

\addplot[mark=*, boxplot, boxplot/draw position=6]
table[row sep=\\, y index=0] {
data
0.69 \\
0.755 \\
0.64 \\
0.49 \\
0.805 \\
0.89 \\
0.75 \\
0.56 \\
0.615 \\
0.595 \\
0.52 \\
0.71 \\
0.815 \\
0.67 \\
0.67 \\
0.555 \\
0.765 \\
0.43 \\
0.46 \\
0.715 \\
0.44 \\
0.495 \\
0.665 \\
0.52 \\
0.6 \\
0.68 \\
0.62 \\
0.655 \\
0.67 \\
0.665 \\
};

\addplot[mark=*, boxplot, boxplot/draw position=10]
table[row sep=\\, y index=0] {
data
0.445 \\
0.445 \\
0.44 \\
0.47 \\
0.445 \\
0.445 \\
0.445 \\
0.505 \\
0.69 \\
0.44 \\
0.405 \\
0.44 \\
0.605 \\
0.48 \\
0.41 \\
0.44 \\
0.445 \\
0.445 \\
0.44 \\
0.44 \\
0.63 \\
0.445 \\
0.44 \\
0.445 \\
0.445 \\
0.42 \\
0.61 \\
0.46 \\
0.44 \\
0.445 \\
};

\addplot[mark=*, boxplot, boxplot/draw position=7]
table[row sep=\\, y index=0] {
data
0.525 \\
0.725 \\
0.745 \\
0.63 \\
0.825 \\
0.45 \\
0.59 \\
0.595 \\
0.44 \\
0.445 \\
0.665 \\
0.735 \\
0.75 \\
0.755 \\
0.5 \\
0.64 \\
0.57 \\
0.725 \\
0.595 \\
0.475 \\
0.72 \\
0.48 \\
0.565 \\
0.62 \\
0.45 \\
0.57 \\
0.74 \\
0.475 \\
0.79 \\
0.815 \\
};

\addplot[mark=*, boxplot, boxplot/draw position=12]
table[row sep=\\, y index=0] {
data
0.5 \\
0.5 \\
0.5 \\
0.5 \\
0.5 \\
0.5 \\
0.5 \\
0.5 \\
0.5 \\
0.5 \\
0.5 \\
0.5 \\
0.5 \\
0.5 \\
0.5 \\
0.5 \\
0.5 \\
0.5 \\
0.5 \\
0.5 \\
0.5 \\
0.5 \\
0.5 \\
0.5 \\
0.5 \\
0.5 \\
0.5 \\
0.5 \\
0.5 \\
0.5 \\
};

\addplot[mark=*, boxplot, boxplot/draw position=0]
table[row sep=\\, y index=0] {
data
0.46 \\
0.48 \\
0.44 \\
0.445 \\
0.545 \\
0.57 \\
0.47 \\
0.46 \\
0.505 \\
0.565 \\
0.545 \\
0.495 \\
0.49 \\
0.5 \\
0.455 \\
0.52 \\
0.485 \\
0.485 \\
0.505 \\
0.565 \\
0.46 \\
0.53 \\
0.505 \\
0.5 \\
0.51 \\
0.59 \\
0.52 \\
0.515 \\
0.54 \\
0.505 \\
};

\addplot[mark=*, boxplot, boxplot/draw position=1]
table[row sep=\\, y index=0] {
data
0.52 \\
0.49 \\
0.475 \\
0.465 \\
0.53 \\
0.51 \\
0.54 \\
0.445 \\
0.49 \\
0.46 \\
0.485 \\
0.525 \\
0.53 \\
0.515 \\
0.465 \\
0.495 \\
0.495 \\
0.48 \\
0.545 \\
0.49 \\
0.46 \\
0.525 \\
0.51 \\
0.47 \\
0.44 \\
0.54 \\
0.425 \\
0.455 \\
0.565 \\
0.51 \\
};

\addplot[mark=*, boxplot, boxplot/draw position=8]
table[row sep=\\, y index=0] {
data
0.495 \\
0.65 \\
0.44 \\
0.9 \\
0.44 \\
0.61 \\
0.455 \\
0.665 \\
0.445 \\
0.61 \\
0.435 \\
0.59 \\
0.47 \\
0.76 \\
0.495 \\
0.65 \\
0.705 \\
0.465 \\
0.525 \\
0.615 \\
0.475 \\
0.91 \\
0.515 \\
0.64 \\
0.485 \\
0.44 \\
0.535 \\
0.585 \\
0.47 \\
0.67 \\
};

\addplot[mark=*, boxplot, boxplot/draw position=4]
table[row sep=\\, y index=0] {
data
0.775 \\
0.62 \\
0.535 \\
0.66 \\
0.46 \\
0.75 \\
0.68 \\
0.505 \\
0.48 \\
0.59 \\
0.6 \\
0.59 \\
0.53 \\
0.79 \\
0.495 \\
0.64 \\
0.61 \\
0.75 \\
0.58 \\
0.6 \\
0.625 \\
0.535 \\
0.535 \\
0.665 \\
0.66 \\
0.55 \\
0.645 \\
0.61 \\
0.81 \\
0.63 \\
};
}{0.2}{Input connectivity}

        }
        \subfloat[N=65]{
            \myboxplot{

\addplot[mark=*, boxplot, boxplot/draw position=5]
table[row sep=\\, y index=0] {
data
0.535 \\
0.6 \\
0.71 \\
0.735 \\
0.685 \\
0.625 \\
0.595 \\
0.47 \\
0.795 \\
0.46 \\
0.63 \\
0.71 \\
0.675 \\
0.5 \\
0.64 \\
0.665 \\
0.62 \\
0.64 \\
0.54 \\
0.825 \\
0.775 \\
0.5 \\
0.555 \\
0.695 \\
0.6 \\
0.655 \\
0.685 \\
0.605 \\
0.68 \\
0.79 \\
};

\addplot[mark=*, boxplot, boxplot/draw position=9]
table[row sep=\\, y index=0] {
data
0.53 \\
0.73 \\
0.445 \\
0.485 \\
0.445 \\
0.57 \\
0.45 \\
0.62 \\
0.48 \\
0.405 \\
0.44 \\
0.615 \\
0.515 \\
0.565 \\
0.535 \\
0.44 \\
0.675 \\
0.435 \\
0.44 \\
0.6 \\
0.64 \\
0.705 \\
0.525 \\
0.545 \\
0.415 \\
0.485 \\
0.545 \\
0.49 \\
0.63 \\
0.57 \\
};

\addplot[mark=*, boxplot, boxplot/draw position=3]
table[row sep=\\, y index=0] {
data
0.49 \\
0.63 \\
0.565 \\
0.485 \\
0.54 \\
0.565 \\
0.565 \\
0.625 \\
0.59 \\
0.57 \\
0.54 \\
0.615 \\
0.55 \\
0.625 \\
0.575 \\
0.54 \\
0.56 \\
0.48 \\
0.585 \\
0.5 \\
0.575 \\
0.59 \\
0.535 \\
0.47 \\
0.625 \\
0.495 \\
0.49 \\
0.465 \\
0.62 \\
0.69 \\
};

\addplot[mark=*, boxplot, boxplot/draw position=2]
table[row sep=\\, y index=0] {
data
0.475 \\
0.48 \\
0.58 \\
0.49 \\
0.535 \\
0.54 \\
0.38 \\
0.57 \\
0.52 \\
0.53 \\
0.525 \\
0.51 \\
0.495 \\
0.545 \\
0.54 \\
0.445 \\
0.62 \\
0.46 \\
0.495 \\
0.45 \\
0.51 \\
0.57 \\
0.49 \\
0.5 \\
0.585 \\
0.43 \\
0.425 \\
0.505 \\
0.49 \\
0.66 \\
};

\addplot[mark=*, boxplot, boxplot/draw position=11]
table[row sep=\\, y index=0] {
data
0.445 \\
0.44 \\
0.57 \\
0.44 \\
0.445 \\
0.44 \\
0.445 \\
0.505 \\
0.44 \\
0.445 \\
0.445 \\
0.445 \\
0.51 \\
0.44 \\
0.475 \\
0.48 \\
0.55 \\
0.445 \\
0.44 \\
0.65 \\
0.445 \\
0.445 \\
0.445 \\
0.52 \\
0.44 \\
0.515 \\
0.44 \\
0.445 \\
0.445 \\
0.585 \\
};

\addplot[mark=*, boxplot, boxplot/draw position=6]
table[row sep=\\, y index=0] {
data
0.715 \\
0.615 \\
0.7 \\
0.735 \\
0.605 \\
0.725 \\
0.83 \\
0.58 \\
0.7 \\
0.625 \\
0.685 \\
0.86 \\
0.54 \\
0.745 \\
0.56 \\
0.695 \\
0.72 \\
0.565 \\
0.58 \\
0.57 \\
0.63 \\
0.645 \\
0.535 \\
0.54 \\
0.81 \\
0.57 \\
0.83 \\
0.78 \\
0.62 \\
0.66 \\
};

\addplot[mark=*, boxplot, boxplot/draw position=10]
table[row sep=\\, y index=0] {
data
0.47 \\
0.595 \\
0.445 \\
0.585 \\
0.445 \\
0.44 \\
0.445 \\
0.455 \\
0.645 \\
0.435 \\
0.505 \\
0.465 \\
0.595 \\
0.445 \\
0.995 \\
0.445 \\
0.495 \\
0.64 \\
0.435 \\
0.44 \\
0.615 \\
0.445 \\
0.55 \\
0.59 \\
0.465 \\
0.425 \\
0.44 \\
0.54 \\
0.44 \\
0.485 \\
};

\addplot[mark=*, boxplot, boxplot/draw position=7]
table[row sep=\\, y index=0] {
data
0.585 \\
0.655 \\
0.6 \\
0.655 \\
0.71 \\
0.765 \\
0.72 \\
0.785 \\
0.645 \\
0.68 \\
0.7 \\
0.63 \\
0.85 \\
0.72 \\
0.7 \\
0.49 \\
0.57 \\
0.57 \\
0.545 \\
0.635 \\
0.515 \\
0.535 \\
0.585 \\
0.71 \\
0.65 \\
0.585 \\
0.54 \\
0.785 \\
0.66 \\
0.395 \\
};

\addplot[mark=*, boxplot, boxplot/draw position=12]
table[row sep=\\, y index=0] {
data
0.445 \\
0.435 \\
0.445 \\
0.44 \\
0.445 \\
0.47 \\
0.44 \\
0.44 \\
0.445 \\
0.5 \\
0.445 \\
0.445 \\
0.445 \\
0.44 \\
0.445 \\
0.545 \\
0.445 \\
0.445 \\
0.445 \\
0.445 \\
0.445 \\
0.445 \\
0.445 \\
0.445 \\
0.51 \\
0.445 \\
0.445 \\
0.44 \\
0.445 \\
0.56 \\
};

\addplot[mark=*, boxplot, boxplot/draw position=0]
table[row sep=\\, y index=0] {
data
0.54 \\
0.555 \\
0.505 \\
0.515 \\
0.455 \\
0.495 \\
0.495 \\
0.45 \\
0.485 \\
0.545 \\
0.48 \\
0.555 \\
0.545 \\
0.595 \\
0.51 \\
0.535 \\
0.47 \\
0.505 \\
0.56 \\
0.475 \\
0.585 \\
0.555 \\
0.495 \\
0.445 \\
0.525 \\
0.46 \\
0.47 \\
0.57 \\
0.46 \\
0.505 \\
};

\addplot[mark=*, boxplot, boxplot/draw position=1]
table[row sep=\\, y index=0] {
data
0.505 \\
0.56 \\
0.465 \\
0.5 \\
0.515 \\
0.495 \\
0.49 \\
0.47 \\
0.52 \\
0.515 \\
0.505 \\
0.46 \\
0.48 \\
0.495 \\
0.52 \\
0.495 \\
0.5 \\
0.5 \\
0.5 \\
0.48 \\
0.495 \\
0.48 \\
0.485 \\
0.52 \\
0.44 \\
0.48 \\
0.52 \\
0.57 \\
0.47 \\
0.48 \\
};

\addplot[mark=*, boxplot, boxplot/draw position=8]
table[row sep=\\, y index=0] {
data
0.58 \\
0.535 \\
0.73 \\
0.605 \\
0.705 \\
0.68 \\
0.5 \\
0.64 \\
0.66 \\
0.615 \\
0.505 \\
0.74 \\
0.545 \\
0.55 \\
0.69 \\
0.415 \\
0.765 \\
0.845 \\
0.51 \\
0.615 \\
0.445 \\
0.755 \\
0.555 \\
0.535 \\
0.71 \\
0.77 \\
0.645 \\
0.585 \\
0.71 \\
0.665 \\
};

\addplot[mark=*, boxplot, boxplot/draw position=4]
table[row sep=\\, y index=0] {
data
0.555 \\
0.73 \\
0.64 \\
0.585 \\
0.65 \\
0.7 \\
0.61 \\
0.575 \\
0.66 \\
0.515 \\
0.775 \\
0.635 \\
0.54 \\
0.55 \\
0.59 \\
0.535 \\
0.705 \\
0.505 \\
0.535 \\
0.555 \\
0.48 \\
0.59 \\
0.49 \\
0.785 \\
0.575 \\
0.415 \\
0.42 \\
0.52 \\
0.705 \\
0.635 \\
};

\addplot[mark=*, boxplot, boxplot/draw position=13]
table[row sep=\\, y index=0] {
data
0.5 \\
0.5 \\
0.5 \\
0.5 \\
0.5 \\
0.5 \\
0.5 \\
0.5 \\
0.5 \\
0.5 \\
0.5 \\
0.5 \\
0.5 \\
0.5 \\
0.5 \\
0.5 \\
0.5 \\
0.5 \\
0.5 \\
0.5 \\
0.5 \\
0.5 \\
0.5 \\
0.5 \\
0.5 \\
0.5 \\
0.5 \\
0.5 \\
0.5 \\
0.5 \\
};
}{0.2}{Input connectivity}

        }
    }
    \caption{Task 2 - Part 2}
\end{figure*}

\begin{figure*}[ht]
    \centering
    \resizebox{\textwidth}{!}{
        \subfloat[N=70]{
            \myboxplot{

\addplot[mark=*, boxplot, boxplot/draw position=5]
table[row sep=\\, y index=0] {
data
0.52 \\
0.66 \\
0.67 \\
0.65 \\
0.69 \\
0.815 \\
0.565 \\
0.62 \\
0.56 \\
0.71 \\
0.565 \\
0.545 \\
0.55 \\
0.72 \\
0.54 \\
0.655 \\
0.635 \\
0.61 \\
0.575 \\
0.61 \\
0.63 \\
0.74 \\
0.735 \\
0.565 \\
0.515 \\
0.885 \\
0.565 \\
0.605 \\
0.825 \\
0.56 \\
};

\addplot[mark=*, boxplot, boxplot/draw position=9]
table[row sep=\\, y index=0] {
data
0.425 \\
0.645 \\
0.515 \\
0.73 \\
0.62 \\
0.47 \\
0.545 \\
0.52 \\
0.705 \\
0.6 \\
0.51 \\
0.555 \\
0.445 \\
0.6 \\
0.655 \\
0.815 \\
0.465 \\
0.62 \\
0.455 \\
0.49 \\
0.56 \\
0.575 \\
0.715 \\
0.535 \\
0.67 \\
0.69 \\
0.765 \\
0.62 \\
0.51 \\
0.725 \\
};

\addplot[mark=*, boxplot, boxplot/draw position=3]
table[row sep=\\, y index=0] {
data
0.49 \\
0.53 \\
0.575 \\
0.52 \\
0.515 \\
0.445 \\
0.52 \\
0.745 \\
0.565 \\
0.505 \\
0.84 \\
0.475 \\
0.485 \\
0.51 \\
0.515 \\
0.425 \\
0.55 \\
0.435 \\
0.565 \\
0.54 \\
0.455 \\
0.505 \\
0.465 \\
0.5 \\
0.54 \\
0.505 \\
0.525 \\
0.435 \\
0.47 \\
0.585 \\
};

\addplot[mark=*, boxplot, boxplot/draw position=2]
table[row sep=\\, y index=0] {
data
0.465 \\
0.49 \\
0.475 \\
0.425 \\
0.515 \\
0.485 \\
0.455 \\
0.44 \\
0.525 \\
0.535 \\
0.52 \\
0.52 \\
0.585 \\
0.545 \\
0.425 \\
0.51 \\
0.49 \\
0.52 \\
0.565 \\
0.47 \\
0.495 \\
0.58 \\
0.54 \\
0.585 \\
0.46 \\
0.475 \\
0.5 \\
0.555 \\
0.54 \\
0.545 \\
};

\addplot[mark=*, boxplot, boxplot/draw position=11]
table[row sep=\\, y index=0] {
data
0.44 \\
0.475 \\
0.44 \\
0.67 \\
0.575 \\
0.575 \\
0.44 \\
0.475 \\
0.465 \\
0.455 \\
0.465 \\
0.445 \\
0.43 \\
0.45 \\
0.49 \\
0.56 \\
0.475 \\
0.495 \\
0.405 \\
0.43 \\
0.45 \\
0.445 \\
0.555 \\
0.465 \\
0.425 \\
0.475 \\
0.445 \\
0.445 \\
0.51 \\
0.5 \\
};

\addplot[mark=*, boxplot, boxplot/draw position=6]
table[row sep=\\, y index=0] {
data
0.85 \\
0.59 \\
0.735 \\
0.625 \\
0.665 \\
0.65 \\
0.655 \\
0.515 \\
0.64 \\
0.88 \\
0.595 \\
0.81 \\
0.59 \\
0.77 \\
0.855 \\
0.715 \\
0.835 \\
0.84 \\
0.5 \\
0.67 \\
0.695 \\
0.615 \\
0.7 \\
0.66 \\
0.66 \\
0.785 \\
0.61 \\
0.655 \\
0.545 \\
0.555 \\
};

\addplot[mark=*, boxplot, boxplot/draw position=10]
table[row sep=\\, y index=0] {
data
0.61 \\
0.425 \\
0.615 \\
0.555 \\
0.585 \\
0.51 \\
0.43 \\
0.57 \\
0.44 \\
0.555 \\
0.67 \\
0.645 \\
0.525 \\
0.655 \\
0.475 \\
0.495 \\
0.44 \\
0.435 \\
0.62 \\
0.53 \\
0.44 \\
0.44 \\
0.565 \\
0.47 \\
0.53 \\
0.515 \\
0.445 \\
0.595 \\
0.47 \\
0.475 \\
};

\addplot[mark=*, boxplot, boxplot/draw position=7]
table[row sep=\\, y index=0] {
data
0.56 \\
0.625 \\
0.625 \\
0.64 \\
0.735 \\
0.64 \\
0.76 \\
0.695 \\
0.64 \\
0.61 \\
0.72 \\
0.54 \\
0.665 \\
0.8 \\
0.805 \\
0.645 \\
0.76 \\
0.705 \\
0.7 \\
0.63 \\
0.62 \\
0.66 \\
0.83 \\
0.555 \\
0.64 \\
0.65 \\
0.73 \\
0.87 \\
0.75 \\
0.885 \\
};

\addplot[mark=*, boxplot, boxplot/draw position=12]
table[row sep=\\, y index=0] {
data
0.44 \\
0.445 \\
0.44 \\
0.445 \\
0.445 \\
0.445 \\
0.44 \\
0.54 \\
0.44 \\
0.445 \\
0.51 \\
0.44 \\
0.44 \\
0.43 \\
0.44 \\
0.445 \\
0.445 \\
0.445 \\
0.505 \\
0.54 \\
0.445 \\
0.44 \\
0.65 \\
0.445 \\
0.44 \\
0.445 \\
0.445 \\
0.445 \\
0.59 \\
0.56 \\
};

\addplot[mark=*, boxplot, boxplot/draw position=0]
table[row sep=\\, y index=0] {
data
0.535 \\
0.565 \\
0.44 \\
0.47 \\
0.49 \\
0.495 \\
0.52 \\
0.45 \\
0.525 \\
0.46 \\
0.5 \\
0.495 \\
0.5 \\
0.545 \\
0.53 \\
0.5 \\
0.475 \\
0.525 \\
0.42 \\
0.495 \\
0.535 \\
0.51 \\
0.51 \\
0.5 \\
0.525 \\
0.44 \\
0.43 \\
0.525 \\
0.52 \\
0.495 \\
};

\addplot[mark=*, boxplot, boxplot/draw position=1]
table[row sep=\\, y index=0] {
data
0.475 \\
0.49 \\
0.475 \\
0.44 \\
0.49 \\
0.49 \\
0.555 \\
0.4 \\
0.465 \\
0.545 \\
0.475 \\
0.45 \\
0.43 \\
0.505 \\
0.485 \\
0.48 \\
0.525 \\
0.53 \\
0.515 \\
0.445 \\
0.48 \\
0.57 \\
0.55 \\
0.495 \\
0.48 \\
0.49 \\
0.5 \\
0.485 \\
0.515 \\
0.465 \\
};

\addplot[mark=*, boxplot, boxplot/draw position=8]
table[row sep=\\, y index=0] {
data
0.78 \\
0.61 \\
0.78 \\
0.47 \\
0.595 \\
0.645 \\
0.685 \\
0.845 \\
0.465 \\
0.685 \\
0.585 \\
0.895 \\
0.54 \\
0.735 \\
0.545 \\
0.775 \\
0.86 \\
0.635 \\
0.52 \\
0.52 \\
0.705 \\
0.73 \\
0.815 \\
0.62 \\
0.79 \\
0.615 \\
0.755 \\
0.635 \\
0.805 \\
0.705 \\
};

\addplot[mark=*, boxplot, boxplot/draw position=14]
table[row sep=\\, y index=0] {
data
0.5 \\
0.5 \\
0.5 \\
0.5 \\
0.5 \\
0.5 \\
0.5 \\
0.5 \\
0.5 \\
0.5 \\
0.5 \\
0.5 \\
0.5 \\
0.5 \\
0.5 \\
0.5 \\
0.5 \\
0.5 \\
0.5 \\
0.5 \\
0.5 \\
0.5 \\
0.5 \\
0.5 \\
0.5 \\
0.5 \\
0.5 \\
0.5 \\
0.5 \\
0.5 \\
};

\addplot[mark=*, boxplot, boxplot/draw position=4]
table[row sep=\\, y index=0] {
data
0.71 \\
0.68 \\
0.605 \\
0.48 \\
0.725 \\
0.495 \\
0.65 \\
0.645 \\
0.49 \\
0.59 \\
0.615 \\
0.62 \\
0.585 \\
0.79 \\
0.615 \\
0.525 \\
0.57 \\
0.54 \\
0.65 \\
0.695 \\
0.51 \\
0.61 \\
0.58 \\
0.71 \\
0.6 \\
0.565 \\
0.675 \\
0.515 \\
0.635 \\
0.725 \\
};

\addplot[mark=*, boxplot, boxplot/draw position=13]
table[row sep=\\, y index=0] {
data
0.445 \\
0.445 \\
0.44 \\
0.51 \\
0.445 \\
0.445 \\
0.445 \\
0.445 \\
0.445 \\
0.445 \\
0.445 \\
0.445 \\
0.445 \\
0.445 \\
0.445 \\
0.445 \\
0.445 \\
0.51 \\
0.5 \\
0.5 \\
0.445 \\
0.445 \\
0.445 \\
0.445 \\
0.44 \\
0.445 \\
0.445 \\
0.445 \\
0.445 \\
0.445 \\
};
}{0.2}{Input connectivity}

        }
        \subfloat[N=75]{
            \myboxplot{

\addplot[mark=*, boxplot, boxplot/draw position=5]
table[row sep=\\, y index=0] {
data
0.685 \\
0.635 \\
0.55 \\
0.625 \\
0.645 \\
0.66 \\
0.705 \\
0.485 \\
0.67 \\
0.725 \\
0.56 \\
0.74 \\
0.645 \\
0.705 \\
0.645 \\
0.575 \\
0.57 \\
0.49 \\
0.7 \\
0.53 \\
0.675 \\
0.595 \\
0.59 \\
0.655 \\
0.585 \\
0.695 \\
0.72 \\
0.83 \\
0.61 \\
0.665 \\
};

\addplot[mark=*, boxplot, boxplot/draw position=9]
table[row sep=\\, y index=0] {
data
0.62 \\
0.61 \\
0.63 \\
0.745 \\
0.78 \\
0.765 \\
0.44 \\
0.795 \\
0.59 \\
0.6 \\
0.695 \\
0.755 \\
0.44 \\
0.435 \\
0.83 \\
0.61 \\
0.61 \\
0.53 \\
0.51 \\
0.53 \\
0.71 \\
0.66 \\
0.51 \\
0.75 \\
0.54 \\
0.67 \\
0.665 \\
0.44 \\
0.61 \\
0.525 \\
};

\addplot[mark=*, boxplot, boxplot/draw position=3]
table[row sep=\\, y index=0] {
data
0.58 \\
0.45 \\
0.51 \\
0.58 \\
0.535 \\
0.535 \\
0.64 \\
0.49 \\
0.53 \\
0.5 \\
0.535 \\
0.61 \\
0.555 \\
0.455 \\
0.72 \\
0.565 \\
0.425 \\
0.475 \\
0.515 \\
0.525 \\
0.48 \\
0.58 \\
0.55 \\
0.71 \\
0.495 \\
0.51 \\
0.485 \\
0.495 \\
0.49 \\
0.75 \\
};

\addplot[mark=*, boxplot, boxplot/draw position=2]
table[row sep=\\, y index=0] {
data
0.575 \\
0.57 \\
0.51 \\
0.45 \\
0.54 \\
0.5 \\
0.52 \\
0.525 \\
0.505 \\
0.49 \\
0.485 \\
0.53 \\
0.47 \\
0.49 \\
0.52 \\
0.44 \\
0.465 \\
0.54 \\
0.575 \\
0.495 \\
0.56 \\
0.57 \\
0.565 \\
0.445 \\
0.5 \\
0.42 \\
0.585 \\
0.51 \\
0.525 \\
0.51 \\
};

\addplot[mark=*, boxplot, boxplot/draw position=11]
table[row sep=\\, y index=0] {
data
0.52 \\
0.58 \\
0.48 \\
0.765 \\
0.44 \\
0.44 \\
0.685 \\
0.515 \\
0.52 \\
0.445 \\
0.515 \\
0.665 \\
0.645 \\
0.445 \\
0.64 \\
0.525 \\
0.545 \\
0.595 \\
0.675 \\
0.44 \\
0.585 \\
0.435 \\
0.445 \\
0.66 \\
0.44 \\
0.615 \\
0.45 \\
0.44 \\
0.595 \\
0.72 \\
};

\addplot[mark=*, boxplot, boxplot/draw position=6]
table[row sep=\\, y index=0] {
data
0.49 \\
0.555 \\
0.64 \\
0.43 \\
0.58 \\
0.675 \\
0.73 \\
0.675 \\
0.505 \\
0.725 \\
0.515 \\
0.805 \\
0.74 \\
0.49 \\
0.62 \\
0.76 \\
0.7 \\
0.725 \\
0.555 \\
0.875 \\
0.54 \\
0.67 \\
0.94 \\
0.775 \\
0.615 \\
0.77 \\
0.59 \\
0.625 \\
0.825 \\
0.645 \\
};

\addplot[mark=*, boxplot, boxplot/draw position=10]
table[row sep=\\, y index=0] {
data
0.615 \\
0.6 \\
0.76 \\
0.665 \\
0.44 \\
0.47 \\
0.6 \\
0.395 \\
0.635 \\
0.595 \\
0.5 \\
0.605 \\
0.635 \\
0.55 \\
0.625 \\
0.79 \\
0.46 \\
0.565 \\
0.595 \\
0.685 \\
0.625 \\
0.795 \\
0.515 \\
0.745 \\
0.64 \\
0.46 \\
0.565 \\
0.585 \\
0.43 \\
0.44 \\
};

\addplot[mark=*, boxplot, boxplot/draw position=7]
table[row sep=\\, y index=0] {
data
0.82 \\
0.755 \\
0.765 \\
0.57 \\
0.7 \\
0.525 \\
0.725 \\
0.82 \\
0.78 \\
0.78 \\
0.585 \\
0.68 \\
0.92 \\
0.46 \\
0.58 \\
0.73 \\
0.75 \\
0.73 \\
0.75 \\
0.825 \\
0.73 \\
0.695 \\
0.515 \\
0.73 \\
0.74 \\
0.735 \\
0.7 \\
0.86 \\
0.72 \\
0.665 \\
};

\addplot[mark=*, boxplot, boxplot/draw position=12]
table[row sep=\\, y index=0] {
data
0.44 \\
0.46 \\
0.635 \\
0.415 \\
0.575 \\
0.46 \\
0.47 \\
0.445 \\
0.545 \\
0.44 \\
0.45 \\
0.44 \\
0.445 \\
0.44 \\
0.445 \\
0.495 \\
0.55 \\
0.63 \\
0.45 \\
0.695 \\
0.43 \\
0.49 \\
0.45 \\
0.44 \\
0.415 \\
0.475 \\
0.44 \\
0.445 \\
0.51 \\
0.43 \\
};

\addplot[mark=*, boxplot, boxplot/draw position=0]
table[row sep=\\, y index=0] {
data
0.555 \\
0.505 \\
0.465 \\
0.5 \\
0.495 \\
0.435 \\
0.55 \\
0.52 \\
0.47 \\
0.455 \\
0.5 \\
0.49 \\
0.51 \\
0.53 \\
0.465 \\
0.48 \\
0.49 \\
0.53 \\
0.44 \\
0.515 \\
0.49 \\
0.49 \\
0.535 \\
0.525 \\
0.485 \\
0.575 \\
0.495 \\
0.455 \\
0.54 \\
0.45 \\
};

\addplot[mark=*, boxplot, boxplot/draw position=15]
table[row sep=\\, y index=0] {
data
0.5 \\
0.5 \\
0.5 \\
0.5 \\
0.5 \\
0.5 \\
0.5 \\
0.5 \\
0.5 \\
0.5 \\
0.5 \\
0.5 \\
0.5 \\
0.5 \\
0.5 \\
0.5 \\
0.5 \\
0.5 \\
0.5 \\
0.5 \\
0.5 \\
0.5 \\
0.5 \\
0.5 \\
0.5 \\
0.5 \\
0.5 \\
0.5 \\
0.5 \\
0.5 \\
};

\addplot[mark=*, boxplot, boxplot/draw position=1]
table[row sep=\\, y index=0] {
data
0.545 \\
0.485 \\
0.455 \\
0.52 \\
0.535 \\
0.5 \\
0.47 \\
0.565 \\
0.47 \\
0.465 \\
0.465 \\
0.495 \\
0.495 \\
0.49 \\
0.455 \\
0.445 \\
0.48 \\
0.445 \\
0.44 \\
0.485 \\
0.465 \\
0.53 \\
0.43 \\
0.52 \\
0.445 \\
0.465 \\
0.415 \\
0.49 \\
0.54 \\
0.555 \\
};

\addplot[mark=*, boxplot, boxplot/draw position=8]
table[row sep=\\, y index=0] {
data
0.825 \\
0.625 \\
0.79 \\
0.63 \\
0.445 \\
0.845 \\
0.77 \\
0.935 \\
0.71 \\
0.675 \\
0.85 \\
0.74 \\
0.735 \\
0.74 \\
0.425 \\
0.72 \\
0.67 \\
0.68 \\
0.51 \\
0.665 \\
0.755 \\
0.86 \\
0.795 \\
0.665 \\
0.76 \\
0.735 \\
0.73 \\
0.695 \\
0.73 \\
0.595 \\
};

\addplot[mark=*, boxplot, boxplot/draw position=14]
table[row sep=\\, y index=0] {
data
0.495 \\
0.445 \\
0.5 \\
0.445 \\
0.445 \\
0.445 \\
0.445 \\
0.445 \\
0.445 \\
0.445 \\
0.445 \\
0.44 \\
0.5 \\
0.44 \\
0.5 \\
0.44 \\
0.445 \\
0.445 \\
0.445 \\
0.445 \\
0.5 \\
0.51 \\
0.5 \\
0.445 \\
0.445 \\
0.445 \\
0.445 \\
0.44 \\
0.445 \\
0.445 \\
};

\addplot[mark=*, boxplot, boxplot/draw position=4]
table[row sep=\\, y index=0] {
data
0.635 \\
0.56 \\
0.615 \\
0.535 \\
0.55 \\
0.465 \\
0.57 \\
0.52 \\
0.535 \\
0.595 \\
0.625 \\
0.49 \\
0.49 \\
0.54 \\
0.5 \\
0.725 \\
0.645 \\
0.545 \\
0.595 \\
0.505 \\
0.695 \\
0.81 \\
0.545 \\
0.575 \\
0.66 \\
0.535 \\
0.615 \\
0.755 \\
0.57 \\
0.605 \\
};

\addplot[mark=*, boxplot, boxplot/draw position=13]
table[row sep=\\, y index=0] {
data
0.44 \\
0.44 \\
0.445 \\
0.445 \\
0.465 \\
0.44 \\
0.44 \\
0.445 \\
0.44 \\
0.445 \\
0.445 \\
0.445 \\
0.445 \\
0.465 \\
0.44 \\
0.44 \\
0.505 \\
0.47 \\
0.44 \\
0.495 \\
0.445 \\
0.44 \\
0.44 \\
0.445 \\
0.41 \\
0.415 \\
0.56 \\
0.445 \\
0.445 \\
0.445 \\
};
}{0.2}{Input connectivity}{20}

        }
    }
    \resizebox{\textwidth}{!}{
        \subfloat[N=80]{
            \myboxplot{

\addplot[mark=*, boxplot, boxplot/draw position=5]
table[row sep=\\, y index=0] {
data
0.54 \\
0.67 \\
0.655 \\
0.575 \\
0.635 \\
0.46 \\
0.675 \\
0.74 \\
0.645 \\
0.67 \\
0.515 \\
0.695 \\
0.66 \\
0.62 \\
0.645 \\
0.5 \\
0.555 \\
0.52 \\
0.625 \\
0.58 \\
0.625 \\
0.725 \\
0.69 \\
0.78 \\
0.61 \\
0.73 \\
0.645 \\
0.605 \\
0.595 \\
0.625 \\
};

\addplot[mark=*, boxplot, boxplot/draw position=9]
table[row sep=\\, y index=0] {
data
0.85 \\
0.84 \\
0.67 \\
0.785 \\
0.48 \\
0.575 \\
0.79 \\
0.49 \\
0.635 \\
0.665 \\
0.78 \\
0.6 \\
0.765 \\
0.685 \\
0.745 \\
0.705 \\
0.605 \\
0.815 \\
0.645 \\
0.425 \\
0.715 \\
0.755 \\
0.89 \\
0.685 \\
0.54 \\
0.825 \\
0.47 \\
0.82 \\
0.715 \\
0.54 \\
};

\addplot[mark=*, boxplot, boxplot/draw position=3]
table[row sep=\\, y index=0] {
data
0.53 \\
0.545 \\
0.49 \\
0.5 \\
0.495 \\
0.56 \\
0.565 \\
0.495 \\
0.735 \\
0.525 \\
0.595 \\
0.535 \\
0.515 \\
0.485 \\
0.585 \\
0.525 \\
0.475 \\
0.435 \\
0.515 \\
0.515 \\
0.515 \\
0.515 \\
0.46 \\
0.56 \\
0.62 \\
0.485 \\
0.585 \\
0.58 \\
0.455 \\
0.48 \\
};

\addplot[mark=*, boxplot, boxplot/draw position=2]
table[row sep=\\, y index=0] {
data
0.475 \\
0.5 \\
0.49 \\
0.515 \\
0.46 \\
0.525 \\
0.475 \\
0.55 \\
0.52 \\
0.505 \\
0.525 \\
0.44 \\
0.42 \\
0.41 \\
0.425 \\
0.515 \\
0.5 \\
0.385 \\
0.455 \\
0.51 \\
0.5 \\
0.485 \\
0.505 \\
0.535 \\
0.465 \\
0.455 \\
0.48 \\
0.45 \\
0.555 \\
0.485 \\
};

\addplot[mark=*, boxplot, boxplot/draw position=11]
table[row sep=\\, y index=0] {
data
0.585 \\
0.44 \\
0.79 \\
0.655 \\
0.47 \\
0.665 \\
0.61 \\
0.575 \\
0.44 \\
0.745 \\
0.505 \\
0.505 \\
0.575 \\
0.75 \\
0.44 \\
0.405 \\
0.66 \\
0.56 \\
0.44 \\
0.735 \\
0.43 \\
0.55 \\
0.635 \\
0.63 \\
0.8 \\
0.505 \\
0.58 \\
0.435 \\
0.44 \\
0.425 \\
};

\addplot[mark=*, boxplot, boxplot/draw position=6]
table[row sep=\\, y index=0] {
data
0.83 \\
0.615 \\
0.73 \\
0.605 \\
0.85 \\
0.845 \\
0.575 \\
0.795 \\
0.75 \\
0.715 \\
0.705 \\
0.77 \\
0.665 \\
0.68 \\
0.71 \\
0.675 \\
0.71 \\
0.785 \\
0.495 \\
0.52 \\
0.5 \\
0.62 \\
0.56 \\
0.62 \\
0.66 \\
0.45 \\
0.595 \\
0.75 \\
0.555 \\
0.83 \\
};

\addplot[mark=*, boxplot, boxplot/draw position=10]
table[row sep=\\, y index=0] {
data
0.38 \\
0.77 \\
0.7 \\
0.695 \\
0.66 \\
0.83 \\
0.88 \\
0.61 \\
0.795 \\
0.69 \\
0.625 \\
0.795 \\
0.765 \\
0.69 \\
0.66 \\
0.76 \\
0.605 \\
0.65 \\
0.525 \\
0.475 \\
0.495 \\
0.65 \\
0.525 \\
0.425 \\
0.635 \\
0.65 \\
0.575 \\
0.445 \\
0.655 \\
0.485 \\
};

\addplot[mark=*, boxplot, boxplot/draw position=7]
table[row sep=\\, y index=0] {
data
0.715 \\
0.5 \\
0.53 \\
0.715 \\
0.675 \\
0.8 \\
0.585 \\
0.745 \\
0.63 \\
0.76 \\
0.81 \\
0.625 \\
0.62 \\
0.67 \\
0.875 \\
0.76 \\
0.835 \\
0.625 \\
0.88 \\
0.88 \\
0.585 \\
0.65 \\
0.77 \\
0.66 \\
0.625 \\
0.655 \\
0.54 \\
0.995 \\
0.665 \\
0.685 \\
};

\addplot[mark=*, boxplot, boxplot/draw position=12]
table[row sep=\\, y index=0] {
data
0.44 \\
0.44 \\
0.725 \\
0.505 \\
0.705 \\
0.455 \\
0.445 \\
0.435 \\
0.535 \\
0.425 \\
0.615 \\
0.56 \\
0.475 \\
0.575 \\
0.425 \\
0.44 \\
0.575 \\
0.465 \\
0.7 \\
0.835 \\
0.74 \\
0.445 \\
0.7 \\
0.45 \\
0.425 \\
0.595 \\
0.505 \\
0.425 \\
0.44 \\
0.44 \\
};

\addplot[mark=*, boxplot, boxplot/draw position=0]
table[row sep=\\, y index=0] {
data
0.54 \\
0.54 \\
0.45 \\
0.505 \\
0.51 \\
0.515 \\
0.51 \\
0.485 \\
0.48 \\
0.525 \\
0.435 \\
0.535 \\
0.475 \\
0.515 \\
0.47 \\
0.45 \\
0.535 \\
0.48 \\
0.5 \\
0.5 \\
0.5 \\
0.515 \\
0.54 \\
0.47 \\
0.51 \\
0.45 \\
0.46 \\
0.49 \\
0.45 \\
0.515 \\
};

\addplot[mark=*, boxplot, boxplot/draw position=15]
table[row sep=\\, y index=0] {
data
0.445 \\
0.445 \\
0.5 \\
0.5 \\
0.445 \\
0.445 \\
0.445 \\
0.445 \\
0.445 \\
0.445 \\
0.445 \\
0.445 \\
0.445 \\
0.445 \\
0.5 \\
0.445 \\
0.445 \\
0.44 \\
0.5 \\
0.445 \\
0.445 \\
0.445 \\
0.445 \\
0.445 \\
0.445 \\
0.5 \\
0.445 \\
0.58 \\
0.445 \\
0.445 \\
};

\addplot[mark=*, boxplot, boxplot/draw position=1]
table[row sep=\\, y index=0] {
data
0.51 \\
0.445 \\
0.485 \\
0.43 \\
0.525 \\
0.425 \\
0.485 \\
0.405 \\
0.46 \\
0.495 \\
0.475 \\
0.535 \\
0.54 \\
0.53 \\
0.495 \\
0.52 \\
0.49 \\
0.44 \\
0.535 \\
0.48 \\
0.55 \\
0.45 \\
0.475 \\
0.51 \\
0.57 \\
0.435 \\
0.49 \\
0.395 \\
0.51 \\
0.495 \\
};

\addplot[mark=*, boxplot, boxplot/draw position=8]
table[row sep=\\, y index=0] {
data
0.48 \\
0.695 \\
0.555 \\
0.65 \\
0.795 \\
0.715 \\
0.745 \\
0.735 \\
0.615 \\
0.895 \\
0.84 \\
0.745 \\
0.78 \\
0.71 \\
0.775 \\
0.75 \\
0.845 \\
0.645 \\
0.83 \\
0.565 \\
0.91 \\
0.58 \\
0.565 \\
0.625 \\
0.66 \\
0.75 \\
0.755 \\
0.825 \\
0.775 \\
0.795 \\
};

\addplot[mark=*, boxplot, boxplot/draw position=14]
table[row sep=\\, y index=0] {
data
0.445 \\
0.44 \\
0.445 \\
0.45 \\
0.44 \\
0.44 \\
0.5 \\
0.445 \\
0.44 \\
0.44 \\
0.44 \\
0.44 \\
0.545 \\
0.445 \\
0.445 \\
0.47 \\
0.445 \\
0.44 \\
0.445 \\
0.57 \\
0.44 \\
0.44 \\
0.445 \\
0.445 \\
0.445 \\
0.5 \\
0.445 \\
0.445 \\
0.55 \\
0.48 \\
};

\addplot[mark=*, boxplot, boxplot/draw position=4]
table[row sep=\\, y index=0] {
data
0.475 \\
0.515 \\
0.565 \\
0.715 \\
0.695 \\
0.49 \\
0.525 \\
0.565 \\
0.62 \\
0.625 \\
0.555 \\
0.59 \\
0.565 \\
0.505 \\
0.515 \\
0.655 \\
0.5 \\
0.55 \\
0.58 \\
0.55 \\
0.43 \\
0.57 \\
0.56 \\
0.66 \\
0.64 \\
0.505 \\
0.65 \\
0.52 \\
0.67 \\
0.455 \\
};

\addplot[mark=*, boxplot, boxplot/draw position=16]
table[row sep=\\, y index=0] {
data
0.5 \\
0.5 \\
0.5 \\
0.5 \\
0.5 \\
0.5 \\
0.5 \\
0.5 \\
0.5 \\
0.5 \\
0.5 \\
0.5 \\
0.5 \\
0.5 \\
0.5 \\
0.5 \\
0.5 \\
0.5 \\
0.5 \\
0.5 \\
0.5 \\
0.5 \\
0.5 \\
0.5 \\
0.5 \\
0.5 \\
0.5 \\
0.5 \\
0.5 \\
0.5 \\
};

\addplot[mark=*, boxplot, boxplot/draw position=13]
table[row sep=\\, y index=0] {
data
0.445 \\
0.44 \\
0.54 \\
0.44 \\
0.575 \\
0.49 \\
0.435 \\
0.44 \\
0.44 \\
0.44 \\
0.44 \\
0.475 \\
0.585 \\
0.445 \\
0.44 \\
0.445 \\
0.465 \\
0.495 \\
0.44 \\
0.445 \\
0.505 \\
0.57 \\
0.995 \\
0.44 \\
0.585 \\
0.465 \\
0.515 \\
0.445 \\
0.44 \\
0.485 \\
};
}{0.2}{Input connectivity}{20}

        }
        \subfloat[N=85]{
            \myboxplot{

\addplot[mark=*, boxplot, boxplot/draw position=5]
table[row sep=\\, y index=0] {
data
0.465 \\
0.49 \\
0.57 \\
0.46 \\
0.76 \\
0.675 \\
0.61 \\
0.69 \\
0.59 \\
0.675 \\
0.74 \\
0.665 \\
0.51 \\
0.625 \\
0.57 \\
0.665 \\
0.57 \\
0.66 \\
0.685 \\
0.63 \\
0.6 \\
0.635 \\
0.455 \\
0.515 \\
0.57 \\
0.65 \\
0.46 \\
0.525 \\
0.635 \\
0.57 \\
};

\addplot[mark=*, boxplot, boxplot/draw position=9]
table[row sep=\\, y index=0] {
data
0.765 \\
0.865 \\
0.705 \\
0.725 \\
0.815 \\
0.9 \\
0.735 \\
0.705 \\
0.795 \\
0.88 \\
0.995 \\
0.7 \\
0.755 \\
0.68 \\
0.825 \\
0.615 \\
0.66 \\
0.925 \\
0.735 \\
0.68 \\
0.75 \\
0.46 \\
0.69 \\
0.71 \\
0.835 \\
0.72 \\
0.57 \\
0.665 \\
0.72 \\
0.785 \\
};

\addplot[mark=*, boxplot, boxplot/draw position=3]
table[row sep=\\, y index=0] {
data
0.505 \\
0.53 \\
0.52 \\
0.51 \\
0.59 \\
0.5 \\
0.49 \\
0.46 \\
0.485 \\
0.445 \\
0.49 \\
0.5 \\
0.525 \\
0.56 \\
0.53 \\
0.655 \\
0.48 \\
0.465 \\
0.495 \\
0.545 \\
0.485 \\
0.53 \\
0.485 \\
0.595 \\
0.53 \\
0.51 \\
0.67 \\
0.44 \\
0.495 \\
0.5 \\
};

\addplot[mark=*, boxplot, boxplot/draw position=2]
table[row sep=\\, y index=0] {
data
0.46 \\
0.515 \\
0.43 \\
0.555 \\
0.465 \\
0.54 \\
0.415 \\
0.51 \\
0.505 \\
0.475 \\
0.55 \\
0.5 \\
0.475 \\
0.455 \\
0.475 \\
0.465 \\
0.46 \\
0.495 \\
0.495 \\
0.58 \\
0.465 \\
0.45 \\
0.495 \\
0.475 \\
0.51 \\
0.545 \\
0.5 \\
0.47 \\
0.495 \\
0.5 \\
};

\addplot[mark=*, boxplot, boxplot/draw position=11]
table[row sep=\\, y index=0] {
data
0.665 \\
0.44 \\
0.65 \\
0.745 \\
0.465 \\
0.575 \\
0.5 \\
0.81 \\
0.48 \\
0.495 \\
0.57 \\
0.43 \\
0.805 \\
0.775 \\
0.72 \\
0.47 \\
0.74 \\
0.865 \\
0.605 \\
0.635 \\
0.525 \\
0.695 \\
0.59 \\
0.45 \\
0.585 \\
0.465 \\
0.775 \\
0.74 \\
0.645 \\
0.45 \\
};

\addplot[mark=*, boxplot, boxplot/draw position=6]
table[row sep=\\, y index=0] {
data
0.695 \\
0.68 \\
0.67 \\
0.525 \\
0.55 \\
0.62 \\
0.77 \\
0.76 \\
0.715 \\
0.545 \\
0.705 \\
0.75 \\
0.455 \\
0.63 \\
0.655 \\
0.735 \\
0.47 \\
0.7 \\
0.76 \\
0.68 \\
0.62 \\
0.655 \\
0.645 \\
0.545 \\
0.67 \\
0.62 \\
0.785 \\
0.73 \\
0.755 \\
0.76 \\
};

\addplot[mark=*, boxplot, boxplot/draw position=10]
table[row sep=\\, y index=0] {
data
0.815 \\
0.695 \\
0.815 \\
0.675 \\
0.68 \\
0.795 \\
0.77 \\
0.71 \\
0.77 \\
0.725 \\
0.855 \\
0.825 \\
0.745 \\
0.65 \\
0.485 \\
0.69 \\
0.715 \\
0.415 \\
0.56 \\
0.665 \\
0.645 \\
0.76 \\
0.82 \\
0.575 \\
0.65 \\
0.8 \\
0.81 \\
0.74 \\
0.69 \\
0.855 \\
};

\addplot[mark=*, boxplot, boxplot/draw position=7]
table[row sep=\\, y index=0] {
data
0.835 \\
0.82 \\
0.64 \\
0.79 \\
0.735 \\
0.715 \\
0.83 \\
0.66 \\
0.725 \\
0.655 \\
0.71 \\
0.635 \\
0.665 \\
0.585 \\
0.72 \\
0.8 \\
0.84 \\
0.69 \\
0.655 \\
0.645 \\
0.735 \\
0.745 \\
0.505 \\
0.655 \\
0.815 \\
0.635 \\
0.755 \\
0.62 \\
0.63 \\
0.74 \\
};

\addplot[mark=*, boxplot, boxplot/draw position=12]
table[row sep=\\, y index=0] {
data
0.585 \\
0.545 \\
0.735 \\
0.52 \\
0.58 \\
0.49 \\
0.515 \\
0.57 \\
0.65 \\
0.46 \\
0.625 \\
0.47 \\
0.44 \\
0.445 \\
0.44 \\
0.44 \\
0.415 \\
0.595 \\
0.43 \\
0.44 \\
0.425 \\
0.665 \\
0.445 \\
0.57 \\
0.44 \\
0.42 \\
0.525 \\
0.535 \\
0.63 \\
0.65 \\
};

\addplot[mark=*, boxplot, boxplot/draw position=0]
table[row sep=\\, y index=0] {
data
0.52 \\
0.49 \\
0.535 \\
0.515 \\
0.47 \\
0.48 \\
0.47 \\
0.45 \\
0.5 \\
0.465 \\
0.505 \\
0.505 \\
0.495 \\
0.47 \\
0.47 \\
0.515 \\
0.515 \\
0.5 \\
0.49 \\
0.56 \\
0.465 \\
0.48 \\
0.485 \\
0.465 \\
0.54 \\
0.55 \\
0.5 \\
0.49 \\
0.52 \\
0.51 \\
};

\addplot[mark=*, boxplot, boxplot/draw position=15]
table[row sep=\\, y index=0] {
data
0.445 \\
0.445 \\
0.46 \\
0.445 \\
0.445 \\
0.44 \\
0.44 \\
0.445 \\
0.45 \\
0.445 \\
0.445 \\
0.445 \\
0.445 \\
0.45 \\
0.45 \\
0.44 \\
0.44 \\
0.545 \\
0.445 \\
0.72 \\
0.445 \\
0.44 \\
0.44 \\
0.445 \\
0.445 \\
0.445 \\
0.66 \\
0.445 \\
0.445 \\
0.525 \\
};

\addplot[mark=*, boxplot, boxplot/draw position=1]
table[row sep=\\, y index=0] {
data
0.53 \\
0.505 \\
0.495 \\
0.54 \\
0.505 \\
0.525 \\
0.51 \\
0.51 \\
0.51 \\
0.475 \\
0.51 \\
0.5 \\
0.485 \\
0.485 \\
0.515 \\
0.48 \\
0.475 \\
0.525 \\
0.44 \\
0.49 \\
0.485 \\
0.48 \\
0.5 \\
0.495 \\
0.485 \\
0.475 \\
0.485 \\
0.465 \\
0.52 \\
0.53 \\
};

\addplot[mark=*, boxplot, boxplot/draw position=8]
table[row sep=\\, y index=0] {
data
0.81 \\
0.655 \\
0.655 \\
0.76 \\
0.765 \\
0.83 \\
0.88 \\
0.7 \\
0.595 \\
0.61 \\
0.775 \\
0.785 \\
0.67 \\
0.79 \\
0.89 \\
0.605 \\
0.555 \\
0.54 \\
0.645 \\
0.685 \\
0.77 \\
0.8 \\
0.67 \\
0.835 \\
0.885 \\
0.785 \\
0.62 \\
0.685 \\
0.605 \\
0.56 \\
};

\addplot[mark=*, boxplot, boxplot/draw position=14]
table[row sep=\\, y index=0] {
data
0.48 \\
0.655 \\
0.565 \\
0.44 \\
0.52 \\
0.445 \\
0.445 \\
0.445 \\
0.44 \\
0.44 \\
0.41 \\
0.44 \\
0.445 \\
0.44 \\
0.44 \\
0.445 \\
0.5 \\
0.645 \\
0.44 \\
0.745 \\
0.44 \\
0.445 \\
0.44 \\
0.44 \\
0.44 \\
0.485 \\
0.445 \\
0.425 \\
0.44 \\
0.455 \\
};

\addplot[mark=*, boxplot, boxplot/draw position=4]
table[row sep=\\, y index=0] {
data
0.495 \\
0.59 \\
0.72 \\
0.58 \\
0.485 \\
0.48 \\
0.83 \\
0.44 \\
0.49 \\
0.535 \\
0.725 \\
0.555 \\
0.565 \\
0.58 \\
0.555 \\
0.7 \\
0.62 \\
0.605 \\
0.565 \\
0.515 \\
0.495 \\
0.57 \\
0.505 \\
0.57 \\
0.52 \\
0.62 \\
0.645 \\
0.53 \\
0.55 \\
0.68 \\
};

\addplot[mark=*, boxplot, boxplot/draw position=16]
table[row sep=\\, y index=0] {
data
0.47 \\
0.445 \\
0.445 \\
0.445 \\
0.445 \\
0.445 \\
0.445 \\
0.445 \\
0.445 \\
0.445 \\
0.445 \\
0.445 \\
0.445 \\
0.445 \\
0.445 \\
0.445 \\
0.445 \\
0.445 \\
0.45 \\
0.445 \\
0.5 \\
0.445 \\
0.445 \\
0.445 \\
0.495 \\
0.445 \\
0.445 \\
0.445 \\
0.445 \\
0.445 \\
};

\addplot[mark=*, boxplot, boxplot/draw position=17]
table[row sep=\\, y index=0] {
data
0.5 \\
0.5 \\
0.5 \\
0.5 \\
0.5 \\
0.5 \\
0.5 \\
0.5 \\
0.5 \\
0.5 \\
0.5 \\
0.5 \\
0.5 \\
0.5 \\
0.5 \\
0.5 \\
0.5 \\
0.5 \\
0.5 \\
0.5 \\
0.5 \\
0.5 \\
0.5 \\
0.5 \\
0.5 \\
0.5 \\
0.5 \\
0.5 \\
0.5 \\
0.5 \\
};

\addplot[mark=*, boxplot, boxplot/draw position=13]
table[row sep=\\, y index=0] {
data
0.45 \\
0.425 \\
0.74 \\
0.44 \\
0.51 \\
0.75 \\
0.62 \\
0.605 \\
0.465 \\
0.44 \\
0.46 \\
0.44 \\
0.515 \\
0.385 \\
0.405 \\
0.495 \\
0.7 \\
0.44 \\
0.55 \\
0.485 \\
0.465 \\
0.47 \\
0.47 \\
0.525 \\
0.44 \\
0.51 \\
0.48 \\
0.445 \\
0.495 \\
0.585 \\
};
}{0.2}{Input connectivity}{20}

        }
    }
    \resizebox{\textwidth}{!}{
        \subfloat[N=90]{
            \myboxplot{

\addplot[mark=*, boxplot, boxplot/draw position=5]
table[row sep=\\, y index=0] {
data
0.685 \\
0.49 \\
0.515 \\
0.605 \\
0.515 \\
0.625 \\
0.75 \\
0.685 \\
0.525 \\
0.61 \\
0.535 \\
0.565 \\
0.56 \\
0.67 \\
0.67 \\
0.715 \\
0.465 \\
0.63 \\
0.505 \\
0.64 \\
0.635 \\
0.56 \\
0.515 \\
0.53 \\
0.62 \\
0.55 \\
0.705 \\
0.5 \\
0.665 \\
0.75 \\
};

\addplot[mark=*, boxplot, boxplot/draw position=9]
table[row sep=\\, y index=0] {
data
0.615 \\
0.64 \\
0.81 \\
0.695 \\
0.735 \\
0.75 \\
0.64 \\
0.795 \\
0.78 \\
0.72 \\
0.735 \\
0.81 \\
0.615 \\
0.745 \\
0.785 \\
0.61 \\
0.425 \\
0.75 \\
0.84 \\
0.76 \\
0.63 \\
0.615 \\
0.62 \\
0.855 \\
0.75 \\
0.52 \\
0.8 \\
0.775 \\
0.855 \\
0.76 \\
};

\addplot[mark=*, boxplot, boxplot/draw position=3]
table[row sep=\\, y index=0] {
data
0.485 \\
0.525 \\
0.455 \\
0.465 \\
0.415 \\
0.53 \\
0.535 \\
0.505 \\
0.76 \\
0.435 \\
0.52 \\
0.45 \\
0.53 \\
0.51 \\
0.585 \\
0.535 \\
0.5 \\
0.595 \\
0.525 \\
0.565 \\
0.575 \\
0.505 \\
0.49 \\
0.455 \\
0.525 \\
0.56 \\
0.505 \\
0.5 \\
0.495 \\
0.54 \\
};

\addplot[mark=*, boxplot, boxplot/draw position=2]
table[row sep=\\, y index=0] {
data
0.51 \\
0.475 \\
0.51 \\
0.45 \\
0.465 \\
0.52 \\
0.45 \\
0.485 \\
0.42 \\
0.5 \\
0.505 \\
0.485 \\
0.49 \\
0.525 \\
0.465 \\
0.455 \\
0.5 \\
0.47 \\
0.555 \\
0.475 \\
0.48 \\
0.48 \\
0.465 \\
0.525 \\
0.52 \\
0.465 \\
0.51 \\
0.48 \\
0.44 \\
0.515 \\
};

\addplot[mark=*, boxplot, boxplot/draw position=11]
table[row sep=\\, y index=0] {
data
0.61 \\
0.9 \\
0.825 \\
0.715 \\
0.815 \\
0.805 \\
0.465 \\
0.665 \\
0.49 \\
0.615 \\
0.745 \\
0.705 \\
0.695 \\
0.665 \\
0.76 \\
0.745 \\
0.565 \\
0.62 \\
0.835 \\
0.73 \\
0.435 \\
0.52 \\
0.86 \\
0.59 \\
0.6 \\
0.475 \\
0.53 \\
0.715 \\
0.805 \\
0.75 \\
};

\addplot[mark=*, boxplot, boxplot/draw position=6]
table[row sep=\\, y index=0] {
data
0.53 \\
0.745 \\
0.815 \\
0.585 \\
0.54 \\
0.74 \\
0.775 \\
0.71 \\
0.595 \\
0.535 \\
0.585 \\
0.62 \\
0.485 \\
0.645 \\
0.425 \\
0.655 \\
0.74 \\
0.605 \\
0.675 \\
0.655 \\
0.76 \\
0.67 \\
0.505 \\
0.71 \\
0.56 \\
0.555 \\
0.825 \\
0.655 \\
0.625 \\
0.57 \\
};

\addplot[mark=*, boxplot, boxplot/draw position=10]
table[row sep=\\, y index=0] {
data
0.56 \\
0.605 \\
0.735 \\
0.63 \\
0.535 \\
0.41 \\
0.845 \\
0.63 \\
0.49 \\
0.535 \\
0.445 \\
0.77 \\
0.72 \\
0.645 \\
0.8 \\
0.675 \\
0.665 \\
0.76 \\
0.83 \\
0.815 \\
0.76 \\
0.765 \\
0.825 \\
0.58 \\
0.55 \\
0.86 \\
0.7 \\
0.815 \\
0.46 \\
0.75 \\
};

\addplot[mark=*, boxplot, boxplot/draw position=7]
table[row sep=\\, y index=0] {
data
0.575 \\
0.68 \\
0.545 \\
0.62 \\
0.735 \\
0.685 \\
0.685 \\
0.525 \\
0.65 \\
0.68 \\
0.855 \\
0.835 \\
0.68 \\
0.825 \\
0.79 \\
0.67 \\
0.98 \\
0.74 \\
0.66 \\
0.64 \\
0.65 \\
0.65 \\
0.61 \\
0.695 \\
0.775 \\
0.62 \\
0.64 \\
0.59 \\
0.51 \\
0.645 \\
};

\addplot[mark=*, boxplot, boxplot/draw position=12]
table[row sep=\\, y index=0] {
data
0.475 \\
0.675 \\
0.49 \\
0.58 \\
0.555 \\
0.62 \\
0.455 \\
0.45 \\
0.83 \\
0.735 \\
0.72 \\
0.715 \\
0.515 \\
0.635 \\
0.49 \\
0.465 \\
0.535 \\
0.44 \\
0.695 \\
0.48 \\
0.79 \\
0.515 \\
0.44 \\
0.445 \\
0.895 \\
0.645 \\
0.835 \\
0.59 \\
0.6 \\
0.63 \\
};

\addplot[mark=*, boxplot, boxplot/draw position=0]
table[row sep=\\, y index=0] {
data
0.54 \\
0.475 \\
0.465 \\
0.545 \\
0.46 \\
0.515 \\
0.545 \\
0.515 \\
0.505 \\
0.475 \\
0.45 \\
0.505 \\
0.48 \\
0.555 \\
0.475 \\
0.5 \\
0.52 \\
0.52 \\
0.52 \\
0.465 \\
0.535 \\
0.425 \\
0.595 \\
0.505 \\
0.525 \\
0.525 \\
0.425 \\
0.47 \\
0.52 \\
0.48 \\
};

\addplot[mark=*, boxplot, boxplot/draw position=15]
table[row sep=\\, y index=0] {
data
0.44 \\
0.625 \\
0.44 \\
0.44 \\
0.44 \\
0.65 \\
0.52 \\
0.425 \\
0.44 \\
0.535 \\
0.535 \\
0.525 \\
0.44 \\
0.44 \\
0.445 \\
0.515 \\
0.445 \\
0.445 \\
0.445 \\
0.44 \\
0.445 \\
0.445 \\
0.54 \\
0.63 \\
0.455 \\
0.445 \\
0.44 \\
0.44 \\
0.45 \\
0.44 \\
};

\addplot[mark=*, boxplot, boxplot/draw position=1]
table[row sep=\\, y index=0] {
data
0.455 \\
0.495 \\
0.48 \\
0.46 \\
0.54 \\
0.475 \\
0.49 \\
0.495 \\
0.495 \\
0.485 \\
0.465 \\
0.535 \\
0.47 \\
0.545 \\
0.475 \\
0.465 \\
0.53 \\
0.46 \\
0.515 \\
0.47 \\
0.435 \\
0.465 \\
0.5 \\
0.46 \\
0.455 \\
0.51 \\
0.535 \\
0.53 \\
0.475 \\
0.5 \\
};

\addplot[mark=*, boxplot, boxplot/draw position=8]
table[row sep=\\, y index=0] {
data
0.63 \\
0.785 \\
0.645 \\
0.775 \\
0.765 \\
0.76 \\
0.505 \\
0.64 \\
0.82 \\
0.785 \\
0.655 \\
0.655 \\
0.885 \\
0.54 \\
0.885 \\
0.655 \\
0.815 \\
0.735 \\
0.77 \\
0.76 \\
0.7 \\
0.7 \\
0.915 \\
0.865 \\
0.62 \\
0.8 \\
0.655 \\
0.72 \\
0.8 \\
0.69 \\
};

\addplot[mark=*, boxplot, boxplot/draw position=14]
table[row sep=\\, y index=0] {
data
0.44 \\
0.425 \\
0.44 \\
0.525 \\
0.44 \\
0.44 \\
0.475 \\
0.41 \\
0.44 \\
0.455 \\
0.44 \\
0.455 \\
0.47 \\
0.45 \\
0.515 \\
0.7 \\
0.86 \\
0.475 \\
0.625 \\
0.51 \\
0.465 \\
0.49 \\
0.625 \\
0.425 \\
0.59 \\
0.51 \\
0.69 \\
0.58 \\
0.585 \\
0.585 \\
};

\addplot[mark=*, boxplot, boxplot/draw position=4]
table[row sep=\\, y index=0] {
data
0.495 \\
0.615 \\
0.615 \\
0.455 \\
0.53 \\
0.54 \\
0.5 \\
0.54 \\
0.515 \\
0.48 \\
0.535 \\
0.545 \\
0.51 \\
0.54 \\
0.525 \\
0.53 \\
0.55 \\
0.64 \\
0.515 \\
0.615 \\
0.53 \\
0.52 \\
0.495 \\
0.54 \\
0.535 \\
0.59 \\
0.475 \\
0.55 \\
0.555 \\
0.55 \\
};

\addplot[mark=*, boxplot, boxplot/draw position=16]
table[row sep=\\, y index=0] {
data
0.445 \\
0.445 \\
0.645 \\
0.455 \\
0.445 \\
0.48 \\
0.445 \\
0.44 \\
0.64 \\
0.445 \\
0.645 \\
0.51 \\
0.445 \\
0.445 \\
0.495 \\
0.44 \\
0.445 \\
0.44 \\
0.63 \\
0.51 \\
0.545 \\
0.405 \\
0.46 \\
0.555 \\
0.445 \\
0.445 \\
0.445 \\
0.445 \\
0.445 \\
0.445 \\
};

\addplot[mark=*, boxplot, boxplot/draw position=17]
table[row sep=\\, y index=0] {
data
0.445 \\
0.445 \\
0.5 \\
0.445 \\
0.445 \\
0.445 \\
0.445 \\
0.445 \\
0.445 \\
0.445 \\
0.445 \\
0.445 \\
0.445 \\
0.51 \\
0.445 \\
0.5 \\
0.445 \\
0.445 \\
0.445 \\
0.445 \\
0.445 \\
0.445 \\
0.5 \\
0.445 \\
0.445 \\
0.445 \\
0.445 \\
0.445 \\
0.445 \\
0.445 \\
};

\addplot[mark=*, boxplot, boxplot/draw position=18]
table[row sep=\\, y index=0] {
data
0.5 \\
0.5 \\
0.5 \\
0.5 \\
0.5 \\
0.5 \\
0.5 \\
0.5 \\
0.5 \\
0.5 \\
0.5 \\
0.5 \\
0.5 \\
0.5 \\
0.5 \\
0.5 \\
0.5 \\
0.5 \\
0.5 \\
0.5 \\
0.5 \\
0.5 \\
0.5 \\
0.5 \\
0.5 \\
0.5 \\
0.5 \\
0.5 \\
0.5 \\
0.5 \\
};

\addplot[mark=*, boxplot, boxplot/draw position=13]
table[row sep=\\, y index=0] {
data
0.44 \\
0.44 \\
0.71 \\
0.69 \\
0.54 \\
0.46 \\
0.755 \\
0.525 \\
0.635 \\
0.525 \\
0.635 \\
0.44 \\
0.54 \\
0.775 \\
0.46 \\
0.58 \\
0.425 \\
0.595 \\
0.7 \\
0.605 \\
0.52 \\
0.59 \\
0.805 \\
0.67 \\
0.53 \\
0.475 \\
0.43 \\
0.61 \\
0.615 \\
0.46 \\
};
}{0.2}{Input connectivity}

        }
        \subfloat[N=95]{
            \myboxplot{

\addplot[mark=*, boxplot, boxplot/draw position=5]
table[row sep=\\, y index=0] {
data
0.565 \\
0.59 \\
0.67 \\
0.7 \\
0.505 \\
0.59 \\
0.65 \\
0.53 \\
0.7 \\
0.625 \\
0.52 \\
0.72 \\
0.525 \\
0.55 \\
0.485 \\
0.605 \\
0.535 \\
0.52 \\
0.63 \\
0.455 \\
0.515 \\
0.735 \\
0.63 \\
0.525 \\
0.625 \\
0.545 \\
0.905 \\
0.6 \\
0.62 \\
0.49 \\
};

\addplot[mark=*, boxplot, boxplot/draw position=9]
table[row sep=\\, y index=0] {
data
0.82 \\
0.67 \\
0.54 \\
0.705 \\
0.595 \\
0.665 \\
0.815 \\
0.72 \\
0.58 \\
0.825 \\
0.595 \\
0.605 \\
0.745 \\
0.785 \\
0.79 \\
0.71 \\
0.635 \\
0.795 \\
0.915 \\
0.775 \\
0.64 \\
0.805 \\
0.775 \\
0.85 \\
0.635 \\
0.81 \\
0.8 \\
0.645 \\
0.635 \\
0.865 \\
};

\addplot[mark=*, boxplot, boxplot/draw position=3]
table[row sep=\\, y index=0] {
data
0.495 \\
0.465 \\
0.56 \\
0.515 \\
0.48 \\
0.58 \\
0.56 \\
0.53 \\
0.445 \\
0.48 \\
0.53 \\
0.52 \\
0.5 \\
0.56 \\
0.485 \\
0.51 \\
0.45 \\
0.505 \\
0.53 \\
0.41 \\
0.485 \\
0.49 \\
0.43 \\
0.565 \\
0.49 \\
0.465 \\
0.485 \\
0.5 \\
0.475 \\
0.425 \\
};

\addplot[mark=*, boxplot, boxplot/draw position=2]
table[row sep=\\, y index=0] {
data
0.61 \\
0.535 \\
0.515 \\
0.465 \\
0.47 \\
0.46 \\
0.415 \\
0.51 \\
0.51 \\
0.47 \\
0.46 \\
0.5 \\
0.56 \\
0.455 \\
0.48 \\
0.475 \\
0.515 \\
0.425 \\
0.545 \\
0.525 \\
0.47 \\
0.53 \\
0.455 \\
0.475 \\
0.51 \\
0.465 \\
0.54 \\
0.515 \\
0.51 \\
0.55 \\
};

\addplot[mark=*, boxplot, boxplot/draw position=11]
table[row sep=\\, y index=0] {
data
0.67 \\
0.685 \\
0.85 \\
0.64 \\
0.775 \\
0.79 \\
0.72 \\
0.57 \\
0.75 \\
0.655 \\
0.73 \\
0.73 \\
0.75 \\
0.715 \\
0.725 \\
0.805 \\
0.77 \\
0.715 \\
0.855 \\
0.725 \\
0.63 \\
0.72 \\
0.6 \\
0.705 \\
0.74 \\
0.66 \\
0.59 \\
0.765 \\
0.63 \\
0.665 \\
};

\addplot[mark=*, boxplot, boxplot/draw position=6]
table[row sep=\\, y index=0] {
data
0.53 \\
0.72 \\
0.665 \\
0.675 \\
0.84 \\
0.725 \\
0.565 \\
0.71 \\
0.68 \\
0.55 \\
0.615 \\
0.825 \\
0.61 \\
0.745 \\
0.73 \\
0.65 \\
0.605 \\
0.675 \\
0.755 \\
0.49 \\
0.62 \\
0.565 \\
0.595 \\
0.66 \\
0.565 \\
0.77 \\
0.605 \\
0.555 \\
0.715 \\
0.575 \\
};

\addplot[mark=*, boxplot, boxplot/draw position=10]
table[row sep=\\, y index=0] {
data
0.785 \\
0.675 \\
0.515 \\
0.625 \\
0.665 \\
0.825 \\
0.705 \\
0.66 \\
0.55 \\
0.705 \\
0.615 \\
0.83 \\
0.91 \\
0.725 \\
0.68 \\
0.705 \\
0.56 \\
0.52 \\
0.85 \\
0.91 \\
0.705 \\
0.705 \\
0.65 \\
0.545 \\
0.555 \\
0.82 \\
0.59 \\
0.88 \\
0.66 \\
0.755 \\
};

\addplot[mark=*, boxplot, boxplot/draw position=7]
table[row sep=\\, y index=0] {
data
0.755 \\
0.725 \\
0.61 \\
0.6 \\
0.58 \\
0.55 \\
0.54 \\
0.785 \\
0.67 \\
0.63 \\
0.62 \\
0.715 \\
0.68 \\
0.61 \\
0.735 \\
0.79 \\
0.675 \\
0.72 \\
0.69 \\
0.74 \\
0.815 \\
0.65 \\
0.72 \\
0.77 \\
0.835 \\
0.7 \\
0.62 \\
0.73 \\
0.685 \\
0.66 \\
};

\addplot[mark=*, boxplot, boxplot/draw position=12]
table[row sep=\\, y index=0] {
data
0.655 \\
0.71 \\
0.87 \\
0.69 \\
0.655 \\
0.675 \\
0.69 \\
0.76 \\
0.475 \\
0.585 \\
0.66 \\
0.81 \\
0.705 \\
0.675 \\
0.95 \\
0.56 \\
0.65 \\
0.92 \\
0.715 \\
0.71 \\
0.7 \\
0.425 \\
0.62 \\
0.65 \\
0.545 \\
0.6 \\
0.61 \\
0.815 \\
0.815 \\
0.67 \\
};

\addplot[mark=*, boxplot, boxplot/draw position=0]
table[row sep=\\, y index=0] {
data
0.49 \\
0.545 \\
0.48 \\
0.51 \\
0.45 \\
0.445 \\
0.45 \\
0.535 \\
0.535 \\
0.465 \\
0.495 \\
0.505 \\
0.465 \\
0.485 \\
0.45 \\
0.42 \\
0.52 \\
0.5 \\
0.525 \\
0.51 \\
0.55 \\
0.45 \\
0.45 \\
0.46 \\
0.46 \\
0.44 \\
0.48 \\
0.515 \\
0.375 \\
0.44 \\
};

\addplot[mark=*, boxplot, boxplot/draw position=15]
table[row sep=\\, y index=0] {
data
0.445 \\
0.44 \\
0.505 \\
0.48 \\
0.485 \\
0.44 \\
0.44 \\
0.7 \\
0.405 \\
0.44 \\
0.44 \\
0.405 \\
0.505 \\
0.415 \\
0.525 \\
0.48 \\
0.44 \\
0.565 \\
0.44 \\
0.465 \\
0.64 \\
0.44 \\
0.425 \\
0.415 \\
0.44 \\
0.6 \\
0.585 \\
0.44 \\
0.555 \\
0.44 \\
};

\addplot[mark=*, boxplot, boxplot/draw position=1]
table[row sep=\\, y index=0] {
data
0.495 \\
0.485 \\
0.505 \\
0.54 \\
0.495 \\
0.48 \\
0.5 \\
0.485 \\
0.52 \\
0.48 \\
0.515 \\
0.495 \\
0.515 \\
0.49 \\
0.565 \\
0.425 \\
0.535 \\
0.51 \\
0.51 \\
0.48 \\
0.545 \\
0.525 \\
0.485 \\
0.48 \\
0.47 \\
0.435 \\
0.53 \\
0.48 \\
0.505 \\
0.445 \\
};

\addplot[mark=*, boxplot, boxplot/draw position=8]
table[row sep=\\, y index=0] {
data
0.695 \\
0.605 \\
0.76 \\
0.56 \\
0.69 \\
0.67 \\
0.7 \\
0.795 \\
0.845 \\
0.765 \\
0.765 \\
0.76 \\
0.78 \\
0.635 \\
0.655 \\
0.905 \\
0.53 \\
0.825 \\
0.805 \\
0.84 \\
0.775 \\
0.6 \\
0.8 \\
0.71 \\
0.72 \\
0.59 \\
0.64 \\
0.52 \\
0.895 \\
0.66 \\
};

\addplot[mark=*, boxplot, boxplot/draw position=14]
table[row sep=\\, y index=0] {
data
0.75 \\
0.535 \\
0.44 \\
0.44 \\
0.505 \\
0.425 \\
0.44 \\
0.445 \\
0.55 \\
0.535 \\
0.44 \\
0.42 \\
0.56 \\
0.605 \\
0.525 \\
0.5 \\
0.44 \\
0.72 \\
0.445 \\
0.495 \\
0.42 \\
0.61 \\
0.445 \\
0.44 \\
0.61 \\
0.515 \\
0.64 \\
0.705 \\
0.54 \\
0.465 \\
};

\addplot[mark=*, boxplot, boxplot/draw position=4]
table[row sep=\\, y index=0] {
data
0.555 \\
0.515 \\
0.555 \\
0.49 \\
0.48 \\
0.52 \\
0.645 \\
0.66 \\
0.535 \\
0.48 \\
0.555 \\
0.44 \\
0.69 \\
0.5 \\
0.495 \\
0.48 \\
0.605 \\
0.475 \\
0.515 \\
0.46 \\
0.575 \\
0.55 \\
0.52 \\
0.55 \\
0.54 \\
0.465 \\
0.59 \\
0.475 \\
0.485 \\
0.51 \\
};

\addplot[mark=*, boxplot, boxplot/draw position=16]
table[row sep=\\, y index=0] {
data
0.445 \\
0.445 \\
0.445 \\
0.44 \\
0.445 \\
0.445 \\
0.44 \\
0.49 \\
0.445 \\
0.445 \\
0.68 \\
0.41 \\
0.425 \\
0.65 \\
0.555 \\
0.435 \\
0.44 \\
0.44 \\
0.545 \\
0.44 \\
0.51 \\
0.665 \\
0.44 \\
0.445 \\
0.47 \\
0.545 \\
0.44 \\
0.44 \\
0.44 \\
0.49 \\
};

\addplot[mark=*, boxplot, boxplot/draw position=17]
table[row sep=\\, y index=0] {
data
0.445 \\
0.445 \\
0.445 \\
0.445 \\
0.445 \\
0.565 \\
0.44 \\
0.44 \\
0.445 \\
0.44 \\
0.44 \\
0.445 \\
0.445 \\
0.445 \\
0.445 \\
0.445 \\
0.47 \\
0.445 \\
0.44 \\
0.44 \\
0.51 \\
0.445 \\
0.445 \\
0.44 \\
0.45 \\
0.445 \\
0.445 \\
0.445 \\
0.44 \\
0.445 \\
};

\addplot[mark=*, boxplot, boxplot/draw position=19]
table[row sep=\\, y index=0] {
data
0.5 \\
0.5 \\
0.5 \\
0.5 \\
0.5 \\
0.5 \\
0.5 \\
0.5 \\
0.5 \\
0.5 \\
0.5 \\
0.5 \\
0.5 \\
0.5 \\
0.5 \\
0.5 \\
0.5 \\
0.5 \\
0.5 \\
0.5 \\
0.5 \\
0.5 \\
0.5 \\
0.5 \\
0.5 \\
0.5 \\
0.5 \\
0.5 \\
0.5 \\
0.5 \\
};

\addplot[mark=*, boxplot, boxplot/draw position=18]
table[row sep=\\, y index=0] {
data
0.445 \\
0.5 \\
0.44 \\
0.445 \\
0.445 \\
0.5 \\
0.445 \\
0.445 \\
0.5 \\
0.445 \\
0.445 \\
0.445 \\
0.445 \\
0.45 \\
0.445 \\
0.445 \\
0.445 \\
0.445 \\
0.445 \\
0.445 \\
0.5 \\
0.5 \\
0.445 \\
0.44 \\
0.445 \\
0.445 \\
0.445 \\
0.445 \\
0.445 \\
0.445 \\
};

\addplot[mark=*, boxplot, boxplot/draw position=13]
table[row sep=\\, y index=0] {
data
0.61 \\
0.555 \\
0.44 \\
0.58 \\
0.515 \\
0.54 \\
0.665 \\
0.52 \\
0.66 \\
0.69 \\
0.5 \\
0.76 \\
0.655 \\
0.44 \\
0.485 \\
0.51 \\
0.64 \\
0.545 \\
0.73 \\
0.625 \\
0.5 \\
0.565 \\
0.43 \\
0.49 \\
0.815 \\
0.535 \\
0.465 \\
0.75 \\
0.685 \\
0.53 \\
};
}{0.2}{Input connectivity}

        }
    }
    \resizebox{0.5\textwidth}{!}{
        \subfloat[N=100]{
            \myboxplot{

\addplot[mark=*, boxplot, boxplot/draw position=5]
table[row sep=\\, y index=0] {
data
0.53 \\
0.635 \\
0.5 \\
0.605 \\
0.625 \\
0.54 \\
0.535 \\
0.575 \\
0.655 \\
0.53 \\
0.63 \\
0.47 \\
0.67 \\
0.56 \\
0.63 \\
0.57 \\
0.615 \\
0.58 \\
0.615 \\
0.585 \\
0.56 \\
0.57 \\
0.49 \\
0.53 \\
0.465 \\
0.545 \\
0.485 \\
0.535 \\
0.64 \\
0.665 \\
};

\addplot[mark=*, boxplot, boxplot/draw position=9]
table[row sep=\\, y index=0] {
data
0.82 \\
0.655 \\
0.825 \\
0.83 \\
0.745 \\
0.805 \\
0.79 \\
0.8 \\
0.69 \\
0.755 \\
0.74 \\
0.73 \\
0.745 \\
0.745 \\
0.775 \\
0.725 \\
0.63 \\
0.725 \\
0.685 \\
0.59 \\
0.775 \\
0.76 \\
0.515 \\
0.715 \\
0.7 \\
0.67 \\
0.805 \\
0.455 \\
0.815 \\
0.735 \\
};

\addplot[mark=*, boxplot, boxplot/draw position=3]
table[row sep=\\, y index=0] {
data
0.5 \\
0.485 \\
0.51 \\
0.495 \\
0.495 \\
0.415 \\
0.475 \\
0.535 \\
0.45 \\
0.52 \\
0.5 \\
0.435 \\
0.575 \\
0.45 \\
0.52 \\
0.48 \\
0.465 \\
0.535 \\
0.49 \\
0.585 \\
0.455 \\
0.435 \\
0.575 \\
0.455 \\
0.54 \\
0.475 \\
0.61 \\
0.585 \\
0.6 \\
0.47 \\
};

\addplot[mark=*, boxplot, boxplot/draw position=2]
table[row sep=\\, y index=0] {
data
0.465 \\
0.515 \\
0.48 \\
0.475 \\
0.48 \\
0.53 \\
0.525 \\
0.485 \\
0.405 \\
0.54 \\
0.495 \\
0.5 \\
0.54 \\
0.47 \\
0.4 \\
0.51 \\
0.485 \\
0.56 \\
0.5 \\
0.485 \\
0.51 \\
0.5 \\
0.48 \\
0.515 \\
0.465 \\
0.46 \\
0.525 \\
0.465 \\
0.445 \\
0.495 \\
};

\addplot[mark=*, boxplot, boxplot/draw position=11]
table[row sep=\\, y index=0] {
data
0.655 \\
0.845 \\
0.625 \\
0.625 \\
0.42 \\
0.745 \\
0.55 \\
0.91 \\
0.84 \\
0.745 \\
0.845 \\
0.785 \\
0.83 \\
0.785 \\
0.995 \\
0.78 \\
0.85 \\
0.92 \\
0.51 \\
0.745 \\
0.85 \\
0.745 \\
0.75 \\
0.85 \\
0.86 \\
0.5 \\
0.55 \\
0.65 \\
0.95 \\
0.82 \\
};

\addplot[mark=*, boxplot, boxplot/draw position=6]
table[row sep=\\, y index=0] {
data
0.5 \\
0.675 \\
0.835 \\
0.69 \\
0.64 \\
0.685 \\
0.63 \\
0.585 \\
0.585 \\
0.71 \\
0.535 \\
0.59 \\
0.555 \\
0.615 \\
0.575 \\
0.585 \\
0.51 \\
0.71 \\
0.58 \\
0.695 \\
0.56 \\
0.665 \\
0.615 \\
0.805 \\
0.725 \\
0.675 \\
0.62 \\
0.685 \\
0.68 \\
0.5 \\
};

\addplot[mark=*, boxplot, boxplot/draw position=10]
table[row sep=\\, y index=0] {
data
0.75 \\
0.92 \\
0.83 \\
0.92 \\
0.63 \\
0.885 \\
0.75 \\
0.705 \\
0.72 \\
0.89 \\
0.615 \\
0.84 \\
0.695 \\
0.92 \\
0.76 \\
0.75 \\
0.715 \\
0.96 \\
0.75 \\
0.885 \\
0.945 \\
0.66 \\
0.845 \\
0.61 \\
0.565 \\
0.59 \\
0.68 \\
0.84 \\
0.64 \\
0.525 \\
};

\addplot[mark=*, boxplot, boxplot/draw position=7]
table[row sep=\\, y index=0] {
data
0.875 \\
0.585 \\
0.54 \\
0.565 \\
0.59 \\
0.66 \\
0.73 \\
0.61 \\
0.685 \\
0.575 \\
0.71 \\
0.615 \\
0.665 \\
0.64 \\
0.635 \\
0.68 \\
0.91 \\
0.745 \\
0.595 \\
0.685 \\
0.62 \\
0.665 \\
0.66 \\
0.795 \\
0.685 \\
0.585 \\
0.58 \\
0.72 \\
0.77 \\
0.67 \\
};

\addplot[mark=*, boxplot, boxplot/draw position=12]
table[row sep=\\, y index=0] {
data
0.555 \\
0.75 \\
0.805 \\
0.545 \\
0.78 \\
0.705 \\
0.475 \\
0.69 \\
0.625 \\
0.5 \\
0.655 \\
0.51 \\
0.535 \\
0.705 \\
0.8 \\
0.79 \\
0.45 \\
0.65 \\
0.575 \\
0.815 \\
0.75 \\
0.89 \\
0.565 \\
0.745 \\
0.87 \\
0.75 \\
0.46 \\
0.875 \\
0.585 \\
0.705 \\
};

\addplot[mark=*, boxplot, boxplot/draw position=0]
table[row sep=\\, y index=0] {
data
0.455 \\
0.475 \\
0.565 \\
0.53 \\
0.47 \\
0.505 \\
0.505 \\
0.48 \\
0.54 \\
0.51 \\
0.445 \\
0.46 \\
0.49 \\
0.515 \\
0.47 \\
0.46 \\
0.505 \\
0.485 \\
0.53 \\
0.49 \\
0.49 \\
0.5 \\
0.54 \\
0.54 \\
0.505 \\
0.495 \\
0.565 \\
0.49 \\
0.42 \\
0.565 \\
};

\addplot[mark=*, boxplot, boxplot/draw position=15]
table[row sep=\\, y index=0] {
data
0.485 \\
0.45 \\
0.485 \\
0.455 \\
0.465 \\
0.56 \\
0.46 \\
0.425 \\
0.48 \\
0.76 \\
0.745 \\
0.44 \\
0.685 \\
0.58 \\
0.625 \\
0.45 \\
0.475 \\
0.465 \\
0.485 \\
0.445 \\
0.655 \\
0.63 \\
0.62 \\
0.575 \\
0.425 \\
0.5 \\
0.455 \\
0.445 \\
0.425 \\
0.515 \\
};

\addplot[mark=*, boxplot, boxplot/draw position=1]
table[row sep=\\, y index=0] {
data
0.475 \\
0.49 \\
0.505 \\
0.485 \\
0.43 \\
0.495 \\
0.47 \\
0.51 \\
0.49 \\
0.435 \\
0.48 \\
0.505 \\
0.475 \\
0.465 \\
0.515 \\
0.55 \\
0.48 \\
0.475 \\
0.44 \\
0.435 \\
0.48 \\
0.455 \\
0.46 \\
0.53 \\
0.465 \\
0.485 \\
0.53 \\
0.465 \\
0.48 \\
0.475 \\
};

\addplot[mark=*, boxplot, boxplot/draw position=8]
table[row sep=\\, y index=0] {
data
0.81 \\
0.63 \\
0.635 \\
0.86 \\
0.575 \\
0.68 \\
0.605 \\
0.685 \\
0.84 \\
0.77 \\
0.84 \\
0.63 \\
0.795 \\
0.81 \\
0.54 \\
0.685 \\
0.815 \\
0.725 \\
0.805 \\
0.725 \\
0.605 \\
0.765 \\
0.81 \\
0.73 \\
0.76 \\
0.915 \\
0.7 \\
0.8 \\
0.67 \\
0.56 \\
};

\addplot[mark=*, boxplot, boxplot/draw position=14]
table[row sep=\\, y index=0] {
data
0.465 \\
0.62 \\
0.41 \\
0.545 \\
0.41 \\
0.595 \\
0.55 \\
0.535 \\
0.785 \\
0.525 \\
0.47 \\
0.465 \\
0.45 \\
0.505 \\
0.59 \\
0.51 \\
0.63 \\
0.61 \\
0.81 \\
0.495 \\
0.435 \\
0.745 \\
0.57 \\
0.44 \\
0.63 \\
0.55 \\
0.645 \\
0.44 \\
0.535 \\
0.69 \\
};

\addplot[mark=*, boxplot, boxplot/draw position=4]
table[row sep=\\, y index=0] {
data
0.5 \\
0.455 \\
0.535 \\
0.515 \\
0.58 \\
0.555 \\
0.51 \\
0.6 \\
0.51 \\
0.44 \\
0.465 \\
0.575 \\
0.415 \\
0.445 \\
0.62 \\
0.495 \\
0.51 \\
0.52 \\
0.49 \\
0.455 \\
0.56 \\
0.585 \\
0.595 \\
0.455 \\
0.52 \\
0.5 \\
0.52 \\
0.535 \\
0.655 \\
0.525 \\
};

\addplot[mark=*, boxplot, boxplot/draw position=16]
table[row sep=\\, y index=0] {
data
0.53 \\
0.43 \\
0.595 \\
0.44 \\
0.43 \\
0.705 \\
0.405 \\
0.485 \\
0.445 \\
0.675 \\
0.435 \\
0.44 \\
0.51 \\
0.485 \\
0.44 \\
0.435 \\
0.575 \\
0.54 \\
0.425 \\
0.58 \\
0.44 \\
0.45 \\
0.545 \\
0.53 \\
0.45 \\
0.465 \\
0.59 \\
0.44 \\
0.445 \\
0.44 \\
};

\addplot[mark=*, boxplot, boxplot/draw position=17]
table[row sep=\\, y index=0] {
data
0.445 \\
0.445 \\
0.445 \\
0.44 \\
0.425 \\
0.48 \\
0.445 \\
0.44 \\
0.64 \\
0.475 \\
0.44 \\
0.505 \\
0.54 \\
0.435 \\
0.44 \\
0.44 \\
0.44 \\
0.44 \\
0.445 \\
0.445 \\
0.465 \\
0.445 \\
0.445 \\
0.44 \\
0.535 \\
0.465 \\
0.41 \\
0.44 \\
0.495 \\
0.445 \\
};

\addplot[mark=*, boxplot, boxplot/draw position=19]
table[row sep=\\, y index=0] {
data
0.445 \\
0.545 \\
0.445 \\
0.445 \\
0.445 \\
0.445 \\
0.445 \\
0.445 \\
0.445 \\
0.44 \\
0.445 \\
0.445 \\
0.445 \\
0.445 \\
0.445 \\
0.445 \\
0.445 \\
0.445 \\
0.445 \\
0.445 \\
0.445 \\
0.5 \\
0.445 \\
0.445 \\
0.445 \\
0.445 \\
0.445 \\
0.445 \\
0.44 \\
0.445 \\
};

\addplot[mark=*, boxplot, boxplot/draw position=20]
table[row sep=\\, y index=0] {
data
0.5 \\
0.5 \\
0.5 \\
0.5 \\
0.5 \\
0.5 \\
0.5 \\
0.5 \\
0.5 \\
0.5 \\
0.5 \\
0.5 \\
0.5 \\
0.5 \\
0.5 \\
0.5 \\
0.5 \\
0.5 \\
0.5 \\
0.5 \\
0.5 \\
0.5 \\
0.5 \\
0.5 \\
0.5 \\
0.5 \\
0.5 \\
0.5 \\
0.5 \\
0.5 \\
};

\addplot[mark=*, boxplot, boxplot/draw position=18]
table[row sep=\\, y index=0] {
data
0.44 \\
0.445 \\
0.44 \\
0.445 \\
0.445 \\
0.44 \\
0.44 \\
0.44 \\
0.47 \\
0.44 \\
0.445 \\
0.445 \\
0.445 \\
0.445 \\
0.445 \\
0.505 \\
0.445 \\
0.445 \\
0.445 \\
0.445 \\
0.575 \\
0.44 \\
0.44 \\
0.44 \\
0.445 \\
0.44 \\
0.44 \\
0.51 \\
0.445 \\
0.445 \\
};

\addplot[mark=*, boxplot, boxplot/draw position=13]
table[row sep=\\, y index=0] {
data
0.615 \\
0.5 \\
0.595 \\
0.63 \\
0.695 \\
0.545 \\
0.59 \\
0.72 \\
0.44 \\
0.81 \\
0.63 \\
0.935 \\
0.645 \\
0.58 \\
0.78 \\
0.57 \\
0.74 \\
0.5 \\
0.495 \\
0.655 \\
0.43 \\
0.7 \\
0.7 \\
0.63 \\
0.455 \\
0.6 \\
0.82 \\
0.625 \\
0.79 \\
0.635 \\
};
}{0.2}{Input connectivity}

        }
    }
    \caption{Task 2 - Part 3}
\end{figure*}

\section{Optimal output connectivity}

To reduce the number of reservoir combinations we create in search of the optimal output connectivity,
we'll use the finding from the previous section that optimal input connectivity most often is found at $IC=Nodes/2$.
We'll then map the accuracies of reservoirs up against their output connectivities, as we did against their input connectivities in the previous section.

\subsection{Task 1}

We now test required output connectivity for task 1 on reservoirs of $size <= 100$.
Seems like it's enough to read around 10 of the nodes in the network,
which is interesting, as this is the same number of nodes that are required to solve the task in the first place!

\begin{figure*}[ht]
    \centering
    \resizebox{\textwidth}{!}{
        \subfloat[N=10]{
            \myboxplot{

\addplot[mark=*, boxplot, boxplot/draw position=0]
table[row sep=\\, y index=0] {
data
0.53 \\
0.525 \\
0.53 \\
0.73 \\
0.72 \\
0.645 \\
0.5 \\
0.855 \\
0.505 \\
0.835 \\
0.675 \\
0.57 \\
0.625 \\
0.63 \\
0.885 \\
0.795 \\
0.915 \\
1.0 \\
0.455 \\
0.87 \\
0.76 \\
0.635 \\
0.735 \\
0.75 \\
0.935 \\
0.615 \\
0.53 \\
0.885 \\
0.5 \\
0.74 \\
};

\addplot[mark=*, boxplot, boxplot/draw position=1]
table[row sep=\\, y index=0] {
data
0.53 \\
0.525 \\
0.53 \\
0.73 \\
0.72 \\
0.645 \\
0.5 \\
0.855 \\
0.505 \\
0.835 \\
0.675 \\
0.57 \\
0.625 \\
0.63 \\
0.885 \\
0.795 \\
0.915 \\
1.0 \\
0.455 \\
0.87 \\
0.76 \\
0.635 \\
0.735 \\
0.75 \\
0.935 \\
0.615 \\
0.53 \\
0.885 \\
0.5 \\
0.74 \\
};
}{0.1}{Output connectivity}{10}

        }
        \subfloat[N=20]{
            \myboxplot{

\addplot[mark=*, boxplot, boxplot/draw position=0]
table[row sep=\\, y index=0] {
data
0.67 \\
0.61 \\
0.82 \\
0.685 \\
0.875 \\
0.765 \\
0.885 \\
0.84 \\
0.735 \\
0.835 \\
0.805 \\
0.905 \\
0.82 \\
0.43 \\
0.705 \\
0.76 \\
0.855 \\
1.0 \\
0.89 \\
0.885 \\
0.73 \\
0.675 \\
0.835 \\
1.0 \\
0.73 \\
0.54 \\
0.765 \\
0.785 \\
0.765 \\
0.68 \\
};

\addplot[mark=*, boxplot, boxplot/draw position=1]
table[row sep=\\, y index=0] {
data
0.74 \\
0.635 \\
0.685 \\
0.53 \\
0.645 \\
0.76 \\
0.53 \\
0.74 \\
0.73 \\
0.725 \\
0.745 \\
0.85 \\
0.82 \\
0.41 \\
0.555 \\
0.745 \\
0.855 \\
0.645 \\
0.82 \\
0.715 \\
0.73 \\
0.545 \\
0.585 \\
0.615 \\
0.645 \\
0.48 \\
0.51 \\
0.45 \\
0.61 \\
0.5 \\
};

\addplot[mark=*, boxplot, boxplot/draw position=2]
table[row sep=\\, y index=0] {
data
0.67 \\
0.61 \\
0.82 \\
0.685 \\
0.875 \\
0.765 \\
0.885 \\
0.84 \\
0.735 \\
0.835 \\
0.805 \\
0.905 \\
0.82 \\
0.43 \\
0.705 \\
0.76 \\
0.855 \\
1.0 \\
0.89 \\
0.885 \\
0.73 \\
0.675 \\
0.835 \\
1.0 \\
0.73 \\
0.54 \\
0.765 \\
0.785 \\
0.765 \\
0.68 \\
};
}{0.1}{Output connectivity}{10}

        }
    }
    \resizebox{\textwidth}{!}{
        \subfloat[N=30]{
            \myboxplot{

\addplot[mark=*, boxplot, boxplot/draw position=0]
table[row sep=\\, y index=0] {
data
1.0 \\
1.0 \\
0.885 \\
0.84 \\
0.955 \\
1.0 \\
0.65 \\
0.765 \\
0.85 \\
0.65 \\
0.875 \\
0.525 \\
0.77 \\
0.925 \\
0.735 \\
0.965 \\
0.865 \\
0.975 \\
0.895 \\
0.99 \\
0.97 \\
1.0 \\
0.935 \\
0.945 \\
0.87 \\
0.755 \\
1.0 \\
0.87 \\
0.915 \\
0.825 \\
};

\addplot[mark=*, boxplot, boxplot/draw position=1]
table[row sep=\\, y index=0] {
data
0.735 \\
0.81 \\
0.485 \\
0.525 \\
0.655 \\
0.855 \\
0.835 \\
0.67 \\
0.76 \\
0.775 \\
0.765 \\
0.385 \\
0.55 \\
0.7 \\
0.735 \\
0.685 \\
0.815 \\
0.63 \\
0.595 \\
0.825 \\
0.76 \\
0.635 \\
0.545 \\
0.7 \\
0.705 \\
0.75 \\
0.615 \\
0.655 \\
1.0 \\
0.815 \\
};

\addplot[mark=*, boxplot, boxplot/draw position=2]
table[row sep=\\, y index=0] {
data
0.845 \\
0.9 \\
0.65 \\
0.67 \\
0.65 \\
0.89 \\
0.84 \\
0.755 \\
0.87 \\
0.835 \\
0.775 \\
0.85 \\
0.755 \\
0.89 \\
0.83 \\
0.845 \\
0.835 \\
0.765 \\
0.81 \\
0.845 \\
0.785 \\
0.735 \\
0.54 \\
0.79 \\
0.73 \\
0.95 \\
0.73 \\
0.81 \\
1.0 \\
0.945 \\
};

\addplot[mark=*, boxplot, boxplot/draw position=3]
table[row sep=\\, y index=0] {
data
0.78 \\
0.925 \\
1.0 \\
0.745 \\
1.0 \\
0.935 \\
0.915 \\
0.98 \\
0.965 \\
0.925 \\
0.685 \\
1.0 \\
0.83 \\
1.0 \\
0.87 \\
0.85 \\
0.925 \\
0.865 \\
0.85 \\
0.63 \\
0.95 \\
0.94 \\
0.94 \\
0.855 \\
0.92 \\
1.0 \\
0.78 \\
0.74 \\
1.0 \\
0.85 \\
};
}{0.1}{Output connectivity}{10}

        }
        \subfloat[N=40]{
            \myboxplot{

\addplot[mark=*, boxplot, boxplot/draw position=0]
table[row sep=\\, y index=0] {
data
0.97 \\
0.9 \\
1.0 \\
0.83 \\
0.995 \\
0.765 \\
0.96 \\
0.955 \\
0.805 \\
0.925 \\
1.0 \\
1.0 \\
1.0 \\
0.83 \\
0.935 \\
0.93 \\
1.0 \\
1.0 \\
0.755 \\
0.845 \\
0.95 \\
0.995 \\
1.0 \\
0.82 \\
0.85 \\
0.965 \\
0.89 \\
0.995 \\
1.0 \\
0.885 \\
};

\addplot[mark=*, boxplot, boxplot/draw position=1]
table[row sep=\\, y index=0] {
data
0.71 \\
0.815 \\
0.84 \\
0.6 \\
0.735 \\
0.805 \\
0.755 \\
0.865 \\
0.51 \\
0.4 \\
0.725 \\
1.0 \\
0.605 \\
0.615 \\
0.775 \\
0.595 \\
0.685 \\
0.8 \\
0.635 \\
0.695 \\
0.635 \\
0.635 \\
0.59 \\
0.775 \\
0.58 \\
0.585 \\
0.665 \\
0.575 \\
0.635 \\
0.59 \\
};

\addplot[mark=*, boxplot, boxplot/draw position=2]
table[row sep=\\, y index=0] {
data
0.835 \\
1.0 \\
0.785 \\
0.895 \\
0.82 \\
0.92 \\
0.78 \\
1.0 \\
0.855 \\
0.85 \\
1.0 \\
0.72 \\
0.9 \\
1.0 \\
0.95 \\
1.0 \\
0.805 \\
0.855 \\
0.89 \\
0.74 \\
0.92 \\
0.93 \\
0.675 \\
0.68 \\
0.89 \\
0.815 \\
0.95 \\
0.82 \\
0.82 \\
0.885 \\
};

\addplot[mark=*, boxplot, boxplot/draw position=3]
table[row sep=\\, y index=0] {
data
0.775 \\
0.935 \\
0.88 \\
0.77 \\
0.835 \\
1.0 \\
0.845 \\
0.975 \\
0.85 \\
0.72 \\
1.0 \\
0.915 \\
0.675 \\
0.775 \\
0.925 \\
0.855 \\
0.785 \\
1.0 \\
0.895 \\
1.0 \\
0.93 \\
0.555 \\
0.77 \\
0.805 \\
0.9 \\
0.935 \\
0.9 \\
0.82 \\
0.89 \\
0.845 \\
};

\addplot[mark=*, boxplot, boxplot/draw position=4]
table[row sep=\\, y index=0] {
data
0.875 \\
0.93 \\
0.88 \\
0.77 \\
0.875 \\
0.985 \\
0.885 \\
0.965 \\
1.0 \\
0.82 \\
1.0 \\
0.915 \\
0.955 \\
0.745 \\
0.95 \\
0.975 \\
0.935 \\
1.0 \\
0.955 \\
1.0 \\
0.93 \\
0.57 \\
0.95 \\
0.815 \\
0.98 \\
0.935 \\
0.835 \\
0.88 \\
0.94 \\
0.99 \\
};
}{0.1}{Output connectivity}{10}

        }
    }
    \resizebox{\textwidth}{!}{
        \subfloat[N=50]{
            \myboxplot{

\addplot[mark=*, boxplot, boxplot/draw position=0]
table[row sep=\\, y index=0] {
data
1.0 \\
1.0 \\
0.98 \\
0.9 \\
0.955 \\
1.0 \\
1.0 \\
0.825 \\
0.985 \\
0.94 \\
0.865 \\
1.0 \\
0.945 \\
0.99 \\
1.0 \\
0.815 \\
0.69 \\
0.99 \\
1.0 \\
1.0 \\
1.0 \\
0.93 \\
0.9 \\
0.91 \\
0.955 \\
0.99 \\
0.935 \\
0.975 \\
0.985 \\
0.855 \\
};

\addplot[mark=*, boxplot, boxplot/draw position=1]
table[row sep=\\, y index=0] {
data
0.725 \\
0.43 \\
0.875 \\
0.635 \\
0.66 \\
0.835 \\
0.89 \\
0.63 \\
0.465 \\
0.53 \\
0.7 \\
0.655 \\
0.805 \\
0.56 \\
0.505 \\
0.69 \\
0.77 \\
0.54 \\
0.665 \\
0.785 \\
0.615 \\
0.535 \\
0.635 \\
0.825 \\
0.84 \\
0.61 \\
0.79 \\
0.57 \\
0.535 \\
0.61 \\
};

\addplot[mark=*, boxplot, boxplot/draw position=2]
table[row sep=\\, y index=0] {
data
0.785 \\
0.685 \\
0.895 \\
0.75 \\
0.785 \\
0.845 \\
0.965 \\
0.6 \\
0.63 \\
1.0 \\
0.745 \\
0.655 \\
1.0 \\
0.705 \\
0.45 \\
0.83 \\
0.77 \\
0.675 \\
0.845 \\
1.0 \\
0.69 \\
0.47 \\
0.755 \\
0.935 \\
0.835 \\
0.825 \\
0.79 \\
0.5 \\
0.585 \\
0.905 \\
};

\addplot[mark=*, boxplot, boxplot/draw position=3]
table[row sep=\\, y index=0] {
data
0.925 \\
1.0 \\
0.865 \\
0.885 \\
0.845 \\
0.745 \\
0.89 \\
1.0 \\
1.0 \\
0.91 \\
0.77 \\
0.95 \\
0.82 \\
0.955 \\
0.85 \\
0.98 \\
0.85 \\
0.985 \\
1.0 \\
0.875 \\
0.8 \\
0.95 \\
0.9 \\
0.89 \\
0.835 \\
0.795 \\
0.96 \\
1.0 \\
0.76 \\
0.88 \\
};

\addplot[mark=*, boxplot, boxplot/draw position=4]
table[row sep=\\, y index=0] {
data
0.985 \\
0.95 \\
0.875 \\
0.955 \\
1.0 \\
0.85 \\
0.945 \\
0.995 \\
0.95 \\
0.965 \\
0.95 \\
0.965 \\
0.96 \\
0.905 \\
0.87 \\
0.995 \\
1.0 \\
0.935 \\
1.0 \\
0.735 \\
0.73 \\
0.975 \\
0.955 \\
1.0 \\
1.0 \\
0.785 \\
0.855 \\
0.87 \\
1.0 \\
0.885 \\
};

\addplot[mark=*, boxplot, boxplot/draw position=5]
table[row sep=\\, y index=0] {
data
1.0 \\
0.93 \\
0.985 \\
0.965 \\
1.0 \\
0.945 \\
1.0 \\
0.995 \\
0.95 \\
0.965 \\
0.995 \\
0.98 \\
0.995 \\
0.92 \\
0.87 \\
0.985 \\
1.0 \\
0.97 \\
1.0 \\
1.0 \\
0.75 \\
1.0 \\
0.985 \\
1.0 \\
1.0 \\
0.9 \\
0.85 \\
0.87 \\
1.0 \\
1.0 \\
};
}{0.1}{Output connectivity}{10}

        }
        \subfloat[N=60]{
            \myboxplot{

\addplot[mark=*, boxplot, boxplot/draw position=0]
table[row sep=\\, y index=0] {
data
1.0 \\
0.985 \\
0.975 \\
0.975 \\
1.0 \\
0.975 \\
1.0 \\
1.0 \\
0.975 \\
1.0 \\
0.935 \\
1.0 \\
0.96 \\
1.0 \\
0.945 \\
1.0 \\
1.0 \\
0.98 \\
1.0 \\
1.0 \\
1.0 \\
0.96 \\
0.98 \\
0.975 \\
0.955 \\
1.0 \\
0.985 \\
1.0 \\
1.0 \\
0.955 \\
};

\addplot[mark=*, boxplot, boxplot/draw position=1]
table[row sep=\\, y index=0] {
data
0.85 \\
0.66 \\
0.86 \\
0.635 \\
0.73 \\
0.75 \\
0.9 \\
0.845 \\
0.5 \\
0.85 \\
0.675 \\
0.865 \\
0.645 \\
1.0 \\
0.745 \\
0.7 \\
0.81 \\
1.0 \\
0.54 \\
0.73 \\
0.65 \\
0.665 \\
0.785 \\
0.755 \\
0.76 \\
0.735 \\
0.6 \\
0.93 \\
0.93 \\
0.735 \\
};

\addplot[mark=*, boxplot, boxplot/draw position=2]
table[row sep=\\, y index=0] {
data
1.0 \\
0.665 \\
0.94 \\
0.935 \\
0.74 \\
0.815 \\
0.925 \\
0.87 \\
0.6 \\
0.865 \\
0.765 \\
1.0 \\
0.715 \\
1.0 \\
0.745 \\
0.85 \\
0.935 \\
1.0 \\
0.61 \\
0.97 \\
0.68 \\
0.89 \\
0.815 \\
0.87 \\
0.76 \\
0.87 \\
0.635 \\
0.865 \\
0.995 \\
0.905 \\
};

\addplot[mark=*, boxplot, boxplot/draw position=3]
table[row sep=\\, y index=0] {
data
0.745 \\
0.785 \\
0.855 \\
1.0 \\
0.96 \\
0.975 \\
0.96 \\
1.0 \\
0.98 \\
0.95 \\
0.905 \\
0.85 \\
0.815 \\
0.89 \\
0.94 \\
0.825 \\
0.97 \\
0.845 \\
1.0 \\
1.0 \\
0.945 \\
0.99 \\
0.875 \\
0.975 \\
0.905 \\
0.915 \\
0.845 \\
0.895 \\
0.97 \\
1.0 \\
};

\addplot[mark=*, boxplot, boxplot/draw position=4]
table[row sep=\\, y index=0] {
data
0.875 \\
1.0 \\
0.985 \\
0.97 \\
1.0 \\
0.795 \\
0.905 \\
1.0 \\
0.965 \\
0.97 \\
1.0 \\
0.99 \\
0.975 \\
0.81 \\
0.965 \\
0.985 \\
0.965 \\
0.98 \\
0.83 \\
0.885 \\
1.0 \\
0.945 \\
0.855 \\
0.975 \\
1.0 \\
0.915 \\
0.95 \\
1.0 \\
0.98 \\
0.895 \\
};

\addplot[mark=*, boxplot, boxplot/draw position=5]
table[row sep=\\, y index=0] {
data
0.905 \\
1.0 \\
1.0 \\
0.99 \\
1.0 \\
0.8 \\
1.0 \\
1.0 \\
0.965 \\
1.0 \\
1.0 \\
0.99 \\
1.0 \\
0.955 \\
1.0 \\
0.99 \\
0.95 \\
0.98 \\
0.88 \\
0.9 \\
1.0 \\
1.0 \\
0.88 \\
0.975 \\
1.0 \\
0.925 \\
1.0 \\
1.0 \\
1.0 \\
1.0 \\
};

\addplot[mark=*, boxplot, boxplot/draw position=6]
table[row sep=\\, y index=0] {
data
0.98 \\
0.985 \\
0.94 \\
1.0 \\
1.0 \\
1.0 \\
1.0 \\
1.0 \\
1.0 \\
1.0 \\
0.99 \\
0.845 \\
0.965 \\
1.0 \\
0.985 \\
1.0 \\
0.87 \\
1.0 \\
1.0 \\
0.94 \\
0.985 \\
0.97 \\
0.98 \\
1.0 \\
1.0 \\
0.92 \\
0.9 \\
0.825 \\
1.0 \\
1.0 \\
};
}{0.1}{Output connectivity}{10}

        }
    }
    \caption{Plots for optimal input task 1 - part 1}
\end{figure*}

\begin{figure*}[ht]
    \resizebox{\textwidth}{!}{
        \subfloat[N=80]{
            \myboxplot{

\addplot[mark=*, boxplot, boxplot/draw position=0]
table[row sep=\\, y index=0] {
data
0.985 \\
1.0 \\
1.0 \\
0.995 \\
0.94 \\
0.98 \\
1.0 \\
0.995 \\
1.0 \\
0.995 \\
1.0 \\
1.0 \\
1.0 \\
1.0 \\
1.0 \\
0.975 \\
1.0 \\
1.0 \\
1.0 \\
1.0 \\
1.0 \\
1.0 \\
1.0 \\
1.0 \\
0.935 \\
1.0 \\
1.0 \\
1.0 \\
1.0 \\
1.0 \\
};

\addplot[mark=*, boxplot, boxplot/draw position=1]
table[row sep=\\, y index=0] {
data
0.525 \\
0.53 \\
0.845 \\
0.825 \\
0.595 \\
0.7 \\
0.91 \\
0.845 \\
0.715 \\
0.73 \\
0.795 \\
0.54 \\
0.555 \\
0.605 \\
0.74 \\
0.655 \\
0.77 \\
0.815 \\
0.62 \\
0.54 \\
1.0 \\
0.645 \\
0.685 \\
0.675 \\
0.86 \\
0.52 \\
0.82 \\
0.64 \\
0.79 \\
0.555 \\
};

\addplot[mark=*, boxplot, boxplot/draw position=2]
table[row sep=\\, y index=0] {
data
0.875 \\
0.78 \\
1.0 \\
0.575 \\
0.86 \\
0.895 \\
0.775 \\
0.985 \\
0.765 \\
0.79 \\
0.925 \\
0.875 \\
0.93 \\
0.785 \\
0.685 \\
0.785 \\
1.0 \\
0.885 \\
1.0 \\
0.83 \\
1.0 \\
0.965 \\
0.795 \\
0.79 \\
0.88 \\
0.99 \\
0.735 \\
0.725 \\
0.975 \\
0.975 \\
};

\addplot[mark=*, boxplot, boxplot/draw position=3]
table[row sep=\\, y index=0] {
data
0.925 \\
0.795 \\
0.925 \\
1.0 \\
0.925 \\
0.815 \\
1.0 \\
0.855 \\
0.595 \\
0.965 \\
1.0 \\
0.8 \\
0.95 \\
0.85 \\
0.9 \\
0.94 \\
0.82 \\
0.985 \\
1.0 \\
1.0 \\
1.0 \\
0.945 \\
0.65 \\
0.925 \\
0.74 \\
1.0 \\
1.0 \\
1.0 \\
0.79 \\
0.92 \\
};

\addplot[mark=*, boxplot, boxplot/draw position=4]
table[row sep=\\, y index=0] {
data
0.93 \\
0.89 \\
1.0 \\
1.0 \\
0.94 \\
0.815 \\
1.0 \\
0.845 \\
0.585 \\
1.0 \\
1.0 \\
0.84 \\
0.97 \\
0.95 \\
0.995 \\
0.98 \\
0.815 \\
1.0 \\
1.0 \\
1.0 \\
1.0 \\
0.97 \\
0.765 \\
0.96 \\
0.715 \\
1.0 \\
1.0 \\
1.0 \\
0.84 \\
0.965 \\
};

\addplot[mark=*, boxplot, boxplot/draw position=5]
table[row sep=\\, y index=0] {
data
1.0 \\
0.975 \\
1.0 \\
0.795 \\
0.885 \\
1.0 \\
1.0 \\
0.975 \\
1.0 \\
0.975 \\
1.0 \\
1.0 \\
0.97 \\
1.0 \\
1.0 \\
0.95 \\
1.0 \\
1.0 \\
0.875 \\
0.97 \\
0.92 \\
1.0 \\
0.95 \\
0.99 \\
1.0 \\
0.94 \\
1.0 \\
1.0 \\
0.975 \\
0.925 \\
};

\addplot[mark=*, boxplot, boxplot/draw position=6]
table[row sep=\\, y index=0] {
data
1.0 \\
0.96 \\
1.0 \\
1.0 \\
1.0 \\
0.985 \\
1.0 \\
0.93 \\
1.0 \\
0.87 \\
1.0 \\
0.9 \\
1.0 \\
0.995 \\
0.995 \\
1.0 \\
1.0 \\
0.935 \\
0.955 \\
1.0 \\
0.985 \\
0.995 \\
0.995 \\
0.89 \\
0.965 \\
0.995 \\
0.965 \\
0.99 \\
0.99 \\
0.9 \\
};

\addplot[mark=*, boxplot, boxplot/draw position=7]
table[row sep=\\, y index=0] {
data
1.0 \\
0.97 \\
1.0 \\
1.0 \\
1.0 \\
0.98 \\
1.0 \\
0.93 \\
1.0 \\
0.875 \\
1.0 \\
0.95 \\
1.0 \\
0.995 \\
0.995 \\
1.0 \\
1.0 \\
1.0 \\
1.0 \\
1.0 \\
0.985 \\
0.995 \\
0.99 \\
0.885 \\
0.99 \\
1.0 \\
0.965 \\
0.99 \\
0.99 \\
0.965 \\
};
}{0.1}{Output connectivity}

        }
        \subfloat[N=90]{
            \myboxplot{

\addplot[mark=*, boxplot, boxplot/draw position=0]
table[row sep=\\, y index=0] {
data
1.0 \\
1.0 \\
1.0 \\
0.895 \\
1.0 \\
1.0 \\
1.0 \\
1.0 \\
0.99 \\
1.0 \\
0.995 \\
1.0 \\
0.97 \\
1.0 \\
1.0 \\
1.0 \\
1.0 \\
0.995 \\
1.0 \\
1.0 \\
1.0 \\
1.0 \\
1.0 \\
1.0 \\
1.0 \\
1.0 \\
1.0 \\
1.0 \\
1.0 \\
1.0 \\
};

\addplot[mark=*, boxplot, boxplot/draw position=1]
table[row sep=\\, y index=0] {
data
0.625 \\
0.59 \\
0.61 \\
0.665 \\
0.585 \\
0.725 \\
0.81 \\
0.73 \\
0.635 \\
0.815 \\
0.525 \\
0.77 \\
0.725 \\
0.7 \\
0.71 \\
0.61 \\
0.735 \\
0.695 \\
0.375 \\
0.65 \\
0.555 \\
0.895 \\
0.765 \\
0.79 \\
0.665 \\
0.69 \\
0.84 \\
0.77 \\
0.78 \\
0.89 \\
};

\addplot[mark=*, boxplot, boxplot/draw position=2]
table[row sep=\\, y index=0] {
data
0.76 \\
0.97 \\
0.8 \\
0.75 \\
0.87 \\
0.85 \\
0.9 \\
0.895 \\
0.85 \\
0.915 \\
0.76 \\
0.935 \\
0.905 \\
0.925 \\
0.81 \\
0.615 \\
0.85 \\
0.725 \\
0.965 \\
0.775 \\
0.745 \\
0.895 \\
0.9 \\
0.895 \\
0.75 \\
0.89 \\
0.985 \\
0.77 \\
0.915 \\
0.97 \\
};

\addplot[mark=*, boxplot, boxplot/draw position=3]
table[row sep=\\, y index=0] {
data
0.83 \\
0.895 \\
0.635 \\
0.92 \\
0.935 \\
1.0 \\
0.785 \\
0.855 \\
0.98 \\
0.62 \\
0.745 \\
0.835 \\
0.985 \\
0.96 \\
0.98 \\
0.94 \\
0.795 \\
0.965 \\
1.0 \\
0.95 \\
0.825 \\
0.805 \\
0.965 \\
0.695 \\
0.95 \\
1.0 \\
0.96 \\
0.855 \\
0.97 \\
0.825 \\
};

\addplot[mark=*, boxplot, boxplot/draw position=4]
table[row sep=\\, y index=0] {
data
1.0 \\
0.97 \\
0.95 \\
0.98 \\
1.0 \\
0.97 \\
0.98 \\
0.91 \\
0.92 \\
0.98 \\
1.0 \\
0.945 \\
1.0 \\
1.0 \\
0.995 \\
1.0 \\
0.995 \\
0.92 \\
1.0 \\
0.895 \\
0.89 \\
1.0 \\
1.0 \\
0.895 \\
0.985 \\
0.94 \\
1.0 \\
0.93 \\
1.0 \\
0.97 \\
};

\addplot[mark=*, boxplot, boxplot/draw position=5]
table[row sep=\\, y index=0] {
data
1.0 \\
0.99 \\
0.96 \\
0.98 \\
1.0 \\
0.995 \\
1.0 \\
0.91 \\
0.93 \\
1.0 \\
1.0 \\
0.975 \\
1.0 \\
1.0 \\
0.995 \\
1.0 \\
0.995 \\
0.94 \\
1.0 \\
0.9 \\
0.9 \\
1.0 \\
1.0 \\
0.98 \\
0.985 \\
0.96 \\
1.0 \\
0.975 \\
1.0 \\
0.975 \\
};

\addplot[mark=*, boxplot, boxplot/draw position=6]
table[row sep=\\, y index=0] {
data
1.0 \\
0.945 \\
0.935 \\
1.0 \\
0.985 \\
1.0 \\
1.0 \\
1.0 \\
0.98 \\
0.89 \\
0.935 \\
1.0 \\
0.905 \\
0.99 \\
1.0 \\
1.0 \\
1.0 \\
0.985 \\
1.0 \\
0.985 \\
1.0 \\
1.0 \\
0.87 \\
0.885 \\
1.0 \\
0.98 \\
0.965 \\
1.0 \\
1.0 \\
1.0 \\
};

\addplot[mark=*, boxplot, boxplot/draw position=7]
table[row sep=\\, y index=0] {
data
1.0 \\
1.0 \\
0.94 \\
1.0 \\
0.98 \\
0.95 \\
1.0 \\
1.0 \\
1.0 \\
1.0 \\
1.0 \\
0.96 \\
1.0 \\
1.0 \\
0.935 \\
1.0 \\
1.0 \\
1.0 \\
0.965 \\
0.985 \\
1.0 \\
1.0 \\
1.0 \\
1.0 \\
0.98 \\
0.945 \\
0.995 \\
1.0 \\
1.0 \\
0.965 \\
};

\addplot[mark=*, boxplot, boxplot/draw position=8]
table[row sep=\\, y index=0] {
data
1.0 \\
1.0 \\
1.0 \\
1.0 \\
0.99 \\
0.95 \\
1.0 \\
1.0 \\
1.0 \\
1.0 \\
1.0 \\
0.96 \\
1.0 \\
1.0 \\
0.96 \\
1.0 \\
1.0 \\
1.0 \\
0.99 \\
0.985 \\
1.0 \\
1.0 \\
1.0 \\
1.0 \\
1.0 \\
0.99 \\
1.0 \\
1.0 \\
1.0 \\
0.98 \\
};

\addplot[mark=*, boxplot, boxplot/draw position=9]
table[row sep=\\, y index=0] {
data
1.0 \\
0.995 \\
1.0 \\
1.0 \\
1.0 \\
1.0 \\
1.0 \\
1.0 \\
0.99 \\
1.0 \\
0.975 \\
1.0 \\
1.0 \\
1.0 \\
1.0 \\
1.0 \\
1.0 \\
1.0 \\
1.0 \\
1.0 \\
1.0 \\
0.97 \\
1.0 \\
0.96 \\
1.0 \\
1.0 \\
0.995 \\
0.99 \\
1.0 \\
1.0 \\
};
}{0.1}{Output connectivity}{10}

        }
    }
    \resizebox{0.5\textwidth}{!}{
        \subfloat[N=100]{
            \myboxplot{

\addplot[mark=*, boxplot, boxplot/draw position=0]
table[row sep=\\, y index=0] {
data
0.985 \\
0.955 \\
0.93 \\
1.0 \\
1.0 \\
1.0 \\
0.945 \\
1.0 \\
1.0 \\
1.0 \\
1.0 \\
1.0 \\
1.0 \\
1.0 \\
0.99 \\
1.0 \\
1.0 \\
0.985 \\
0.93 \\
1.0 \\
1.0 \\
1.0 \\
1.0 \\
1.0 \\
1.0 \\
1.0 \\
1.0 \\
1.0 \\
1.0 \\
1.0 \\
};

\addplot[mark=*, boxplot, boxplot/draw position=1]
table[row sep=\\, y index=0] {
data
0.64 \\
0.705 \\
0.575 \\
0.77 \\
0.675 \\
0.665 \\
0.61 \\
0.565 \\
0.59 \\
0.595 \\
0.605 \\
0.6 \\
0.595 \\
0.79 \\
0.77 \\
0.965 \\
0.785 \\
0.61 \\
0.815 \\
0.725 \\
0.42 \\
0.73 \\
0.88 \\
0.89 \\
0.715 \\
0.825 \\
0.585 \\
0.55 \\
0.585 \\
0.87 \\
};

\addplot[mark=*, boxplot, boxplot/draw position=2]
table[row sep=\\, y index=0] {
data
0.97 \\
0.68 \\
0.705 \\
0.71 \\
0.95 \\
0.64 \\
0.785 \\
0.81 \\
0.665 \\
0.75 \\
0.9 \\
0.85 \\
0.845 \\
0.8 \\
0.97 \\
0.735 \\
0.83 \\
0.67 \\
0.8 \\
1.0 \\
0.65 \\
0.625 \\
0.985 \\
0.995 \\
0.8 \\
0.865 \\
0.885 \\
0.825 \\
0.98 \\
1.0 \\
};

\addplot[mark=*, boxplot, boxplot/draw position=3]
table[row sep=\\, y index=0] {
data
0.875 \\
0.77 \\
0.99 \\
1.0 \\
0.89 \\
0.825 \\
0.995 \\
0.915 \\
0.965 \\
0.99 \\
0.89 \\
0.94 \\
0.745 \\
0.905 \\
0.785 \\
1.0 \\
0.615 \\
1.0 \\
0.755 \\
0.7 \\
0.98 \\
1.0 \\
0.965 \\
0.795 \\
0.83 \\
0.885 \\
0.975 \\
1.0 \\
1.0 \\
0.865 \\
};

\addplot[mark=*, boxplot, boxplot/draw position=4]
table[row sep=\\, y index=0] {
data
0.875 \\
0.995 \\
0.96 \\
1.0 \\
0.915 \\
0.815 \\
1.0 \\
0.95 \\
0.985 \\
1.0 \\
0.905 \\
0.955 \\
0.895 \\
0.94 \\
1.0 \\
1.0 \\
0.65 \\
1.0 \\
0.9 \\
0.9 \\
0.97 \\
1.0 \\
0.975 \\
0.795 \\
0.875 \\
0.89 \\
0.98 \\
1.0 \\
1.0 \\
0.93 \\
};

\addplot[mark=*, boxplot, boxplot/draw position=5]
table[row sep=\\, y index=0] {
data
0.845 \\
0.875 \\
0.915 \\
1.0 \\
1.0 \\
0.88 \\
0.995 \\
0.92 \\
0.945 \\
0.87 \\
0.995 \\
0.91 \\
0.935 \\
0.965 \\
1.0 \\
0.945 \\
0.985 \\
1.0 \\
1.0 \\
0.995 \\
0.99 \\
0.94 \\
1.0 \\
0.99 \\
0.905 \\
0.995 \\
0.955 \\
0.995 \\
0.99 \\
0.995 \\
};

\addplot[mark=*, boxplot, boxplot/draw position=6]
table[row sep=\\, y index=0] {
data
1.0 \\
0.94 \\
1.0 \\
0.895 \\
1.0 \\
1.0 \\
1.0 \\
0.96 \\
1.0 \\
1.0 \\
0.995 \\
1.0 \\
1.0 \\
1.0 \\
0.995 \\
1.0 \\
1.0 \\
1.0 \\
1.0 \\
0.995 \\
0.995 \\
0.755 \\
1.0 \\
0.98 \\
1.0 \\
1.0 \\
1.0 \\
1.0 \\
0.885 \\
1.0 \\
};

\addplot[mark=*, boxplot, boxplot/draw position=7]
table[row sep=\\, y index=0] {
data
1.0 \\
0.985 \\
1.0 \\
0.985 \\
1.0 \\
1.0 \\
1.0 \\
0.96 \\
1.0 \\
1.0 \\
1.0 \\
1.0 \\
1.0 \\
1.0 \\
1.0 \\
1.0 \\
1.0 \\
1.0 \\
1.0 \\
0.995 \\
0.995 \\
0.85 \\
1.0 \\
0.98 \\
1.0 \\
1.0 \\
1.0 \\
1.0 \\
0.96 \\
1.0 \\
};

\addplot[mark=*, boxplot, boxplot/draw position=8]
table[row sep=\\, y index=0] {
data
0.995 \\
1.0 \\
1.0 \\
0.995 \\
1.0 \\
1.0 \\
1.0 \\
0.995 \\
1.0 \\
1.0 \\
1.0 \\
1.0 \\
1.0 \\
1.0 \\
1.0 \\
1.0 \\
1.0 \\
1.0 \\
1.0 \\
1.0 \\
0.99 \\
1.0 \\
1.0 \\
1.0 \\
0.985 \\
1.0 \\
1.0 \\
0.925 \\
1.0 \\
1.0 \\
};

\addplot[mark=*, boxplot, boxplot/draw position=9]
table[row sep=\\, y index=0] {
data
0.99 \\
1.0 \\
1.0 \\
0.99 \\
0.96 \\
0.94 \\
1.0 \\
1.0 \\
0.985 \\
0.86 \\
0.98 \\
1.0 \\
1.0 \\
0.995 \\
1.0 \\
1.0 \\
1.0 \\
1.0 \\
1.0 \\
1.0 \\
1.0 \\
1.0 \\
0.985 \\
1.0 \\
1.0 \\
1.0 \\
1.0 \\
1.0 \\
1.0 \\
1.0 \\
};

\addplot[mark=*, boxplot, boxplot/draw position=10]
table[row sep=\\, y index=0] {
data
0.995 \\
1.0 \\
1.0 \\
0.99 \\
0.96 \\
0.965 \\
1.0 \\
1.0 \\
0.985 \\
0.88 \\
0.985 \\
1.0 \\
1.0 \\
0.995 \\
1.0 \\
1.0 \\
1.0 \\
1.0 \\
1.0 \\
1.0 \\
1.0 \\
1.0 \\
1.0 \\
1.0 \\
1.0 \\
1.0 \\
1.0 \\
1.0 \\
1.0 \\
1.0 \\
};
}{0.1}{Output connectivity}

        }
    }
    \caption{Plots for optimal input task 1 - part 1}
\end{figure*}

\subsection{Task 2}

We now test the required output connectivity for reservoirs working on task 2.
Sample size has been increased to 50 due to me being unsure if 30 might have been to little,
especially considering the randomness involved when sampling a subset of nodes for regression.

It seems that the same finding from task 1 holds, that as soon as your reservoir is larger than the one required to solve a task (in this case somewhere around 80-90) you no longer need more readout than that to solve the task for even bigger reservoirs.
Whoop de doo!

\begin{figure*}[ht]
    \centering
    \resizebox{\textwidth}{!}{
        \subfloat[N=10]{
            \myboxplot{

\addplot[mark=*, boxplot, boxplot/draw position=1]
table[row sep=\\, y index=0] {
data
0.52 \\
0.54 \\
0.56 \\
0.525 \\
0.51 \\
0.525 \\
0.565 \\
0.57 \\
0.57 \\
0.565 \\
0.505 \\
0.665 \\
0.535 \\
0.53 \\
0.61 \\
0.625 \\
0.59 \\
0.605 \\
0.585 \\
0.525 \\
0.55 \\
0.56 \\
0.575 \\
0.505 \\
0.535 \\
0.59 \\
0.485 \\
0.645 \\
0.525 \\
0.51 \\
0.57 \\
0.52 \\
0.525 \\
0.535 \\
0.505 \\
0.52 \\
0.525 \\
0.525 \\
0.58 \\
0.65 \\
0.55 \\
0.55 \\
0.66 \\
0.525 \\
0.58 \\
0.565 \\
0.53 \\
0.555 \\
0.52 \\
0.525 \\
};
}{0.1}{Output connectivity}

        }
        \subfloat[N=20]{
            \myboxplot{

\addplot[mark=*, boxplot, boxplot/draw position=1]
table[row sep=\\, y index=0] {
data
0.525 \\
0.515 \\
0.53 \\
0.65 \\
0.55 \\
0.65 \\
0.5 \\
0.56 \\
0.555 \\
0.525 \\
0.555 \\
0.52 \\
0.53 \\
0.58 \\
0.54 \\
0.605 \\
0.615 \\
0.53 \\
0.6 \\
0.625 \\
0.515 \\
0.56 \\
0.615 \\
0.515 \\
0.505 \\
0.53 \\
0.58 \\
0.56 \\
0.5 \\
0.52 \\
0.535 \\
0.57 \\
0.495 \\
0.51 \\
0.53 \\
0.575 \\
0.415 \\
0.52 \\
0.5 \\
0.53 \\
0.525 \\
0.54 \\
0.54 \\
0.585 \\
0.5 \\
0.54 \\
0.575 \\
0.605 \\
0.565 \\
0.585 \\
};

\addplot[mark=*, boxplot, boxplot/draw position=2]
table[row sep=\\, y index=0] {
data
0.525 \\
0.53 \\
0.505 \\
0.64 \\
0.56 \\
0.655 \\
0.6 \\
0.555 \\
0.64 \\
0.525 \\
0.62 \\
0.535 \\
0.62 \\
0.555 \\
0.64 \\
0.615 \\
0.655 \\
0.575 \\
0.69 \\
0.65 \\
0.54 \\
0.6 \\
0.65 \\
0.495 \\
0.54 \\
0.53 \\
0.58 \\
0.715 \\
0.54 \\
0.655 \\
0.525 \\
0.535 \\
0.56 \\
0.575 \\
0.69 \\
0.585 \\
0.475 \\
0.515 \\
0.495 \\
0.54 \\
0.585 \\
0.65 \\
0.625 \\
0.585 \\
0.525 \\
0.52 \\
0.605 \\
0.59 \\
0.575 \\
0.585 \\
};
}{0.1}{Output connectivity}{14}

        }
    }
    \resizebox{\textwidth}{!}{
        \subfloat[N=30]{
            \myboxplot{

\addplot[mark=*, boxplot, boxplot/draw position=1]
table[row sep=\\, y index=0] {
data
0.54 \\
0.555 \\
0.53 \\
0.585 \\
0.575 \\
0.555 \\
0.59 \\
0.435 \\
0.53 \\
0.49 \\
0.52 \\
0.55 \\
0.625 \\
0.535 \\
0.525 \\
0.525 \\
0.53 \\
0.47 \\
0.54 \\
0.61 \\
0.545 \\
0.545 \\
0.625 \\
0.515 \\
0.535 \\
0.59 \\
0.565 \\
0.53 \\
0.585 \\
0.6 \\
0.525 \\
0.595 \\
0.525 \\
0.52 \\
0.555 \\
0.51 \\
0.56 \\
0.55 \\
0.49 \\
0.58 \\
0.555 \\
0.495 \\
0.535 \\
0.56 \\
0.535 \\
0.515 \\
0.65 \\
0.52 \\
0.575 \\
0.515 \\
};

\addplot[mark=*, boxplot, boxplot/draw position=2]
table[row sep=\\, y index=0] {
data
0.615 \\
0.575 \\
0.525 \\
0.7 \\
0.665 \\
0.615 \\
0.545 \\
0.51 \\
0.525 \\
0.52 \\
0.49 \\
0.54 \\
0.74 \\
0.505 \\
0.58 \\
0.555 \\
0.7 \\
0.545 \\
0.58 \\
0.56 \\
0.525 \\
0.62 \\
0.655 \\
0.53 \\
0.535 \\
0.635 \\
0.66 \\
0.615 \\
0.61 \\
0.715 \\
0.525 \\
0.645 \\
0.545 \\
0.65 \\
0.5 \\
0.59 \\
0.53 \\
0.635 \\
0.54 \\
0.635 \\
0.565 \\
0.515 \\
0.675 \\
0.57 \\
0.535 \\
0.61 \\
0.675 \\
0.51 \\
0.55 \\
0.615 \\
};

\addplot[mark=*, boxplot, boxplot/draw position=3]
table[row sep=\\, y index=0] {
data
0.71 \\
0.565 \\
0.505 \\
0.705 \\
0.66 \\
0.615 \\
0.55 \\
0.55 \\
0.495 \\
0.505 \\
0.57 \\
0.56 \\
0.81 \\
0.54 \\
0.58 \\
0.6 \\
0.695 \\
0.585 \\
0.65 \\
0.575 \\
0.55 \\
0.555 \\
0.62 \\
0.63 \\
0.51 \\
0.68 \\
0.65 \\
0.56 \\
0.595 \\
0.705 \\
0.755 \\
0.67 \\
0.53 \\
0.655 \\
0.48 \\
0.55 \\
0.495 \\
0.56 \\
0.575 \\
0.635 \\
0.65 \\
0.465 \\
0.715 \\
0.57 \\
0.535 \\
0.575 \\
0.71 \\
0.55 \\
0.53 \\
0.58 \\
};
}{0.1}{Output connectivity}{14}

        }
        \subfloat[N=40]{
            \myboxplot{

\addplot[mark=*, boxplot, boxplot/draw position=1]
table[row sep=\\, y index=0] {
data
0.495 \\
0.52 \\
0.575 \\
0.62 \\
0.62 \\
0.525 \\
0.58 \\
0.61 \\
0.515 \\
0.59 \\
0.58 \\
0.55 \\
0.56 \\
0.475 \\
0.585 \\
0.63 \\
0.64 \\
0.535 \\
0.54 \\
0.525 \\
0.53 \\
0.58 \\
0.545 \\
0.615 \\
0.595 \\
0.545 \\
0.535 \\
0.53 \\
0.585 \\
0.585 \\
0.535 \\
0.545 \\
0.52 \\
0.485 \\
0.545 \\
0.53 \\
0.55 \\
0.67 \\
0.5 \\
0.49 \\
0.625 \\
0.565 \\
0.56 \\
0.66 \\
0.495 \\
0.51 \\
0.505 \\
0.54 \\
0.51 \\
0.62 \\
};

\addplot[mark=*, boxplot, boxplot/draw position=2]
table[row sep=\\, y index=0] {
data
0.605 \\
0.565 \\
0.58 \\
0.525 \\
0.59 \\
0.575 \\
0.57 \\
0.595 \\
0.62 \\
0.465 \\
0.52 \\
0.635 \\
0.505 \\
0.66 \\
0.52 \\
0.545 \\
0.51 \\
0.65 \\
0.62 \\
0.505 \\
0.585 \\
0.56 \\
0.6 \\
0.53 \\
0.655 \\
0.51 \\
0.675 \\
0.515 \\
0.625 \\
0.505 \\
0.585 \\
0.695 \\
0.56 \\
0.52 \\
0.525 \\
0.62 \\
0.505 \\
0.525 \\
0.66 \\
0.53 \\
0.56 \\
0.475 \\
0.63 \\
0.555 \\
0.59 \\
0.575 \\
0.575 \\
0.6 \\
0.535 \\
0.54 \\
};

\addplot[mark=*, boxplot, boxplot/draw position=3]
table[row sep=\\, y index=0] {
data
0.68 \\
0.53 \\
0.585 \\
0.51 \\
0.595 \\
0.55 \\
0.575 \\
0.775 \\
0.635 \\
0.505 \\
0.485 \\
0.67 \\
0.43 \\
0.65 \\
0.565 \\
0.515 \\
0.485 \\
0.655 \\
0.64 \\
0.525 \\
0.575 \\
0.57 \\
0.61 \\
0.53 \\
0.645 \\
0.61 \\
0.62 \\
0.525 \\
0.685 \\
0.55 \\
0.56 \\
0.715 \\
0.58 \\
0.515 \\
0.53 \\
0.675 \\
0.755 \\
0.54 \\
0.695 \\
0.59 \\
0.525 \\
0.535 \\
0.675 \\
0.555 \\
0.5 \\
0.52 \\
0.71 \\
0.53 \\
0.685 \\
0.565 \\
};

\addplot[mark=*, boxplot, boxplot/draw position=4]
table[row sep=\\, y index=0] {
data
0.685 \\
0.69 \\
0.525 \\
0.505 \\
0.675 \\
0.555 \\
0.725 \\
0.61 \\
0.565 \\
0.57 \\
0.645 \\
0.555 \\
0.72 \\
0.715 \\
0.575 \\
0.645 \\
0.67 \\
0.59 \\
0.57 \\
0.755 \\
0.59 \\
0.625 \\
0.62 \\
0.62 \\
0.625 \\
0.69 \\
0.625 \\
0.55 \\
0.59 \\
0.505 \\
0.45 \\
0.515 \\
0.7 \\
0.835 \\
0.575 \\
0.575 \\
0.535 \\
0.595 \\
0.715 \\
0.65 \\
0.7 \\
0.585 \\
0.585 \\
0.615 \\
0.695 \\
0.525 \\
0.695 \\
0.615 \\
0.585 \\
0.58 \\
};
}{0.1}{Output connectivity}

        }
    }
    \resizebox{\textwidth}{!}{
        \subfloat[N=50]{
            \myboxplot{

\addplot[mark=*, boxplot, boxplot/draw position=1]
table[row sep=\\, y index=0] {
data
0.55 \\
0.66 \\
0.525 \\
0.58 \\
0.565 \\
0.625 \\
0.605 \\
0.565 \\
0.515 \\
0.645 \\
0.555 \\
0.64 \\
0.56 \\
0.49 \\
0.63 \\
0.66 \\
0.53 \\
0.505 \\
0.475 \\
0.51 \\
0.525 \\
0.55 \\
0.715 \\
0.56 \\
0.55 \\
0.595 \\
0.58 \\
0.675 \\
0.61 \\
0.57 \\
0.55 \\
0.54 \\
0.525 \\
0.535 \\
0.535 \\
0.575 \\
0.55 \\
0.655 \\
0.525 \\
0.63 \\
0.63 \\
0.53 \\
0.5 \\
0.565 \\
0.58 \\
0.525 \\
0.58 \\
0.61 \\
0.595 \\
0.52 \\
};

\addplot[mark=*, boxplot, boxplot/draw position=2]
table[row sep=\\, y index=0] {
data
0.575 \\
0.665 \\
0.535 \\
0.705 \\
0.535 \\
0.63 \\
0.645 \\
0.56 \\
0.64 \\
0.67 \\
0.57 \\
0.63 \\
0.685 \\
0.465 \\
0.565 \\
0.625 \\
0.53 \\
0.565 \\
0.455 \\
0.51 \\
0.525 \\
0.52 \\
0.675 \\
0.545 \\
0.58 \\
0.59 \\
0.565 \\
0.765 \\
0.73 \\
0.625 \\
0.585 \\
0.53 \\
0.545 \\
0.63 \\
0.525 \\
0.575 \\
0.565 \\
0.665 \\
0.495 \\
0.63 \\
0.67 \\
0.545 \\
0.495 \\
0.645 \\
0.585 \\
0.54 \\
0.6 \\
0.56 \\
0.59 \\
0.65 \\
};

\addplot[mark=*, boxplot, boxplot/draw position=3]
table[row sep=\\, y index=0] {
data
0.57 \\
0.535 \\
0.77 \\
0.505 \\
0.535 \\
0.755 \\
0.63 \\
0.685 \\
0.53 \\
0.575 \\
0.51 \\
0.625 \\
0.69 \\
0.61 \\
0.595 \\
0.645 \\
0.715 \\
0.605 \\
0.695 \\
0.655 \\
0.67 \\
0.57 \\
0.6 \\
0.635 \\
0.54 \\
0.76 \\
0.545 \\
0.59 \\
0.74 \\
0.58 \\
0.67 \\
0.59 \\
0.59 \\
0.535 \\
0.61 \\
0.64 \\
0.6 \\
0.565 \\
0.55 \\
0.535 \\
0.565 \\
0.565 \\
0.625 \\
0.555 \\
0.545 \\
0.545 \\
0.63 \\
0.735 \\
0.655 \\
0.575 \\
};

\addplot[mark=*, boxplot, boxplot/draw position=4]
table[row sep=\\, y index=0] {
data
0.595 \\
0.625 \\
0.6 \\
0.795 \\
0.715 \\
0.63 \\
0.575 \\
0.6 \\
0.61 \\
0.64 \\
0.6 \\
0.725 \\
0.605 \\
0.645 \\
0.545 \\
0.68 \\
0.71 \\
0.59 \\
0.49 \\
0.79 \\
0.655 \\
0.59 \\
0.735 \\
0.675 \\
0.645 \\
0.62 \\
0.685 \\
0.735 \\
0.54 \\
0.66 \\
0.555 \\
0.7 \\
0.525 \\
0.565 \\
0.72 \\
0.785 \\
0.8 \\
0.59 \\
0.725 \\
0.855 \\
0.515 \\
0.725 \\
0.545 \\
0.695 \\
0.685 \\
0.555 \\
0.635 \\
0.56 \\
0.56 \\
0.66 \\
};

\addplot[mark=*, boxplot, boxplot/draw position=5]
table[row sep=\\, y index=0] {
data
0.63 \\
0.55 \\
0.65 \\
0.8 \\
0.71 \\
0.73 \\
0.6 \\
0.595 \\
0.7 \\
0.66 \\
0.615 \\
0.735 \\
0.695 \\
0.76 \\
0.59 \\
0.725 \\
0.77 \\
0.585 \\
0.565 \\
0.795 \\
0.695 \\
0.58 \\
0.81 \\
0.68 \\
0.605 \\
0.64 \\
0.75 \\
0.755 \\
0.62 \\
0.69 \\
0.54 \\
0.68 \\
0.525 \\
0.645 \\
0.695 \\
0.795 \\
0.865 \\
0.57 \\
0.71 \\
0.83 \\
0.51 \\
0.715 \\
0.555 \\
0.7 \\
0.635 \\
0.565 \\
0.62 \\
0.555 \\
0.585 \\
0.68 \\
};
}{0.1}{Output connectivity}

        }
        \subfloat[N=60]{
            \myboxplot{

\addplot[mark=*, boxplot, boxplot/draw position=1]
table[row sep=\\, y index=0] {
data
0.59 \\
0.535 \\
0.495 \\
0.565 \\
0.56 \\
0.525 \\
0.525 \\
0.65 \\
0.55 \\
0.53 \\
0.515 \\
0.595 \\
0.58 \\
0.645 \\
0.61 \\
0.58 \\
0.6 \\
0.53 \\
0.545 \\
0.6 \\
0.535 \\
0.53 \\
0.605 \\
0.67 \\
0.54 \\
0.515 \\
0.61 \\
0.545 \\
0.53 \\
0.61 \\
0.655 \\
0.495 \\
0.525 \\
0.575 \\
0.535 \\
0.6 \\
0.555 \\
0.485 \\
0.57 \\
0.535 \\
0.5 \\
0.605 \\
0.575 \\
0.54 \\
0.5 \\
0.525 \\
0.535 \\
0.55 \\
0.52 \\
0.495 \\
};

\addplot[mark=*, boxplot, boxplot/draw position=2]
table[row sep=\\, y index=0] {
data
0.605 \\
0.595 \\
0.58 \\
0.64 \\
0.45 \\
0.54 \\
0.58 \\
0.58 \\
0.545 \\
0.58 \\
0.51 \\
0.535 \\
0.53 \\
0.49 \\
0.6 \\
0.505 \\
0.445 \\
0.48 \\
0.525 \\
0.63 \\
0.565 \\
0.58 \\
0.62 \\
0.51 \\
0.695 \\
0.62 \\
0.605 \\
0.585 \\
0.625 \\
0.54 \\
0.65 \\
0.58 \\
0.62 \\
0.655 \\
0.555 \\
0.5 \\
0.58 \\
0.635 \\
0.615 \\
0.61 \\
0.635 \\
0.67 \\
0.54 \\
0.505 \\
0.695 \\
0.525 \\
0.645 \\
0.53 \\
0.54 \\
0.685 \\
};

\addplot[mark=*, boxplot, boxplot/draw position=3]
table[row sep=\\, y index=0] {
data
0.64 \\
0.555 \\
0.6 \\
0.685 \\
0.425 \\
0.6 \\
0.745 \\
0.6 \\
0.62 \\
0.595 \\
0.62 \\
0.535 \\
0.53 \\
0.535 \\
0.64 \\
0.475 \\
0.485 \\
0.555 \\
0.63 \\
0.605 \\
0.56 \\
0.6 \\
0.655 \\
0.5 \\
0.715 \\
0.69 \\
0.58 \\
0.53 \\
0.665 \\
0.585 \\
0.65 \\
0.79 \\
0.665 \\
0.665 \\
0.57 \\
0.585 \\
0.575 \\
0.695 \\
0.635 \\
0.66 \\
0.625 \\
0.665 \\
0.51 \\
0.545 \\
0.71 \\
0.525 \\
0.61 \\
0.585 \\
0.62 \\
0.66 \\
};

\addplot[mark=*, boxplot, boxplot/draw position=4]
table[row sep=\\, y index=0] {
data
0.615 \\
0.77 \\
0.67 \\
0.51 \\
0.565 \\
0.59 \\
0.495 \\
0.575 \\
0.87 \\
0.62 \\
0.695 \\
0.62 \\
0.755 \\
0.735 \\
0.75 \\
0.855 \\
0.725 \\
0.55 \\
0.62 \\
0.65 \\
0.765 \\
0.605 \\
0.63 \\
0.735 \\
0.59 \\
0.515 \\
0.74 \\
0.58 \\
0.715 \\
0.65 \\
0.6 \\
0.605 \\
0.64 \\
0.725 \\
0.675 \\
0.515 \\
0.75 \\
0.765 \\
0.615 \\
0.615 \\
0.655 \\
0.79 \\
0.585 \\
0.7 \\
0.585 \\
0.695 \\
0.635 \\
0.635 \\
0.72 \\
0.555 \\
};

\addplot[mark=*, boxplot, boxplot/draw position=5]
table[row sep=\\, y index=0] {
data
0.61 \\
0.555 \\
0.49 \\
0.66 \\
0.655 \\
0.63 \\
0.635 \\
0.555 \\
0.655 \\
0.665 \\
0.555 \\
0.665 \\
0.8 \\
0.525 \\
0.49 \\
0.69 \\
0.73 \\
0.69 \\
0.745 \\
0.68 \\
0.615 \\
0.64 \\
0.575 \\
0.865 \\
0.73 \\
0.635 \\
0.605 \\
0.545 \\
0.77 \\
0.72 \\
0.545 \\
0.525 \\
0.66 \\
0.59 \\
0.81 \\
0.575 \\
0.795 \\
0.605 \\
0.67 \\
0.675 \\
0.51 \\
0.535 \\
0.65 \\
0.615 \\
0.58 \\
0.74 \\
0.65 \\
0.515 \\
0.53 \\
0.73 \\
};

\addplot[mark=*, boxplot, boxplot/draw position=6]
table[row sep=\\, y index=0] {
data
0.6 \\
0.59 \\
0.49 \\
0.65 \\
0.895 \\
0.635 \\
0.665 \\
0.555 \\
0.695 \\
0.665 \\
0.57 \\
0.675 \\
0.775 \\
0.555 \\
0.59 \\
0.7 \\
0.755 \\
0.71 \\
0.76 \\
0.685 \\
0.66 \\
0.655 \\
0.605 \\
0.87 \\
0.795 \\
0.685 \\
0.595 \\
0.68 \\
0.77 \\
0.71 \\
0.565 \\
0.53 \\
0.705 \\
0.625 \\
0.795 \\
0.665 \\
0.795 \\
0.6 \\
0.62 \\
0.675 \\
0.505 \\
0.525 \\
0.74 \\
0.68 \\
0.47 \\
0.75 \\
0.645 \\
0.68 \\
0.505 \\
0.72 \\
};
}{0.1}{Output connectivity}{14}

        }
    }
    \caption{Plots for optimal input task 2 - part 1}
\end{figure*}

\begin{figure*}[ht]
    \centering
    \resizebox{\textwidth}{!}{
        \subfloat[N=70]{
            \myboxplot{

\addplot[mark=*, boxplot, boxplot/draw position=1]
table[row sep=\\, y index=0] {
data
0.625 \\
0.545 \\
0.495 \\
0.485 \\
0.615 \\
0.535 \\
0.525 \\
0.495 \\
0.585 \\
0.565 \\
0.535 \\
0.565 \\
0.53 \\
0.63 \\
0.565 \\
0.54 \\
0.485 \\
0.465 \\
0.54 \\
0.5 \\
0.485 \\
0.63 \\
0.545 \\
0.54 \\
0.53 \\
0.55 \\
0.5 \\
0.56 \\
0.465 \\
0.54 \\
0.575 \\
0.505 \\
0.53 \\
0.505 \\
0.54 \\
0.515 \\
0.585 \\
0.525 \\
0.455 \\
0.525 \\
0.585 \\
0.6 \\
0.57 \\
0.635 \\
0.57 \\
0.675 \\
0.49 \\
0.525 \\
0.585 \\
0.525 \\
};

\addplot[mark=*, boxplot, boxplot/draw position=2]
table[row sep=\\, y index=0] {
data
0.645 \\
0.575 \\
0.605 \\
0.63 \\
0.585 \\
0.77 \\
0.62 \\
0.615 \\
0.565 \\
0.555 \\
0.565 \\
0.545 \\
0.575 \\
0.595 \\
0.63 \\
0.575 \\
0.595 \\
0.615 \\
0.545 \\
0.45 \\
0.54 \\
0.57 \\
0.465 \\
0.62 \\
0.485 \\
0.595 \\
0.535 \\
0.595 \\
0.565 \\
0.58 \\
0.57 \\
0.635 \\
0.58 \\
0.595 \\
0.58 \\
0.565 \\
0.665 \\
0.55 \\
0.525 \\
0.725 \\
0.485 \\
0.625 \\
0.53 \\
0.57 \\
0.63 \\
0.54 \\
0.62 \\
0.615 \\
0.565 \\
0.49 \\
};

\addplot[mark=*, boxplot, boxplot/draw position=3]
table[row sep=\\, y index=0] {
data
0.58 \\
0.56 \\
0.445 \\
0.55 \\
0.63 \\
0.645 \\
0.605 \\
0.475 \\
0.64 \\
0.46 \\
0.655 \\
0.54 \\
0.56 \\
0.645 \\
0.74 \\
0.555 \\
0.61 \\
0.575 \\
0.545 \\
0.705 \\
0.755 \\
0.71 \\
0.57 \\
0.485 \\
0.56 \\
0.665 \\
0.62 \\
0.645 \\
0.61 \\
0.54 \\
0.69 \\
0.615 \\
0.665 \\
0.595 \\
0.615 \\
0.64 \\
0.61 \\
0.73 \\
0.645 \\
0.515 \\
0.61 \\
0.585 \\
0.605 \\
0.615 \\
0.615 \\
0.55 \\
0.855 \\
0.525 \\
0.56 \\
0.585 \\
};

\addplot[mark=*, boxplot, boxplot/draw position=4]
table[row sep=\\, y index=0] {
data
0.88 \\
0.565 \\
0.615 \\
0.6 \\
0.535 \\
0.745 \\
0.78 \\
0.63 \\
0.51 \\
0.625 \\
0.57 \\
0.62 \\
0.67 \\
0.665 \\
0.805 \\
0.53 \\
0.595 \\
0.65 \\
0.59 \\
0.525 \\
0.565 \\
0.595 \\
0.65 \\
0.67 \\
0.55 \\
0.665 \\
0.525 \\
0.67 \\
0.615 \\
0.615 \\
0.56 \\
0.69 \\
0.695 \\
0.63 \\
0.615 \\
0.625 \\
0.805 \\
0.565 \\
0.615 \\
0.715 \\
0.53 \\
0.7 \\
0.6 \\
0.45 \\
0.715 \\
0.635 \\
0.655 \\
0.66 \\
0.67 \\
0.43 \\
};

\addplot[mark=*, boxplot, boxplot/draw position=5]
table[row sep=\\, y index=0] {
data
0.8 \\
0.71 \\
0.555 \\
0.745 \\
0.605 \\
0.84 \\
0.755 \\
0.75 \\
0.72 \\
0.665 \\
0.62 \\
0.625 \\
0.565 \\
0.74 \\
0.75 \\
0.575 \\
0.56 \\
0.51 \\
0.715 \\
0.66 \\
0.675 \\
0.59 \\
0.705 \\
0.47 \\
0.71 \\
0.74 \\
0.615 \\
0.645 \\
0.465 \\
0.67 \\
0.645 \\
0.625 \\
0.62 \\
0.56 \\
0.665 \\
0.66 \\
0.65 \\
0.52 \\
0.6 \\
0.71 \\
0.66 \\
0.63 \\
0.645 \\
0.62 \\
0.655 \\
0.69 \\
0.66 \\
0.585 \\
0.58 \\
0.775 \\
};

\addplot[mark=*, boxplot, boxplot/draw position=6]
table[row sep=\\, y index=0] {
data
0.87 \\
0.7 \\
0.71 \\
0.75 \\
0.61 \\
0.85 \\
0.795 \\
0.75 \\
0.755 \\
0.695 \\
0.555 \\
0.635 \\
0.59 \\
0.745 \\
0.785 \\
0.575 \\
0.66 \\
0.555 \\
0.69 \\
0.66 \\
0.68 \\
0.55 \\
0.77 \\
0.47 \\
0.71 \\
0.745 \\
0.615 \\
0.61 \\
0.505 \\
0.7 \\
0.665 \\
0.795 \\
0.6 \\
0.555 \\
0.665 \\
0.77 \\
0.58 \\
0.53 \\
0.63 \\
0.86 \\
0.68 \\
0.615 \\
0.73 \\
0.6 \\
0.675 \\
0.745 \\
0.69 \\
0.67 \\
0.565 \\
0.765 \\
};

\addplot[mark=*, boxplot, boxplot/draw position=7]
table[row sep=\\, y index=0] {
data
0.75 \\
0.735 \\
0.565 \\
0.81 \\
0.705 \\
0.74 \\
0.7 \\
0.785 \\
0.56 \\
0.815 \\
0.815 \\
0.785 \\
0.765 \\
0.73 \\
0.815 \\
0.635 \\
0.64 \\
0.805 \\
0.665 \\
0.685 \\
0.72 \\
0.765 \\
0.585 \\
0.725 \\
0.635 \\
0.715 \\
0.74 \\
0.59 \\
0.78 \\
0.81 \\
0.515 \\
0.85 \\
0.67 \\
0.735 \\
0.68 \\
0.615 \\
0.8 \\
0.755 \\
0.55 \\
0.65 \\
0.69 \\
0.725 \\
0.75 \\
0.825 \\
0.665 \\
0.735 \\
0.59 \\
0.825 \\
0.805 \\
0.71 \\
};
}{0.1}{Output connectivity}

        }
        \subfloat[N=80]{
            \myboxplot{

\addplot[mark=*, boxplot, boxplot/draw position=1]
table[row sep=\\, y index=0] {
data
0.565 \\
0.505 \\
0.51 \\
0.52 \\
0.53 \\
0.59 \\
0.55 \\
0.535 \\
0.525 \\
0.51 \\
0.575 \\
0.63 \\
0.58 \\
0.515 \\
0.575 \\
0.545 \\
0.465 \\
0.52 \\
0.495 \\
0.49 \\
0.6 \\
0.525 \\
0.495 \\
0.58 \\
0.57 \\
0.485 \\
0.57 \\
0.5 \\
0.645 \\
0.57 \\
0.435 \\
0.525 \\
0.62 \\
0.645 \\
0.6 \\
0.555 \\
0.58 \\
0.52 \\
0.565 \\
0.52 \\
0.51 \\
0.515 \\
0.51 \\
0.535 \\
0.525 \\
0.575 \\
0.56 \\
0.625 \\
0.545 \\
0.59 \\
};

\addplot[mark=*, boxplot, boxplot/draw position=2]
table[row sep=\\, y index=0] {
data
0.605 \\
0.5 \\
0.47 \\
0.735 \\
0.55 \\
0.64 \\
0.53 \\
0.57 \\
0.55 \\
0.685 \\
0.56 \\
0.58 \\
0.58 \\
0.56 \\
0.515 \\
0.61 \\
0.435 \\
0.655 \\
0.56 \\
0.55 \\
0.57 \\
0.69 \\
0.675 \\
0.575 \\
0.575 \\
0.51 \\
0.585 \\
0.465 \\
0.645 \\
0.67 \\
0.55 \\
0.47 \\
0.635 \\
0.605 \\
0.6 \\
0.61 \\
0.68 \\
0.525 \\
0.605 \\
0.55 \\
0.615 \\
0.585 \\
0.55 \\
0.57 \\
0.555 \\
0.63 \\
0.61 \\
0.685 \\
0.56 \\
0.605 \\
};

\addplot[mark=*, boxplot, boxplot/draw position=3]
table[row sep=\\, y index=0] {
data
0.62 \\
0.535 \\
0.665 \\
0.685 \\
0.56 \\
0.645 \\
0.65 \\
0.615 \\
0.57 \\
0.54 \\
0.65 \\
0.53 \\
0.605 \\
0.7 \\
0.58 \\
0.555 \\
0.58 \\
0.74 \\
0.56 \\
0.47 \\
0.6 \\
0.6 \\
0.575 \\
0.585 \\
0.68 \\
0.57 \\
0.515 \\
0.735 \\
0.535 \\
0.62 \\
0.665 \\
0.7 \\
0.6 \\
0.535 \\
0.66 \\
0.5 \\
0.565 \\
0.655 \\
0.575 \\
0.575 \\
0.62 \\
0.59 \\
0.46 \\
0.63 \\
0.555 \\
0.72 \\
0.56 \\
0.59 \\
0.555 \\
0.555 \\
};

\addplot[mark=*, boxplot, boxplot/draw position=4]
table[row sep=\\, y index=0] {
data
0.72 \\
0.68 \\
0.71 \\
0.6 \\
0.6 \\
0.625 \\
0.62 \\
0.64 \\
0.7 \\
0.545 \\
0.555 \\
0.625 \\
0.635 \\
0.71 \\
0.675 \\
0.625 \\
0.68 \\
0.52 \\
0.535 \\
0.6 \\
0.6 \\
0.655 \\
0.815 \\
0.575 \\
0.605 \\
0.785 \\
0.79 \\
0.575 \\
0.7 \\
0.68 \\
0.565 \\
0.59 \\
0.565 \\
0.655 \\
0.705 \\
0.615 \\
0.585 \\
0.67 \\
0.595 \\
0.555 \\
0.64 \\
0.55 \\
0.76 \\
0.825 \\
0.51 \\
0.56 \\
0.665 \\
0.53 \\
0.745 \\
0.71 \\
};

\addplot[mark=*, boxplot, boxplot/draw position=5]
table[row sep=\\, y index=0] {
data
0.74 \\
0.61 \\
0.765 \\
0.62 \\
0.76 \\
0.67 \\
0.6 \\
0.74 \\
0.74 \\
0.59 \\
0.57 \\
0.615 \\
0.67 \\
0.755 \\
0.685 \\
0.63 \\
0.67 \\
0.51 \\
0.53 \\
0.64 \\
0.61 \\
0.62 \\
0.83 \\
0.635 \\
0.59 \\
0.795 \\
0.87 \\
0.645 \\
0.725 \\
0.69 \\
0.615 \\
0.61 \\
0.565 \\
0.65 \\
0.695 \\
0.645 \\
0.605 \\
0.725 \\
0.62 \\
0.625 \\
0.665 \\
0.62 \\
0.795 \\
0.835 \\
0.465 \\
0.635 \\
0.615 \\
0.56 \\
0.785 \\
0.845 \\
};

\addplot[mark=*, boxplot, boxplot/draw position=6]
table[row sep=\\, y index=0] {
data
0.64 \\
0.635 \\
0.725 \\
0.725 \\
0.76 \\
0.78 \\
0.725 \\
0.86 \\
0.605 \\
0.74 \\
0.495 \\
0.775 \\
0.725 \\
0.55 \\
0.64 \\
0.625 \\
0.645 \\
0.77 \\
0.685 \\
0.71 \\
0.65 \\
0.675 \\
0.735 \\
0.65 \\
0.625 \\
0.545 \\
0.72 \\
0.615 \\
0.605 \\
0.65 \\
0.595 \\
0.695 \\
0.715 \\
0.63 \\
0.825 \\
0.8 \\
0.635 \\
0.81 \\
0.705 \\
0.765 \\
0.77 \\
0.705 \\
0.735 \\
0.78 \\
0.74 \\
0.59 \\
0.585 \\
0.745 \\
0.845 \\
0.59 \\
};

\addplot[mark=*, boxplot, boxplot/draw position=7]
table[row sep=\\, y index=0] {
data
0.8 \\
0.645 \\
0.5 \\
0.78 \\
0.585 \\
0.635 \\
0.81 \\
0.845 \\
0.7 \\
0.755 \\
0.78 \\
0.605 \\
0.755 \\
0.79 \\
0.825 \\
0.835 \\
0.665 \\
0.73 \\
0.625 \\
0.91 \\
0.61 \\
0.695 \\
0.66 \\
0.595 \\
0.575 \\
0.725 \\
0.66 \\
0.54 \\
0.76 \\
0.835 \\
0.585 \\
0.875 \\
0.795 \\
0.71 \\
0.555 \\
0.675 \\
0.65 \\
0.825 \\
0.585 \\
0.84 \\
0.805 \\
0.695 \\
0.715 \\
0.75 \\
0.81 \\
0.635 \\
0.715 \\
0.535 \\
0.525 \\
0.68 \\
};

\addplot[mark=*, boxplot, boxplot/draw position=8]
table[row sep=\\, y index=0] {
data
0.845 \\
0.685 \\
0.725 \\
0.785 \\
0.845 \\
0.805 \\
0.805 \\
0.89 \\
0.58 \\
0.77 \\
0.475 \\
0.78 \\
0.94 \\
0.6 \\
0.67 \\
0.62 \\
0.74 \\
0.775 \\
0.64 \\
0.735 \\
0.655 \\
0.72 \\
0.805 \\
0.675 \\
0.66 \\
0.64 \\
0.735 \\
0.67 \\
0.635 \\
0.675 \\
0.665 \\
0.745 \\
0.745 \\
0.685 \\
0.9 \\
0.775 \\
0.725 \\
0.84 \\
0.745 \\
0.845 \\
0.795 \\
0.75 \\
0.76 \\
0.84 \\
0.795 \\
0.655 \\
0.72 \\
0.8 \\
0.845 \\
0.77 \\
};
}{0.1}{Output connectivity}{14}

        }
    }
    \resizebox{\textwidth}{!}{
        \subfloat[N=90]{
            \myboxplot{

\addplot[mark=*, boxplot, boxplot/draw position=1]
table[row sep=\\, y index=0] {
data
0.545 \\
0.53 \\
0.6 \\
0.52 \\
0.5 \\
0.495 \\
0.58 \\
0.565 \\
0.575 \\
0.575 \\
0.49 \\
0.59 \\
0.59 \\
0.555 \\
0.605 \\
0.505 \\
0.61 \\
0.52 \\
0.54 \\
0.54 \\
0.58 \\
0.56 \\
0.57 \\
0.51 \\
0.595 \\
0.48 \\
0.505 \\
0.615 \\
0.51 \\
0.535 \\
0.55 \\
0.515 \\
0.53 \\
0.58 \\
0.51 \\
0.495 \\
0.54 \\
0.55 \\
0.545 \\
0.62 \\
0.525 \\
0.605 \\
0.62 \\
0.535 \\
0.555 \\
0.525 \\
0.44 \\
0.585 \\
0.54 \\
0.53 \\
};

\addplot[mark=*, boxplot, boxplot/draw position=2]
table[row sep=\\, y index=0] {
data
0.615 \\
0.51 \\
0.575 \\
0.57 \\
0.525 \\
0.615 \\
0.62 \\
0.51 \\
0.555 \\
0.59 \\
0.63 \\
0.635 \\
0.475 \\
0.63 \\
0.52 \\
0.62 \\
0.565 \\
0.54 \\
0.54 \\
0.56 \\
0.605 \\
0.62 \\
0.5 \\
0.605 \\
0.525 \\
0.59 \\
0.72 \\
0.525 \\
0.515 \\
0.54 \\
0.63 \\
0.53 \\
0.58 \\
0.605 \\
0.535 \\
0.57 \\
0.54 \\
0.46 \\
0.545 \\
0.485 \\
0.53 \\
0.51 \\
0.605 \\
0.53 \\
0.595 \\
0.515 \\
0.52 \\
0.525 \\
0.575 \\
0.565 \\
};

\addplot[mark=*, boxplot, boxplot/draw position=3]
table[row sep=\\, y index=0] {
data
0.66 \\
0.49 \\
0.635 \\
0.625 \\
0.61 \\
0.64 \\
0.635 \\
0.54 \\
0.54 \\
0.565 \\
0.55 \\
0.625 \\
0.52 \\
0.675 \\
0.525 \\
0.63 \\
0.5 \\
0.545 \\
0.56 \\
0.485 \\
0.635 \\
0.675 \\
0.545 \\
0.59 \\
0.47 \\
0.72 \\
0.72 \\
0.53 \\
0.495 \\
0.67 \\
0.57 \\
0.565 \\
0.69 \\
0.605 \\
0.575 \\
0.68 \\
0.65 \\
0.435 \\
0.575 \\
0.545 \\
0.59 \\
0.505 \\
0.635 \\
0.53 \\
0.695 \\
0.51 \\
0.575 \\
0.545 \\
0.56 \\
0.79 \\
};

\addplot[mark=*, boxplot, boxplot/draw position=4]
table[row sep=\\, y index=0] {
data
0.72 \\
0.585 \\
0.585 \\
0.57 \\
0.555 \\
0.615 \\
0.57 \\
0.535 \\
0.58 \\
0.695 \\
0.505 \\
0.6 \\
0.6 \\
0.585 \\
0.54 \\
0.615 \\
0.565 \\
0.775 \\
0.74 \\
0.69 \\
0.765 \\
0.605 \\
0.655 \\
0.46 \\
0.61 \\
0.595 \\
0.61 \\
0.625 \\
0.615 \\
0.7 \\
0.565 \\
0.59 \\
0.55 \\
0.68 \\
0.765 \\
0.66 \\
0.69 \\
0.57 \\
0.615 \\
0.59 \\
0.57 \\
0.62 \\
0.64 \\
0.6 \\
0.755 \\
0.72 \\
0.62 \\
0.525 \\
0.81 \\
0.655 \\
};

\addplot[mark=*, boxplot, boxplot/draw position=5]
table[row sep=\\, y index=0] {
data
0.81 \\
0.61 \\
0.615 \\
0.575 \\
0.59 \\
0.635 \\
0.595 \\
0.73 \\
0.57 \\
0.71 \\
0.555 \\
0.59 \\
0.635 \\
0.595 \\
0.555 \\
0.595 \\
0.615 \\
0.815 \\
0.755 \\
0.67 \\
0.795 \\
0.675 \\
0.645 \\
0.525 \\
0.68 \\
0.535 \\
0.625 \\
0.7 \\
0.6 \\
0.675 \\
0.59 \\
0.62 \\
0.56 \\
0.75 \\
0.815 \\
0.655 \\
0.805 \\
0.605 \\
0.58 \\
0.595 \\
0.62 \\
0.665 \\
0.63 \\
0.6 \\
0.75 \\
0.73 \\
0.645 \\
0.525 \\
0.845 \\
0.65 \\
};

\addplot[mark=*, boxplot, boxplot/draw position=6]
table[row sep=\\, y index=0] {
data
0.8 \\
0.72 \\
0.595 \\
0.67 \\
0.6 \\
0.86 \\
0.645 \\
0.75 \\
0.565 \\
0.61 \\
0.67 \\
0.615 \\
0.68 \\
0.765 \\
0.67 \\
0.665 \\
0.755 \\
0.745 \\
0.865 \\
0.59 \\
0.855 \\
0.945 \\
0.62 \\
0.665 \\
0.61 \\
0.635 \\
0.865 \\
0.655 \\
0.53 \\
0.6 \\
0.615 \\
0.485 \\
0.68 \\
0.54 \\
0.54 \\
0.66 \\
0.68 \\
0.825 \\
0.65 \\
0.795 \\
0.68 \\
0.735 \\
0.745 \\
0.655 \\
0.845 \\
0.67 \\
0.55 \\
0.655 \\
0.67 \\
0.77 \\
};

\addplot[mark=*, boxplot, boxplot/draw position=7]
table[row sep=\\, y index=0] {
data
0.74 \\
0.825 \\
0.67 \\
0.49 \\
0.545 \\
0.75 \\
0.615 \\
0.82 \\
0.675 \\
0.785 \\
0.555 \\
0.67 \\
0.565 \\
0.93 \\
0.66 \\
0.565 \\
0.66 \\
0.505 \\
0.65 \\
0.56 \\
0.815 \\
0.815 \\
0.55 \\
0.79 \\
0.93 \\
0.64 \\
0.725 \\
0.69 \\
0.785 \\
0.65 \\
0.66 \\
0.745 \\
0.68 \\
0.645 \\
0.545 \\
0.725 \\
0.745 \\
0.575 \\
0.8 \\
0.66 \\
0.86 \\
0.74 \\
0.695 \\
0.73 \\
0.81 \\
0.85 \\
0.64 \\
0.615 \\
0.74 \\
0.685 \\
};

\addplot[mark=*, boxplot, boxplot/draw position=8]
table[row sep=\\, y index=0] {
data
0.84 \\
0.825 \\
0.83 \\
0.71 \\
0.735 \\
0.66 \\
0.845 \\
0.79 \\
0.685 \\
0.795 \\
0.91 \\
0.76 \\
0.735 \\
0.845 \\
0.725 \\
0.735 \\
0.615 \\
0.74 \\
0.75 \\
0.815 \\
0.685 \\
0.81 \\
0.75 \\
0.72 \\
0.74 \\
0.65 \\
0.815 \\
0.86 \\
0.705 \\
0.675 \\
0.635 \\
0.705 \\
0.565 \\
0.79 \\
0.86 \\
0.63 \\
0.565 \\
0.76 \\
0.515 \\
0.625 \\
0.545 \\
0.835 \\
0.735 \\
0.7 \\
0.775 \\
0.7 \\
0.625 \\
0.885 \\
0.635 \\
0.515 \\
};

\addplot[mark=*, boxplot, boxplot/draw position=9]
table[row sep=\\, y index=0] {
data
0.82 \\
0.81 \\
0.83 \\
0.78 \\
0.765 \\
0.655 \\
0.845 \\
0.79 \\
0.68 \\
0.805 \\
0.9 \\
0.755 \\
0.765 \\
0.845 \\
0.74 \\
0.735 \\
0.765 \\
0.775 \\
0.855 \\
0.815 \\
0.665 \\
0.81 \\
0.75 \\
0.745 \\
0.895 \\
0.61 \\
0.84 \\
0.835 \\
0.71 \\
0.685 \\
0.66 \\
0.76 \\
0.68 \\
0.785 \\
0.85 \\
0.645 \\
0.61 \\
0.77 \\
0.595 \\
0.595 \\
0.565 \\
0.865 \\
0.72 \\
0.645 \\
0.77 \\
0.73 \\
0.615 \\
0.885 \\
0.67 \\
0.515 \\
};
}{0.1}{Output connectivity}

        }
        \subfloat[N=100]{
            \myboxplot{

\addplot[mark=*, boxplot, boxplot/draw position=1]
table[row sep=\\, y index=0] {
data
0.605 \\
0.57 \\
0.54 \\
0.595 \\
0.575 \\
0.53 \\
0.555 \\
0.61 \\
0.55 \\
0.51 \\
0.525 \\
0.505 \\
0.565 \\
0.595 \\
0.605 \\
0.59 \\
0.555 \\
0.53 \\
0.525 \\
0.55 \\
0.57 \\
0.69 \\
0.535 \\
0.605 \\
0.595 \\
0.545 \\
0.5 \\
0.505 \\
0.66 \\
0.495 \\
0.59 \\
0.52 \\
0.55 \\
0.585 \\
0.51 \\
0.655 \\
0.53 \\
0.61 \\
0.56 \\
0.44 \\
0.525 \\
0.555 \\
0.565 \\
0.56 \\
0.52 \\
0.56 \\
0.705 \\
0.495 \\
0.545 \\
0.63 \\
};

\addplot[mark=*, boxplot, boxplot/draw position=2]
table[row sep=\\, y index=0] {
data
0.595 \\
0.57 \\
0.6 \\
0.595 \\
0.545 \\
0.525 \\
0.625 \\
0.635 \\
0.645 \\
0.575 \\
0.56 \\
0.485 \\
0.62 \\
0.615 \\
0.6 \\
0.625 \\
0.5 \\
0.565 \\
0.52 \\
0.545 \\
0.615 \\
0.725 \\
0.605 \\
0.59 \\
0.675 \\
0.535 \\
0.6 \\
0.52 \\
0.67 \\
0.56 \\
0.595 \\
0.61 \\
0.665 \\
0.625 \\
0.595 \\
0.675 \\
0.525 \\
0.555 \\
0.635 \\
0.515 \\
0.52 \\
0.58 \\
0.575 \\
0.625 \\
0.665 \\
0.61 \\
0.72 \\
0.585 \\
0.61 \\
0.575 \\
};

\addplot[mark=*, boxplot, boxplot/draw position=3]
table[row sep=\\, y index=0] {
data
0.645 \\
0.52 \\
0.52 \\
0.535 \\
0.615 \\
0.585 \\
0.615 \\
0.635 \\
0.655 \\
0.665 \\
0.605 \\
0.485 \\
0.63 \\
0.565 \\
0.615 \\
0.695 \\
0.49 \\
0.545 \\
0.555 \\
0.525 \\
0.67 \\
0.7 \\
0.56 \\
0.69 \\
0.54 \\
0.635 \\
0.54 \\
0.625 \\
0.585 \\
0.56 \\
0.55 \\
0.68 \\
0.625 \\
0.535 \\
0.65 \\
0.72 \\
0.615 \\
0.535 \\
0.565 \\
0.625 \\
0.625 \\
0.685 \\
0.59 \\
0.575 \\
0.56 \\
0.57 \\
0.495 \\
0.46 \\
0.65 \\
0.705 \\
};

\addplot[mark=*, boxplot, boxplot/draw position=4]
table[row sep=\\, y index=0] {
data
0.78 \\
0.55 \\
0.545 \\
0.715 \\
0.555 \\
0.765 \\
0.695 \\
0.585 \\
0.68 \\
0.6 \\
0.66 \\
0.64 \\
0.695 \\
0.64 \\
0.635 \\
0.675 \\
0.63 \\
0.725 \\
0.61 \\
0.74 \\
0.635 \\
0.535 \\
0.665 \\
0.61 \\
0.64 \\
0.655 \\
0.665 \\
0.665 \\
0.63 \\
0.675 \\
0.685 \\
0.745 \\
0.635 \\
0.705 \\
0.64 \\
0.73 \\
0.655 \\
0.64 \\
0.595 \\
0.655 \\
0.62 \\
0.525 \\
0.66 \\
0.585 \\
0.79 \\
0.695 \\
0.545 \\
0.655 \\
0.52 \\
0.545 \\
};

\addplot[mark=*, boxplot, boxplot/draw position=5]
table[row sep=\\, y index=0] {
data
0.65 \\
0.605 \\
0.715 \\
0.625 \\
0.725 \\
0.65 \\
0.63 \\
0.655 \\
0.625 \\
0.7 \\
0.655 \\
0.66 \\
0.705 \\
0.58 \\
0.605 \\
0.62 \\
0.535 \\
0.695 \\
0.775 \\
0.625 \\
0.615 \\
0.62 \\
0.7 \\
0.61 \\
0.775 \\
0.63 \\
0.72 \\
0.665 \\
0.745 \\
0.855 \\
0.805 \\
0.54 \\
0.695 \\
0.605 \\
0.645 \\
0.7 \\
0.59 \\
0.7 \\
0.675 \\
0.555 \\
0.66 \\
0.73 \\
0.785 \\
0.665 \\
0.685 \\
0.71 \\
0.74 \\
0.565 \\
0.675 \\
0.565 \\
};

\addplot[mark=*, boxplot, boxplot/draw position=6]
table[row sep=\\, y index=0] {
data
0.63 \\
0.625 \\
0.725 \\
0.675 \\
0.75 \\
0.675 \\
0.685 \\
0.665 \\
0.64 \\
0.64 \\
0.81 \\
0.72 \\
0.72 \\
0.615 \\
0.6 \\
0.735 \\
0.67 \\
0.685 \\
0.735 \\
0.645 \\
0.66 \\
0.65 \\
0.725 \\
0.705 \\
0.78 \\
0.715 \\
0.72 \\
0.685 \\
0.77 \\
0.865 \\
0.8 \\
0.575 \\
0.675 \\
0.605 \\
0.685 \\
0.81 \\
0.595 \\
0.735 \\
0.63 \\
0.53 \\
0.65 \\
0.755 \\
0.845 \\
0.645 \\
0.685 \\
0.74 \\
0.73 \\
0.515 \\
0.645 \\
0.58 \\
};

\addplot[mark=*, boxplot, boxplot/draw position=7]
table[row sep=\\, y index=0] {
data
0.66 \\
0.835 \\
0.755 \\
0.7 \\
0.575 \\
0.695 \\
0.575 \\
0.835 \\
0.865 \\
0.75 \\
0.74 \\
0.545 \\
0.715 \\
0.81 \\
0.795 \\
0.675 \\
0.645 \\
0.505 \\
0.87 \\
0.985 \\
0.65 \\
0.71 \\
0.68 \\
0.77 \\
0.82 \\
0.76 \\
0.585 \\
0.55 \\
0.59 \\
0.635 \\
0.79 \\
0.575 \\
0.64 \\
0.565 \\
0.635 \\
0.695 \\
0.71 \\
0.72 \\
0.725 \\
0.75 \\
0.78 \\
0.575 \\
0.705 \\
0.66 \\
0.845 \\
0.585 \\
0.74 \\
0.78 \\
0.765 \\
0.78 \\
};

\addplot[mark=*, boxplot, boxplot/draw position=8]
table[row sep=\\, y index=0] {
data
0.71 \\
0.84 \\
0.74 \\
0.72 \\
0.575 \\
0.74 \\
0.57 \\
0.835 \\
0.88 \\
0.745 \\
0.785 \\
0.605 \\
0.76 \\
0.815 \\
0.785 \\
0.715 \\
0.645 \\
0.51 \\
0.865 \\
0.99 \\
0.705 \\
0.735 \\
0.72 \\
0.785 \\
0.83 \\
0.835 \\
0.605 \\
0.605 \\
0.605 \\
0.69 \\
0.79 \\
0.655 \\
0.635 \\
0.565 \\
0.65 \\
0.67 \\
0.705 \\
0.705 \\
0.725 \\
0.77 \\
0.8 \\
0.605 \\
0.73 \\
0.735 \\
0.83 \\
0.6 \\
0.76 \\
0.855 \\
0.765 \\
0.77 \\
};

\addplot[mark=*, boxplot, boxplot/draw position=9]
table[row sep=\\, y index=0] {
data
0.825 \\
0.765 \\
0.745 \\
0.655 \\
0.52 \\
0.85 \\
0.99 \\
0.73 \\
0.75 \\
0.7 \\
0.81 \\
0.85 \\
0.83 \\
0.655 \\
0.605 \\
0.61 \\
0.675 \\
0.81 \\
0.635 \\
0.67 \\
0.565 \\
0.625 \\
0.75 \\
0.73 \\
0.705 \\
0.735 \\
0.775 \\
0.805 \\
0.605 \\
0.7 \\
0.715 \\
0.825 \\
0.58 \\
0.825 \\
0.9 \\
0.765 \\
0.81 \\
0.8 \\
0.78 \\
0.855 \\
0.725 \\
0.8 \\
0.84 \\
0.665 \\
0.645 \\
0.87 \\
0.925 \\
0.845 \\
0.735 \\
0.705 \\
};

\addplot[mark=*, boxplot, boxplot/draw position=10]
table[row sep=\\, y index=0] {
data
0.725 \\
0.635 \\
0.815 \\
0.64 \\
0.88 \\
0.855 \\
0.67 \\
0.865 \\
0.625 \\
0.84 \\
0.825 \\
0.905 \\
0.85 \\
0.82 \\
0.85 \\
0.82 \\
0.7 \\
0.705 \\
0.655 \\
0.77 \\
0.905 \\
0.795 \\
0.84 \\
0.705 \\
0.715 \\
0.99 \\
0.75 \\
0.7 \\
0.755 \\
0.675 \\
0.825 \\
0.855 \\
0.72 \\
0.77 \\
0.765 \\
0.76 \\
0.65 \\
0.77 \\
0.755 \\
0.77 \\
0.795 \\
0.81 \\
0.675 \\
0.75 \\
0.885 \\
0.825 \\
0.81 \\
0.535 \\
0.72 \\
0.705 \\
};
}{0.1}{Output connectivity}

        }
    }
    \resizebox{\textwidth}{!}{
        \subfloat[N=110]{
            \myboxplot{

\addplot[mark=*, boxplot, boxplot/draw position=1]
table[row sep=\\, y index=0] {
data
0.635 \\
0.51 \\
0.6 \\
0.495 \\
0.595 \\
0.49 \\
0.54 \\
0.585 \\
0.54 \\
0.48 \\
0.505 \\
0.65 \\
0.56 \\
0.585 \\
0.56 \\
0.62 \\
0.555 \\
0.52 \\
0.54 \\
0.56 \\
0.49 \\
0.61 \\
0.535 \\
0.505 \\
0.52 \\
0.475 \\
0.55 \\
0.525 \\
0.58 \\
0.61 \\
0.515 \\
0.565 \\
0.53 \\
0.55 \\
0.5 \\
0.505 \\
0.57 \\
0.585 \\
0.545 \\
0.465 \\
0.525 \\
0.535 \\
0.56 \\
0.505 \\
0.505 \\
0.51 \\
0.505 \\
0.54 \\
0.585 \\
0.55 \\
};

\addplot[mark=*, boxplot, boxplot/draw position=2]
table[row sep=\\, y index=0] {
data
0.655 \\
0.53 \\
0.615 \\
0.54 \\
0.615 \\
0.58 \\
0.565 \\
0.605 \\
0.565 \\
0.485 \\
0.585 \\
0.65 \\
0.55 \\
0.565 \\
0.52 \\
0.665 \\
0.54 \\
0.515 \\
0.67 \\
0.585 \\
0.49 \\
0.605 \\
0.66 \\
0.52 \\
0.555 \\
0.5 \\
0.6 \\
0.52 \\
0.68 \\
0.59 \\
0.515 \\
0.62 \\
0.515 \\
0.59 \\
0.6 \\
0.595 \\
0.61 \\
0.59 \\
0.64 \\
0.555 \\
0.55 \\
0.55 \\
0.575 \\
0.505 \\
0.515 \\
0.625 \\
0.565 \\
0.6 \\
0.685 \\
0.615 \\
};

\addplot[mark=*, boxplot, boxplot/draw position=3]
table[row sep=\\, y index=0] {
data
0.58 \\
0.505 \\
0.625 \\
0.6 \\
0.505 \\
0.725 \\
0.57 \\
0.755 \\
0.515 \\
0.5 \\
0.6 \\
0.6 \\
0.515 \\
0.765 \\
0.645 \\
0.6 \\
0.675 \\
0.545 \\
0.57 \\
0.64 \\
0.565 \\
0.785 \\
0.59 \\
0.705 \\
0.575 \\
0.525 \\
0.65 \\
0.57 \\
0.64 \\
0.4 \\
0.555 \\
0.705 \\
0.57 \\
0.535 \\
0.6 \\
0.59 \\
0.545 \\
0.56 \\
0.625 \\
0.595 \\
0.625 \\
0.745 \\
0.68 \\
0.675 \\
0.635 \\
0.565 \\
0.61 \\
0.615 \\
0.54 \\
0.58 \\
};

\addplot[mark=*, boxplot, boxplot/draw position=4]
table[row sep=\\, y index=0] {
data
0.645 \\
0.675 \\
0.56 \\
0.725 \\
0.545 \\
0.755 \\
0.665 \\
0.75 \\
0.515 \\
0.5 \\
0.52 \\
0.63 \\
0.585 \\
0.79 \\
0.635 \\
0.57 \\
0.795 \\
0.605 \\
0.775 \\
0.655 \\
0.59 \\
0.945 \\
0.595 \\
0.74 \\
0.585 \\
0.57 \\
0.715 \\
0.605 \\
0.625 \\
0.535 \\
0.575 \\
0.73 \\
0.575 \\
0.49 \\
0.605 \\
0.56 \\
0.575 \\
0.555 \\
0.65 \\
0.605 \\
0.665 \\
0.73 \\
0.705 \\
0.685 \\
0.665 \\
0.645 \\
0.51 \\
0.59 \\
0.62 \\
0.635 \\
};

\addplot[mark=*, boxplot, boxplot/draw position=5]
table[row sep=\\, y index=0] {
data
0.65 \\
0.61 \\
0.77 \\
0.81 \\
0.755 \\
0.665 \\
0.73 \\
0.67 \\
0.715 \\
0.775 \\
0.845 \\
0.505 \\
0.86 \\
0.685 \\
0.68 \\
0.635 \\
0.625 \\
0.645 \\
0.73 \\
0.685 \\
0.67 \\
0.745 \\
0.59 \\
0.76 \\
0.605 \\
0.7 \\
0.65 \\
0.61 \\
0.75 \\
0.605 \\
0.6 \\
0.66 \\
0.655 \\
0.775 \\
0.58 \\
0.615 \\
0.815 \\
0.71 \\
0.72 \\
0.77 \\
0.695 \\
0.8 \\
0.675 \\
0.88 \\
0.625 \\
0.58 \\
0.745 \\
0.625 \\
0.805 \\
0.795 \\
};

\addplot[mark=*, boxplot, boxplot/draw position=6]
table[row sep=\\, y index=0] {
data
0.66 \\
0.645 \\
0.685 \\
0.705 \\
0.625 \\
0.655 \\
0.515 \\
0.735 \\
0.73 \\
0.64 \\
0.845 \\
0.725 \\
0.72 \\
0.575 \\
0.72 \\
0.815 \\
0.655 \\
0.64 \\
0.765 \\
0.725 \\
0.705 \\
0.705 \\
0.72 \\
0.7 \\
0.64 \\
0.72 \\
0.585 \\
0.735 \\
0.9 \\
0.73 \\
0.675 \\
0.7 \\
0.56 \\
0.67 \\
0.715 \\
0.655 \\
0.685 \\
0.725 \\
0.58 \\
0.705 \\
0.67 \\
0.885 \\
0.675 \\
0.795 \\
0.725 \\
0.775 \\
0.61 \\
0.64 \\
0.525 \\
0.625 \\
};

\addplot[mark=*, boxplot, boxplot/draw position=7]
table[row sep=\\, y index=0] {
data
0.66 \\
0.655 \\
0.73 \\
0.695 \\
0.63 \\
0.775 \\
0.78 \\
0.485 \\
0.54 \\
0.645 \\
0.69 \\
0.85 \\
0.815 \\
0.79 \\
0.76 \\
0.77 \\
0.74 \\
0.785 \\
0.66 \\
0.895 \\
0.835 \\
0.63 \\
0.79 \\
0.755 \\
0.795 \\
0.52 \\
0.79 \\
0.64 \\
0.515 \\
0.625 \\
0.64 \\
0.7 \\
0.645 \\
0.61 \\
0.71 \\
0.71 \\
0.825 \\
0.725 \\
0.595 \\
0.82 \\
0.76 \\
0.835 \\
0.895 \\
0.605 \\
0.61 \\
0.65 \\
0.715 \\
0.66 \\
0.81 \\
0.845 \\
};

\addplot[mark=*, boxplot, boxplot/draw position=8]
table[row sep=\\, y index=0] {
data
0.67 \\
0.705 \\
0.75 \\
0.71 \\
0.595 \\
0.835 \\
0.79 \\
0.55 \\
0.595 \\
0.71 \\
0.8 \\
0.905 \\
0.81 \\
0.815 \\
0.73 \\
0.79 \\
0.75 \\
0.81 \\
0.66 \\
0.895 \\
0.845 \\
0.645 \\
0.78 \\
0.855 \\
0.825 \\
0.565 \\
0.835 \\
0.69 \\
0.525 \\
0.575 \\
0.64 \\
0.7 \\
0.7 \\
0.62 \\
0.715 \\
0.74 \\
0.84 \\
0.715 \\
0.6 \\
0.835 \\
0.75 \\
0.825 \\
0.895 \\
0.805 \\
0.65 \\
0.675 \\
0.8 \\
0.685 \\
0.835 \\
0.85 \\
};

\addplot[mark=*, boxplot, boxplot/draw position=9]
table[row sep=\\, y index=0] {
data
0.805 \\
0.665 \\
0.795 \\
0.86 \\
0.885 \\
0.81 \\
0.76 \\
0.595 \\
0.835 \\
0.61 \\
0.835 \\
0.725 \\
0.71 \\
0.545 \\
0.695 \\
0.795 \\
0.58 \\
0.765 \\
0.745 \\
0.68 \\
0.56 \\
0.68 \\
0.775 \\
0.795 \\
0.64 \\
0.57 \\
0.76 \\
0.725 \\
0.81 \\
0.655 \\
0.865 \\
0.595 \\
0.615 \\
0.635 \\
0.735 \\
0.745 \\
0.88 \\
0.89 \\
0.87 \\
0.99 \\
0.805 \\
0.925 \\
0.84 \\
0.775 \\
0.64 \\
0.605 \\
0.915 \\
0.725 \\
0.79 \\
0.755 \\
};

\addplot[mark=*, boxplot, boxplot/draw position=10]
table[row sep=\\, y index=0] {
data
0.785 \\
0.725 \\
0.61 \\
0.88 \\
0.82 \\
0.64 \\
0.805 \\
0.665 \\
0.73 \\
0.92 \\
0.805 \\
0.92 \\
0.84 \\
0.695 \\
0.715 \\
0.875 \\
0.725 \\
0.73 \\
0.595 \\
0.81 \\
0.86 \\
0.82 \\
0.79 \\
0.775 \\
0.775 \\
0.79 \\
0.885 \\
0.83 \\
0.72 \\
0.755 \\
0.765 \\
0.71 \\
0.885 \\
0.675 \\
0.72 \\
0.705 \\
0.705 \\
0.885 \\
0.9 \\
0.65 \\
0.72 \\
0.685 \\
0.775 \\
0.68 \\
0.72 \\
0.685 \\
0.91 \\
0.85 \\
0.83 \\
0.73 \\
};

\addplot[mark=*, boxplot, boxplot/draw position=11]
table[row sep=\\, y index=0] {
data
0.79 \\
0.63 \\
0.855 \\
0.86 \\
0.915 \\
0.855 \\
0.85 \\
0.635 \\
0.865 \\
0.62 \\
0.85 \\
0.755 \\
0.79 \\
0.605 \\
0.78 \\
0.85 \\
0.59 \\
0.8 \\
0.77 \\
0.75 \\
0.56 \\
0.715 \\
0.78 \\
0.83 \\
0.695 \\
0.56 \\
0.795 \\
0.735 \\
0.88 \\
0.75 \\
0.895 \\
0.745 \\
0.62 \\
0.745 \\
0.775 \\
0.875 \\
0.89 \\
0.92 \\
0.87 \\
0.99 \\
0.83 \\
0.935 \\
0.88 \\
0.72 \\
0.71 \\
0.695 \\
0.92 \\
0.79 \\
0.815 \\
0.78 \\
};
}{0.1}{Output connectivity}

        }
        \subfloat[N=120]{
            \myboxplot{

\addplot[mark=*, boxplot, boxplot/draw position=1]
table[row sep=\\, y index=0] {
data
0.66 \\
0.51 \\
0.55 \\
0.545 \\
0.525 \\
0.515 \\
0.58 \\
0.575 \\
0.54 \\
0.595 \\
0.52 \\
0.595 \\
0.525 \\
0.55 \\
0.575 \\
0.6 \\
0.59 \\
0.515 \\
0.56 \\
0.675 \\
0.51 \\
0.52 \\
0.5 \\
0.615 \\
0.615 \\
0.515 \\
0.525 \\
0.51 \\
0.585 \\
0.505 \\
0.495 \\
0.575 \\
0.6 \\
0.505 \\
0.46 \\
0.59 \\
0.545 \\
0.55 \\
0.535 \\
0.575 \\
0.605 \\
0.525 \\
0.52 \\
0.53 \\
0.63 \\
0.58 \\
0.545 \\
0.61 \\
0.51 \\
0.505 \\
};

\addplot[mark=*, boxplot, boxplot/draw position=2]
table[row sep=\\, y index=0] {
data
0.525 \\
0.58 \\
0.665 \\
0.52 \\
0.6 \\
0.585 \\
0.485 \\
0.555 \\
0.555 \\
0.6 \\
0.625 \\
0.62 \\
0.66 \\
0.6 \\
0.56 \\
0.545 \\
0.615 \\
0.565 \\
0.685 \\
0.73 \\
0.58 \\
0.595 \\
0.575 \\
0.655 \\
0.51 \\
0.56 \\
0.53 \\
0.725 \\
0.54 \\
0.625 \\
0.585 \\
0.575 \\
0.525 \\
0.67 \\
0.59 \\
0.675 \\
0.68 \\
0.57 \\
0.57 \\
0.545 \\
0.67 \\
0.55 \\
0.505 \\
0.66 \\
0.585 \\
0.52 \\
0.515 \\
0.595 \\
0.705 \\
0.535 \\
};

\addplot[mark=*, boxplot, boxplot/draw position=3]
table[row sep=\\, y index=0] {
data
0.6 \\
0.555 \\
0.66 \\
0.53 \\
0.59 \\
0.58 \\
0.49 \\
0.595 \\
0.64 \\
0.68 \\
0.67 \\
0.625 \\
0.71 \\
0.62 \\
0.595 \\
0.58 \\
0.55 \\
0.625 \\
0.67 \\
0.87 \\
0.55 \\
0.595 \\
0.51 \\
0.73 \\
0.565 \\
0.58 \\
0.55 \\
0.735 \\
0.53 \\
0.63 \\
0.62 \\
0.585 \\
0.525 \\
0.85 \\
0.53 \\
0.675 \\
0.74 \\
0.63 \\
0.575 \\
0.52 \\
0.72 \\
0.66 \\
0.53 \\
0.645 \\
0.545 \\
0.54 \\
0.575 \\
0.625 \\
0.775 \\
0.615 \\
};

\addplot[mark=*, boxplot, boxplot/draw position=4]
table[row sep=\\, y index=0] {
data
0.61 \\
0.6 \\
0.565 \\
0.72 \\
0.805 \\
0.64 \\
0.515 \\
0.59 \\
0.785 \\
0.595 \\
0.64 \\
0.635 \\
0.69 \\
0.705 \\
0.64 \\
0.635 \\
0.83 \\
0.605 \\
0.46 \\
0.765 \\
0.65 \\
0.74 \\
0.55 \\
0.645 \\
0.745 \\
0.595 \\
0.605 \\
0.595 \\
0.76 \\
0.59 \\
0.605 \\
0.59 \\
0.61 \\
0.655 \\
0.59 \\
0.715 \\
0.625 \\
0.59 \\
0.71 \\
0.48 \\
0.65 \\
0.535 \\
0.69 \\
0.57 \\
0.575 \\
0.685 \\
0.52 \\
0.665 \\
0.595 \\
0.59 \\
};

\addplot[mark=*, boxplot, boxplot/draw position=5]
table[row sep=\\, y index=0] {
data
0.88 \\
0.65 \\
0.5 \\
0.635 \\
0.615 \\
0.57 \\
0.805 \\
0.485 \\
0.725 \\
0.62 \\
0.66 \\
0.64 \\
0.605 \\
0.555 \\
0.765 \\
0.63 \\
0.835 \\
0.62 \\
0.775 \\
0.635 \\
0.65 \\
0.6 \\
0.64 \\
0.75 \\
0.73 \\
0.655 \\
0.78 \\
0.86 \\
0.75 \\
0.875 \\
0.72 \\
0.59 \\
0.64 \\
0.84 \\
0.75 \\
0.775 \\
0.83 \\
0.74 \\
0.66 \\
0.62 \\
0.575 \\
0.585 \\
0.72 \\
0.655 \\
0.57 \\
0.635 \\
0.815 \\
0.565 \\
0.565 \\
0.55 \\
};

\addplot[mark=*, boxplot, boxplot/draw position=6]
table[row sep=\\, y index=0] {
data
0.895 \\
0.67 \\
0.535 \\
0.65 \\
0.595 \\
0.6 \\
0.83 \\
0.53 \\
0.73 \\
0.625 \\
0.66 \\
0.675 \\
0.69 \\
0.64 \\
0.795 \\
0.645 \\
0.875 \\
0.63 \\
0.77 \\
0.69 \\
0.73 \\
0.61 \\
0.68 \\
0.735 \\
0.73 \\
0.635 \\
0.78 \\
0.865 \\
0.73 \\
0.875 \\
0.77 \\
0.59 \\
0.7 \\
0.835 \\
0.845 \\
0.8 \\
0.805 \\
0.72 \\
0.765 \\
0.635 \\
0.705 \\
0.64 \\
0.755 \\
0.66 \\
0.67 \\
0.65 \\
0.815 \\
0.725 \\
0.62 \\
0.6 \\
};

\addplot[mark=*, boxplot, boxplot/draw position=7]
table[row sep=\\, y index=0] {
data
0.665 \\
0.695 \\
0.745 \\
0.77 \\
0.58 \\
0.76 \\
0.73 \\
0.64 \\
0.7 \\
0.7 \\
0.775 \\
0.735 \\
0.59 \\
0.705 \\
0.675 \\
0.735 \\
0.565 \\
0.745 \\
0.72 \\
0.775 \\
0.705 \\
0.8 \\
0.695 \\
0.73 \\
0.66 \\
0.775 \\
0.69 \\
0.655 \\
0.71 \\
0.69 \\
0.775 \\
0.66 \\
0.51 \\
0.635 \\
0.58 \\
0.725 \\
0.705 \\
0.69 \\
0.59 \\
0.55 \\
0.75 \\
0.575 \\
0.45 \\
0.715 \\
0.635 \\
0.525 \\
0.505 \\
0.765 \\
0.64 \\
0.565 \\
};

\addplot[mark=*, boxplot, boxplot/draw position=8]
table[row sep=\\, y index=0] {
data
0.68 \\
0.73 \\
0.775 \\
0.565 \\
0.58 \\
0.89 \\
0.715 \\
0.88 \\
0.8 \\
0.75 \\
0.805 \\
0.795 \\
0.585 \\
0.69 \\
0.61 \\
0.89 \\
0.775 \\
0.745 \\
0.915 \\
0.54 \\
0.83 \\
0.68 \\
0.835 \\
0.67 \\
0.72 \\
0.915 \\
0.9 \\
0.715 \\
0.645 \\
0.84 \\
0.565 \\
0.68 \\
0.69 \\
0.65 \\
0.81 \\
0.765 \\
0.875 \\
0.78 \\
0.62 \\
0.675 \\
0.815 \\
0.645 \\
0.66 \\
0.87 \\
0.71 \\
0.735 \\
0.7 \\
0.875 \\
0.74 \\
0.83 \\
};

\addplot[mark=*, boxplot, boxplot/draw position=9]
table[row sep=\\, y index=0] {
data
0.66 \\
0.81 \\
0.69 \\
0.795 \\
0.875 \\
0.76 \\
0.635 \\
0.81 \\
0.83 \\
0.64 \\
0.69 \\
0.735 \\
0.645 \\
0.79 \\
0.885 \\
0.88 \\
0.775 \\
0.725 \\
0.595 \\
0.625 \\
0.71 \\
0.71 \\
0.77 \\
0.735 \\
0.66 \\
0.685 \\
0.705 \\
0.745 \\
0.805 \\
0.77 \\
0.73 \\
0.79 \\
0.695 \\
0.86 \\
0.875 \\
0.775 \\
0.755 \\
0.72 \\
0.785 \\
0.69 \\
0.83 \\
0.81 \\
0.755 \\
0.77 \\
0.675 \\
0.715 \\
0.68 \\
0.72 \\
0.825 \\
0.685 \\
};

\addplot[mark=*, boxplot, boxplot/draw position=10]
table[row sep=\\, y index=0] {
data
0.775 \\
0.75 \\
0.755 \\
0.61 \\
0.705 \\
0.89 \\
0.82 \\
0.945 \\
0.82 \\
0.775 \\
0.805 \\
0.835 \\
0.59 \\
0.7 \\
0.67 \\
0.905 \\
0.83 \\
0.84 \\
0.93 \\
0.515 \\
0.85 \\
0.705 \\
0.78 \\
0.71 \\
0.775 \\
0.925 \\
0.9 \\
0.77 \\
0.705 \\
0.85 \\
0.595 \\
0.695 \\
0.81 \\
0.705 \\
0.845 \\
0.8 \\
0.88 \\
0.79 \\
0.64 \\
0.745 \\
0.74 \\
0.615 \\
0.905 \\
0.89 \\
0.735 \\
0.785 \\
0.81 \\
0.885 \\
0.745 \\
0.905 \\
};

\addplot[mark=*, boxplot, boxplot/draw position=11]
table[row sep=\\, y index=0] {
data
0.835 \\
0.745 \\
0.99 \\
0.84 \\
0.74 \\
0.85 \\
0.73 \\
0.765 \\
0.835 \\
0.8 \\
0.85 \\
0.8 \\
0.895 \\
0.765 \\
0.91 \\
0.845 \\
0.815 \\
0.765 \\
0.665 \\
0.635 \\
0.675 \\
0.76 \\
0.7 \\
0.71 \\
0.805 \\
0.64 \\
0.875 \\
0.565 \\
0.7 \\
0.81 \\
0.865 \\
0.845 \\
0.63 \\
0.93 \\
0.7 \\
0.99 \\
0.705 \\
0.745 \\
0.85 \\
0.895 \\
0.855 \\
0.885 \\
0.8 \\
0.935 \\
0.85 \\
0.84 \\
0.78 \\
0.76 \\
0.8 \\
0.69 \\
};

\addplot[mark=*, boxplot, boxplot/draw position=12]
table[row sep=\\, y index=0] {
data
0.84 \\
0.76 \\
0.99 \\
0.955 \\
0.72 \\
0.85 \\
0.78 \\
0.8 \\
0.865 \\
0.78 \\
0.85 \\
0.93 \\
0.895 \\
0.775 \\
0.92 \\
0.85 \\
0.84 \\
0.775 \\
0.655 \\
0.635 \\
0.675 \\
0.765 \\
0.705 \\
0.67 \\
0.81 \\
0.655 \\
0.885 \\
0.61 \\
0.735 \\
0.795 \\
0.845 \\
0.84 \\
0.625 \\
0.93 \\
0.725 \\
0.99 \\
0.73 \\
0.77 \\
0.875 \\
0.895 \\
0.87 \\
0.915 \\
0.82 \\
0.935 \\
0.87 \\
0.88 \\
0.805 \\
0.795 \\
0.815 \\
0.7 \\
};
}{0.1}{Output connectivity}

        }
    }
    \caption{Plots for optimal input task 2 - part 2}
\end{figure*}

\begin{figure*}[ht]
    \centering
    \resizebox{\textwidth}{!}{
        \subfloat[N=130]{
            \myboxplot{

\addplot[mark=*, boxplot, boxplot/draw position=1]
table[row sep=\\, y index=0] {
data
0.505 \\
0.51 \\
0.54 \\
0.465 \\
0.64 \\
0.525 \\
0.525 \\
0.575 \\
0.555 \\
0.54 \\
0.575 \\
0.515 \\
0.525 \\
0.505 \\
0.54 \\
0.535 \\
0.535 \\
0.54 \\
0.605 \\
0.555 \\
0.49 \\
0.555 \\
0.605 \\
0.505 \\
0.585 \\
0.505 \\
0.53 \\
0.56 \\
0.61 \\
0.535 \\
0.64 \\
0.545 \\
0.54 \\
0.575 \\
0.595 \\
0.55 \\
0.56 \\
0.47 \\
0.56 \\
0.59 \\
0.48 \\
0.53 \\
0.51 \\
0.635 \\
0.525 \\
0.545 \\
0.53 \\
0.585 \\
0.495 \\
0.47 \\
};

\addplot[mark=*, boxplot, boxplot/draw position=2]
table[row sep=\\, y index=0] {
data
0.72 \\
0.535 \\
0.59 \\
0.615 \\
0.575 \\
0.565 \\
0.595 \\
0.61 \\
0.455 \\
0.555 \\
0.585 \\
0.59 \\
0.56 \\
0.545 \\
0.605 \\
0.495 \\
0.545 \\
0.79 \\
0.505 \\
0.615 \\
0.56 \\
0.57 \\
0.72 \\
0.56 \\
0.51 \\
0.58 \\
0.525 \\
0.675 \\
0.65 \\
0.63 \\
0.64 \\
0.545 \\
0.61 \\
0.515 \\
0.58 \\
0.575 \\
0.62 \\
0.56 \\
0.57 \\
0.55 \\
0.55 \\
0.52 \\
0.51 \\
0.6 \\
0.51 \\
0.65 \\
0.585 \\
0.68 \\
0.535 \\
0.555 \\
};

\addplot[mark=*, boxplot, boxplot/draw position=3]
table[row sep=\\, y index=0] {
data
0.62 \\
0.48 \\
0.625 \\
0.625 \\
0.69 \\
0.52 \\
0.67 \\
0.525 \\
0.585 \\
0.6 \\
0.55 \\
0.645 \\
0.685 \\
0.585 \\
0.6 \\
0.66 \\
0.67 \\
0.535 \\
0.635 \\
0.63 \\
0.685 \\
0.55 \\
0.505 \\
0.675 \\
0.595 \\
0.56 \\
0.575 \\
0.55 \\
0.545 \\
0.71 \\
0.575 \\
0.555 \\
0.71 \\
0.855 \\
0.73 \\
0.49 \\
0.51 \\
0.645 \\
0.705 \\
0.615 \\
0.685 \\
0.555 \\
0.585 \\
0.55 \\
0.675 \\
0.58 \\
0.575 \\
0.61 \\
0.56 \\
0.5 \\
};

\addplot[mark=*, boxplot, boxplot/draw position=4]
table[row sep=\\, y index=0] {
data
0.805 \\
0.585 \\
0.64 \\
0.6 \\
0.555 \\
0.595 \\
0.595 \\
0.655 \\
0.655 \\
0.68 \\
0.63 \\
0.685 \\
0.605 \\
0.57 \\
0.685 \\
0.66 \\
0.615 \\
0.84 \\
0.565 \\
0.78 \\
0.62 \\
0.64 \\
0.845 \\
0.71 \\
0.535 \\
0.625 \\
0.68 \\
0.71 \\
0.745 \\
0.68 \\
0.675 \\
0.65 \\
0.695 \\
0.585 \\
0.62 \\
0.585 \\
0.71 \\
0.68 \\
0.51 \\
0.655 \\
0.675 \\
0.53 \\
0.615 \\
0.76 \\
0.51 \\
0.71 \\
0.54 \\
0.65 \\
0.52 \\
0.7 \\
};

\addplot[mark=*, boxplot, boxplot/draw position=5]
table[row sep=\\, y index=0] {
data
0.645 \\
0.52 \\
0.7 \\
0.605 \\
0.625 \\
0.7 \\
0.585 \\
0.72 \\
0.67 \\
0.705 \\
0.56 \\
0.645 \\
0.655 \\
0.73 \\
0.645 \\
0.49 \\
0.635 \\
0.59 \\
0.705 \\
0.665 \\
0.58 \\
0.775 \\
0.64 \\
0.57 \\
0.645 \\
0.8 \\
0.635 \\
0.72 \\
0.615 \\
0.78 \\
0.65 \\
0.785 \\
0.525 \\
0.55 \\
0.79 \\
0.715 \\
0.805 \\
0.735 \\
0.715 \\
0.66 \\
0.745 \\
0.86 \\
0.71 \\
0.715 \\
0.59 \\
0.73 \\
0.69 \\
0.575 \\
0.66 \\
0.71 \\
};

\addplot[mark=*, boxplot, boxplot/draw position=6]
table[row sep=\\, y index=0] {
data
0.625 \\
0.49 \\
0.71 \\
0.615 \\
0.63 \\
0.755 \\
0.575 \\
0.76 \\
0.73 \\
0.71 \\
0.625 \\
0.645 \\
0.675 \\
0.785 \\
0.67 \\
0.475 \\
0.68 \\
0.58 \\
0.72 \\
0.71 \\
0.65 \\
0.795 \\
0.72 \\
0.66 \\
0.71 \\
0.82 \\
0.62 \\
0.72 \\
0.625 \\
0.805 \\
0.63 \\
0.775 \\
0.515 \\
0.54 \\
0.74 \\
0.76 \\
0.89 \\
0.745 \\
0.71 \\
0.745 \\
0.8 \\
0.87 \\
0.75 \\
0.7 \\
0.685 \\
0.645 \\
0.675 \\
0.58 \\
0.66 \\
0.73 \\
};

\addplot[mark=*, boxplot, boxplot/draw position=7]
table[row sep=\\, y index=0] {
data
0.575 \\
0.64 \\
0.74 \\
0.63 \\
0.86 \\
0.72 \\
0.76 \\
0.815 \\
0.78 \\
0.675 \\
0.655 \\
0.87 \\
0.905 \\
0.705 \\
0.72 \\
0.775 \\
0.765 \\
0.805 \\
0.895 \\
0.735 \\
0.65 \\
0.57 \\
0.58 \\
0.835 \\
0.675 \\
0.785 \\
0.765 \\
0.66 \\
0.655 \\
0.635 \\
0.64 \\
0.865 \\
0.775 \\
0.64 \\
0.58 \\
0.575 \\
0.75 \\
0.81 \\
0.63 \\
0.715 \\
0.845 \\
0.735 \\
0.675 \\
0.86 \\
0.705 \\
0.66 \\
0.96 \\
0.605 \\
0.73 \\
0.7 \\
};

\addplot[mark=*, boxplot, boxplot/draw position=8]
table[row sep=\\, y index=0] {
data
0.65 \\
0.685 \\
0.595 \\
0.845 \\
0.8 \\
0.595 \\
0.735 \\
0.77 \\
0.67 \\
0.67 \\
0.845 \\
0.78 \\
0.765 \\
0.57 \\
0.615 \\
0.715 \\
0.745 \\
0.92 \\
0.735 \\
0.68 \\
0.84 \\
0.655 \\
0.705 \\
0.595 \\
0.685 \\
0.615 \\
0.72 \\
0.83 \\
0.855 \\
0.92 \\
0.665 \\
0.87 \\
0.79 \\
0.61 \\
0.705 \\
0.735 \\
0.51 \\
0.72 \\
0.63 \\
0.775 \\
0.8 \\
0.805 \\
0.91 \\
0.815 \\
0.7 \\
0.82 \\
0.645 \\
0.495 \\
0.855 \\
0.695 \\
};

\addplot[mark=*, boxplot, boxplot/draw position=9]
table[row sep=\\, y index=0] {
data
0.77 \\
0.88 \\
0.725 \\
0.775 \\
0.88 \\
0.59 \\
0.845 \\
0.795 \\
0.76 \\
0.755 \\
0.79 \\
0.875 \\
0.825 \\
0.72 \\
0.705 \\
0.6 \\
0.705 \\
0.72 \\
0.76 \\
0.69 \\
0.58 \\
0.75 \\
0.69 \\
0.665 \\
0.59 \\
0.71 \\
0.775 \\
0.64 \\
0.87 \\
0.705 \\
0.805 \\
0.68 \\
0.805 \\
0.74 \\
0.69 \\
0.575 \\
0.825 \\
0.815 \\
0.78 \\
0.845 \\
0.66 \\
0.745 \\
0.67 \\
0.62 \\
0.78 \\
0.705 \\
0.78 \\
0.8 \\
0.935 \\
0.84 \\
};

\addplot[mark=*, boxplot, boxplot/draw position=10]
table[row sep=\\, y index=0] {
data
0.835 \\
0.93 \\
0.9 \\
0.825 \\
0.88 \\
0.705 \\
0.87 \\
0.74 \\
0.85 \\
0.69 \\
0.78 \\
0.82 \\
0.965 \\
0.68 \\
0.89 \\
0.835 \\
0.795 \\
0.795 \\
0.865 \\
0.7 \\
0.94 \\
0.575 \\
0.57 \\
0.87 \\
0.77 \\
0.665 \\
0.77 \\
0.76 \\
0.91 \\
0.755 \\
0.72 \\
0.745 \\
0.8 \\
0.785 \\
0.89 \\
0.715 \\
0.915 \\
0.76 \\
0.59 \\
0.865 \\
0.845 \\
0.825 \\
0.685 \\
0.81 \\
0.835 \\
0.785 \\
0.875 \\
0.775 \\
0.545 \\
0.85 \\
};

\addplot[mark=*, boxplot, boxplot/draw position=11]
table[row sep=\\, y index=0] {
data
0.74 \\
0.89 \\
0.8 \\
0.795 \\
0.895 \\
0.675 \\
0.86 \\
0.71 \\
0.84 \\
0.75 \\
0.88 \\
0.89 \\
0.85 \\
0.73 \\
0.745 \\
0.735 \\
0.755 \\
0.715 \\
0.79 \\
0.765 \\
0.6 \\
0.905 \\
0.685 \\
0.67 \\
0.75 \\
0.77 \\
0.81 \\
0.71 \\
0.885 \\
0.74 \\
0.795 \\
0.67 \\
0.885 \\
0.745 \\
0.705 \\
0.62 \\
0.84 \\
0.835 \\
0.79 \\
0.86 \\
0.72 \\
0.745 \\
0.66 \\
0.62 \\
0.775 \\
0.695 \\
0.795 \\
0.88 \\
0.93 \\
0.86 \\
};

\addplot[mark=*, boxplot, boxplot/draw position=12]
table[row sep=\\, y index=0] {
data
0.885 \\
0.89 \\
0.91 \\
0.84 \\
0.9 \\
0.655 \\
0.885 \\
0.8 \\
0.84 \\
0.71 \\
0.8 \\
0.83 \\
0.955 \\
0.695 \\
0.91 \\
0.85 \\
0.82 \\
0.87 \\
0.845 \\
0.81 \\
0.945 \\
0.605 \\
0.605 \\
0.86 \\
0.795 \\
0.705 \\
0.9 \\
0.74 \\
0.95 \\
0.81 \\
0.75 \\
0.765 \\
0.825 \\
0.805 \\
0.925 \\
0.755 \\
0.945 \\
0.82 \\
0.91 \\
0.87 \\
0.88 \\
0.845 \\
0.685 \\
0.815 \\
0.85 \\
0.79 \\
0.885 \\
0.765 \\
0.56 \\
0.845 \\
};

\addplot[mark=*, boxplot, boxplot/draw position=13]
table[row sep=\\, y index=0] {
data
0.74 \\
0.73 \\
0.865 \\
0.76 \\
0.89 \\
0.66 \\
0.82 \\
0.96 \\
0.86 \\
0.685 \\
0.89 \\
0.815 \\
0.93 \\
0.53 \\
0.745 \\
0.895 \\
0.825 \\
0.71 \\
0.825 \\
0.675 \\
0.645 \\
0.605 \\
0.835 \\
0.83 \\
0.885 \\
0.845 \\
0.81 \\
0.8 \\
0.89 \\
0.76 \\
0.87 \\
0.865 \\
0.87 \\
0.51 \\
0.73 \\
0.88 \\
0.785 \\
0.755 \\
0.685 \\
0.975 \\
0.87 \\
0.71 \\
0.875 \\
0.86 \\
0.865 \\
0.59 \\
0.77 \\
0.765 \\
0.775 \\
0.765 \\
};
}{0.1}{Output connectivity}{14}

        }
        \subfloat[N=140]{
            \myboxplot{

\addplot[mark=*, boxplot, boxplot/draw position=1]
table[row sep=\\, y index=0] {
data
0.525 \\
0.55 \\
0.495 \\
0.565 \\
0.53 \\
0.63 \\
0.495 \\
0.495 \\
0.56 \\
0.585 \\
0.465 \\
0.68 \\
0.465 \\
0.43 \\
0.58 \\
0.495 \\
0.435 \\
0.53 \\
0.56 \\
0.55 \\
0.5 \\
0.545 \\
0.6 \\
0.51 \\
0.53 \\
0.495 \\
0.45 \\
0.57 \\
0.515 \\
0.5 \\
0.44 \\
0.535 \\
0.565 \\
0.485 \\
0.575 \\
0.51 \\
0.65 \\
0.595 \\
0.6 \\
0.57 \\
0.54 \\
0.55 \\
0.595 \\
0.51 \\
0.65 \\
0.505 \\
0.645 \\
0.545 \\
0.545 \\
0.585 \\
};

\addplot[mark=*, boxplot, boxplot/draw position=2]
table[row sep=\\, y index=0] {
data
0.515 \\
0.63 \\
0.54 \\
0.55 \\
0.495 \\
0.49 \\
0.495 \\
0.52 \\
0.515 \\
0.53 \\
0.575 \\
0.66 \\
0.475 \\
0.595 \\
0.56 \\
0.73 \\
0.55 \\
0.54 \\
0.59 \\
0.595 \\
0.755 \\
0.7 \\
0.67 \\
0.53 \\
0.57 \\
0.645 \\
0.57 \\
0.59 \\
0.515 \\
0.63 \\
0.53 \\
0.53 \\
0.74 \\
0.55 \\
0.425 \\
0.655 \\
0.78 \\
0.745 \\
0.59 \\
0.545 \\
0.54 \\
0.55 \\
0.615 \\
0.66 \\
0.59 \\
0.625 \\
0.535 \\
0.59 \\
0.6 \\
0.53 \\
};

\addplot[mark=*, boxplot, boxplot/draw position=3]
table[row sep=\\, y index=0] {
data
0.625 \\
0.585 \\
0.695 \\
0.69 \\
0.58 \\
0.505 \\
0.59 \\
0.6 \\
0.735 \\
0.54 \\
0.56 \\
0.595 \\
0.585 \\
0.675 \\
0.465 \\
0.61 \\
0.62 \\
0.515 \\
0.625 \\
0.68 \\
0.535 \\
0.625 \\
0.54 \\
0.605 \\
0.565 \\
0.595 \\
0.595 \\
0.715 \\
0.605 \\
0.64 \\
0.74 \\
0.45 \\
0.515 \\
0.63 \\
0.555 \\
0.565 \\
0.525 \\
0.645 \\
0.705 \\
0.605 \\
0.685 \\
0.65 \\
0.55 \\
0.6 \\
0.595 \\
0.56 \\
0.695 \\
0.69 \\
0.58 \\
0.605 \\
};

\addplot[mark=*, boxplot, boxplot/draw position=4]
table[row sep=\\, y index=0] {
data
0.59 \\
0.705 \\
0.565 \\
0.58 \\
0.585 \\
0.63 \\
0.49 \\
0.67 \\
0.575 \\
0.595 \\
0.71 \\
0.715 \\
0.54 \\
0.64 \\
0.615 \\
0.81 \\
0.695 \\
0.62 \\
0.615 \\
0.835 \\
0.755 \\
0.785 \\
0.65 \\
0.53 \\
0.535 \\
0.73 \\
0.76 \\
0.545 \\
0.52 \\
0.665 \\
0.65 \\
0.54 \\
0.815 \\
0.615 \\
0.585 \\
0.64 \\
0.78 \\
0.78 \\
0.605 \\
0.7 \\
0.65 \\
0.54 \\
0.67 \\
0.725 \\
0.665 \\
0.54 \\
0.65 \\
0.595 \\
0.66 \\
0.59 \\
};

\addplot[mark=*, boxplot, boxplot/draw position=5]
table[row sep=\\, y index=0] {
data
0.605 \\
0.61 \\
0.695 \\
0.765 \\
0.6 \\
0.505 \\
0.605 \\
0.655 \\
0.76 \\
0.74 \\
0.585 \\
0.58 \\
0.63 \\
0.67 \\
0.565 \\
0.68 \\
0.63 \\
0.555 \\
0.66 \\
0.675 \\
0.6 \\
0.75 \\
0.565 \\
0.69 \\
0.555 \\
0.615 \\
0.56 \\
0.76 \\
0.68 \\
0.635 \\
0.785 \\
0.77 \\
0.555 \\
0.675 \\
0.59 \\
0.655 \\
0.705 \\
0.63 \\
0.715 \\
0.61 \\
0.725 \\
0.71 \\
0.615 \\
0.655 \\
0.655 \\
0.715 \\
0.695 \\
0.78 \\
0.65 \\
0.67 \\
};

\addplot[mark=*, boxplot, boxplot/draw position=6]
table[row sep=\\, y index=0] {
data
0.655 \\
0.59 \\
0.855 \\
0.62 \\
0.735 \\
0.65 \\
0.5 \\
0.88 \\
0.65 \\
0.755 \\
0.54 \\
0.735 \\
0.77 \\
0.9 \\
0.71 \\
0.66 \\
0.62 \\
0.6 \\
0.645 \\
0.725 \\
0.69 \\
0.65 \\
0.97 \\
0.68 \\
0.88 \\
0.895 \\
0.89 \\
0.72 \\
0.805 \\
0.735 \\
0.715 \\
0.46 \\
0.675 \\
0.615 \\
0.63 \\
0.62 \\
0.695 \\
0.535 \\
0.695 \\
0.665 \\
0.565 \\
0.885 \\
0.71 \\
0.525 \\
0.67 \\
0.77 \\
0.81 \\
0.68 \\
0.6 \\
0.66 \\
};

\addplot[mark=*, boxplot, boxplot/draw position=7]
table[row sep=\\, y index=0] {
data
0.665 \\
0.61 \\
0.675 \\
0.765 \\
0.73 \\
0.73 \\
0.8 \\
0.665 \\
0.83 \\
0.665 \\
0.57 \\
0.91 \\
0.605 \\
0.64 \\
0.855 \\
0.71 \\
0.795 \\
0.76 \\
0.845 \\
0.675 \\
0.71 \\
0.6 \\
0.83 \\
0.685 \\
0.76 \\
0.65 \\
0.615 \\
0.65 \\
0.84 \\
0.59 \\
0.62 \\
0.535 \\
0.835 \\
0.615 \\
0.835 \\
0.5 \\
0.75 \\
0.785 \\
0.81 \\
0.535 \\
0.585 \\
0.755 \\
0.77 \\
0.82 \\
0.68 \\
0.81 \\
0.825 \\
0.655 \\
0.67 \\
0.875 \\
};

\addplot[mark=*, boxplot, boxplot/draw position=8]
table[row sep=\\, y index=0] {
data
0.785 \\
0.645 \\
0.84 \\
0.59 \\
0.765 \\
0.615 \\
0.66 \\
0.675 \\
0.875 \\
0.69 \\
0.625 \\
0.54 \\
0.55 \\
0.635 \\
0.6 \\
0.825 \\
0.73 \\
0.64 \\
0.675 \\
0.86 \\
0.745 \\
0.865 \\
0.65 \\
0.67 \\
0.64 \\
0.77 \\
0.72 \\
0.745 \\
0.795 \\
0.52 \\
0.805 \\
0.91 \\
0.76 \\
0.68 \\
0.7 \\
0.935 \\
0.785 \\
0.91 \\
0.825 \\
0.715 \\
0.815 \\
0.64 \\
0.745 \\
0.655 \\
0.74 \\
0.545 \\
0.68 \\
0.83 \\
0.81 \\
0.675 \\
};

\addplot[mark=*, boxplot, boxplot/draw position=9]
table[row sep=\\, y index=0] {
data
0.88 \\
0.935 \\
0.745 \\
0.64 \\
0.67 \\
0.83 \\
0.825 \\
0.615 \\
0.66 \\
0.81 \\
0.625 \\
0.83 \\
0.9 \\
0.66 \\
0.79 \\
0.755 \\
0.835 \\
0.835 \\
0.715 \\
0.71 \\
0.665 \\
0.665 \\
0.585 \\
0.715 \\
0.765 \\
0.74 \\
0.485 \\
0.705 \\
0.76 \\
0.86 \\
0.745 \\
0.68 \\
0.765 \\
0.885 \\
0.94 \\
0.8 \\
0.555 \\
0.61 \\
0.575 \\
0.76 \\
0.665 \\
0.895 \\
0.825 \\
0.94 \\
0.795 \\
0.84 \\
0.67 \\
0.695 \\
0.87 \\
0.855 \\
};

\addplot[mark=*, boxplot, boxplot/draw position=10]
table[row sep=\\, y index=0] {
data
0.825 \\
0.665 \\
0.865 \\
0.58 \\
0.78 \\
0.66 \\
0.7 \\
0.725 \\
0.88 \\
0.72 \\
0.715 \\
0.595 \\
0.6 \\
0.755 \\
0.565 \\
0.905 \\
0.86 \\
0.685 \\
0.695 \\
0.85 \\
0.805 \\
0.895 \\
0.715 \\
0.695 \\
0.675 \\
0.785 \\
0.855 \\
0.785 \\
0.81 \\
0.54 \\
0.805 \\
0.95 \\
0.825 \\
0.68 \\
0.69 \\
0.955 \\
0.835 \\
0.93 \\
0.8 \\
0.76 \\
0.83 \\
0.85 \\
0.765 \\
0.675 \\
0.795 \\
0.58 \\
0.645 \\
0.89 \\
0.835 \\
0.735 \\
};

\addplot[mark=*, boxplot, boxplot/draw position=11]
table[row sep=\\, y index=0] {
data
0.885 \\
0.95 \\
0.725 \\
0.72 \\
0.74 \\
0.865 \\
0.805 \\
0.62 \\
0.67 \\
0.92 \\
0.71 \\
0.82 \\
0.915 \\
0.71 \\
0.815 \\
0.755 \\
0.85 \\
0.865 \\
0.8 \\
0.74 \\
0.715 \\
0.715 \\
0.555 \\
0.715 \\
0.83 \\
0.675 \\
0.49 \\
0.72 \\
0.805 \\
0.865 \\
0.805 \\
0.68 \\
0.76 \\
0.89 \\
0.915 \\
0.8 \\
0.6 \\
0.565 \\
0.625 \\
0.755 \\
0.705 \\
0.895 \\
0.84 \\
0.945 \\
0.905 \\
0.84 \\
0.66 \\
0.79 \\
0.915 \\
0.91 \\
};

\addplot[mark=*, boxplot, boxplot/draw position=12]
table[row sep=\\, y index=0] {
data
0.735 \\
0.88 \\
0.675 \\
0.805 \\
0.79 \\
0.69 \\
0.91 \\
0.635 \\
0.76 \\
0.855 \\
0.77 \\
0.635 \\
0.905 \\
0.935 \\
0.85 \\
0.755 \\
0.795 \\
0.84 \\
0.885 \\
0.87 \\
0.805 \\
0.76 \\
0.99 \\
0.73 \\
0.68 \\
0.845 \\
0.92 \\
0.655 \\
0.83 \\
0.795 \\
0.73 \\
0.79 \\
0.715 \\
0.78 \\
0.875 \\
0.875 \\
0.835 \\
0.86 \\
0.735 \\
0.89 \\
0.825 \\
0.86 \\
0.835 \\
0.64 \\
0.825 \\
0.63 \\
0.695 \\
0.84 \\
0.71 \\
0.825 \\
};

\addplot[mark=*, boxplot, boxplot/draw position=13]
table[row sep=\\, y index=0] {
data
0.875 \\
0.835 \\
0.7 \\
0.75 \\
0.72 \\
0.815 \\
0.845 \\
0.715 \\
0.725 \\
0.86 \\
0.865 \\
0.815 \\
0.77 \\
0.875 \\
0.68 \\
0.935 \\
0.87 \\
0.865 \\
0.88 \\
0.885 \\
0.895 \\
0.795 \\
0.85 \\
0.84 \\
0.785 \\
0.97 \\
0.795 \\
0.655 \\
0.785 \\
0.85 \\
0.84 \\
0.84 \\
0.665 \\
0.805 \\
0.865 \\
0.88 \\
0.945 \\
0.885 \\
0.91 \\
0.725 \\
0.815 \\
0.835 \\
0.645 \\
0.64 \\
0.935 \\
0.805 \\
0.69 \\
0.825 \\
0.695 \\
0.79 \\
};

\addplot[mark=*, boxplot, boxplot/draw position=14]
table[row sep=\\, y index=0] {
data
0.99 \\
0.84 \\
0.925 \\
0.915 \\
0.845 \\
0.775 \\
0.88 \\
0.925 \\
0.8 \\
0.835 \\
0.91 \\
0.955 \\
0.75 \\
0.79 \\
0.95 \\
0.73 \\
0.635 \\
0.88 \\
0.765 \\
0.865 \\
0.86 \\
0.85 \\
0.69 \\
0.75 \\
0.905 \\
0.93 \\
0.645 \\
0.735 \\
0.925 \\
0.805 \\
0.97 \\
0.725 \\
0.69 \\
0.8 \\
0.925 \\
0.925 \\
0.715 \\
0.82 \\
0.745 \\
0.75 \\
0.91 \\
0.89 \\
0.675 \\
0.805 \\
0.835 \\
0.845 \\
0.74 \\
0.865 \\
0.925 \\
0.95 \\
};
}{0.1}{Output connectivity}

        }
    }
    \resizebox{0.5\textwidth}{!}{
        \subfloat[N=150]{
            \myboxplot{

\addplot[mark=*, boxplot, boxplot/draw position=1]
table[row sep=\\, y index=0] {
data
0.515 \\
0.595 \\
0.515 \\
0.47 \\
0.57 \\
0.54 \\
0.655 \\
0.595 \\
0.68 \\
0.575 \\
0.55 \\
0.545 \\
0.565 \\
0.52 \\
0.615 \\
0.525 \\
0.47 \\
0.425 \\
0.485 \\
0.515 \\
0.5 \\
0.485 \\
0.61 \\
0.495 \\
0.595 \\
0.52 \\
0.575 \\
0.615 \\
0.495 \\
0.5 \\
0.655 \\
0.535 \\
0.535 \\
0.6 \\
0.595 \\
0.66 \\
0.505 \\
0.585 \\
0.535 \\
0.545 \\
0.545 \\
0.485 \\
0.585 \\
0.61 \\
0.575 \\
0.605 \\
0.53 \\
0.615 \\
0.54 \\
0.49 \\
};

\addplot[mark=*, boxplot, boxplot/draw position=2]
table[row sep=\\, y index=0] {
data
0.565 \\
0.59 \\
0.62 \\
0.605 \\
0.64 \\
0.465 \\
0.585 \\
0.64 \\
0.67 \\
0.49 \\
0.575 \\
0.52 \\
0.675 \\
0.525 \\
0.625 \\
0.61 \\
0.515 \\
0.525 \\
0.555 \\
0.575 \\
0.545 \\
0.52 \\
0.58 \\
0.5 \\
0.585 \\
0.6 \\
0.525 \\
0.695 \\
0.59 \\
0.57 \\
0.615 \\
0.495 \\
0.5 \\
0.675 \\
0.505 \\
0.55 \\
0.555 \\
0.57 \\
0.485 \\
0.585 \\
0.515 \\
0.49 \\
0.625 \\
0.635 \\
0.63 \\
0.655 \\
0.59 \\
0.685 \\
0.54 \\
0.545 \\
};

\addplot[mark=*, boxplot, boxplot/draw position=3]
table[row sep=\\, y index=0] {
data
0.635 \\
0.67 \\
0.51 \\
0.575 \\
0.57 \\
0.675 \\
0.46 \\
0.59 \\
0.545 \\
0.605 \\
0.595 \\
0.615 \\
0.645 \\
0.63 \\
0.705 \\
0.565 \\
0.615 \\
0.685 \\
0.625 \\
0.555 \\
0.64 \\
0.565 \\
0.635 \\
0.635 \\
0.74 \\
0.595 \\
0.61 \\
0.59 \\
0.655 \\
0.515 \\
0.565 \\
0.53 \\
0.59 \\
0.54 \\
0.505 \\
0.585 \\
0.49 \\
0.635 \\
0.6 \\
0.56 \\
0.63 \\
0.66 \\
0.565 \\
0.6 \\
0.575 \\
0.605 \\
0.8 \\
0.55 \\
0.775 \\
0.555 \\
};

\addplot[mark=*, boxplot, boxplot/draw position=4]
table[row sep=\\, y index=0] {
data
0.745 \\
0.65 \\
0.675 \\
0.56 \\
0.545 \\
0.575 \\
0.465 \\
0.75 \\
0.655 \\
0.665 \\
0.72 \\
0.56 \\
0.64 \\
0.63 \\
0.5 \\
0.605 \\
0.64 \\
0.545 \\
0.59 \\
0.66 \\
0.64 \\
0.695 \\
0.62 \\
0.605 \\
0.745 \\
0.605 \\
0.675 \\
0.63 \\
0.575 \\
0.655 \\
0.665 \\
0.695 \\
0.685 \\
0.58 \\
0.72 \\
0.62 \\
0.5 \\
0.56 \\
0.81 \\
0.695 \\
0.54 \\
0.82 \\
0.725 \\
0.695 \\
0.61 \\
0.585 \\
0.58 \\
0.705 \\
0.58 \\
0.66 \\
};

\addplot[mark=*, boxplot, boxplot/draw position=5]
table[row sep=\\, y index=0] {
data
0.635 \\
0.8 \\
0.655 \\
0.75 \\
0.655 \\
0.775 \\
0.705 \\
0.66 \\
0.74 \\
0.68 \\
0.56 \\
0.68 \\
0.765 \\
0.74 \\
0.46 \\
0.63 \\
0.63 \\
0.55 \\
0.71 \\
0.69 \\
0.725 \\
0.66 \\
0.655 \\
0.665 \\
0.575 \\
0.79 \\
0.665 \\
0.61 \\
0.64 \\
0.79 \\
0.715 \\
0.665 \\
0.805 \\
0.835 \\
0.54 \\
0.71 \\
0.73 \\
0.58 \\
0.835 \\
0.89 \\
0.56 \\
0.845 \\
0.745 \\
0.74 \\
0.55 \\
0.605 \\
0.68 \\
0.69 \\
0.775 \\
0.57 \\
};

\addplot[mark=*, boxplot, boxplot/draw position=6]
table[row sep=\\, y index=0] {
data
0.685 \\
0.58 \\
0.955 \\
0.755 \\
0.815 \\
0.71 \\
0.62 \\
0.625 \\
0.91 \\
0.62 \\
0.66 \\
0.6 \\
0.71 \\
0.69 \\
0.575 \\
0.715 \\
0.64 \\
0.745 \\
0.825 \\
0.84 \\
0.625 \\
0.675 \\
0.805 \\
0.6 \\
0.795 \\
0.75 \\
0.645 \\
0.66 \\
0.55 \\
0.8 \\
0.63 \\
0.6 \\
0.57 \\
0.63 \\
0.605 \\
0.79 \\
0.805 \\
0.685 \\
0.62 \\
0.615 \\
0.73 \\
0.74 \\
0.69 \\
0.86 \\
0.58 \\
0.685 \\
0.76 \\
0.895 \\
0.65 \\
0.89 \\
};

\addplot[mark=*, boxplot, boxplot/draw position=7]
table[row sep=\\, y index=0] {
data
0.5 \\
0.83 \\
0.755 \\
0.785 \\
0.75 \\
0.84 \\
0.595 \\
0.69 \\
0.54 \\
0.825 \\
0.785 \\
0.86 \\
0.82 \\
0.61 \\
0.71 \\
0.7 \\
0.515 \\
0.73 \\
0.99 \\
0.86 \\
0.865 \\
0.62 \\
0.715 \\
0.725 \\
0.68 \\
0.755 \\
0.585 \\
0.815 \\
0.71 \\
0.82 \\
0.56 \\
0.67 \\
0.77 \\
0.7 \\
0.68 \\
0.675 \\
0.83 \\
0.755 \\
0.715 \\
0.725 \\
0.775 \\
0.815 \\
0.825 \\
0.84 \\
0.925 \\
0.8 \\
0.835 \\
0.78 \\
0.61 \\
0.795 \\
};

\addplot[mark=*, boxplot, boxplot/draw position=8]
table[row sep=\\, y index=0] {
data
0.46 \\
0.88 \\
0.81 \\
0.785 \\
0.75 \\
0.86 \\
0.615 \\
0.67 \\
0.58 \\
0.805 \\
0.785 \\
0.84 \\
0.875 \\
0.61 \\
0.66 \\
0.72 \\
0.505 \\
0.76 \\
0.99 \\
0.885 \\
0.85 \\
0.63 \\
0.735 \\
0.73 \\
0.7 \\
0.75 \\
0.605 \\
0.82 \\
0.77 \\
0.815 \\
0.545 \\
0.665 \\
0.805 \\
0.69 \\
0.695 \\
0.68 \\
0.815 \\
0.81 \\
0.74 \\
0.755 \\
0.79 \\
0.775 \\
0.84 \\
0.89 \\
0.925 \\
0.87 \\
0.83 \\
0.815 \\
0.59 \\
0.795 \\
};

\addplot[mark=*, boxplot, boxplot/draw position=9]
table[row sep=\\, y index=0] {
data
0.63 \\
0.695 \\
0.795 \\
0.755 \\
0.72 \\
0.61 \\
0.915 \\
0.925 \\
0.71 \\
0.73 \\
0.655 \\
0.775 \\
0.63 \\
0.65 \\
0.85 \\
0.805 \\
0.735 \\
0.71 \\
0.82 \\
0.745 \\
0.7 \\
0.775 \\
0.915 \\
0.785 \\
0.76 \\
0.72 \\
0.68 \\
0.855 \\
0.685 \\
0.855 \\
0.885 \\
0.61 \\
0.795 \\
0.91 \\
0.775 \\
0.755 \\
0.72 \\
0.76 \\
0.61 \\
0.77 \\
0.84 \\
0.7 \\
0.585 \\
0.79 \\
0.83 \\
0.71 \\
0.845 \\
0.81 \\
0.78 \\
0.965 \\
};

\addplot[mark=*, boxplot, boxplot/draw position=10]
table[row sep=\\, y index=0] {
data
0.66 \\
0.525 \\
0.71 \\
0.625 \\
0.72 \\
0.905 \\
0.77 \\
0.825 \\
0.66 \\
0.705 \\
0.685 \\
0.81 \\
0.9 \\
0.625 \\
0.925 \\
0.69 \\
0.685 \\
0.79 \\
0.76 \\
0.965 \\
0.655 \\
0.865 \\
0.725 \\
0.925 \\
0.82 \\
0.745 \\
0.855 \\
0.65 \\
0.72 \\
0.89 \\
0.675 \\
0.75 \\
0.73 \\
0.62 \\
0.845 \\
0.8 \\
0.83 \\
0.86 \\
0.69 \\
0.805 \\
0.905 \\
0.745 \\
0.84 \\
0.65 \\
0.815 \\
0.88 \\
0.915 \\
0.78 \\
0.705 \\
0.815 \\
};

\addplot[mark=*, boxplot, boxplot/draw position=11]
table[row sep=\\, y index=0] {
data
0.77 \\
0.775 \\
0.79 \\
0.525 \\
0.75 \\
0.66 \\
0.835 \\
0.515 \\
0.74 \\
0.835 \\
0.87 \\
0.745 \\
0.84 \\
0.75 \\
0.855 \\
0.9 \\
0.81 \\
0.66 \\
0.695 \\
0.79 \\
0.895 \\
0.895 \\
0.78 \\
0.64 \\
0.81 \\
0.89 \\
0.915 \\
0.755 \\
0.78 \\
0.8 \\
0.755 \\
0.86 \\
0.65 \\
0.75 \\
0.745 \\
0.765 \\
0.845 \\
0.845 \\
0.735 \\
0.88 \\
0.615 \\
0.68 \\
0.89 \\
0.9 \\
0.705 \\
0.62 \\
0.855 \\
0.895 \\
0.635 \\
0.84 \\
};

\addplot[mark=*, boxplot, boxplot/draw position=12]
table[row sep=\\, y index=0] {
data
0.785 \\
0.73 \\
0.695 \\
0.65 \\
0.625 \\
0.895 \\
0.61 \\
0.685 \\
0.79 \\
0.765 \\
0.9 \\
0.96 \\
0.855 \\
0.88 \\
0.73 \\
0.875 \\
0.785 \\
0.785 \\
0.9 \\
0.825 \\
0.845 \\
0.845 \\
0.81 \\
0.745 \\
0.62 \\
0.505 \\
0.895 \\
0.81 \\
0.895 \\
0.78 \\
0.87 \\
0.965 \\
0.76 \\
0.705 \\
0.925 \\
0.825 \\
0.82 \\
0.885 \\
0.935 \\
0.82 \\
0.805 \\
0.77 \\
0.845 \\
0.575 \\
0.78 \\
0.865 \\
0.78 \\
0.92 \\
0.935 \\
0.87 \\
};

\addplot[mark=*, boxplot, boxplot/draw position=13]
table[row sep=\\, y index=0] {
data
0.98 \\
0.825 \\
0.89 \\
0.66 \\
0.655 \\
0.765 \\
0.735 \\
0.97 \\
0.82 \\
0.895 \\
0.955 \\
0.975 \\
0.77 \\
0.55 \\
0.935 \\
0.795 \\
0.62 \\
0.915 \\
0.92 \\
0.855 \\
0.885 \\
0.935 \\
0.915 \\
0.665 \\
0.79 \\
0.58 \\
0.805 \\
0.945 \\
0.895 \\
0.76 \\
0.75 \\
0.835 \\
0.705 \\
0.885 \\
0.81 \\
0.665 \\
0.775 \\
0.795 \\
0.87 \\
0.78 \\
0.76 \\
0.74 \\
0.815 \\
0.905 \\
0.87 \\
0.86 \\
0.84 \\
0.805 \\
0.79 \\
0.92 \\
};

\addplot[mark=*, boxplot, boxplot/draw position=14]
table[row sep=\\, y index=0] {
data
0.98 \\
0.87 \\
0.895 \\
0.69 \\
0.67 \\
0.755 \\
0.77 \\
0.97 \\
0.825 \\
0.885 \\
0.95 \\
0.98 \\
0.78 \\
0.58 \\
0.92 \\
0.87 \\
0.62 \\
0.92 \\
0.915 \\
0.93 \\
0.89 \\
0.98 \\
0.92 \\
0.67 \\
0.835 \\
0.62 \\
0.8 \\
0.96 \\
0.93 \\
0.77 \\
0.805 \\
0.825 \\
0.75 \\
0.88 \\
0.81 \\
0.665 \\
0.795 \\
0.78 \\
0.85 \\
0.785 \\
0.765 \\
0.76 \\
0.85 \\
0.905 \\
0.905 \\
0.905 \\
0.855 \\
0.895 \\
0.79 \\
0.93 \\
};

\addplot[mark=*, boxplot, boxplot/draw position=15]
table[row sep=\\, y index=0] {
data
0.775 \\
0.78 \\
0.75 \\
0.945 \\
0.835 \\
0.645 \\
0.895 \\
0.72 \\
0.7 \\
0.745 \\
0.86 \\
0.68 \\
0.865 \\
0.97 \\
0.87 \\
0.94 \\
0.975 \\
0.745 \\
0.97 \\
0.925 \\
0.75 \\
0.78 \\
0.935 \\
0.785 \\
0.785 \\
0.84 \\
0.675 \\
0.87 \\
0.95 \\
0.91 \\
0.735 \\
0.935 \\
0.825 \\
0.865 \\
0.865 \\
0.74 \\
0.92 \\
0.725 \\
0.865 \\
0.77 \\
0.825 \\
0.78 \\
0.86 \\
0.745 \\
0.98 \\
0.945 \\
0.825 \\
0.71 \\
0.635 \\
0.695 \\
};
}{0.1}{Output connectivity}

        }
    }
    \caption{Plots for optimal input task 2 - part 3}
\end{figure*}
