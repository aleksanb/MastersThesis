\section*{\Huge Abstract}
\addcontentsline{toc}{chapter}{Abstract}
$\\[0.5cm]$
%Working title: "RBN as A Model for Matter Reservoirs".

% Introduction
\noindent Reservoir Computing, a relatively new approach to machine learning,
utilizes untrained recurrent neural nets as a reservoir of dynamics to preprocess some temporal task,
making it separable with a linear readout layer
Originating from the study of Liquid State Machines and Echo State Networks,
potentially any sparsely connected network containing feedforward and feedback loops can be a reservoir.
Random Boolean Networks are such sparsely connected network that may be suitable for Reservoir Computing.

% Problemstilling
Reservoir Computing can be used with physical or simulated reservoirs.
When using nontraditional physical devices for computation,
as in evolution-in-materio,
one is often restricted in how one can perturb and read out the underlying substrate.
Random Boolean network reservoir computing systems (RRC systems) may be a useful abstraction over a physical device.

% Method
The properties of these RRC systems are investigated with a series of experiments simulated in a self developed reservoir computing toolbox.
The findings are discussed in light of physical reservoirs,
with the end goal of using RBNs as a model for matter reservoirs.

% Resultater
Experiments confirm that the required reservoir size increases with the difficulty of the task at hand,
with the largest factor being how many bits of input the reservoir is required to remember.
Simulation of RRC systems can therefore aid in deciding the optimal size of physical reservoirs,
given a bridge between the computational power of the reservoir and RBNs can be deduced.

Optimal reservoir perturbance is found to lie at roughly 50\% of the size of the reservoir for RBNs with $K=3$.
When using smaller slices of a reservoir for computation,
lower amounts of total perturbation will be required as long as these perturbations are located within the same topological area.

Results also show that subsets of larger reservoirs will perform at least as well as a separate reservoir of equal size.
Any interference from the unused parts of the reservoir is either minimal or slightly positive.

Finally, no relationship is found between the attractors of an RBN and its performance in a RRC system.
It can therefore not be used for guiding the construction of accurate RBNs.

\cleardoublepage
