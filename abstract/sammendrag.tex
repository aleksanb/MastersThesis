\section*{\Huge Sammendrag}
\addcontentsline{toc}{chapter}{Sammendrag}
$\\[0.5cm]$

\noindent Reservoir computing (RC) er en relativt ny teknikk innen maskinlæring.
Et utrent rekurrent nevralt nettverk benyttes som et reservoar fyllt av dynamikk for å preprossesere et temporært problem,
og dermed gjøre det separerbart med et lineært avlesingslag.
Alle glissne nettverk med tilbakekoblinger kan potensielt benyttes som et reservoar.
Et slikt nettverk det tilfeldige Boolske nettverket (RBN),
som tidligere har blitt brukt i RC-systemer.

Reservoir computing kan benytte både fysiske og simulerte reservoar.
Når et fysisk reservoar brukes for beregning, som i evolution-in-materio,
har man ofte begrenset adgang til å påvirke samt lese ut verdier fra materialet.
I denne avhandlingen vil følgende egenskaper ved RBN RC-systemer bli undersøkt og relatert til fysiske reservoarer:
Hvordan kompleksiteten til oppgaven påvirker størrelsen på reservoaret,
hvor mye påvirkning som er optimalt,
hva ytelsen på subsett av større rerservoarer er,
og hva forbindelsen mellom et RBNs tiltrekkere og dets ytelse som reservoar er.

Våre eksperimenter bekrefter at den nødvendige størrelsen på reservoaret øker i takt med vanskelighetsgraden på oppgaven som skal løses.
Den største faktoren er hvor mange bits med input som må huskes.
Simuleringer av RBN RC-systemer kan derfor hjelpe i å avgjøre den optimale størrelsen på et fysisk reservoar,
gitt man finner en bro mellom målet for beregningskraft på det simulerte og fysiske reservoaret.
Optimal reservoarpåvirkning er funnet til å ligge på rundt 50\% av størrelsen på reservoaret (for RBN med konnektivitet 3).
Når man bruker et subsett av et reservoar for beregning så trengs det mindre mengder med påvirkning, gitt at påvirkingen skjer i samme del av reservoaret som leses av.
Våre resultater viser også at subsett av større reservoar yter like bra som separate reservoar av den størrelsen.
En eventuell interferens mellom den ubrukte delen av reservoaret og den brukte er enten neglisjerbar eller svakt positiv.
Det viste seg ingen sammenheng mellom tiltrekkerne til et RBN og dets ytelse i et RBN RC-system.
Dette kan dermed ikke brukes for å guide konstruksjonen av nøyaktige RBN.

\cleardoublepage
