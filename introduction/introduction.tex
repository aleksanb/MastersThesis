\chapter{Introduction}
%Working title: "RBN as A Model for Matter Reservoirs".

%\textbf{Introducing reservoir computing}
Reservoir Computing (RC) is a category of machine learning that sprang out from the study of recurrent neural networks (RNNs).
In short, it utilizes the dynamics of some complex system dubbed a 'reservoir' to preprocess a timeseries problem,
transforming it from a temporal to a spacial one in the reservoir,
making it separable with a simple readout layer \cite{lukovsevivcius2012reservoir}.

%\textbf{Theoretical RESERVOIRS}
Reservoir Computing can be used both with physical and theoretical reservoirs.
The original theoretical Echo State Networks \cite{jaeger2002adaptive} and Liquid State Machines \cite{natschlager2002liquid} provided the original basis for theoretical Reservoir Computing systems.
As an alternative abstraction to RRNs,
Random Boolean Networks \cite{gershenson2004introduction} were introduced and sucessfully used in \cite{rbn-reservoir}.
RBNs are a useful abstraction over many physical phenomena, most famously used by Kauffman \cite{kauffman1969metabolic} as a model for the genetic regulatory network.
In my pre-thesis paper, the author reproduces findings from \cite{rbn-reservoir} as well as investigating the reusability of trained readout layers,
discovering a neturality in the state-space of possible RBNs.

%\textbf{PHYSICAL RESERVOIRS}
Physical reservoirs take many shapes and forms.
In the ever-so-cited 'Pattern recognition in a bucket' paper \cite{fernando2003pattern},
the authors use a bucket of water to successfully perform speech recognition.
In \cite{farstad2015evolving},
a carbon-nanotube based substrate is successfully used to evolve binary logic gates as well as stable cellular automata.
These can then be used to build a CA-based computing paradigm over this unconventional hardware.
As Cellular Automata are a special case of Random Boolean Networks,
any general findings of the latter may be applicable to RC systems using the former.
The wider category of attempting to harness the power of unconventional physical devices for computation,
usually aided by artificial evolution,
is known as evolution-in-materio \cite{miller2002evolution}.

%\textbf{PROBLEM}
When using nontraditional physical devices for computation, as in evo-materio,
one is often restricted in how one can perturb and read out the underlying substrate.
It might be prohibitively expensive or technologically infeasible to read the entire state of the entire reservoir,
and the values sampled might be affected by its neighborhood in unknown ways to us.
In a stem-cell based system \cm{Har du no sitering for dette, gunnar. Tenker St.Olavs},
the topology and size of the network might change over time.
Having to reinstrument such a system might be impractical,
so the question becomes how the unused parts of the reservoir impact the parts used for computation.
Finding the minimum required size of a reservoir for given tasks allows us to create physical reservoirs with no-more-than-needed computational abilities,
saving on physical cost.

%\textbf{HYPOTHESIS}
It has been shown that Reservoir Computing using Random Boolean Networks is a feasible approach to solving binary time-series problems,
as well as being a potentially useful abstraction over physical Reservoir Computing devices.
In this thesis, the properties of these RRC systems are further investigated,
with the motivation of using the results as a template for instrumenting and performing computations on physical Reservoir Computing systems.

%\textbf{MY WORK}
The following questions will be answered in this thesis:
\begin{enumerate}
    \item How small can a RRC system be while still solving its task at $ \geq 98\% $ accuracy?
    \item Does the optimal amount of reservoir perturbation depend on the task at hand?
    \item Does one have to read the state of the entire reservoir to maintain task accuracy?
    \item Is there a correlation between the topological characteristics of the RBN (number of attractors, attractor length) and its performance as a reservoir?
\end{enumerate}

The thesis is structured as follows:
Chapter \ref{chapter:background} provides background information on reservoir computing, random boolean networks, evo-materio, and using random boolean networks for reservoir computing (my pre-project thesis amongst others).
Chapter \ref{chapter:methodology} describes the experimental setup and methodology used for performing experiments.
Chapter \ref{chapter:experiments} contains an in-depth descriptions of each experiment,
the results obtained, and a discussion thereof.
Chapter \ref{chapter:conclusion} contains further work and concludes the thesis.

\cleardoublepage
