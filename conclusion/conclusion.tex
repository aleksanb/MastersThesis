\chapter{Conclusion}

This thesis covered three main topics within the field of RRC systems,
all by experimentation and empirical observations:

\begin{enumerate}
    \item Required reservoir sizes and optimal input connectivities.
    \item Required output connectivities.
    \item Reservoir dynamics and their relation to task accuracy.
\end{enumerate}

Required reservoir sizes varied with the difficulty of the task at hand,
the largest factor being how many timesteps the reservoir was required to remember back in time.

It was found that for larger reservoirs, reading out a subset of equal size to the smallest sufficient reservoir is enough to complete the same task.
Little interference from the unused part of the reservoir was observed,
laying empirical evidence behind the fact that one doesn't have to read out an entire reservoir .

Finally, there wasn't found a relationship between an RBNs attractors and its performance in a RRC system.
One will have to continue to rely on RBN criticality as an indicator of RBN performance.

\todo{Stuff about the connection to physical systems i guess?}
