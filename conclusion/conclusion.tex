\chapter{Conclusion}
\label{chapter:conclusion}

\section{Conclusion}

This thesis covered four main topics within the field of RRC systems,
all by experimentation and empirical observations:

\begin{enumerate}
    \item Required reservoir sizes and optimal input connectivities.
    \item Reservoir subset performance
    \item Reservoir dynamics and their relation to task accuracy.
\end{enumerate}

Required reservoir sizes varied with the difficulty of the task at hand,
the largest factor being how many timesteps the reservoir was required to remember back in time.

It was found that for larger reservoirs, reading out a subset of equal size to the smallest sufficient reservoir is enough to complete the same task.
Little interference from the unused part of the reservoir was observed,
laying empirical evidence behind the fact that one doesn't have to read out an entire reservoir .

Finally, there wasn't found a relationship between an RBNs attractors and its performance in a RRC system.
One will have to continue to rely on RBN criticality as an indicator of RBN performance.

\todo{Stuff about the connection to physical systems i guess?}

\section{Further work}

While looking at the performance of reservoir subsets in section \ref{section:output_connectivity-discussion},
changes to reservoir topology during computation was discussed.
When using neurons for computation,
as in the rat neuron airplane automation experiment \cite{demarse2005adaptive},
neurons would mature, grow, and change connections by themselves over time.
It would be interesting to study how much of an RBN in a RC system can be manipulated before its predictive power collapses,
and the readout layer has to be retrained.
These changes would include flipping bits in transition tables and changing internal edges in the network.
If one can relate the network's robustness to perturbations to a regression model's resistance to variance in the predictor variables,
a metric of RRC robustness can be developed.
It would relate the amount of change in the RBN to the expected reduction of reservoir accuracy.

Another question is in what degree does a reservoirs optimal input connectivity change with regards to what physical subset of the substrate is used?
The topology of an RBN is randomly generated and doesn't inherently have a physical mapping,
so one would have to be created for this experiment.
In the water bucket RC system \cite{fernando2003pattern},
one would expect that perturbing one part of the reservoir would result in a larger effect in that area rather than the other side of the bucket.
If one were to observe this smaller part of the total reservoir only,
the hypothesis would be that the perturbance within this restricted area would have to be smaller than if one were to use the entire reservoir for computation.
